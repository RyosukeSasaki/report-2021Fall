\documentclass[uplatex,a4j,11pt,dvipdfmx]{jsarticle}
\usepackage{listings,jvlisting}
\bibliographystyle{junsrt}

\usepackage{url}

\usepackage{graphicx}
\usepackage{gnuplot-lua-tikz}
\usepackage{pgfplots}
\usepackage{tikz}
\usepackage{amsmath,amsfonts,amssymb}
\usepackage{bm}
\usepackage{siunitx}

\makeatletter
\def\fgcaption{\def\@captype{figure}\caption}
\makeatother
\newcommand{\setsections}[3]{
\setcounter{section}{#1}
\setcounter{subsection}{#2}
\setcounter{subsubsection}{#3}
}
\newcommand{\mfig}[3][width=15cm]{
\begin{center}
\includegraphics[#1]{#2}
\fgcaption{#3 \label{fig:#2}}
\end{center}
}
\newcommand{\gnu}[2]{
\begin{figure}[hptb]
\begin{center}
\input{#2}
\caption{#1}
\label{fig:#2}
\end{center}
\end{figure}
}

\begin{document}
\title{セキュリティ総論D レポートNo.3}
\author{慶應義塾大学 61908697 佐々木良輔}
\date{}
\maketitle
\subsection*{CSIRTにおける自分の役割}
私がCISRTにおいて最もふさわしいと思う役割は「ソリューションアナリスト」または「システム運用担当」であると考える.
その理由として,現在私がアルバイトで所属している企業での役割が挙げられる.

その企業は公共交通システム系のスタートアップであり,
ハードウェアや制御システムの設計などを行い,
それを地方公共団体や施設などに販売するベンダーであると言える.
私はそこで主に制御システムの設計や実装を行っているが,
組み込み機器である以上設計段階である程度セキュリティを考慮する必要が出てくる.
以上の経緯から,私は自組織のシステムについて多少の知識があり,
また小規模なスタートアップであるため,
現段階では設計・実装者がその機器の取り扱いに最も詳しい状況があり,
運用開始初期においては担当箇所の維持管理は私がある程度考える必要があると思われる.
また組織の規模が拡大した場合,
私の開発担当箇所に関してはドキュメントの構築や社内での教育なども行う可能性もある.
またビジネスビジョンについても組織内で十分に共有されている.

また,将来的には開発した交通システムのパッケージを販売し,運用は各地方公共団体や運用を行う企業に任せるといったことも考えられる.
そういった場合は運用主体からの外注を引き受けると言ったことも考えられ,
ベンダー側としてセキュリティの保全に関わるといったことも考えられる.

以上から現時点では私の参加する企業にはまだ明確にCISRTに関連する部門は存在しないが,
現状の自分の役割を当てはめるならば「ソリューションアナリスト」または「システム運用担当」であると考える.
\end{document}