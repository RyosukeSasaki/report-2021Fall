\documentclass[uplatex,a4j,11pt,dvipdfmx]{jsarticle}
\usepackage{listings,jvlisting}
\bibliographystyle{jplain}

\usepackage{url}

\usepackage{graphicx}
\usepackage{gnuplot-lua-tikz}
\usepackage{pgfplots}
\usepackage{tikz}
\usepackage{amsmath,amsfonts,amssymb}
\usepackage{bm}
\usepackage{siunitx}

\makeatletter
\def\fgcaption{\def\@captype{figure}\caption}
\makeatother
\newcommand{\setsections}[3]{
\setcounter{section}{#1}
\setcounter{subsection}{#2}
\setcounter{subsubsection}{#3}
}
\newcommand{\mfig}[3][width=15cm]{
\begin{center}
\includegraphics[#1]{#2}
\fgcaption{#3 \label{fig:#2}}
\end{center}
}
\newcommand{\gnu}[2]{
\begin{figure}[hptb]
\begin{center}
\input{#2}
\caption{#1}
\label{fig:#2}
\end{center}
\end{figure}
}

\begin{document}
\title{パワーエレクトロニクス No.5}
\author{61908697 佐々木良輔}
\date{}
\maketitle
\subsubsection*{問1.(1)}
図\ref{fig:pwm.png}に抵抗$R$の両端電圧$V$を示す.
グラフはQucs-Sを用いて描画した.
\mfig[width=10cm]{pwm.png}{電圧$V$}
\subsubsection*{問1.(2)}
平均電圧は
\begin{align}
  V_{\rm ave}=0.4\times100=40\ \si{\volt}
\end{align}
\subsubsection*{問2}
以下の表\ref{tab:cond}の各条件の波形を図\ref{fig:4e-3_20_05.png}から図\ref{fig:2e-3_20_05.png}に示す.
条件(1)と条件(4)では電源電圧が$3/4$倍になっているがduty比がおよそ$4/3$倍になっており,平均電圧が等しくなっていることがわかる.
また条件(1)と条件(5)では周期が異なるがduty比と電源電圧が同じため平均電圧が等しいことがわかる.
\begin{table}[h]
\caption{条件}
\label{tab:cond}
\centering
\begin{tabular}{cccc}
\hline
条件&周期&電源電圧&duty比\\
\hline \hline
1&0.004&20&0.5\\
1&0.004&20&0.2\\
1&0.004&20&0.8\\
4&0.004&15&0.66\\
5&0.002&20&0.5\\
\hline
\end{tabular}
\end{table}
\newpage
\mfig[width=10cm]{4e-3_20_05.png}{条件(1)の波形}
\mfig[width=10cm]{4e-3_20_02.png}{条件(2)の波形}
\newpage
\mfig[width=10cm]{4e-3_20_08.png}{条件(3)の波形}
\mfig[width=10cm]{4e-3_15_066.png}{条件(4)の波形}
\newpage
\mfig[width=10cm]{2e-3_20_05.png}{条件(5)の波形}
\end{document}