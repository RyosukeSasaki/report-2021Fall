\documentclass[uplatex,a4j,11pt,dvipdfmx]{jsarticle}
\usepackage{listings,jvlisting}
\bibliographystyle{jplain}

\usepackage{url}

\usepackage{graphicx}
\usepackage{gnuplot-lua-tikz}
\usepackage{pgfplots}
\usepackage{tikz}
\usepackage{amsmath,amsfonts,amssymb}
\usepackage{bm}
\usepackage{siunitx}

\makeatletter
\def\fgcaption{\def\@captype{figure}\caption}
\makeatother
\newcommand{\setsections}[3]{
\setcounter{section}{#1}
\setcounter{subsection}{#2}
\setcounter{subsubsection}{#3}
}
\newcommand{\mfig}[3][width=15cm]{
\begin{center}
\includegraphics[#1]{#2}
\fgcaption{#3 \label{fig:#2}}
\end{center}
}
\newcommand{\gnu}[2]{
\begin{figure}[hptb]
\begin{center}
\input{#2}
\caption{#1}
\label{fig:#2}
\end{center}
\end{figure}
}

\begin{document}
\title{パワーエレクトロニクス No.10}
\author{61908697 佐々木良輔}
\date{}
\maketitle
\subsection*{フォワードコンバーター}
$S$の導通時には$v_{n_1}=E$となる.したがって$n_2$の電圧は$v_{n_2}=E\cdot n_2/n_1$となる.
また$D_2$には順方向の電流が流れているので$v_{D_2}=0$となる.
したがって$D_2$が導通であるから$D$には電圧$v_{n_2}$の逆バイアスがかかるので$v_D=-E\cdot n_2/n_1$となる.
また$n_3$には$D_3$によって電流が流れないので$v_{n_3}=0$, $v_{D_3}=-E$となる.
以上から
\begin{align*}
  v_{n_1}=100\ \si{\volt},\ \ v_{n_2}=20\ \si{\volt},\ \ v_{n_3}=0\ \si{\volt}\\
  v_D=-20\ \si{\volt},\ \ v_{D_2}=0\ \si{\volt}\ \ v_{D_3}=-100\ \si{\volt}
\end{align*}
となる.
次に$S$の非導通時には$n_1$に逆起電力によって$v_{n_1}=-E$の電圧が発生する.
このとき$n_3$の負側には$-E$の電圧が生じているため$D_3$には順バイアスがかかり$v_{D_3}=0$となる.
また$v_{n_3}=0-(-E)=E$となる.
また$n_2$も逆起電力によって$v_{n_2}=-E\cdot n_2/n_1$となっている.
また$D$には順バイアスが加わるので$v_D=0$となる
したがって$v_{D_2}=-E\cdot n_2/n_1$となる.
以上から
\begin{align*}
  v_{n_1}=-100\ \si{\volt},\ \ v_{n_2}=-20\ \si{\volt},\ \ v_{n_3}=100\ \si{\volt}\\
  v_D=0\ \si{\volt},\ \ v_{D_2}=-20\ \si{\volt}\ \ v_{D_3}=0\ \si{\volt}
\end{align*}
となる.
\subsection*{フライバックコンバーター}
$S$の導通時は明らかに$v_S=0$, $v_{n_1}=E$である.
また$n_2$の極性が逆なので$v_{n_2}=-E\cdot n_2/n_1$となる.
$C$が十分大きいことから$v_R$は定常であると考えるが,その値は後ほど計算する.
これがOFFになった瞬間を考えると$v_{n_1}=E$, $v_S=-2E$, $v_{n_2}=E\cdot n_2/n_1$となる.
このとき$D$は導通状態になるので$v_D=0$,となる.
したがって$v_R=d\cdot E\cdot n_2/n_1$である.
以上から導通時は
\begin{align*}
  v_{n_1}=140\ \si{\volt},\ \ v_{n_2}=-14\ \si{\volt},\ \ v_{S}=0\ \si{\volt}\\
  v_{D}=-21\ \si{\volt},\ \ v_{R}=7\ \si{\volt}
\end{align*}
非導通時は
\begin{align*}
  v_{n_1}=140\ \si{\volt},\ \ v_{n_2}=14\ \si{\volt},\ \ v_{S}=-280\ \si{\volt}\\
  v_{D}=0\ \si{\volt},\ \ v_{R}=7\ \si{\volt}
\end{align*}
となる.
\end{document}