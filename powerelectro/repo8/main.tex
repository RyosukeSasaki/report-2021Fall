\documentclass[uplatex,a4j,11pt,dvipdfmx]{jsarticle}
\usepackage{listings,jvlisting}
\bibliographystyle{jplain}

\usepackage{url}

\usepackage{graphicx}
\usepackage{gnuplot-lua-tikz}
\usepackage{pgfplots}
\usepackage{tikz}
\usepackage{amsmath,amsfonts,amssymb}
\usepackage{bm}
\usepackage{siunitx}

\makeatletter
\def\fgcaption{\def\@captype{figure}\caption}
\makeatother
\newcommand{\setsections}[3]{
\setcounter{section}{#1}
\setcounter{subsection}{#2}
\setcounter{subsubsection}{#3}
}
\newcommand{\mfig}[3][width=15cm]{
\begin{center}
\includegraphics[#1]{#2}
\fgcaption{#3 \label{fig:#2}}
\end{center}
}
\newcommand{\gnu}[2]{
\begin{figure}[hptb]
\begin{center}
\input{#2}
\caption{#1}
\label{fig:#2}
\end{center}
\end{figure}
}

\begin{document}
\title{パワーエレクトロニクス No.8}
\author{61908697 佐々木良輔}
\date{}
\maketitle
\subsection*{演習1}
\subsubsection*{(1)}
ダイオードはサイリスタと異なりゲート信号に関わらず順バイアスに対しては導通する.
したがって$\alpha=0$とみなすことができるので,
電圧の平均値は
\begin{align}
  E_d=\frac{3}{\pi}\int^{5\pi/6}_{\pi/6}\sqrt{2}V\sin\theta d\theta=\frac{3\sqrt{6}V}{\pi}\simeq 270.2\ \si{\volt}
\end{align}
平均電流は
\begin{align}
  I_d=\frac{E_d}{R}\simeq 5.403\ \si{\ampere}
\end{align}
\subsubsection*{(2)}
$L/R\gg 1$から直流電流が一定と考えると,相電流$i_1$は導通時には$I_d$に一致すると考える.
したがって相電流の実効値は
\begin{align}
  \begin{split}
    I_e&=\sqrt{\frac{1}{2\pi}\int_0^{2\pi}i_1^2(\theta)d\theta}\\
    &=\sqrt{\frac{1}{3}(I_d)^2+\frac{1}{3}(-I_d)^2}\\
    &=I_d\sqrt{\frac{2}{3}}\simeq4.412\ \si{\ampere}
  \end{split}
\end{align}

次に$i_1$をフーリエ変換する.
$i_1$が寄関数になるように位相を決める.
このときフーリエ係数は
\begin{align}
  a_n&=0\\
  \begin{split}
    b_n&=\frac{1}{\pi}\int_0^{2\pi}i_1\sin n\theta d\theta\\
    &=\frac{I_d}{\pi} \int_{\pi/6}^{5\pi/6}\sin n\theta d\theta-\frac{I_d}{\pi} \int_{7\pi/6}^{11\pi/6}\sin n\theta d\theta\\
    &=\frac{2I_d}{\pi}\int_{\pi/6}^{5\pi/6}\sin n\theta d\theta\\
    &=\frac{2I_d}{n\pi}\left(\cos\frac{n\pi}{6}-\cos\frac{5n\pi}{6}\right)
  \end{split}
\end{align}
よって基本波実効値は
\begin{align}
  \begin{split}
    I_1&=\sqrt{\frac{1}{2\pi}\frac{12I_d^2}{\pi^2}\int_0^{2\pi}\sin^2\theta d\theta}\\
    &=\frac{\sqrt{6}I_d}{\pi}=4.213\ \si{\ampere}
  \end{split}
\end{align}
また位相遅れ角は0なので基本波力率は
\begin{align}
  \cos 0=1
\end{align}
\subsubsection*{(3)}
両辺を$\pi/6+\alpha$から$\pi/6+\alpha+u$まで積分する.
\begin{align}
  \begin{split}
    (左辺)=\omega l_s\int_{\pi/6+\alpha}^{\pi/6+\alpha+u}\frac{d i_1}{d\theta}d\theta
    =\omega l_sI_d
  \end{split}
\end{align}
右辺について$v_1=\sqrt{2}V\sin \theta$, $v_3=\sqrt{2}V\sin(\theta-4\pi/3)$と置くと
\begin{align}
  \begin{split}
    (右辺)&=\frac{\sqrt{2}V}{2}\int_{\pi/6+\alpha}^{\pi/6+\alpha+u}\sin \theta-\sin\left(\theta-\frac{4\pi}{3}\right)d\theta\\
    &=\frac{\sqrt{6}V}{2}\int_{\pi/6+\alpha}^{\pi/6+\alpha+u}\sin\left(\theta-\frac{\pi}{6}\right)d\theta\\
    &=\frac{\sqrt{6}V}{2}(\cos \alpha-\cos(\alpha+u))
  \end{split}
\end{align}
以上から
\begin{align}
  I_d=\frac{\sqrt{6}V}{2\omega l_s}(\cos \alpha-\cos(\alpha+u))
\end{align}
また$E_d=3\sqrt{6}V/\pi\cos\alpha$なので
\begin{align}
  \frac{E_d}{I_d}=\frac{6\omega l_s}{\pi}\frac{1}{1-\frac{\cos(\alpha+u)}{\cos\alpha}}
\end{align}
となり,これはインピーダンスのように捉えられる.
ここで$u\rightarrow0$のときは$E_d/I_d\rightarrow\infty$となる.
対して$u$が大きくなると$E_d/I_d$が有限の値を持つことがわかる.
特に$\cos(\alpha+u)/\cos\alpha=-1$のとき,極小となる.
\subsection*{演習2}
図1から図5に各条件でのシミュレーション結果を示す.
これらは正弦波信号と三角波信号の比較で生成されたPWM信号で,正弦波の振幅をduty比に変調したものである.
実際それぞれの図を見ると正弦波の電圧が大きいところではパルス幅が大きく,小さいところではパルス幅が小さくなっていることがわかる.
また条件1から条件3を比較すると三角波の周波数が高ければ高いほどPWMの周波数が高いことがわかる.
また条件1と条件4を比較すると正弦波周波数が高いと変調された信号もそれに伴っていることがわかる.
\newpage
\begin{figure}[htbp]
  \begin{minipage}[b]{0.45\linewidth}
    \centering
    \mfig[width=6cm]{fig/fig1.png}{条件1の結果}
  \end{minipage}
  \begin{minipage}[b]{0.45\linewidth}
    \centering
    \mfig[width=6cm]{fig/fig2.png}{条件2の結果}
  \end{minipage}
\end{figure}
\newpage
\begin{figure}[htbp]
  \begin{minipage}[b]{0.45\linewidth}
    \centering
    \mfig[width=6cm]{fig/fig3.png}{条件3の結果}
  \end{minipage}
  \begin{minipage}[b]{0.45\linewidth}
    \centering
    \mfig[width=6cm]{fig/fig4.png}{条件4の結果}
  \end{minipage}
\end{figure}
\newpage
\mfig[width=6cm]{fig/fig5.png}{条件5の結果}
\end{document}