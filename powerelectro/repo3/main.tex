\documentclass[uplatex,a4j,11pt,dvipdfmx]{jsarticle}
\usepackage{listings,jvlisting}
\bibliographystyle{jplain}

\usepackage{url}

\usepackage{graphicx}
\usepackage{gnuplot-lua-tikz}
\usepackage{pgfplots}
\usepackage{tikz}
\usepackage{amsmath,amsfonts,amssymb}
\usepackage{bm}
\usepackage{siunitx}

\makeatletter
\def\fgcaption{\def\@captype{figure}\caption}
\makeatother
\newcommand{\setsections}[3]{
\setcounter{section}{#1}
\setcounter{subsection}{#2}
\setcounter{subsubsection}{#3}
}
\newcommand{\mfig}[3][width=15cm]{
\begin{center}
\includegraphics[#1]{#2}
\fgcaption{#3 \label{fig:#2}}
\end{center}
}
\newcommand{\gnu}[2]{
\begin{figure}[hptb]
\begin{center}
\input{#2}
\caption{#1}
\label{fig:#2}
\end{center}
\end{figure}
}

\begin{document}
\title{パワーエレクトロニクス No.3}
\author{61908697 佐々木良輔}
\date{}
\maketitle
図\ref{fig:1.jpg}にRCスナバ回路の回路図を示す.
RCスナバ回路はリレーやサイリスタなどのスイッチング素子に並列に用いられる.
スイッチング素子のON時に流れる電流は直流または低周波なので,インピーダンスが高いスナバ回路には流れ込まない.
一方でターンオフ時に発生するサージ電流は高周波なのでスナバ回路に流れ込み,抵抗で熱に変換される.
これによりターンオフ時のスパイクを減衰させ,素子に高電圧がかかるのを防止することができる.\cite{RC}
\mfig[width=6cm]{1.jpg}{RCスナバ回路}
\bibliography{ref.bib}
\end{document}