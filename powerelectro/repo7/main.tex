\documentclass[uplatex,a4j,11pt,dvipdfmx]{jsarticle}
\usepackage{listings,jvlisting}
\bibliographystyle{jplain}

\usepackage{url}

\usepackage{graphicx}
\usepackage{gnuplot-lua-tikz}
\usepackage{pgfplots}
\usepackage{tikz}
\usepackage{amsmath,amsfonts,amssymb}
\usepackage{bm}
\usepackage{siunitx}

\makeatletter
\def\fgcaption{\def\@captype{figure}\caption}
\makeatother
\newcommand{\setsections}[3]{
\setcounter{section}{#1}
\setcounter{subsection}{#2}
\setcounter{subsubsection}{#3}
}
\newcommand{\mfig}[3][width=15cm]{
\begin{center}
\includegraphics[#1]{#2}
\fgcaption{#3 \label{fig:#2}}
\end{center}
}
\newcommand{\gnu}[2]{
\begin{figure}[hptb]
\begin{center}
\input{#2}
\caption{#1}
\label{fig:#2}
\end{center}
\end{figure}
}

\begin{document}
\title{パワーエレクトロニクス No.6}
\author{61908697 佐々木良輔}
\date{}
\maketitle
\subsubsection*{(1)-1}
$R$にかかる電圧$e_d$はダイオードの順方向電圧を無視すれば
\begin{align}
  e_d=|v_s|=|200\sqrt{2}\sin\omega t|
\end{align}
また$L=0$において流れる電流$i_d$は
\begin{align}
  i_d=\frac{|v_s|}{R}
\end{align}
であるので$R$に消費される平均電力$P$は
\begin{align}
  \begin{split}
    P&=\frac{\omega}{2\pi}\int_0^{2\pi/\omega}\frac{e_d^2}{R}dt\\
    &=\frac{1}{2\pi}\frac{2\cdot(200)^2}{50}\int_0^{2\pi}\sin^2\theta d\theta\\
    &=800\ \si{\watt}
  \end{split}
\end{align}
\subsubsection*{(1)-2}
インダクタンス両端の電圧は
\begin{align}
  \omega L\frac{di_d}{d\theta}
\end{align}
であったので$e_d$の平均値$E_d$は
\begin{align}
  E_d=\frac{1}{2\pi}\int_0^{2\pi}e_d d\theta=\frac{1}{2\pi}\left(\int_0^{2\pi}Ri_dd\theta+\omega L\int_0^{2\pi}\frac{di_d}{d\theta}d\theta\right)
\end{align}
ここで$L/R\gg 1$であり$i_d$が定常であると見なせるならば$di_d/d\theta=0$なので
\begin{align}
  E_d=\frac{1}{2\pi}\int_0^{2\pi}Ri_dd\theta=RI_d
\end{align}
となる.したがって平均電力は
\begin{align}
  P=E_dI_d=\frac{E_d^2}{R}
\end{align}
であり$E_d$は
\begin{align}
  E_d=\frac{1}{2\pi}\int_0^{2\pi}|200\sqrt{2}\sin\theta|d\theta=\frac{400\sqrt{2}}{\pi}
\end{align}
であるので
\begin{align}
  P=\frac{1}{50}\left(\frac{400\sqrt{2}}{\pi}\right)^2=648\ \si{\watt}
\end{align}
\subsubsection*{(1)-3}
$L=0$の場合は$e_d$と$i_d$が同位相であるので,電圧が最大のときに電流も最大になっている.
一方で$L/R\gg 1$の場合においては$i_d$が平滑化されたことによりそのピークは$2/\pi$倍に小さくなり,また$e_d$が小さい領域でも値を持つようになった.
これにより電力$e_di_d$の平均は全体として小さくなったと考えられる.
このことはインダクタンスに蓄えられたエネルギーが電圧の小さい期間に放出されたと理解できる.
\subsubsection*{(2)}
関数同士が直交するとは,その関数同士の内積が0になることである.すなわち互いに異なる関数$f(x)$と$g(x)$が直交するとは
\begin{align}
  \langle f(x), g(x)\rangle=\int_{-\infty}^\infty f(x)g(x)dx=0
\end{align}
と表される.このように互いに直交する関数からなる集合を直交関数系と呼び,三角関数やLegendre多項式, Hermite多項式などが該当する.
特に関数が規格化された直交関数系は規格直交系と呼ばれる.
特に正弦関数の直交性を示す.以下の内積を考える.
\begin{align}
  \langle \sin nx,\sin mx\rangle=\int_{-\pi}^{\pi}\sin nx\sin mx\ dx\qquad (n,m\in\mathbb{Z})
\end{align}
正弦波の周期性から積分は1周期のみ行えば十分である.ここで$n\neq m$のとき三角関数の積和公式を用いれば
\begin{align}
  \begin{split}
    \int_{-\pi}^{\pi}\sin nx\cdot\sin mx\ dx&=\int_{-\pi}^{\pi}\frac{\cos(n-m)x-\cos(n+m)x}{2}dx\\
    &=\frac{1}{2}\left(\frac{1}{n-m}\left[\sin(n-m)x\right]_{-\pi}^{\pi}-\frac{1}{n+m}\left[\sin(n+m)x\right]_{-\pi}^{\pi}\right)\\
    &=0
  \end{split}
\end{align}
また$n=m$のとき
\begin{align}
  \begin{split}
    \int_{-\pi}^{\pi}\sin nx\cdot\sin mx\ dx=\int_{-\pi}^\pi\sin^2 nx\ dx=\pi
  \end{split}
\end{align}
以上より
\begin{align}
  \langle \sin nx,\sin mx\rangle=\pi\delta_{n,m}
\end{align}
であり,直交していることがわかる.

上記で述べた通り正弦関数は直交系をなしていることから異なる角周波数成分同士の内積は$0$になる.
したがって$i_s$が
\begin{align}
  i_s=\frac{I_d}{2}+\sum_{n=1}^\infty a_n\sin nx
\end{align}
と展開されるのならば$v_s=A\sin x$との内積は$n=1$の成分を除いて$0$になる.
\end{document}