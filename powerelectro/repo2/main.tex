\documentclass[uplatex,a4j,11pt,dvipdfmx]{jsarticle}
\usepackage{listings,jvlisting}
\bibliographystyle{jplain}

\usepackage{url}

\usepackage{graphicx}
\usepackage{gnuplot-lua-tikz}
\usepackage{pgfplots}
\usepackage{tikz}
\usepackage{amsmath,amsfonts,amssymb}
\usepackage{bm}
\usepackage{siunitx}

\makeatletter
\def\fgcaption{\def\@captype{figure}\caption}
\makeatother
\newcommand{\setsections}[3]{
\setcounter{section}{#1}
\setcounter{subsection}{#2}
\setcounter{subsubsection}{#3}
}
\newcommand{\mfig}[3][width=15cm]{
\begin{center}
\includegraphics[#1]{#2}
\fgcaption{#3 \label{fig:#2}}
\end{center}
}
\newcommand{\gnu}[2]{
\begin{figure}[hptb]
\begin{center}
\input{#2}
\caption{#1}
\label{fig:#2}
\end{center}
\end{figure}
}

\begin{document}
\title{パワーエレクトロニクス No.2}
\author{佐々木良輔}
\date{}
\maketitle
\subsection*{演習1}
実効値$V_{\rm eff}$は
\begin{align}
  \begin{split}
    V_{\rm eff}&=\sqrt{\frac{1}{2\pi}\int_0^{2\pi}v^2(\theta)\ {\rm d}\theta}\\
    &=\sqrt{\frac{1}{2\pi}}\sqrt{\int_\theta^{\pi-\theta}V^2\ {\rm d}\theta+\int^{2\pi-\theta}_{\pi+\theta}V^2\ {\rm d}\theta}\\
    &=V\sqrt{1-\frac{2\theta}{\pi}}
  \end{split}
\end{align}
また平均値は明らかに0なので$V_0=0$,奇関数なので$a_n=0$である.
したがってフーリエ係数$b_n$は
\begin{align}
  \begin{split}
    b_n&=\int_0^{2\pi}v(\theta')\sin n\theta'\ {\rm d}\theta\\
    &=\frac{V}{\pi}\left(\int_\theta^{\pi-\theta}\sin n\theta'\ {\rm d}\theta'-\int_{\pi+\theta}^{2\pi-\theta}\sin n\theta'\ {\rm d}\theta'\right)\\
    &=\frac{V}{n\pi}\left(-\cos n(\pi-\theta)+\cos n\theta+\cos n(2\pi-\theta)-\cos n(\pi+\theta)\right)
  \end{split}
\end{align}
ここで$n$が奇数の時$\cos n\theta=\cos n(2\pi-\theta)=-\cos n(\pi-\theta)=-\cos n(\pi+\theta)$,
$n$が偶数の時$\cos n\theta=\cos n(2\pi-\theta)=\cos n(\pi-\theta)=\cos n(\pi+\theta)$なので
\begin{align}
  \begin{split}
    b_n=\begin{cases}
      \frac{4V}{n\pi}\cos n\theta & (n:奇数)\\
      0 & (n:偶数)
    \end{cases}
  \end{split}
\end{align}
以上から
\begin{align}
  v(\theta')=\sum_{n=1}^{\infty} \frac{4V}{(2n-1)\pi}\cos(2n-1)\theta\sin(2n-1)\theta'
\end{align}
したがって基本波は
\begin{align}
  v_1(\theta')=\frac{4V}{\pi}\cos\theta\sin\theta'
\end{align}
高調波は
\begin{align}
  v_h(\theta')=\sum_{n=2}^{\infty} \frac{4V}{(2n-1)\pi}\cos(2n-1)\theta\sin(2n-1)\theta'
\end{align}
である.よって基本波の実効値は
\begin{align}
  V_1=\frac{4V}{\pi}\cos\theta\frac{1}{\sqrt{2}}=\frac{2\sqrt{2}V\cos\theta}{\pi}
\end{align}
したがってひずみ率は
\begin{align}
  \begin{split}
    \frac{V_H}{V_1}&=\frac{\sqrt{V^2\left(1-\frac{2\theta}{\pi}\right)^2-\frac{8V^2\cos\theta}{\pi^2}}}{\frac{2\sqrt{2}V\cos\theta}{\pi}}\\
    &=\frac{\sqrt{(\pi-2\theta)^2-8\cos^2\theta}}{2\sqrt{2}\cos\theta}    
  \end{split}
\end{align}
となる.
\subsection*{演習2}
直流の場合の消費電力$P_D$は
\begin{align}
  P_D=\frac{V^2}{R}
\end{align}
である.また振幅$\sqrt{2}V$の交流電源の実効値は$V$である.
この電源の電力の平均値は
\begin{align}
  P_A&=\frac{1}{T}\int_0^T{\rm d}t\sqrt{2}V\sin\left(\frac{2\pi t}{T}\right)\cdot\sqrt{2}\frac{V}{R}\sin\left(\frac{2\pi t}{T}\right)\\
  &=\frac{2V^2}{R}\frac{1}{2\pi^2}\int_0^{2\pi}\sin x\ {\rm d}x\\
  &=\frac{V^2}{R}
\end{align}
となる.このことから実効値とはある電圧のひずみの無い交流信号と同じ仕事をできる直流の電圧と言える.
\end{document}