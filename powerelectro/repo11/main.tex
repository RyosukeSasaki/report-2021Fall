\documentclass[uplatex,a4j,11pt,dvipdfmx]{jsarticle}
\usepackage{listings,jvlisting}
\bibliographystyle{jplain}

\usepackage{url}

\usepackage{graphicx}
\usepackage{gnuplot-lua-tikz}
\usepackage{pgfplots}
\usepackage{tikz}
\usepackage{amsmath,amsfonts,amssymb}
\usepackage{bm}
\usepackage{siunitx}

\makeatletter
\def\fgcaption{\def\@captype{figure}\caption}
\makeatother
\newcommand{\setsections}[3]{
\setcounter{section}{#1}
\setcounter{subsection}{#2}
\setcounter{subsubsection}{#3}
}
\newcommand{\mfig}[3][width=15cm]{
\begin{center}
\includegraphics[#1]{#2}
\fgcaption{#3 \label{fig:#2}}
\end{center}
}
\newcommand{\gnu}[2]{
\begin{figure}[hptb]
\begin{center}
\input{#2}
\caption{#1}
\label{fig:#2}
\end{center}
\end{figure}
}

\begin{document}
\title{パワーエレクトロニクス No.10}
\author{61908697 佐々木良輔}
\date{}
\maketitle
\subsubsection*{(1)}
$v_{un}$は奇関数なので余弦波成分は$0$である.
正弦波成分は周期を$T$とすれば
\begin{align}
  \begin{split}
    b_n=&\frac{2}{T}\int_0^Tv_{un}(t)\sin\frac{2\pi nt}{T}\ dt\\
    =&\frac{1}{\pi}\int_0^{2\pi}\tilde{v}_{un}(x)\sin nx\ dx\\
    =&\frac{1}{\pi}\int_0^{\pi/3}\frac{V_s}{3}\sin nx\ dx+\frac{1}{\pi}\int_{\pi/3}^{2\pi/3}\frac{2V_s}{3}\sin nx\ dx
    +\frac{1}{\pi}\int_{2\pi/3}^{\pi}\frac{V_s}{3}\sin nx\ dx\\
    &-\frac{1}{\pi}\int_\pi^{4\pi/3}\frac{V_s}{3}\sin nx\ dx-\frac{1}{\pi}\int_{4\pi/3}^{5\pi/3}\frac{2V_s}{3}\sin nx\ dx
    -\frac{1}{\pi}\int_{5\pi/3}^{2\pi}\frac{V_s}{3}\sin nx\ dx\\
    =&\frac{V_s}{3}\frac{2}{\pi n}\left(1+\cos\frac{\pi n}{3}-\cos\frac{2\pi n}{3}-\cos\pi n\right)\\
    =&\left\{
    \begin{array}{ll}
      \cfrac{2V_s}{\pi n}&(n=1+6k,\ 5+6k,\ k\in\mathbb{N})\\
      0&({\rm otherwise})
    \end{array}
    \right.
  \end{split}
\end{align}
したがって$v_{un}$のフーリエ級数展開は
\begin{align}
  \begin{split}
    v_{un}=\sum_{k=0}^\infty\left(\frac{2V_s}{\pi(1+6k)}\sin\left((1+6k)\frac{2\pi}{T}t\right)+\frac{2V_s}{\pi(5+6k)}\sin\left((5+6k)\frac{2\pi}{T}t\right)\right)
  \end{split}
\end{align}
と表される.
\subsubsection*{(2)}
$v_{uo}$のフーリエ級数展開は
\begin{align}
  v_{uo}=\sum_{n=1}^{\infty}\frac{2V_s}{(2n-1)\pi}\sin\left((2n-1)\frac{2\pi}{T}t\right)
\end{align}
である.一方で$v_{no}$は奇関数なので,そのフーリエ係数は
\begin{align}
  \begin{split}
    b_n=&\frac{2}{T}\int_0^Tv_{no}(t)\sin\frac{2\pi nt}{T}\ dt\\
    =&\frac{1}{\pi}\int_0^{2\pi}\tilde{v}_{no}(x)\sin nx\ dx\\
    =&\frac{V_s}{3\pi n}\left(1-\cos\frac{\pi n}{3}+\cos\frac{2\pi n}{3}-\cos\pi n+\cos\frac{4\pi n}{3}-\cos\frac{5\pi n}{3}\right)\\
    =&\left\{
      \begin{array}{ll}
        \cfrac{2V_s}{\pi n}&(n=3+6k,\ k\in\mathbb{N})\\
        0&({\rm otherwise})
      \end{array}
    \right.
  \end{split}
\end{align}
したがって
\begin{align}
  \begin{split}
    v_{no}&=\sum_{k=0}^\infty\frac{2V_s}{(3+6k)\pi}\sin\left((3k+6)\frac{2\pi}{T}t\right)\\
    &=\frac{2V_s}{3}\sin\left(3\frac{2\pi}{T}t\right)
    +\frac{2V_s}{9}\sin\left(9\frac{2\pi}{T}t\right)
    +\frac{2V_s}{15}\sin\left(15\frac{2\pi}{T}t\right)+\cdots
  \end{split}
\end{align}
となり,これは(3)式の$3,\ 9,\ 15,\ldots$次高調波に等しいことがわかる.
\end{document}