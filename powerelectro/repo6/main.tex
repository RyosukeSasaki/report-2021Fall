\documentclass[uplatex,a4j,11pt,dvipdfmx]{jsarticle}
\usepackage{listings,jvlisting}
\bibliographystyle{jplain}

\usepackage{url}

\usepackage{graphicx}
\usepackage{gnuplot-lua-tikz}
\usepackage{pgfplots}
\usepackage{tikz}
\usepackage{amsmath,amsfonts,amssymb}
\usepackage{bm}
\usepackage{siunitx}

\makeatletter
\def\fgcaption{\def\@captype{figure}\caption}
\makeatother
\newcommand{\setsections}[3]{
\setcounter{section}{#1}
\setcounter{subsection}{#2}
\setcounter{subsubsection}{#3}
}
\newcommand{\mfig}[3][width=15cm]{
\begin{center}
\includegraphics[#1]{#2}
\fgcaption{#3 \label{fig:#2}}
\end{center}
}
\newcommand{\gnu}[2]{
\begin{figure}[hptb]
\begin{center}
\input{#2}
\caption{#1}
\label{fig:#2}
\end{center}
\end{figure}
}

\begin{document}
\title{パワーエレクトロニクス No.5}
\author{61908697 佐々木良輔}
\date{}
\maketitle
\subsubsection*{(1)}
オン電圧が$1\ \si{\volt}$のとき$V_R=99\ \si{\volt}$であり流れる電流は$9.9\ \si{\ampere}$である.
したがってオン損失は
\begin{align}
  P_{on}=1\times9.9=9.9\ \si{\watt}
\end{align}
\subsubsection*{(2)}
$\Delta T=1\ \si{\micro\second}$のとき
\begin{align}
  \begin{split}
    P_{sw}=\frac{EI}{6}\Delta T\times 2f&=\frac{100\times 10}{6}\times10^{-6}\times 2\times25\times10^3\\
    &=8.3\ \si{\watt}
  \end{split}
\end{align}
\end{document}