\documentclass[uplatex,a4j,11pt,dvipdfmx]{jsarticle}
\usepackage{listings,jvlisting}
\bibliographystyle{jplain}

\usepackage{url}

\usepackage{graphicx}
\usepackage{gnuplot-lua-tikz}
\usepackage{pgfplots}
\usepackage{tikz}
\usepackage{amsmath,amsfonts,amssymb}
\usepackage{bm}
\usepackage{siunitx}

\makeatletter
\def\fgcaption{\def\@captype{figure}\caption}
\makeatother
\newcommand{\setsections}[3]{
\setcounter{section}{#1}
\setcounter{subsection}{#2}
\setcounter{subsubsection}{#3}
}
\newcommand{\mfig}[3][width=15cm]{
\begin{center}
\includegraphics[#1]{#2}
\fgcaption{#3 \label{fig:#2}}
\end{center}
}
\newcommand{\gnu}[2]{
\begin{figure}[hptb]
\begin{center}
\input{#2}
\caption{#1}
\label{fig:#2}
\end{center}
\end{figure}
}
\renewcommand{\thesubsubsection}{(\alph{subsubsection})}
\renewcommand{\thesubsection}{(\roman{subsection})}
\begin{document}
\title{天体物理学レポート No.5}
\author{61908697 佐々木良輔}
\date{}
\maketitle
Friedmann方程式は
\begin{align}
  \left(\frac{\dot{a}}{a}\right)^2+\frac{Kc^2}{a^2}-\frac{\Lambda c^2}{3}=\frac{8\pi G}{3c^2}\rho
\end{align}
であり両辺に$a^2$を掛けて$\tau=H_0t$と変換すれば
\begin{align}
  \begin{split}
    \left(\frac{da}{d\tau}\right)^2&=-\frac{Kc^2}{H_0^2}+\frac{\Lambda c^2}{3H_0^2}+\frac{8\pi G}{3c^2H_0^2}\rho a^2\\
    &=-k_0+\lambda_0a^2+\frac{\rho}{\rho_c}a^2
  \end{split}
\end{align}
ここでエネルギー方程式
\begin{align}
  \dot{\rho}=-3\left(\frac{\dot{a}}{a}\right)(\rho+P)
\end{align}
を考える,相対論的(すなわち輻射的)な時には$P=\rho/3$であるので
\begin{align}
  \begin{split}
    \dot{\rho}&=-3\left(\frac{\dot{a}}{a}\right)\frac{4\rho}{3}\\
    \frac{\dot{\rho}}{\rho}&=-4\frac{\dot{a}}{a}\\
    \frac{d}{dt}\log\rho&=-4\frac{d}{dt}\log a\\
    \therefore\ \rho&\propto a^{-4}
  \end{split}
\end{align}
一方で非相対論的(すなわち物質的)な時には$\rho\gg P$として
\begin{align}
  \begin{split}
    \dot{\rho}&=-3\left(\frac{\dot{a}}{a}\right)\rho\\
    \frac{\dot{\rho}}{\rho}&=-3\frac{\dot{a}}{a}\\
    \therefore\ \rho&\propto a^{-3}
  \end{split}
\end{align}
となり$\rho$は$a^{-4}$の輻射成分と$a^{-3}$の物質成分に分離できることが期待できるので
\begin{align}
  \rho=\rho_{0r}a^{-4}+\rho_{0m}a^{-3}
\end{align}
と置く.さらに
\begin{align}
  \Omega_{0r}=\frac{\rho_{0r}}{\rho_c},\ \Omega_{0m}=\frac{\rho_{0m}}{\rho_c}
\end{align}
とすれば(2)式は
\begin{align}
  \left(\frac{da}{d\tau}\right)^2=\Omega_{0r}a^{-2}+\Omega_{0m}a^{-1}-k_0+\lambda_0a^2
\end{align}
となる.以下では$\Lambda=0$の場合を取り扱うので
\begin{align}
  \left(\frac{da}{d\tau}\right)^2=\Omega_{0r}a^{-2}+\Omega_{0m}a^{-1}-k_0
\end{align}
となる.
\subsection{$k_0=0$のとき}
\subsubsection{輻射優勢なとき}
輻射優勢なときには$\Omega_{0r}\gg\Omega_{0m}$なので
\begin{align}
  \begin{split}
    \left(\frac{da}{d\tau}\right)^2&=\Omega_{0r}a^{-2}\\
    \frac{da}{d\tau}a&=\sqrt{\Omega_{0r}}
  \end{split}
\end{align}
両辺積分すれば
\begin{align}
  \begin{split}
    \frac{a^2}{2}&=\sqrt{\Omega_{0r}}\tau+C\\
    a(\tau)&=\sqrt{2(\sqrt{\Omega_{0r}}\tau+C)}
  \end{split}
\end{align}
ここで現在時刻を$\tau=0$とすると現在のスケール因子は$1$と定義されるので
\begin{align}
  C=\frac{1}{2}
\end{align}
となる.以上から
\begin{align}
  a(t)&=\sqrt{2H_0\sqrt{\Omega_{0r}}t+1}
\end{align}
である.したがって宇宙は時間の$1/2$乗程度のスケールで膨張する.
\subsubsection{物質優勢なとき}
物質優勢なときには$\Omega_{0m}\gg\Omega_{0r}$なので
\begin{align}
  \begin{split}
    \left(\frac{da}{d\tau}\right)^2&=\Omega_{0m}a^{-1}\\
    \frac{da}{d\tau}\sqrt{a}&=\sqrt{\Omega_{0m}}\\
    \frac{2}{3}a^{3/2}&=\sqrt{\Omega_{0m}}\tau+C\\
  \end{split}
\end{align}
$a(0)=1$とすれば
\begin{align}
  a(\tau)=\left(\frac{3}{2}\sqrt{\Omega_{0m}}\tau+1\right)^{2/3}
\end{align}
したがって
\begin{align}
  a(t)=\left(\frac{3}{2}H_0\sqrt{\Omega_{0m}}t+1\right)^{2/3}
\end{align}
したがって宇宙は時間の$2/3$乗程度のスケールで膨張する.
\subsection{$k_0\neq 0$のとき}
Friedmann方程式は
\begin{align}
  \begin{split}
    \left(\frac{da}{d\tau}\right)^2-\frac{\Omega_{0r}}{a^2}-\frac{\Omega_{0m}}{a}=-k_0
  \end{split}
\end{align}
ここで
\begin{align}
  U(a)=-\frac{\Omega_{0r}}{a^2}-\frac{\Omega_{0m}}{a}
\end{align}
とすると(17)はポテンシャル$U(a)$下での質点$a$の運動と考えられる.
ここで$\Omega_{0r},\ \Omega_{0m}>0$とすると$U(a)$は図\ref{fig:fig1.jpg}のような概形になる.
したがって$k_0>0$のとき$a(t)$は最初増加し,途中で減少に転じるとわかる.
一方で$k_0<0$のとき$a(t)$は増加し続けると考えられる.
実際$\Lambda=0$のときの解はFriedmannモデルと呼ばれ,曲率に応じて閉じた宇宙,平坦な宇宙,開いた宇宙が解として存在する.\cite{Cosmolog77:online}
\mfig[width=8cm]{fig1.jpg}{$U(a)$の概形}
\bibliography{ref.bib}
\end{document}