\documentclass[uplatex,a4j,11pt,dvipdfmx]{jsarticle}
\usepackage{listings,jvlisting}
\bibliographystyle{jplain}

\usepackage{url}

\usepackage{graphicx}
\usepackage{gnuplot-lua-tikz}
\usepackage{pgfplots}
\usepackage{tikz}
\usepackage{amsmath,amsfonts,amssymb}
\usepackage{bm}
\usepackage{siunitx}

\makeatletter
\def\fgcaption{\def\@captype{figure}\caption}
\makeatother
\newcommand{\setsections}[3]{
\setcounter{section}{#1}
\setcounter{subsection}{#2}
\setcounter{subsubsection}{#3}
}
\newcommand{\mfig}[3][width=15cm]{
\begin{center}
\includegraphics[#1]{#2}
\fgcaption{#3 \label{fig:#2}}
\end{center}
}
\newcommand{\gnu}[2]{
\begin{figure}[hptb]
\begin{center}
\input{#2}
\caption{#1}
\label{fig:#2}
\end{center}
\end{figure}
}

\begin{document}
\title{天体物理学レポート No.3}
\author{61908697 佐々木良輔}
\date{}
\maketitle
炭素原子の微細構造線の光度は
\begin{align}
  L_{10}=\frac{N_cg_1}{Z}{\rm e}^{-\beta h\nu_{10}}A_{10}h\nu_{10}\\
  L_{21}=\frac{N_cg_2}{Z}{\rm e}^{-\beta h\nu_{21}}A_{21}h\nu_{21}
\end{align}
ここで$A_{10}=7.93\time10^{-8}\ \si{\second^{-1}}$, $A_{21}=2.68\time10^{-7}\ \si{\second^{-1}}$はアインシュタインのA係数,
$\nu_{10}=492\ \si{\giga\hertz}$, $\nu_{21}=809\ \si{\giga\hertz}$は微細構造線のスペクトルである.
(1), (2)式から
\begin{align}
  \begin{split}
    \frac{L_{10}}{L_{21}}=5&=\frac{g_1{\rm e}^{-\beta h\nu_{10}}c}{g_2{\rm e}^{-\beta h\nu_{21}}A_{21}h\nu_{21}}\\
    {\rm e}^{-\beta(E_1-E_2)}&=\frac{25}{3}\frac{A_{21}\nu_{21}}{A_{10}\nu_{10}}\\
    \beta&=\frac{1}{E_2-E_1}\log\left(\frac{25}{3}\frac{A_{21}\nu_{21}}{A_{10}\nu_{10}}\right)\\
    &\simeq1.826\times10^{22}
  \end{split}
\end{align}
したがって(1)式から炭素原子数は
\begin{align}
  \begin{split}
    N_C&=L_{10}\frac{g_1{\rm e}^{-\beta E_1}+g_2{\rm e}^{-\beta E_2}}{g_1{\rm e}^{-\beta E_1}A_{10}h\nu_{10}}\\
    &=4.629\times10^{58}
  \end{split}
\end{align}
また炭素原子数$N_C$に対する水素原子数$N_H$の比は$N_H/N_C=10^4$なので全原子数は$N\simeq N_C\times10^4$とする.
水素原子質量を$1.674\times10^{-27}$, 平均分子量を1.36とすれば全質量$M$は
\begin{align}
  M=1.054\times 10^{36}=5.299\times10^5 M_{\odot}
\end{align}
となる.
\end{document}