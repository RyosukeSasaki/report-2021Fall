\documentclass[uplatex,a4j,11pt,dvipdfmx]{jsarticle}
\usepackage{listings,jvlisting}
\bibliographystyle{jplain}

\usepackage{url}

\usepackage{graphicx}
\usepackage{gnuplot-lua-tikz}
\usepackage{pgfplots}
\usepackage{tikz}
\usepackage{amsmath,amsfonts,amssymb}
\usepackage{bm}
\usepackage{siunitx}

\makeatletter
\def\fgcaption{\def\@captype{figure}\caption}
\makeatother
\newcommand{\setsections}[3]{
\setcounter{section}{#1}
\setcounter{subsection}{#2}
\setcounter{subsubsection}{#3}
}
\newcommand{\mfig}[3][width=15cm]{
\begin{center}
\includegraphics[#1]{#2}
\fgcaption{#3 \label{fig:#2}}
\end{center}
}
\newcommand{\gnu}[2]{
\begin{figure}[hptb]
\begin{center}
\input{#2}
\caption{#1}
\label{fig:#2}
\end{center}
\end{figure}
}

\begin{document}
\title{天体物理学レポート No.2}
\author{61908697 佐々木良輔}
\date{}
\maketitle
\subsubsection*{問1}
温度$T$,半径$R$の球体の光度$L$は
\begin{align}
  L=4\pi R^2\sigma T^4
\end{align}
ただし$\sigma$はStefan-Boltzmann定数である.
したがってこの球体を距離$D$離れた位置からみたときの輻射流速$F$は
\begin{align}
  F=\frac{L}{4\pi D^2}=\frac{R^2}{D^2}\sigma T^4
\end{align}
となる.したがって地球の半径を$R_E$,太陽の半径を$\theta_\odot$,太陽の平均温度を$T_\odot$とすると
地球が単位時間に受ける熱量$Q_{in}$は太陽からのエネルギーを断面積$\pi R_E^2$で受けることから
\begin{align}
  Q_{in}=\theta_\odot^2\sigma T_\odot^4\cdot \pi R_E^2
\end{align}
地球が単位時間あたりに放出する熱量$Q_{out}$は地球の平均温度を$T_E$とすると
\begin{align}
  Q_{out}=4\pi R_E^2\sigma T_E^4
\end{align}
熱平衡を仮定すると
\begin{align}
  \begin{split}
    \theta_\odot^2\sigma T_\odot^4\cdot \pi R_E^2&=4\pi R_E^2\sigma T_E^4\\
    \therefore\ T_E&=\left(\frac{\theta_\odot^2}{4}\right)^{1/4}T_\odot\\
    &=\left(\frac{R_\odot^2}{4D^2}\right)^{1/4}T_\odot
  \end{split}
\end{align}
となる.また,地球近傍を周回する人工衛星を考えると人工衛星(半径$r_s$)が受ける太陽からの輻射エネルギー$Q_{in-\odot}$は
(3)と同様に求められるので
\begin{align}
  Q_{in-\odot}=\theta_\odot^2\sigma T_\odot^4\cdot\pi r_s^2
\end{align}
また,地球からの輻射エネルギー$Q_{in-E}$は衛星の軌道半径$\simeq R_E$であり地球の視半径$\theta_E\simeq 1$であることから
\begin{align}
  Q_{in-E}=\sigma T_E^4\cdot\pi r_s^2
\end{align}
また衛星が放出する輻射エネルギー$Q_{out}$は衛星の温度を$T_s$とすれば(4)と同様に
\begin{align}
  Q_{out}=4\pi r_s^2\sigma T_s^4
\end{align}
以上から熱平衡を仮定し
\begin{align}
  \begin{split}
    Q_{out}&=Q_{in-\odot}+Q_{in-E}\\
    T_s&=\left(\frac{1}{4}(\theta_\odot^2 T_\odot^4+T_E^4)\right)^{1/4}\\
    &=\left(\frac{5\theta_\odot^2}{16}\right)^{1/4}T_\odot\simeq 295\ \si{\kelvin}
  \end{split}
\end{align}
となる.
\subsubsection*{問2}
軌道上で半分の時間日陰にいるという状況は前問において$Q_{in-\odot}$の寄与が半分になった場合に相当するので
熱平衡は
\begin{align}
  \begin{split}
    Q_{out}&=\frac{1}{2}Q_{in-\odot}+Q_{in-E}\\
    T_s&=\left(\frac{1}{4}(\frac{\theta_\odot^2 T_\odot^4}{2}+T_E^4)\right)^{1/4}\\
    &=\left(\frac{3\theta_\odot^2}{16}\right)^{1/4}T_\odot\simeq 260\ \si{\kelvin}
  \end{split}
\end{align}
となる.
\end{document}