\documentclass[uplatex,a4j,11pt,dvipdfmx]{jsarticle}
\usepackage{listings,jvlisting}
\bibliographystyle{junsrt}

\usepackage{url}

\usepackage{graphicx}
\usepackage{gnuplot-lua-tikz}
\usepackage{pgfplots}
\usepackage{tikz}
\usepackage{amsmath,amsfonts,amssymb}
\usepackage{bm}
\usepackage{siunitx}

\makeatletter
\def\fgcaption{\def\@captype{figure}\caption}
\makeatother
\newcommand{\setsections}[3]{
\setcounter{section}{#1}
\setcounter{subsection}{#2}
\setcounter{subsubsection}{#3}
}
\newcommand{\mfig}[3][width=15cm]{
\begin{center}
\includegraphics[#1]{#2}
\fgcaption{#3 \label{fig:#2}}
\end{center}
}
\newcommand{\gnu}[2]{
\begin{figure}[hptb]
\begin{center}
\input{#2}
\caption{#1}
\label{fig:#2}
\end{center}
\end{figure}
}

\begin{document}
\title{流体弾性体力学 期末レポート}
\author{61908697 佐々木良輔}
\date{}
\maketitle
\subsection*{レポート1}
ナビエ方程式は
\begin{align}
  \rho({\bm r})\partial_t^2{\bm u}=(\lambda+\mu)\nabla(\nabla\cdot{\bm u})+\mu\nabla^2{\bm u}
\end{align}
である.また変位ベクトルは
\begin{align}
  {\bm u}(x,z)=\left(
    \begin{array}{c}
      \sum_\nu A{\rm e}^{\nu z}{\rm e}^{ik(x-vt)}\\
      0\\
      \sum_\nu B{\rm e}^{\nu z}{\rm e}^{ik(x-vt)}
    \end{array}
  \right)
  &=:\left(
    \begin{array}{c}
      \sum_\nu Af(\nu,x,z)\\
      0\\
      \sum_\nu Bf(\nu,x,z)
    \end{array}
  \right)
\end{align}
の実部である.ここで$f(\nu,x,z)={\rm e}^{\nu z}{\rm e}^{ik(x-vt)}$とした.
このとき
\begin{align}
  \begin{split}
    \partial_t^2{\bm u}&=\left(
      \begin{array}{c}
        \sum_\nu A(-ikv)^2f(\nu,x,z)\\
        0\\
        \sum_\nu B(-ikv)^2f(\nu,x,z)
      \end{array}
    \right)
  \end{split}
\end{align}
\begin{align}
  \begin{split}
    \nabla(\nabla\cdot{\bm u})&=\left(
      \begin{array}{c}
        \partial_x(\partial_xu_x+\partial_zu_z)\\
        0\\
        \partial_z(\partial_xu_x+\partial_zu_z)
      \end{array}
    \right)\\
    &=\left(
      \begin{array}{c}
        \sum_\nu ik(ikA+\nu B)f(\nu,x,z)\\
        0\\
        \sum_\nu \nu(ikA+\nu B)f(\nu,x,z)
      \end{array}
    \right)
  \end{split}
\end{align}
\begin{align}
  \begin{split}
    \nabla^2{\bm u}&=\left(
      \begin{array}{c}
        (\partial_x^2+\partial_z^2)u_x\\
        0\\
        (\partial_x^2+\partial_z^2)u_z
      \end{array}
    \right)\\
    &=\left(
      \begin{array}{c}
        \sum_\nu ((ik)^2+\nu^2)Af(\nu,x,z)\\
        0\\
        \sum_\nu ((ik)^2+\nu^2)Bf(\nu,x,z)
      \end{array}
    \right)
  \end{split}
\end{align}
これらを(1)に代入して$f(\nu,x,z)$の係数を比較すると$x$成分については
\begin{align}
  \sum_\nu\left(\rho k^2v^2A+(\lambda+\mu)(-k^2A+ik\nu B)+\mu(\nu^2-k^2)A\right)=0
\end{align}
同様に$z$成分については
\begin{align}
  \sum_\nu\left(\rho k^2v^2B+(\lambda+\mu)(ik\nu A+\nu^2B)+\mu(\nu^2-k^2)B\right)
\end{align}
それぞれ$\nu$について独立であるとすると
\begin{align}
  \rho k^2v^2A+(\lambda+\mu)(-k^2A+ik\nu B)+\mu(\nu^2-k^2)A&=0\\
  \rho k^2v^2B+(\lambda+\mu)(ik\nu A+\nu^2B)+\mu(\nu^2-k^2)B&=0
\end{align}
すなわち
\begin{align}
  \begin{split}
    \left(
      \begin{array}{cc}
      \rho k^2v^2-k^2(\lambda+\mu)+\mu(\nu^2-k^2)&ik\nu(\lambda+\mu)\\
      ik\nu(\lambda+\mu)&\rho k^2v^2+\nu^2(\lambda+\mu)+\mu(\nu^2-k^2)
    \end{array}
    \right)
    \left(
      \begin{array}{c}
        A\\B
      \end{array}
    \right)
    ={\bm 0}
  \end{split}
\end{align}
となる.これが非自明な解を持つためには
\begin{align}
  \left|
    \begin{array}{cc}
    \rho k^2v^2-k^2(\lambda+\mu)+\mu(\nu^2-k^2)&ik\nu(\lambda+\mu)\\
    ik\nu(\lambda+\mu)&\rho k^2v^2+\nu^2(\lambda+\mu)+\mu(\nu^2-k^2)
  \end{array}
  \right|=0
\end{align}
これを整理すると
\begin{align}
  \left(\rho k^2v^2+(\nu^2-k^2)\mu\right)\left(\rho k^2v^2+(\nu^2-k^2)(\lambda+2\mu)\right)=0
\end{align}
となる.
ここで
\begin{align}
  c_l^2=\frac{\lambda+2\mu}{\rho},\ c_t^2=\frac{\mu}{\rho}
\end{align}
を用いると
\begin{align}
  \left(c_t^2\nu^2-k^2(c_t^2-v^2)\right)\left(c_l^2\nu^2-k^2(c_l^2-v^2)\right)=0
\end{align}
となるので
\begin{align}
  \nu_1^2=k^2\left(1-\frac{v^2}{c_l^2}\right),\ \nu_2^2=k^2\left(1-\frac{v^2}{c_t^2}\right)
\end{align}
となる.これを用いると(8), (9)は
\begin{align}
  \frac{B}{A}&=\frac{k^2(v^2-c_l^2)+\nu^2c_t^2}{-ik\nu(c_l^2-c_t^2)}
\end{align}
\begin{align}
  \frac{B}{A}&=\frac{-ik\nu(c_l^2-c_t^2)}{k^2(v^2-c_t^2)+\nu^2c_l^2}
\end{align}
更に$\nu_1$, $\nu_2$を代入したときの$A$, $B$を$A_1$, $B_1$及び$A_2$, $B_2$とすると
\begin{align}
  \frac{B_1}{A_1}=-\frac{i\nu_1}{k},\ \frac{B_2}{A_2}=\frac{k}{i\nu_2}
\end{align}
となる.以上から$\bm u$はこれらの線形結合で表されるので
\begin{align}
  {\bm u}_x&=\left(A_1{\rm e}^{\nu_1 z}+A_2{\rm e}^{\nu_2 z}\right){\rm e}^{ik(x-vt)}\\
  {\bm u}_z&=\left(-\frac{i\nu_1}{k}A_1{\rm e}^{\nu_1 z}+\frac{k}{i\nu_2}A_2{\rm e}^{\nu_2 z}\right){\rm e}^{ik(x-vt)}
\end{align}
となる.
また表面においては$z$方向の応力が0となるべきなので
\begin{align}
  \sigma_{xz}=\sigma_{yz}=\sigma_{zz}=0
\end{align}
すなわち
\begin{align}
  \lambda(u_{xx}+u_{yy}+u_{zz})+2\mu u_{zz}=(\lambda+2\mu)u_{zz}+\lambda u_{xx}=0
\end{align}
および
\begin{align}
  u_{xz}+u_{zx}=0
\end{align}
となる.これに(19), (20)を用いると表面では$z=0$であることから
\begin{align}
  (\lambda+2\mu)\left(-\frac{i\nu_1^2}{k}A_1+\frac{k}{i}A_2\right)+\lambda ik(A_1+A_2)=0
\end{align}
\begin{align}
  2\nu_1A_1+\left(\nu_2+\frac{k^2}{\nu_2}\right)A_2=0
\end{align}
(24)に$c_l$, $c_t$を用いると
\begin{align}
  \begin{split}
    c_l^2\left(k\left(1-\frac{v^2}{c_l^2}\right)A_1+kA_2\right)-k(c_l^2+2c_t^2)(A_1+A_2)&=0\\
    (2-\frac{v^2}{c_t^2})A_1+2A_2&=0
  \end{split}
\end{align}
また(25)は
\begin{align}
  2\nu_1A_1+\frac{k^2}{\nu_2}\left(2-\frac{v^2}{c_t^2}\right)A_2=0
\end{align}
以上から(26), (27)は
\begin{align}
  \begin{split}
    \left(
      \begin{array}{cc}
        2-\frac{v^2}{c_t^2}&2\nu_2\\
        2\nu_1&k^2\left(2-\frac{v^2}{c_t^2}\right)
      \end{array}
    \right)\left(
      \begin{array}{c}
        A_1\\A_2/\nu_2
      \end{array}
    \right)={\bm 0}
  \end{split}
\end{align}
これが非自明な解を持つためには
\begin{align}
  \begin{split}
    k^2\left(2-\frac{v^2}{c_t^2}\right)^2&=4\nu_1\nu_2\\
    k^4\left(2-\frac{v^2}{c_t^2}\right)^4&=16k^4\left(1-\frac{v^2}{c_l^2}\right)\left(1-\frac{v^2}{c_t^2}\right)
  \end{split}
\end{align}
ここで$v=c_tx$, $c_t/c_l=\gamma$とすると
\begin{align}
  \begin{split}
    0&=16(1-x^2)(1-\gamma^2x^2)-(2-x^2)^4\\
    &=-x^8+8x^6-(16a^2-24)x^4+16(1-a^2)x^2
  \end{split}
\end{align}
したがって
\begin{align}
  0=x^6-8x^4-8x^2(2\gamma^2-3)+16(\gamma^2-1)
\end{align}
を得る.
%両辺$i=j$で和を取ると${\rm Tr}\ {\bm \epsilon}=\nabla\cdot{\bm u}$から
%\begin{align}
%    0&=(3\lambda+2\mu)\nabla\cdot{\bm u}|_{z=0}
%\end{align}
%したがって
%\begin{align}
%  \begin{split}
%    \nabla\cdot{\bm u}|_{z=0}&=0\\
%    ikA+\nu B&=0
%  \end{split}
%\end{align}
%となる.以上から
%\begin{align}
%  \begin{split}
%    0&=\left(\rho k^2v^2+(\nu^2-k^2)\mu\right)\left(\rho k^2v^2+(\nu^2-k^2)(\lambda+2\mu)\right)\\
%    0&=ikA+\nu B
%  \end{split}
%\end{align}
%が満たすべき方程式である.
\subsection*{レポート2}
流体に粘性が無いことから動径方向の流速が一定とする.さらに流体の速度が$x$方向のみであることから,
流体は渦なしである.したがって管内の流体は渦なし非圧縮流体となり,ベルヌーイの定理を適用できる.
断面積の変化は十分遅く,流れポテンシャルの変化は十分小さいものとする.すなわち
\begin{align}
  \frac{\partial \phi}{\partial t}\simeq0
\end{align}
とする.ここで図\ref{fig:fig1.jpg}のように断面積が$S_0$の位置で速度$v_0$,
断面積が$S_0+\delta S(x,t)$の位置で速度$v$とおく.するとベルヌーイの定理から
\begin{align*}
  \frac{v_0^2}{2}+\frac{P_0}{\rho}=\frac{v(x)^2}{2}+\frac{P_0+\nu\delta S(x,t)}{\rho}    
\end{align*}
\begin{align}
  \label{equ:2-1}
  v(x)^2=v_0^2-\frac{2\nu\delta S(x,t)}{\rho}
\end{align}
よって
\begin{align}
  v(x)=\pm\sqrt{v_0^2-\frac{2\nu\delta S(x,t)}{\rho}}
\end{align}
となる.ここで十分長い距離$L$で$v^2$を平均すると(\ref{equ:2-1})式から
\begin{align}
  \begin{split}
    \frac{1}{L}\int_0^Lv(x)^2dx&=\frac{1}{L}\int_0^Lv_0^2-\frac{2\nu\delta S(x,t)}{\rho}dx\\
    &=v_0^2-\frac{1}{L}\int_0^L\frac{2\nu\delta S(x,t)}{\rho}dx
  \end{split}
\end{align}
$\delta S$は十分な長さに渡って平均すれば0になると考えられるので
\begin{align}
  \frac{1}{L}\int_0^Lv(x)^2dx=v_0^2
\end{align}
となる.このことから断面積$S_0$の位置での速度$v_0$は位置$x$での速度$v$の平均値に一致すると考えられる.
速度の平均値が0であると仮定すると
\begin{align}
  \begin{split}
    v(x)&=\pm\sqrt{0-\frac{2\nu\delta S(x,t)}{\rho}}\\
    &=\pm\sqrt{\frac{2\nu|\delta S(x,t)|}{\rho}}
  \end{split}
\end{align}
となる.
\mfig[width=8cm]{fig1.jpg}{断面積の変化}
\subsection*{レポート3}
$d$次元空間の任意の閉曲面$S$に囲まれた領域$D$を考える.
$D$に含まれる粒子の全運動量の$i$成分は,速度場と密度場の存在を仮定すれば
\begin{align}
  \int_D\rho({\bm r},t)v_i({\bm r},t)dV
\end{align}
となる.この時間変化は
\begin{align}
  \int_D\frac{\partial}{\partial t}(\rho v_i)dV
\end{align}
である.
一方で領域$D$を出入りする粒子により増減する運動量の$i$成分はその流束が
\begin{align}
  {\bm j}_i={\bm v}\rho v_i
\end{align}
と書けることから曲面$S$の法線ベクトル$\hat{n}$を用いて
\begin{align}
  -\int_S\rho v_i{\bm v}\cdot\hat{n}dS
\end{align}
と書ける.ただし符号は流入を正とするためにつけている.
ガウスの発散定理を用いると
\begin{align}
  \label{equ:3-1}
  -\int_S\rho v_i{\bm v}\cdot\hat{n}dS=-\int_D\nabla\cdot({\bm v}\rho v_i)dV
\end{align}
となる.ここで$\nabla$は$d$次元でのナブラベクトルであり第$n$次元の基底ベクトルを${\bm e}_n$として
\begin{align}
  \nabla=\sum_{n=1}^d\frac{\partial}{\partial x_n}{\bm e}_n
\end{align}
と表される.また運動量は外力によっても変化するので,これを計算する.
まず体積力を$\bm g$とすると領域$D$にかかる力は
\begin{align}
  \label{equ:3-2}
  \int_D\rho({\bm r},t){\bm g}dV
\end{align}
次に曲面$S$を通じて加わる圧力は
\begin{align}
  \label{equ:3-3}
  -\int_SP({\bm r})dS=-\int_D\nabla P({\bm r})dV
\end{align}
となる.以上から(\ref{equ:3-1}), (\ref{equ:3-2}), (\ref{equ:3-3})の総和が運動量の時間変化に等しいので
\begin{align}
  \begin{split}
    \int_D\frac{\partial}{\partial t}(\rho v_i)dV&=
    \int_D-\nabla\cdot({\bm v}\rho v_i)+\rho g_i-(\nabla P)_idV\\
    \int_D\frac{\partial}{\partial t}(\rho v_i)+\nabla\cdot({\bm v}\rho v_i)dV&=\int_D\rho g_i-(\nabla P)_idV
  \end{split}
\end{align}
積分領域が等しいことから
\begin{align}
  \label{equ:3-4}
  \frac{\partial}{\partial t}(\rho v_i)+\nabla\cdot({\bm v}\rho v_i)=\rho g_i-(\nabla P)_i
\end{align}
ここで左辺について
\begin{align}
  \frac{\partial}{\partial t}(\rho v_i)+\nabla\cdot({\bm v}\rho v_i)&=
  \left(\frac{\partial \rho}{\partial t}+\nabla\cdot({\bm v}\rho)\right)v_i+\rho\frac{\partial v_i}{\partial t}+({\bm v}\rho)\cdot\nabla v_i
\end{align}
この右辺第1項は連続の式から0になるので(\ref{equ:3-4})は
\begin{align}
  \rho\frac{\partial v_i}{\partial t}+({\bm v}\rho)\cdot\nabla v_i=\rho g_i-(\nabla P)_i
\end{align}
以上から
\begin{align}
  \rho\left(\frac{\partial}{\partial t}+{\bm v}\cdot\nabla\right){\bm v}=-\nabla P+\rho{\bm g}
\end{align}
を得る.
\subsection*{レポート4}
非圧縮渦なし完全流体であることから流れポテンシャルは
\begin{align}
  \Delta \phi=0
\end{align}
のラプラス方程式を満たす.またベルヌーイの定理が適用でき
\begin{align}
  \frac{\partial\phi}{\partial t}+\frac{v^2}{2}+\frac{P}{\rho}+gz={\rm Const}
\end{align}
ここで定数として$P_0/\rho+gh_0$と置けば
\begin{align}
  \begin{split}
    \frac{\partial\phi}{\partial t}+\frac{v^2}{2}+g(z-h_0)&=\frac{1}{\rho}(P_0-P)\\
  \end{split}
\end{align}
となる.これは水面において
\begin{align}
  \frac{\partial\phi}{\partial t}+\frac{v^2}{2}+gh(x,t)&=\frac{f}{\rho}\frac{\partial^2 h(x,t)}{\partial x^2}
\end{align}
となる.ここで$v$が十分小さいとして左辺第2項を無視すると
\begin{align}
  \frac{\partial\phi}{\partial t}=-gh+\frac{f}{\rho}\frac{\partial^2 h}{\partial x^2}
\end{align}
両辺$t$で微分すると
\begin{align}
  \label{equ:4-1}
  \frac{\partial^2\phi}{\partial t^2}=-g\frac{\partial h}{\partial t}+\frac{f}{\rho}\frac{\partial^2}{\partial x^2}\frac{\partial h}{\partial t}
\end{align}
また水面付近での$z$, $x$方向の速度を$\tilde{v}_z$, $\tilde{v}_x$とすると
\begin{align}
  \frac{\partial{h}}{\partial t}=\tilde{v}_z-\frac{\partial{h}}{\partial x}\tilde{v}_x
\end{align}
ここで右辺第2項は微小量の2次なので無視すると
\begin{align}
  \frac{\partial{h}}{\partial t}=\tilde{v}_z=\frac{\partial\tilde{\phi}}{\partial z}
\end{align}
よって(\ref{equ:4-1})式は水面近傍においては
\begin{align}
  \label{equ:4-2}
  \frac{\partial^2\tilde{\phi}}{\partial t^2}=-g\frac{\partial\tilde{\phi}}{\partial z}+\frac{f}{\rho}\frac{\partial^2}{\partial x^2}\frac{\partial\tilde{\phi}}{\partial z}
\end{align}
ここで$\phi$を
\begin{align}
  \phi({\bm r},t)=R(z)\sin(k(x-ct))
\end{align}
と展開する.ラプラス方程式から
\begin{align}
  \begin{split}
    \frac{\partial^2 R}{\partial z^2}\sin(k(x-ct))-k^2R(z)\sin(k(x-ct))&=0\\
    \frac{\partial^2 R}{\partial z^2}-k^2R&=0
  \end{split}
\end{align}
したがって
\begin{align}
  R(z)=A{\rm e}^{kz}+B{\rm e}^{-kz}
\end{align}
ここで水底においては流体の速度の鉛直成分は0であるべきなので
\begin{align}
  \begin{split}
  \left.\frac{\partial\phi}{\partial z}\right|_{z=0}&=0\\
  (A-B)k\sin(k(x-ct))&=0
  \end{split}
\end{align}
したがって
\begin{align}
  A=B
\end{align}
であり
\begin{align}
  \phi({\bm r},t)=A\cosh(kz)\sin(k(x-ct))
\end{align}
となる.これと(\ref{equ:4-2})式から
\begin{align}
  k^2c^2\phi=gkA\sinh(kh_0)\sin(k(x-ct))+\frac{f}{\rho}k^3A\sinh(kh_0)\sin(k(x-ct))
\end{align}
両辺$\phi$で割ると
\begin{align}
  k^2c^2=\left(gk+\frac{f}{\rho}k^3\right)\tanh(kh_0)
\end{align}
\begin{align}
  \therefore\ c=\sqrt{\left(\frac{g}{k}+\frac{f}{\rho}k\right)\tanh(kh_0)}
\end{align}
を得る.
\subsection*{レポート5}
ナビエストークス方程式は動粘性係数$\nu$を用いて
\begin{align}
  \frac{\partial \rho}{\partial t}+({\bm v}\cdot\nabla){\bm v}=-\frac{1}{\rho}\nabla P+g+\nu\Delta{\bm v}
\end{align}
ストークス近似を用いると
\begin{align}
  \frac{\partial \rho}{\partial t}=-\frac{1}{\rho}\nabla P+g+\nu\Delta{\bm v}
\end{align}
となる.更に外力が存在せず,定常であるとすると
\begin{align}
  \label{equ:5-1}
  0=-\frac{1}{\rho}\nabla P+\nu\Delta{\bm v}
\end{align}
ここで一様な流れ$\bm V$の中の原点に半径$a$の球を設置する.
境界条件として非圧縮性から
\begin{align}
  \label{equ:5-2}
  \nabla\cdot{\bm v}=0
\end{align}
物体表面では
\begin{align}
  \label{equ:5-3}
  v=0
\end{align}
とする.無限遠での圧力を$P_0$として速度場と圧力場は以下で与えられる.
\begin{align}
  {\bm v}&={\bm V}\left(1-\frac{3a}{4r}-\frac{a^3}{4r^3}\right)-\frac{3}{4}\frac{{\bm V}\cdot{\bm r}}{r^2}{\bm r}\left(\frac{a}{r}-\frac{a^3}{r^3}\right)\\
  \label{equ:5P0}
  P&=P_0-\frac{3a\rho\nu}{2}\frac{{\bm V}\cdot{\bm r}}{r^3}
\end{align}
実際${\bm r}\rightarrow\infty$で
\begin{align}
  {\bm v}\rightarrow{\bm V},\ P\rightarrow P_0
\end{align}
を満たす.また物体表面($r=a$)では
\begin{align}
  {\bm v}|_{r=a}={\bm V}\left(1-\frac{3}{4}-\frac{1}{4}\right)-\frac{3}{4}\frac{{\bm V}\cdot{\bm r}}{a^2}{\bm r}\left(1-1\right)=0
\end{align}
から(\ref{equ:5-3})を満たす.また
\begin{align}
  \nabla\cdot{\bm v}=\left(\frac{3{\bm V}\cdot{\bm r}}{4r^3}a-\frac{3{\bm V}\cdot{\bm r}}{4r^5}a^3\right)-\frac{3}{4}\frac{{\bm V}\cdot{\bm r}}{r^2}\left(\frac{a}{r}-\frac{a^3}{r^3}\right)=0
\end{align}
から(\ref{equ:5-2})を満たす.また
\begin{align}
  \begin{split}
    -\frac{1}{\rho}\nabla P&=\frac{3a\nu}{2}\nabla\frac{V\cdot{\bm r}}{r^3}\\
    &=\frac{3a\nu}{2r^5}\left(
      \begin{array}{c}
        V_xr^2-3x{\bm V}\cdot{\bm r}\\
        V_yr^2-3y{\bm V}\cdot{\bm r}\\
        V_zr^2-3z{\bm V}\cdot{\bm r}
      \end{array}
    \right)\\
    &=\frac{3a\nu}{2r^5}\left({\bm V}r^2-3({\bm V}\cdot{\bm r}){\bm r}\right)
  \end{split}
\end{align}
\begin{align}
  \Delta\frac{1}{r}=4\pi\delta({\bm r})
\end{align}
ただし球の外部の流れを考えるので
\begin{align}
  \Delta\frac{1}{r}=0
\end{align}
\begin{align}
  \Delta \frac{1}{r^3}=\frac{6}{r^5}
\end{align}
また
\begin{align*}
  \left(\Delta \frac{{\bm V}\cdot{\bm r}}{r^3}{\bm r}\right)_x
  =&\left(\partial_x^2+\partial_y^2+\partial_z^2\right)\frac{V_xx+V_yy+V_zz}{r^3}x\\
  =&\frac{2V_x}{r^3}-3x\frac{5V_xx+3V_yy+3V_zz}{r^5}+15x\frac{x^2(V_xx+V_yy+V_zz)}{r^7}\\
  &-3x\frac{V_xx+3V_yy+V_zz}{r^5}+15x\frac{y^2(V_xx+V_yy+V_zz)}{r^7}\\
  &-3x\frac{V_xx+V_yy+3V_zz}{r^5}+15x\frac{z^2(V_xx+V_yy+V_zz)}{r^7}\\
  =&\frac{2V_x}{r^3}-x\frac{6(V_xx+V_yy+V_zz)}{r^5}
\end{align*}
より
\begin{align}
  \Delta \frac{{\bm V}\cdot{\bm r}}{r^3}{\bm r}=\frac{2}{r^5}\left({\bm V}r^2-3({\bm V}\cdot{\bm r}){\bm r}\right)
\end{align}
また
\begin{align*}
  \left(\Delta\frac{{\bm V}\cdot{\bm r}}{r^5}{\bm r}\right)_x
  =&\left(\partial_x^2+\partial_y^2+\partial_z^2\right)\frac{V_xx+V_yy+V_zz}{r^5}x\\
  =&\frac{2V_x}{r^5}-5x\frac{V_xx+V_yy+V_zz}{r^7}-10x\frac{2V_xx+V_yy+V_zz}{r^7}+35x\frac{x^2(V_xx+V_yy+V_zz)}{r^9}\\
  &-5x\frac{V_xx+V_yy+V_zz}{r^7}-10x\frac{V_yy}{r^7}+35x\frac{y^2(V_xx+V_yy+V_zz)}{r^9}\\
  &-5x\frac{V_xx+V_yy+V_zz}{r^7}-10x\frac{V_zz}{r^7}+35x\frac{z^2(V_xx+V_yy+V_zz)}{r^9}\\
  =&\frac{2V_x}{r^5}
\end{align*}
より
\begin{align}
  \Delta\frac{{\bm V}\cdot{\bm r}}{r^5}{\bm r}=\frac{2{\bm V}}{r^5}
\end{align}
以上から(\ref{equ:5-1})は
\begin{align}
  \begin{split}
    -\frac{1}{\rho}\nabla P+\nu\Delta{\bm v}
    =&\frac{3a\nu}{2r^5}\left({\bm V}r^2-3({\bm V}\cdot{\bm r}){\bm r}\right)-\frac{a^3\nu{\bm V}}{4}\frac{6}{r^5}\\
    &-\frac{3a\nu}{4}\frac{2}{r^5}\left({\bm V}r^2-3({\bm V}\cdot{\bm r}){\bm r}\right)
    +\frac{3a^3\nu}{4}\frac{2{\bm V}}{r^5}\\
    =&0
  \end{split}
\end{align}
となり,この解はナビエストークス方程式を満たす.

次にストークス抵抗を求める.
ここで${\bm r}$と${\bm V}$のなす各を$\theta$とすると(\ref{equ:5P0})は
\begin{align}
  P=P_0-\frac{3a\rho\nu}{2}\frac{V\cos\theta}{r^2}
\end{align}
となる.
球にかかる力は対称性から${\bm V}$に平行な方向にかかるべきである.
よって球にかかる全圧力は圧力の${\bm V}$に平行な成分$P\cos\theta$を球面で積分すれば良いので
\begin{align*}
  \begin{split}
    F_p&=\int_{球表面}-P\cos\theta\ dS\\
    &=-\int_0^{2\pi}\int_0^\pi\left(P_0-\frac{3a\rho\nu}{2}\frac{V\cos\theta}{r^2}\right)\cos\theta a^2\sin\theta d\theta d\phi\\
    &=-2\pi a^2P_0\int_{0}^\pi\cos\theta\sin\theta d\theta+2\pi\frac{3\rho\nu a}{2}V\int_0^\pi\cos^2\theta\sin\theta d\theta\\
    &=-2\pi a^2P_0\left[\frac{\sin^2\theta}{2}\right]^\pi_0+3\pi\rho\nu aV\left[-\frac{\cos^3\theta}{3}\right]_0^\pi\\
    &=0+3\pi\rho\nu aV\frac{2}{3}=2\pi\rho\nu aV
  \end{split}
\end{align*}
となる.

次に粘性応力を考える.
${\bm V}$を$z$方向にとっても一般性を失わないため,以下では${\bm V}\parallel z$とする.
$\bm v$の$r$成分は
\begin{align}
  \begin{split}
    v_r={\bm v}\cdot\hat{e}_r&=V\cos\theta\left(1-\frac{3a}{4r}-\frac{a^3}{4r^3}\right)-\frac{3}{4}\frac{Vr\cos\theta}{r^2}r\left(\frac{a}{r}-\frac{a^3}{r^3}\right)\\
    &=V\cos\theta\left(1-\frac{3a}{2r}+\frac{a^3}{2r^3}\right)
  \end{split}
\end{align}
$\theta$成分は
\begin{align}
  v_\theta={\bm v}\cdot\hat{e}_\theta=V\sin\theta\left(1-\frac{3a}{4r}-\frac{a^3}{4r^3}\right)
\end{align}
$\phi$成分は
\begin{align}
  v_\phi={\bm v}\cdot\hat{e}_\phi=0
\end{align}
となる.球面において応力は常に$r$面に加わる.よって球面上での接線方向の応力は
\begin{align}
  \begin{split}
    \sigma_{\theta r}&=\eta\left.\left(\frac{1}{r}\frac{\partial v_r}{\partial\theta}+r\frac{\partial}{\partial r}\frac{v_\theta}{r}\right)\right|_{r=a}\\
    &=\eta\left(\frac{1}{a}\right)(-V\sin\theta)\left(1-\frac{3a}{2a}+\frac{a^3}{2a^3}\right)+a(-V\sin\theta)\left(-\frac{1}{a^2}+\frac{3a}{2a^3}+\frac{a^3}{a^5}\right)\\
    &=-\frac{3\eta}{2a}V\sin\theta
  \end{split}
\end{align}
この応力の$\bm V$に平行な成分$-\sigma_{\theta r}\sin\theta$を面積分すると,ワイスの公式を用いて
\begin{align}
  \begin{split}
    F_v&=\int_{球表面}-\sigma_{\theta r}\sin\theta dS\\
    &=2\pi\frac{3\eta}{2a}V\int_0^\pi\sin^2\theta a^2\sin\theta d\theta\\
    &=3\pi\rho\nu aV\left(\left[-\frac{\cos\theta\sin^2\theta}{3}\right]_0^\pi+2\frac{2}{3}\right)\\
    &=4\pi\rho\nu aV
  \end{split}
\end{align}
以上から球の受ける全力は
\begin{align}
  F=F_p+F_v=6\pi\rho\nu aV
\end{align}
となる.\cite{sec8201073:online}
\bibliography{ref.bib}
\end{document}