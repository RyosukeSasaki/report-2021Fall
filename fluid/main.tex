\documentclass[uplatex,a4j,11pt,dvipdfmx]{jsarticle}
\usepackage{listings,jvlisting}
\bibliographystyle{junsrt}

\usepackage{url}

\usepackage{graphicx}
\usepackage{gnuplot-lua-tikz}
\usepackage{pgfplots}
\usepackage{tikz}
\usepackage{amsmath,amsfonts,amssymb}
\usepackage{bm}
\usepackage{siunitx}

\makeatletter
\def\fgcaption{\def\@captype{figure}\caption}
\makeatother
\newcommand{\setsections}[3]{
\setcounter{section}{#1}
\setcounter{subsection}{#2}
\setcounter{subsubsection}{#3}
}
\newcommand{\mfig}[3][width=15cm]{
\begin{center}
\includegraphics[#1]{#2}
\fgcaption{#3 \label{fig:#2}}
\end{center}
}
\newcommand{\gnu}[2]{
\begin{figure}[hptb]
\begin{center}
\input{#2}
\caption{#1}
\label{fig:#2}
\end{center}
\end{figure}
}

\begin{document}
\title{流体弾性体力学 期末レポート}
\author{61908697 佐々木良輔}
\date{}
\maketitle
\subsection*{レポート1}
ナビエ方程式は
\begin{align}
  \rho({\bm r})\partial_t^2{\bm u}=(\lambda+\mu)\nabla(\nabla\cdot{\bm u})+\mu\nabla^2{\bm u}
\end{align}
である.また変位ベクトルは
\begin{align}
  {\bm u}(x,z)=\left(
    \begin{array}{c}
      \sum_\nu A_{x,\nu}{\rm e}^{\nu z}{\rm e}^{ik(x-vt)}\\
      0\\
      \sum_\nu A_{z,\nu}{\rm e}^{\nu z}{\rm e}^{ik(x-vt)}
    \end{array}
  \right)
  &=:\left(
    \begin{array}{c}
      \sum_\nu A_{x,\nu}f(\nu,x,z)\\
      0\\
      \sum_\nu A_{z,\nu}f(\nu,x,z)
    \end{array}
  \right)
\end{align}
の実部である.ここで$f(\nu,x,z)={\rm e}^{\nu z}{\rm e}^{ik(x-vt)}$とした.
このとき
\begin{align}
  \begin{split}
    \partial_t^2{\bm u}&=\left(
      \begin{array}{c}
        \sum_\nu A_{x,\nu}(-ikv)^2f(\nu,x,z)\\
        0\\
        \sum_\nu A_{z,\nu}(-ikv)^2f(\nu,x,z)
      \end{array}
    \right)
  \end{split}
\end{align}
\begin{align}
  \begin{split}
    \nabla(\nabla\cdot{\bm u})&=\left(
      \begin{array}{c}
        \partial_x(\partial_xu_x+\partial_zu_z)\\
        0\\
        \partial_z(\partial_xu_x+\partial_zu_z)
      \end{array}
    \right)\\
    &=\left(
      \begin{array}{c}
        \sum_\nu ik(ikA_{x,\nu}+\nu A_{z,\nu})f(\nu,x,z)\\
        0\\
        \sum_\nu \nu(ikA_{x,\nu}+\nu A_{z,\nu})f(\nu,x,z)
      \end{array}
    \right)
  \end{split}
\end{align}
\begin{align}
  \begin{split}
    \nabla^2{\bm u}&=\left(
      \begin{array}{c}
        (\partial_x^2+\partial_z^2)u_x\\
        0\\
        (\partial_x^2+\partial_z^2)u_z
      \end{array}
    \right)\\
    &=\left(
      \begin{array}{c}
        \sum_\nu ((ik)^2+\nu^2)A_{x,\nu}f(\nu,x,z)\\
        0\\
        \sum_\nu ((ik)^2+\nu^2)A_{z,\nu}f(\nu,x,z)
      \end{array}
    \right)
  \end{split}
\end{align}
これらを(1)に代入して$f(\nu,x,z)$の係数を比較すると$x$成分については
\begin{align}
  \sum_\nu\left(\rho k^2v^2A_{x,\nu}+(\lambda+\mu)(-k^2A_{x,\nu}+ik\nu A_{z,\nu})+\mu(\nu^2-k^2)A_{x,\nu}\right)=0
\end{align}
同様に$z$成分については
\begin{align}
  \sum_\nu\left(\rho k^2v^2A_{z,\nu}+(\lambda+\mu)(ik\nu A_{x,\nu}+\nu^2A_{z,\nu})+\mu(\nu^2-k^2)A_{z,\nu}\right)
\end{align}
それぞれ$\nu$について独立であるとすると
\begin{align*}
  \rho k^2v^2A_{x,\nu}+(\lambda+\mu)(-k^2A_{x,\nu}+ik\nu A_{z,\nu})+\mu(\nu^2-k^2)A_{x,\nu}&=0\\
  \rho k^2v^2A_{z,\nu}+(\lambda+\mu)(ik\nu A_{x,\nu}+\nu^2A_{z,\nu})+\mu(\nu^2-k^2)A_{z,\nu}&=0
\end{align*}
すなわち
\begin{align}
  \begin{split}
    \left(
      \begin{array}{cc}
      \rho k^2v^2-k^2(\lambda+\mu)+\mu(\nu^2-k^2)&ik\nu(\lambda+\mu)\\
      ik\nu(\lambda+\mu)&\rho k^2v^2+\nu^2(\lambda+\mu)+\mu(\nu^2-k^2)
    \end{array}
    \right)
    \left(
      \begin{array}{c}
        A_{x,\nu}\\A_{z,\nu}
      \end{array}
    \right)
    ={\bm 0}
  \end{split}
\end{align}
となる.これが非自明な解を持つためには
\begin{align}
  \left|
    \begin{array}{cc}
    \rho k^2v^2-k^2(\lambda+\mu)+\mu(\nu^2-k^2)&ik\nu(\lambda+\mu)\\
    ik\nu(\lambda+\mu)&\rho k^2v^2+\nu^2(\lambda+\mu)+\mu(\nu^2-k^2)
  \end{array}
  \right|=0
\end{align}
これを整理すると
\begin{align}
  \left(\rho k^2v^2+(\nu^2-k^2)\mu\right)\left(\rho k^2v^2+(\nu^2-k^2)(\lambda+2\mu)\right)=0
\end{align}
となる.また境界条件から
\begin{align}
  \sigma_{ij}=0=\lambda\nabla\cdot{\bm u}\delta_{ij}+2\mu{\bm \epsilon}_{ij}|_{z=0}
\end{align}
両辺$i=j$で和を取ると${\rm Tr}\ {\bm \epsilon}=\nabla\cdot{\bm u}$から
\begin{align}
    0&=(3\lambda+2\mu)\nabla\cdot{\bm u}|_{z=0}
\end{align}
したがって
\begin{align}
  \begin{split}
    \nabla\cdot{\bm u}|_{z=0}&=0\\
    ikA_{x,\nu}+\nu A_{z,\nu}&=0
  \end{split}
\end{align}
となる.以上から
\begin{align}
  \begin{split}
    0&=\left(\rho k^2v^2+(\nu^2-k^2)\mu\right)\left(\rho k^2v^2+(\nu^2-k^2)(\lambda+2\mu)\right)\\
    0&=ikA_{x,\nu}+\nu A_{z,\nu}
  \end{split}
\end{align}
が満たすべき方程式である.
\subsection*{レポート2}
流体に粘性が無いことから動径方向の流速が一定とする.さらに流体の速度が$x$方向のみであることから,
流体は渦なしである.したがって管内の流体は渦なし非圧縮流体となり,ベルヌーイの定理を適用できる.
断面積の変化は十分遅く,流れポテンシャルの変化は十分小さいものとする.すなわち
\begin{align}
  \frac{\partial \phi}{\partial t}\simeq0
\end{align}
とする.ここで図\ref{fig:fig1.jpg}のように断面積が$S_0$の位置で速度$v_0$,
断面積が$S_0+\delta S(x,t)$の位置で速度$v$とおく.するとベルヌーイの定理から
\begin{align*}
  \frac{v_0^2}{2}+\frac{P_0}{\rho}=\frac{v(x)^2}{2}+\frac{P_0+\nu\delta S(x,t)}{\rho}    
\end{align*}
\begin{align}
  v(x)^2=v_0^2-\frac{2\nu\delta S(x,t)}{\rho}
\end{align}
よって
\begin{align}
  v(x)=\pm\sqrt{v_0^2-\frac{2\nu\delta S(x,t)}{\rho}}
\end{align}
となる.ここで十分長い距離$L$で$v^2$を平均すると(16)式から
\begin{align}
  \begin{split}
    \frac{1}{L}\int_0^Lv(x)^2dx&=\frac{1}{L}\int_0^Lv_0^2-\frac{2\nu\delta S(x,t)}{\rho}dx\\
    &=v_0^2-\frac{1}{L}\int_0^L\frac{2\nu\delta S(x,t)}{\rho}dx
  \end{split}
\end{align}
$\delta S$は十分な長さに渡って平均すれば0になると考えられるので
\begin{align}
  \frac{1}{L}\int_0^Lv(x)^2dx=v_0^2
\end{align}
となる.このことから断面積$S_0$の位置での速度$v_0$は位置$x$での速度$v$の平均値に一致すると考えられる.
速度の平均値が0であると仮定すると
\begin{align}
  \begin{split}
    v(x)&=\pm\sqrt{0-\frac{2\nu\delta S(x,t)}{\rho}}\\
    &=\pm\sqrt{\frac{2\nu|\delta S(x,t)|}{\rho}}
  \end{split}
\end{align}
となる.
\mfig[width=8cm]{fig1.jpg}{断面積の変化}
\subsection*{レポート3}
$d$次元空間の任意の閉曲面$S$に囲まれた領域$D$を考える.
$D$に含まれる粒子の全運動量の$i$成分は,速度場と密度場の存在を仮定すれば
\begin{align}
  \int_D\rho({\bm r},t)v_i({\bm r},t)dV
\end{align}
となる.この時間変化は
\begin{align}
  \int_D\frac{\partial}{\partial t}(\rho v_i)dV
\end{align}
である.
一方で領域$D$を出入りする粒子により増減する運動量の$i$成分はその流束が
\begin{align}
  {\bm j}_i={\bm v}\rho v_i
\end{align}
と書けることから曲面$S$の法線ベクトル$\hat{n}$を用いて
\begin{align}
  -\int_S\rho v_i{\bm v}\cdot\hat{n}dS
\end{align}
と書ける.ただし符号は流入を正とするためにつけている.
ガウスの発散定理を用いると
\begin{align}
  -\int_S\rho v_i{\bm v}\cdot\hat{n}dS=-\int_D\nabla\cdot({\bm v}\rho v_i)dV
\end{align}
となる.ここで$\nabla$は$d$次元でのナブラベクトルであり第$n$次元の基底ベクトルを${\bm e}_n$として
\begin{align}
  \nabla=\sum_{n=1}^d\frac{\partial}{\partial x_n}{\bm e}_n
\end{align}
と表される.また運動量は外力によっても変化するので,これを計算する.
まず体積力を$\bm g$とすると領域$D$にかかる力は
\begin{align}
  \int_D\rho({\bm r},t){\bm g}dV
\end{align}
次に曲面$S$を通じて加わる圧力は
\begin{align}
  -\int_SP({\bm r})dS=-\int_D\nabla P({\bm r})dV
\end{align}
となる.以上から(25), (27), (28)の総和が運動量の時間変化に等しいので
\begin{align}
  \begin{split}
    \int_D\frac{\partial}{\partial t}(\rho v_i)dV&=
    \int_D-\nabla\cdot({\bm v}\rho v_i)+\rho g_i-(\nabla P)_idV\\
    \int_D\frac{\partial}{\partial t}(\rho v_i)+\nabla\cdot({\bm v}\rho v_i)dV&=\int_D\rho g_i-(\nabla P)_idV
  \end{split}
\end{align}
積分領域が等しいことから
\begin{align}
  \frac{\partial}{\partial t}(\rho v_i)+\nabla\cdot({\bm v}\rho v_i)=\rho g_i-(\nabla P)_i
\end{align}
ここで左辺について
\begin{align}
  \frac{\partial}{\partial t}(\rho v_i)+\nabla\cdot({\bm v}\rho v_i)&=
  \left(\frac{\partial \rho}{\partial t}+\nabla\cdot({\bm v}\rho)\right)v_i+\rho\frac{\partial v_i}{\partial t}+({\bm v}\rho)\cdot\nabla v_i
\end{align}
この右辺第1項は連続の式から0になるので(30)は
\begin{align}
  \rho\frac{\partial v_i}{\partial t}+({\bm v}\rho)\cdot\nabla v_i=\rho g_i-(\nabla P)_i
\end{align}
以上から
\begin{align}
  \rho\left(\frac{\partial}{\partial t}+{\bm v}\cdot\nabla\right){\bm v}=-\nabla P+\rho{\bm g}
\end{align}
を得る.
\subsection*{レポート4}
非圧縮渦なし完全流体であることから流れポテンシャルは
\begin{align}
  \Delta \phi=0
\end{align}
のラプラス方程式を満たす.またベルヌーイの定理が適用でき
\begin{align}
  \frac{\partial\phi}{\partial t}+\frac{v^2}{2}+\frac{P}{\rho}+gz={\rm Const}
\end{align}
ここで定数として$P_0/\rho+gh_0$と置けば
\begin{align}
  \begin{split}
    \frac{\partial\phi}{\partial t}+\frac{v^2}{2}+g(z-h_0)&=\frac{1}{\rho}(P_0-P)\\
  \end{split}
\end{align}
となる.(36)式は水面付近において
\begin{align}
  \frac{\partial\phi}{\partial t}+\frac{v^2}{2}+gh(x,t)&=\frac{f}{\rho}\frac{\partial^2 h(x,t)}{\partial x^2}
\end{align}
となる.ここで$v$が十分小さいとして左辺第2項を無視すると
\begin{align}
  \frac{\partial\phi}{\partial t}=-gh+\frac{f}{\rho}\frac{\partial^2 h}{\partial x^2}
\end{align}
両辺$t$で微分すると
\begin{align}
  \frac{\partial^2\phi}{\partial t^2}=-g\frac{\partial h}{\partial t}+\frac{f}{\rho}\frac{\partial^2}{\partial x^2}\frac{\partial h}{\partial t}
\end{align}
また水面付近での$z$, $x$方向の速度を$\tilde{v}_z$, $\tilde{v}_x$とすると
\begin{align}
  \frac{\partial{h}}{\partial t}=\tilde{v}_z-\frac{\partial{h}}{\partial x}\tilde{v}_x
\end{align}
ここで右辺第2項は微小量の2次なので無視すると
\begin{align}
  \frac{\partial{h}}{\partial t}=\tilde{v}_z=\frac{\partial\tilde{\phi}}{\partial z}
\end{align}
よって(39)式は水面近傍においては
\begin{align}
  \frac{\partial^2\tilde{\phi}}{\partial t^2}=-g\frac{\partial\tilde{\phi}}{\partial z}+\frac{f}{\rho}\frac{\partial^2}{\partial x^2}\frac{\partial\tilde{\phi}}{\partial z}
\end{align}
ここで$\phi$を
\begin{align}
  \phi({\bm r},t)=R(z)\sin(k(x-ct))
\end{align}
と展開する.ラプラス方程式から
\begin{align}
  \begin{split}
    \frac{\partial^2 R}{\partial z^2}\sin(k(x-ct))-k^2R(z)\sin(k(x-ct))&=0\\
    \frac{\partial^2 R}{\partial z^2}-k^2R&=0
  \end{split}
\end{align}
したがって
\begin{align}
  R(z)=A{\rm e}^{kz}+B{\rm e}^{-kz}
\end{align}
ここで水底においては流体の速度の鉛直成分は0であるべきなので
\begin{align}
  \begin{split}
  \left.\frac{\partial\phi}{\partial z}\right|_{z=0}&=0\\
  (A-B)k\sin(k(x-ct))&=0
  \end{split}
\end{align}
したがって
\begin{align}
  A=B
\end{align}
であり
\begin{align}
  \phi({\bm r},t)=A\cosh(kz)\sin(k(x-ct))
\end{align}
となる.これと(42)式から
\begin{align}
  k^2c^2\phi=gkA\sinh(kh_0)\sin(k(x-ct))+\frac{f}{\rho}k^3A\sinh(kh_0)\sin(k(x-ct))
\end{align}
両辺$\phi$で割ると
\begin{align}
  k^2c^2=\left(gk+\frac{f}{\rho}k^3\right)\tanh(kh_0)
\end{align}
\begin{align}
  \therefore\ c=\sqrt{\left(\frac{g}{k}+\frac{f}{\rho}k\right)\tanh(kh_0)}
\end{align}
を得る.
\subsection*{レポート5}
\bibliography{ref.bib}
\end{document}