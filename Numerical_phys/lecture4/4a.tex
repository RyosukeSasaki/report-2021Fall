\section{5-A.}
\subsection{プログラムについて}
プログラムはサンプルコードを元に各ブロック毎に計算された物理量をさらに平均化するように修正したものを用いている.
変数sume, sumc, summ, sumxは各ブロックで計算された内部エネルギー,比熱,磁化,帯磁率の合計値を格納し,最後にブロック数で割ることで
物理量の平均値を計算している.
プログラムの修正部分をソースコード\ref{src5a}に示す.
\subsection{結果}
\subsubsection{乱数の検証}
乱数が十分高いエントロピーを持っていることを確認するため,いくつかのシード値に対してシミュレーションを行い
シード値と物理量(内部エネルギー)との相関を調べた.相関係数はgnuplotのstatsコマンドを用いて計算した.
計算条件は表\ref{tab:5a1}の通りである.シード値は3から601までの全奇数(300点)とした.
結果は図\ref{fig:graph/entropy.tex}のようになった.シード値と物理量の相関係数は$R=0.02$であった.
また標準偏差は
\begin{align}
  {\rm sume}=-1.9511(1\pm0.0002)
\end{align}
となった.
\begin{table}[h]
  \caption{計算条件}
  \label{tab:5a1}
  \centering
  \begin{tabular}{cc}
    \hline
    無次元量&値\\
    \hline \hline
    格子サイズ&$8\times 8$\\
    温度&$1.5$\\
    $M_i$&10000\\
    $M_f$&20000\\
    ブロック数&10\\
    \hline
  \end{tabular}
\end{table}
\gnu{シード値と内部エネルギーの関係}{graph/entropy.tex}
\clearpage
\subsubsection{シミュレーション結果}
いくつかの温度で内部エネルギーと比熱のシミュレーションを行った.
また同じ温度での厳密解の値を計算した.
その結果は表\ref{tab:5a3}のとおりである.
また計算条件は表\ref{tab:5a2}の通りである.
\begin{table}[h]
\caption{計算条件}
\label{tab:5a2}
\centering
\begin{tabular}{cc}
\hline
無次元量&値\\
\hline \hline
格子サイズ&$8\times 8$\\
$M_i$&10000\\
$M_f$&20000\\
ブロック数&10\\
シード値&601\\
\hline
\end{tabular}
\end{table}
\begin{table}[h]
\caption{計算結果}
\label{tab:5a3}
\centering
\begin{tabular}{cc|ll|ll}
\hline
\multirow{2}{*}{温度}&\multirow{2}{*}{スピン初期状態}&\multicolumn{2}{c|}{シミュレーション}&\multicolumn{2}{c}{厳密解}\\
&&内部エネルギー&比熱&内部エネルギー&比熱\\
\hline \hline
1.5&all up&$-1.951$&$0.1952$&$-1.951$&$0.1972$\\
2.3&all up&$-1.452$&$1.176$&$-1.456$&$1.170$\\
2.3&random&$-1.453$&$1.174$&$-1.456$&$1.170$\\
3.0&random&$-0.8420$&$0.4850$&$-0.8413$&$0.4840$\\
\hline
\end{tabular}
\end{table}
\subsection{考察}
\subsubsection{乱数の検証}
前節で述べた通り乱数のシード値と計算される物理量の間の相関係数は$R=0.02$と非常に小さく,
シード値は条件を満たす限りにおいて任意に取って良いものと考えられる.

またシード値の変化による物理量の標準偏差も$0.02\%$と非常に小さく,
このことからも計算結果に与えるシード値の影響は十分小さいと考えられる.

以上から今後の計算においては乱数は十分高いエントロピーを持ち,
計算結果にシード値による誤差などは現れていないものと考える.
\subsubsection{シミュレーション結果}
各条件でのシミュレーション結果の厳密解との相対誤差は表\ref{tab:5a4}の通りである.
このことから全ての条件でシミュレーションの結果と厳密解が良く一致しており,
シミュレーションの結果が妥当であると考えられる.

またシミュレーションの結果から比熱は臨界温度付近で大きくなっていることがわかる.
これについては次節の課題5-B.で詳細に検討する,

また臨海温度($2.3$)の2つシミュレーションではスピン初期状態によらずどちらも厳密解に良く一致しており,
初期条件によらず平衡状態に収束していると考えられる.
\begin{table}[h]
\caption{相対誤差}
\label{tab:5a4}
\centering
\begin{tabular}{cc|cc}
\hline
\multirow{2}{*}{温度}&\multirow{2}{*}{スピン初期状態}&\multicolumn{2}{c}{相対誤差 / \%}\\
&&内部エネルギー&比熱\\
\hline \hline
1.5&all up&$0.0$&$1.0$\\
2.3&all up&$0.2$&$0.5$\\
2.3&random&$0.2$&$0.4$\\
3.0&random&$0.1$&$0.2$\\
\hline
\end{tabular}
\end{table}
