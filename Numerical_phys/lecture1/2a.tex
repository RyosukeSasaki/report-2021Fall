\subsection*{2-A.}
\subsubsection*{プログラムについて}
図\ref{fig:fig/2a.png}にPAD図を示す.
PAD図はPadTools (\url{https://naoblo.net/misc/padtools/})を用いて作図した.
このプログラムではdouble precision型のOnePlusEpsilon変数に$1+2^{-n}$を順次代入し,
これが$1$と一致したときにループを抜ける.このとき$2^{-(n-1)}$が計算機イプシロンになる.
ソースコード\ref{src2a}にfortranによるソースコードを示す.
\mfig[width=10cm]{fig/2a.png}{2-A.のPAD図}
\subsubsection*{出力結果}
実行すると$n = 52,\ \varepsilon = 2.2\times10^{-16}$という結果を得た.
またOnePlusEpsilonをreal型に変えると$n = 23,\ \varepsilon = 1.2\times10^{-7}$となった.
これらは倍精度の仮数部が53 bit,単精度の仮数部が24 bitであることとも整合し,
プログラムが正常に動作していると考えられる.\cite{floating}