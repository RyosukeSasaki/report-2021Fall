\subsection*{2-B.}
\subsubsection*{公式の無次元化}
各公式を無次元数$\tilde{\omega}=\beta\hbar\omega$で無次元化する.まずPlanckの公式について
\begin{align*}
  u(\omega,T)&=\frac{\omega^2}{\pi^2c^3}\frac{\hbar\omega}{{\rm e}^{\beta\hbar\omega}-1}\\
  &=\frac{1}{\pi^2c^3}\frac{\tilde{\omega^2}}{\beta^2\hbar^2}\frac{\tilde{\omega}}{\beta({\rm e}^{\tilde{\omega}}-1)}\\
\end{align*}
\begin{align*}
  f_{\rm planck}(\tilde{\omega}):=\beta^3\pi^2\hbar^2c^3u(\omega,T)=\frac{\tilde{\omega}^3}{{\rm e}^{\tilde{\omega}}-1}
\end{align*}
同様にRayleigh-Jeansの公式, Wienの公式について
\begin{align*}
  f_{\rm rayleigh-jeans}(\tilde{\omega})&:=\tilde{\omega}^2\\
  f_{\rm wien}(\tilde{\omega})&:=\tilde{\omega}^3{\rm e}^{-\tilde{\omega}}
\end{align*}
として無次元化できた.
\subsubsection*{プログラムについて}
図\ref{fig:fig/2b.png}にPAD図を示す.このプログラムではx\_startからx\_endまでをnum\_samples分割し,
それぞれの点でのplanck(x), rayleigh\_jeans(x), wien(x)関数の値を出力している.
これらの関数はそれぞれ上で無次元化したPlanckの公式, Rayleigh-Jeansの公式, Wienの公式の値を返す関数である.
ソースコード\ref{src2b}にfortranによるソースコードを示す.
\mfig[width=10cm]{fig/2b.png}{2-B.のPAD図}
\clearpage
\subsubsection*{出力結果}
図\ref{fig:graph/2b.tex}にプロットしたグラフを示す.
Rayleigh-Jeansの公式, Wienの公式はそれぞれPlanckの公式の低周波数,高周波数での近似である.
図\ref{fig:graph/2b.tex}ではたしかにそれぞれの公式が低周波数,高周波数でのPlanckの公式をよく近似していることがわかる.
\gnu{出力結果}{graph/2b.tex}