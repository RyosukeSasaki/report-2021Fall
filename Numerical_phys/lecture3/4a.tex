\subsection*{4-A.(e), (f)}
\subsubsection*{プログラムについて}
プログラムは基本的に教材として配布されたwave.f90を用いている.
ただし標準入力によるデータ入力は使い勝手が悪かったため,
定数はソースコードにベタ書きへ変更した
変更した.ソースコードはソースコード\ref{src4a}に示す.
また厳密解を出力するプログラムのPAD図は図\ref{fig:fig/4ad.png}のようになる.
そのソースコードはソースコード\ref{src4ad}である.
\mfig[width=8cm]{fig/4ad.png}{厳密解の出力プログラムのPAD図}
\subsubsection*{結果}
表\ref{tab:4aef_cond}に計算条件を示す.ここで$dt$は数値解の安定性条件が最も厳しくなる条件3で安定となるように定めた.
図\ref{fig:graph/4ae1.tex},図\ref{fig:graph/4ae2.tex}に各条件での$\langle x\rangle$, $\langle (x-\langle x\rangle)^2\rangle$の数値解及び厳密解を示す.
また図\ref{fig:graph/4af1.tex},図\ref{fig:graph/4af2.tex}に$t=0.05$での$\Delta x^2$対$\langle x\rangle$, $\langle (x-\langle x\rangle)^2\rangle$
の数値解及び厳密解のグラフを示す.
\begin{table}[h]
  \caption{計算条件}
  \label{tab:4aef_cond}
  \centering
  \begin{tabular}{ccccccc}
  \hline
  条件&$\Delta x$&$dt$&$\sigma$&$k_0$&$L$&$x_0$\\
  \hline \hline
  1&$1/64 $&$1.0\times 10^{-5}$&0.1&20&4&$-0.5$\\
  2&$1/128$&〃&〃&〃&〃&〃\\
  3&$1/256$&〃&〃&〃&〃&〃\\
  \hline
  \end{tabular}
\end{table}
\gnu{$\langle x\rangle$の数値解と厳密解}{graph/4ae1.tex}
\gnu{$\langle (x-\langle x\rangle)^2\rangle$の数値解と厳密解}{graph/4ae2.tex}
\gnu{$\Delta x^2$対$\langle x\rangle$}{graph/4af1.tex}
\gnu{$\Delta x^2$対$\langle (x-\langle x\rangle)^2\rangle$}{graph/4af2.tex}
\newpage
\subsubsection*{考察}
Gauss波束の期待値,分散の厳密解は
\begin{align}
  \langle x\rangle=x_0+k_0t\\
  \langle (x-\langle x\rangle)^2\rangle=\sigma^2\left(1+\frac{t^2}{4\sigma^4}\right)
\end{align}
であった.図\ref{fig:graph/4ae1.tex},図\ref{fig:graph/4ae2.tex}から
期待値と分散がそれぞれ線形,二次曲線的な振る舞いをしていることがわかる.
また$\Delta x$が小さくなるほど数値解が厳密解に近づいていることもわかる.

また図\ref{fig:graph/4af1.tex}と図\ref{fig:graph/4af2.tex}を見ると数値解の誤差が$\Delta x^2$
に比例している様子がわかる.また数値解と$\Delta x^2$の関係はそれぞれ最小二乗法により
\begin{align}
  y=-7.856\times10^3x+0.5000\\
  y=-2.541\times10^3x+0.07247
\end{align}
となったので$\Delta x\rightarrow 0$での期待値と分散は
\begin{align}
  \langle x\rangle=0.5000\\
  \langle (x-\langle x\rangle)^2\rangle=0.07247
\end{align}
となる.一方で厳密解から得られる期待値と分散は
\begin{align}
  \langle x\rangle=-0.5+20\times0.05=0.5\\
  \langle (x-\langle x\rangle)^2\rangle=0.1^2\left(1+\frac{0.05^2}{4\times0.1^4}\right)=7.25\times10^{-2}
\end{align}
となり,それぞれ相対誤差は$0.000\ \%$, $4.138\times10^{-2}\ \%$となり,良く一致していることがわかる.
\newpage
\subsection*{4-A.追加問題}
\subsubsection*{プログラムについて}
基本的には前問のプログラムを流用しているが,出力部をソースコード\ref{src4aADD}のように変更し
各離散点における$|\psi(x)|^2$, ${\rm Re}(\psi(x))$, ${\rm Im}(\psi(x))$を出力するようにした.
また出力も$t=0$と$t=0.05$でのみ行うように変更した.
\begin{lstlisting}[caption=出力部の改変,label=src4aADD]
  do j=1, jmax+1
    write(*,'(4e18.8e3)') x(j), r(j), real(cp(j)), aimag(cp(j))
  end do
\end{lstlisting}
\subsubsection*{結果}
図\ref{fig:graph/out/4aADD1.pdf},図\ref{fig:graph/out/4aADD2.pdf}に$t=0$と$t=0.05$におけるグラフを示す.
計算条件は表\ref{tab:4aef_cond}の条件3と同等である.
それぞれ高さ方向に実軸,奥行き方向に虚軸を取っている.
\mfig[width=12cm]{graph/out/4aADD1.pdf}{$t=0$での波形}
\mfig[width=12cm]{graph/out/4aADD2.pdf}{$t=0.05$での波形}
\subsubsection*{考察}
図\ref{fig:graph/out/4aADD1.pdf},図\ref{fig:graph/out/4aADD2.pdf}から時間発展に伴って
波束が進行しながら,広がり鈍くなっていることがわかる.
これは(1), (2)式において期待値が時間とともに進行し,分散が広がることと整合する.

このことはGauss波束をFourier変換することで様々な波数$k$を持った平面波に展開できることから説明できる.
平面波の位相速度は
\begin{align}
  v_k=\frac{\hbar k}{m}
\end{align}
であり,波数ごとに異なった位相速度を持つことがわかる.これにより時間発展と共に波数の大きい成分と小さい成分が
空間的に分離することによってGauss波束の広がりが大きくなる.
