\appendix
\def\thesection{補遺\Alph{section}}
\renewcommand*{\thelstlisting}{\Alph{section}.\arabic{lstlisting}}
\section{ソースコード}
\begin{lstlisting}[caption=4-Aのソースコード,label=src4a]
program wave
  !=======================================================-
  !  sengen
  !=======================================================-
  implicit none
  integer,parameter:: idim=4097
  real(kind=8)::  a(4), b(4), r(idim), x(idim)
  complex(kind=8)::  cp(idim), cs(idim), cps(idim), cdp(idim)
  !
  integer:: iok,jmax,nmax,n1max,j,n,n1,m
  real(kind=8):: rl,sig,rk,x0,dt,dx,w,xj,sum,rnf,xav,xs
  complex(kind=8):: ci
  !=======================================================-
  !  nyuuryoku
  !=======================================================-
  Nmax = 100
  rl = 4d0
  sig = 0.1d0
  rk = 20d0
  x0 = -0.5d0
  jmax = 2*int(rl)*256
  dt = 1d0*10d0**(-5d0)
  n1max = int(0.05d0/dt/dble(nmax))
  write(*,'(''# Jmax  = '',i12)') jmax
  write(*,'(''# Nmax  = '',i12)') nmax
  write(*,'(''# N1max = '',i12)') n1max
  write(*,'(''# L     = '',e18.8e3)') rl
  write(*,'(''# sig   = '',e18.8e3)') sig
  write(*,'(''# k0    = '',e18.8e3)') rk
  write(*,'(''# x0    = '',e18.8e3)') x0
  write(*,'(''# dt    = '',e18.8e3)') dt
  !=======================================================-
  ! junbi
  !=======================================================-
  a(1)=0.0d0
  a(2)=0.5d0
  a(3)=0.5d0
  a(4)=1.0d0
  b(1)=1.0d0/6.0d0
  b(2)=1.0d0/3.0d0
  b(3)=1.0d0/3.0d0
  b(4)=1.0d0/6.0d0
  dx=2.0d0*rl/dble(jmax)
  w =0.5d0/dx**2
  ci=(0.0d0,1.0d0)
  do j=1,jmax+1
    x(j)=dx*dble(j-1)-rl
  enddo
  !=======================================================-
  ! shoki-jooken
  !=======================================================-
  sum=0.0d0
  do j=2,jmax
    xj=x(j)
    cp(j)=cdexp(ci*rk*xj-((xj-x0)/(2.0d0*sig))**2)
    sum=sum+cp(j)*dconjg(cp(j))
  enddo
  cp(1)     =(0.d0,0.d0)
  cp(jmax+1)=(0.d0,0.d0)
  rnf=1.0d0/dsqrt(sum*dx)
  do j=2,jmax
    cp(j)=rnf*cp(j)
  enddo
  !
  n=0
  write(*,'( ''#''/ ''# time'',12x,''  <x> '',12x, ''  <(x-<x>)**2>'' )')
  !=======================================================-
  ! shuukei
  !=======================================================-
  ! r=:|psi|^2, cp=:psi
  do j=2,jmax
    r(j)=cp(j)*dconjg(cp(j))
  enddo
  ! mean x
  xav=0.0d0
  do j=2,jmax
    xav=xav+x(j)*r(j)*dx
  enddo
  ! variance x
  xs=0.0d0
  do j=2,jmax
    xs=xs+((x(j)-xav)**2)*r(j)*dx
  enddo
  write(*,'(3e18.8e3)') dt*n1max*n,xav,xs
  !
  cdp(:)=0.0d0
  !=======================================================-
  do n=1,Nmax
    do n1=1,N1max
      !==============================================-
      ! 1 step sekibun
      !==============================================-
      do j=2,jmax
        cs(j)=(0.0d0,0.0d0)
      enddo
      do m=1,4
        do j=2,jmax
          cps(j)=cp(j)+a(m)*cdp(j)
        enddo
        do j=2,jmax
          cdp(j)=(cps(j+1)-2.0d0*cps(j)+cps(j-1))*w*ci*dt
        enddo
        do j=2,jmax
          cs(j)=cs(j)+b(m)*cdp(j)
        enddo
      enddo
      do j=2,jmax
        cp(j)=cp(j)+cs(j)
      enddo
    enddo
    !=======================================================-
    ! shuukei
    !=======================================================-
    do j=2,jmax
      r(j)=cp(j)*dconjg(cp(j))
    enddo
    xav=0.0d0
    do j=2,jmax
      xav=xav+x(j)*r(j)*dx
    enddo
    xs=0.0d0
    do j=2,jmax
      xs=xs+((x(j)-xav)**2)*r(j)*dx
    enddo
    write(*,'(3e18.8e3)') dt*n1max*n,xav,xs
  enddo
end program wave
\end{lstlisting}
\newpage
\begin{lstlisting}[caption=厳密解の計算プログラム,label=src4ad]
program main
  implicit none
  DOUBLE PRECISION, PARAMETER :: dt = 1d0*10d0**(-5d0)
  DOUBLE PRECISION, PARAMETER :: x0 = -0.5d0, k0 = 20d0, sigma=0.1d0
  DOUBLE PRECISION :: xav, xs, t
  INTEGER, PARAMETER :: nmax = 100
  INTEGER :: n1max = int(0.05d0/dt/dble(nmax)), n
  write(*,'( ''#''/ ''# time'',12x,''  <x> '',12x, ''  <(x-<x>)**2>'' )')
  do n=0,nmax
    t = dt*n1max*n
    xav = -0.5d0 + k0 * t
    xs = sigma**2d0 +t**2d0 / 4d0 / sigma**2d0
    write(*,'(3e18.8e3)') t,xav,xs
  end do
end program main
\end{lstlisting}
\newpage
\begin{lstlisting}[caption=Runge-Kutta関数のソースコード,label=srcRK4]
module differential
  implicit none
    
  contains
  !runge_kutta の計算を 1 ステップ進める
  !arg: 微分方程式; n: 連立する数; init: 初期値; t_begin: 計算開始; t_end: 計算終了; tau: 刻み幅;
  !const: 定数(optional); boundary: 境界条件(optional)
  function runge_kutta(arg, n, init, t_begin, tau, const, boundary)
    implicit none
    interface
      ! 微分方程式
      function arg(t, x, n, const)
        INTEGER, INTENT(IN) :: n
        DOUBLE PRECISION :: arg(n)
        DOUBLE PRECISION, INTENT(IN) :: x(:), t
        DOUBLE PRECISION, OPTIONAL :: const(:)
      end function arg
      ! 境界条件の計算関数
      function boundary(x, n)
        implicit none
        INTEGER, INTENT(IN) :: n
        DOUBLE PRECISION, INTENT(IN) :: x(:)
        DOUBLE PRECISION, OPTIONAL :: boundary(n)
      end function boundary
    end interface
    INTEGER :: i
    ! 段数
    INTEGER, PARAMETER :: order = 4
    INTEGER, INTENT(IN) :: n
    DOUBLE PRECISION, INTENT(in) :: init(:), tau
    DOUBLE PRECISION, INTENT(INOUT) :: t_begin
    DOUBLE PRECISION :: runge_kutta(n), x(n), t, s(n), delta(n)
    DOUBLE PRECISION, OPTIONAL :: const(:)
    DOUBLE PRECISION :: a(order), b(order)
    a(1)=0d0; a(2)=0.5d0; a(3)=0.5d0; a(4)=1d0
    b(1)=1d0/6d0; b(2)=1d0/3d0; b(3)=1d0/3d0; b(4)=1d0/6d0
    x(:) = init(:)
    t = t_begin

    s = 0; delta = 0
    do i = 1, order
      ! optional の定数が与えられている場合はそれを含む計算を実行
      if (PRESENT(const)) then 
        delta(:) = arg(t+a(i)*tau, x(:) + a(i) * delta, n, const)*tau
      else
        delta(:) = arg(t+a(i)*tau, x(:) + a(i) * delta, n)*tau
      end if
      s(:) = s(:) + b(i) * delta(:)
    end do
    x(:) = x(:) + s(:)
    t = t + tau
    t_begin = t
    ! optional の境界条件が与えられている場合はそれを考慮
    if (PRESENT(boundary)) then
      x(:) = boundary(x, n)
    end if
    runge_kutta(:) = x(:)
  end function runge_kutta
end module differential
\end{lstlisting}
