\section{結果}
\subsection{g因子の測定}
図\ref{fig:graph/out/g_factor_17M.pdf}から図\ref{fig:graph/out/g_factor_23M.pdf}に$17\ \si{\mega\hertz}$, $20\ \si{\mega\hertz}$, $23\ \si{\mega\hertz}$の場合の吸収曲線を示す.
ただし$17\ \si{\mega\hertz}$の場合は雑音が大きいためウィンドウサイズ8での移動平均を掛けた場合を図\ref{fig:graph/out/g_factor_17M_mean.pdf}に示す.
各周波数について測定結果は表\ref{tab:ginsi}のようになる.
\begin{table}[h]
\caption{g因子の測定結果}
\label{tab:ginsi}
\centering
\begin{tabular}{ccccc}
\hline
\multirow{2}{*}{共振周波数 / MHz}&\multicolumn{2}{c}{ピーク電圧 / mV}&\multirow{2}{*}{ピーク電圧(外部磁場除去後) / mV}&\multirow{2}{*}{g因子 / $\si{\joule.\tesla^{-1}}$}\\
&ピーク1&ピーク2\\
\hline \hline
17.178&60&-58&59&-1.7\\
20.865&64&-70&67&-1.9\\
17.178&60&-58&59&-1.8\\
\hline
\end{tabular}
\end{table}
\newpage
\begin{figure}[htbp]
  \begin{minipage}{0.5\hsize}
    \mfig[width=8cm]{graph/out/g_factor_17M.pdf}{$17\ \si{\mega\hertz}$の時の吸収曲線}
  \end{minipage}
  \begin{minipage}{0.5\hsize}
    \mfig[width=8cm]{graph/out/g_factor_20M.pdf}{$20\ \si{\mega\hertz}$の時の吸収曲線}
  \end{minipage}
\end{figure}
\mfig[width=8cm]{graph/out/g_factor_23M.pdf}{$23\ \si{\mega\hertz}$の時の吸収曲線}
\newpage
\mfig[width=8cm]{graph/out/g_factor_17M_mean.pdf}{$17\ \si{\mega\hertz}$の時の吸収曲線(移動平均)}
\subsection{地球磁場の測定}
表に各$\theta$での測定結果を示す.ここでは地磁気以外の外部磁場は存在しないものとした.
また図に$\theta$と地磁気の関係,並びにcos関数でのfitting結果を示す.
fittingにより地磁気の絶対値は$(-2.9\pm0.2)\times10^4\ \si{\nano\tesla}$となった.
\begin{table}[h]
\caption{$\theta$と地磁気の関係}
\label{tab:tijiki}
\centering
\begin{tabular}{ccccc}
\hline
\multirow{2}{*}{$\theta$ / $\si{\degree}$}&\multicolumn{2}{c}{ピーク電圧 / mV}&\multirow{2}{*}{$\overline{V_{ch1}}$ / mV}&\multirow{2}{*}{地磁気 / nT}\\
&ピーク1&ピーク2\\
\hline \hline
0 & 64 & -70 & -3 & $-3.6\times10^4$ \\
30 & 66 & -70 & -2 & $-2.4\times10^4$ \\
60 & 66 & -70 & -2 & $-2.4\times10^4$ \\
90 & 68 & -68 & 0 & 0.0 \\
120 & 68 & -66 & 1 & $1.2\times10^4$ \\
150 & 70 & -66 & 2 & $2.4\times10^4$ \\
180 & 70 & -66 & 2 & $2.4\times10^4$ \\
\hline
\end{tabular}
\end{table}
\gnu{方位と磁場強度の関係}{graph/azimuth.tex}