\documentclass[uplatex,a4j,11pt,dvipdfmx]{jsarticle}
\usepackage{listings,jvlisting}
\bibliographystyle{jplain}

\usepackage{url}
\usepackage{chemfig}
\usepackage{graphicx}
\usepackage{gnuplot-lua-tikz}
\usepackage{pgfplots}
\usepackage{tikz}
\usepackage{amsmath,amsfonts,amssymb}
\usepackage{bm}
\usepackage{siunitx}
\usepackage{multirow}

% 使用する括弧を定義
\newcommand\setpolymerdelim[2]{\def\delimleft{#1}\def\delimright{#2}}
% 実際に括弧を描くコマンドの定義
\def\makebraces[#1,#2]#3#4#5{%
  \edef\delimhalfdim{\the\dimexpr(#1+#2)/2}%
  \edef\delimvshift{\the\dimexpr(#1-#2)/2}%
  \chemmove{%
    \node[at=(#4),yshift=(\delimvshift)]
    {$\left\delimleft\vrule height\delimhalfdim depth\delimhalfdim
        width0pt\right.$};%
    \node[at=(#5),yshift=(\delimvshift)]
    {$\left.\vrule height\delimhalfdim depth\delimhalfdim
        width0pt\right\delimright_{\rlap{$\scriptstyle#3$}}$};}}

\makeatletter
\def\fgcaption{\def\@captype{figure}\caption}
\makeatother
\newcommand{\setsections}[3]{
\setcounter{section}{#1}
\setcounter{subsection}{#2}
\setcounter{subsubsection}{#3}
}
\newcommand{\mfig}[3][width=15cm]{
\begin{center}
\includegraphics[#1]{#2}
\fgcaption{#3 \label{fig:#2}}
\end{center}
}
\newcommand{\gnu}[2]{
\begin{figure}[hptb]
\begin{center}
\input{#2}
\caption{#1}
\label{fig:#2}
\end{center}
\end{figure}
}

\begin{document}
\section{目的}
スピン共鳴装置は医療や物質の分析などにおいて広く用いられる手法である.
当実験では電子スピン共鳴を用いて1-ジフェニル-2-ピクリルヒドラジルのg因子の測定,
共鳴曲線の決定,温度依存性の測定,地磁気の測定,またヘルムホルツコイル内の磁場分布の測定を行う.
また$^{1}{\rm H}$, $^{19}{\rm H}$の核磁気共鳴を測定し,それぞれのg因子を測定する.
\section{原理}
\subsection{X線の発生}
電磁波の発生過程には黒体放射,制動放射,遷移放射などがある.
今回の実験で用いたX線発生装置では制動放射と遷移輻射が支配的になっているのでこれらについて述べる.
\subsubsection{連続X線}
連続X線とは制動放射により生じるX線である.
連続X線は図\ref{fig:fig/XraySpectrum.jpg}のスペクトルの連続的な部分である.
図\ref{fig:fig/Xray_gen.png}にX線発生装置の概略を示す.
電子銃から放射された電子は銅のターゲットに衝突する.
ターゲットに衝突した電子は様々な方向に散乱されるが,
その程度によって制動の具合が異なってくる.これによって様々なエネルギーのX線が放射されることになる.
ここで電子銃の加速電圧を$V$としよう.
最も制動が大きいのは1回の衝突で全エネルギーを失うことに相当するので$hc/\lambda=eV$から
\begin{align}
  \lambda_{m}=\frac{hc}{eV}
\end{align}
となる.この波長が加速電圧$V$の連続X線で得られる最も高エネルギーな(すなわち波長が短い)光になる.
連続X線の全強度は管電流を$i$, ターゲット原子の原子番号を$Z$とすれば
\begin{align}
  I\propto iV^mZ
\end{align}
となり,ターゲットに重元素を用いると効率よくX線を得ることができる.\cite{XrayS}
連続X線スペクトルを求める方法としてはBirch-Marshallのモデルが用いられる.\cite{Xraymodel}
\subsubsection{特性X線}
特性X線(または固有X線)とは遷移放射により生じるX線である.
特性X線は図\ref{fig:fig/XraySpectrum.jpg}の鋭いピークの部分である.
特性X線は主量子数$n$の高い準位にいる電子が低い準位に遷移する際に放射されるX線である.
X線の波長は準位間のエネルギーの差を$\Delta E$とすれば
\begin{align}
  \lambda=\frac{hc}{\Delta E}
\end{align}
となり,同じ遷移による放射は常に同じ波長となることから鋭いピークが現れる.
原子では主量子数が0の軌道をK殻, 1の軌道をL殻, 2の軌道をM殻と呼ぶが,
それぞれの遷移について名前がついている.
図\ref{fig:fig/KLM.jpg}のようにL殻からK殻への遷移によるX線を$\rm K_{\alpha}$線,
M殻からK殻への遷移によるX線を$\rm K_{\beta}$線と呼ぶ.また角運動量量子数$j$の違いなどによってL, M殻は微細構造を持つので波長が僅かに異なるX線が発生する.
特に$\rm L_{III}$殻からKへの遷移によるX線は$\rm K_{\alpha 1}$線, 特に$\rm L_{II}$殻からKへの遷移によるX線は$\rm K_{\alpha 2}$線などと呼ばれる.
これらの遷移確率は約$2:1$であり,すなわち強度比も$2:1$となる.
これらのX線の波長は非常に近接しており,実験的には分離が困難なため,その波長を加重平均した
\begin{align}
  \lambda_{\rm K_{\alpha}}=\frac{2\lambda_{{\rm K}_{\alpha 1}}+\lambda_{{\rm K}_{\alpha 2}}}{3}
\end{align}
を用いる場合が多い.\cite{XrayS}
ターゲットとして銅を用いた場合,これらのX線の波長は表\ref{tab:lambda_xray}のようになる.
\begin{table}[h]
\caption{X線の波長}
\label{tab:lambda_xray}
\centering
\begin{tabular}{cc}
\hline
&$波長\ /\ \si{\angstrom}$\\
\hline \hline
${\rm K}_{\alpha 1}$&1.5405\\
${\rm K}_{\alpha 2}$&1.5443\\
${\rm K}_{\alpha}$&1.5418\\
${\rm K}_{\beta}$&1.3922\\
\hline
\end{tabular}
\end{table}
\mfig[width=8cm]{fig/XraySpectrum.jpg}{モリブデンターゲットによるX線のスペクトル\cite{alma990007897430204034}}
\mfig[width=6cm]{fig/Xray_gen.png}{X線発生装置の概略(実験テキストから引用)}
\mfig[width=10cm]{fig/KLM.jpg}{準位間の遷移\cite{XrayS}}

\section{実験方法}
\subsection{実験装置}
\subsubsection{真空装置の構成}
図\ref{fig:fig/fig7.png}に真空装置の構成図を示す.
RPは油回転ポンプDPは油拡散ポンプである.また${\rm V}_n$はバルブである.
コンダクタンス管はフレキシブルパイプで接続され,交換が可能である.
真空チャンバーは内径$216.3\ \si{\milli\metre}$,高さ$250\ \si{\milli\metre}$の円筒容器である.
またコンダクタンス管は内径$10\ \si{\milli\metre}$のものと$15\ \si{\milli\metre}$のものがあり,
共に長さは$200\ \si{\milli\metre}$である.
フランジは旧JIS規格のVF/VGフランジでありOリングと共に用いることで$10\ \si{\micro\pascal}$程度の高真空を実現できる.\cite{VacuumTe71:online}
油回転ポンプは大気圧から$1\ \si{\pascal}$程度まで減圧することができる.
一方で油拡散ポンプは低圧($6\ \si{\pascal}$程度)から高真空まで減圧することができる.
ただし停止中にチャンバー側が真空になっていると油が逆流するため,吸気側をリークする必要がある.
油拡散ポンプを使用する際は${\rm V}_2$を開き,排気された気体を油回転ポンプで更に排気する必要がある.
また電離真空計は$0.6\ \si{\pascal}$以上で使用することはできない.
\mfig[width=12cm]{fig/fig7.png}{真空装置の構成}
\subsubsection{熱電子電流測定装置の構成}
図\ref{fig:fig/fig12.png}に熱電子放出の実験系の構成を示す.
フィラメントはTaであり,電流$I_f$を流すとジュール熱により発熱する.
アノードには引き込み電圧$V_A$が掛けられている.
熱電子がアノードに到達すると電流計が熱電子電流を検出する.
\mfig[width=6cm]{fig/fig12.png}{熱電子電流測定装置の構成}
\subsection{実験手順}
\subsubsection{排気手順}
図\ref{fig:fig/fig8.png}に粗引き手順,図\ref{fig:fig/fig9.png}に油回転ポンプの起動手順
図\ref{fig:fig/fig10.png}に真空引きの手順のフローチャートを示す.
油回転ポンプによる粗引きの後油拡散ポンプを起動,高真空までの真空引きを行っている.
これにより油拡散ポンプへ大気圧を流入させることなく高真空を実現できる.
\begin{figure}[htbp]
  \begin{minipage}{0.5\hsize}
    \centering
    \mfig[width=3cm]{fig/fig8.png}{粗引き手順}
  \end{minipage}
  \begin{minipage}{0.5\hsize}
    \centering
    \mfig[width=3cm]{fig/fig9.png}{油回転ポンプの起動手順}
  \end{minipage}
\end{figure}
\mfig[width=6cm]{fig/fig10.png}{真空引き手順}
\newpage
\subsubsection{真空装置のシャットダウン手順}
図\ref{fig:fig/fig11.png}に真空装置のシャットダウン手順を示す.
シャットダウンの順番は真空引きの際の逆順である.
\mfig[width=6cm]{fig/fig11.png}{シャットダウン手順}
\subsubsection{コンダクタンスの測定}
コンダクタンス管がない場合と2種のコンダクタンス管それぞれについて図\ref{fig:fig/fig13.png}の手順で圧力の時間変化を各5回ずつ測定した.
(\ref{equ:216-pt})によれば圧力の時間変化は指数関数的であり,その時定数は排気速度に依存した.
したがって得られた圧力の時間変化を指数関数でfittingし,その時定数を測定することで排気速度を得ることができる.
コンダクタンス管を取り付けた場合,合成コンダクタンスは(\ref{equ:216-conduct})で表されるので,
コンダクタンス管がない場合の排気速度からコンダクタンス管のコンダクタンスを求める.
以上からコンダクタンスの半径依存性を求め,チャンバーの気体が粘性流か分子流かを判定する.
\mfig[width=6cm]{fig/fig13.png}{コンダクタンスの測定}
\subsubsection{熱電子電流の測定}
図\ref{fig:fig/fig14.png}の手順で熱電子電流$I_P$を測定した.
また有限要素法解析によりフィラメント電流$I_f$とフィラメント温度の関係を計算した.
これによりSchottkyプロットを作成し,外挿により$I_S$を求めた.
(\ref{equ:221-RDmann})から$\log(I_S/T^2)=a-\phi_0/k_BT$なので縦軸を$\log(I_ST^2)$,
横軸を$1/T$とすればその傾きが$-\phi_0/k_B$となり,仕事関数を求められる.
\mfig[width=3cm]{fig/fig14.png}{熱電子電流の測定}
\newpage
\subsubsection{質量分析}
図\ref{fig:fig/fig15.png}の手順で質量スペクトル及び吸着,脱離時の圧力変化を測定した.
質量スペクトルから残留気体の組成を同定した.
\mfig[width=7cm]{fig/fig15.png}{熱電子電流の測定}

\subsection{NMR測定実験}
\subsubsection{NMR測定装置概略}
図\ref{fig:fig/fig5.png}にNMR測定装置の概略を示す.基本的な構成はESR測定装置と同じだが,
ヘルムホルツコイル(b)と同軸に
直流電源と接続されたヘルムホルツコイル(c)が配置されている.
(20)式より核子の共鳴条件は
\begin{align}
  |B_0|=\frac{\omega_0\hbar}{g\mu_n}=\frac{2m_n\omega_0}{ge}
\end{align}
ここで$\mu_n$はボーア磁子の質量を核子の質量$m_n$で置き換えたものである.
以上からわかるとおり質量の大きい核子の共鳴吸収を測定するにはより強い磁場が必要になる.
これを達成するためヘルムホルツコイル(c)を用いて変動磁場をオフセットし,磁場を強くしている.
またヘルムホルツコイル(c)は大電流が流れるため冷却水を流す必要がある.
\mfig[width=10cm]{fig/fig5.png}{NMR測定装置}
\subsubsection{試料}
ここでは水素($^1{\rm H}$)とフッ素同位体($^{19}{\rm F}$)の核磁気共鳴を測定した.
それぞれ水($H_2O$)及びフッ素樹脂であるTeflon
($
\setpolymerdelim()
\chemfig{\vphantom{CF_2}-[@{op,.75}]CF_2-CF_2-[@{cl,.25}]}
\makebraces[3pt,3pt]{\!\!n}{op}{cl}
$)を試料として用いた.また純粋な水の水素の共鳴線幅は非常に狭く,検出が困難なので常磁性イオン(硫酸銅:$\rm CuSO_4\cdot5H_2O$及び鉄アルミニウムミョウバン:$\rm FeNH_4(SO_4)_2\cdot12H_2O$)
を用いて共鳴線幅を広げた.
\subsubsection{準備}
NMR測定装置の配線は事前に行われていた.
電源投入前に冷却水の循環を始める必要がある.
冷却水の循環開始後,全機器の電源を投入した.
試料は装置内に固定されている.以上で測定準備は完了した.
\subsubsection{測定手順}
直流電源の電流を$1.2\ \si{\ampere}$から徐々に減らしていき,ピークが現れる上限と下限の電流を記録した.
この中央の電流値が共鳴吸収が最大になっている電流とする.
同様の測定を$^1{\rm H}$及び$^{19}{\rm F}$それぞれのピークについて行った.
それぞれの共鳴周波数と磁場の強さからg因子を測定した.
ただし電流$I$とヘルムホルツコイル(c)の磁場$B$は以下の式で対応付けられる.
\begin{align}
  B=0.4667I+0.029I^2-0.0157I^3
\end{align}
\section{結果}
\subsection{ホログラフィー干渉}
pythonスクリプトを用いて生成した平板回転による干渉のシミュレーション結果を図\ref{fig:fig/zrot.png},図\ref{fig:fig/yrot.png}に示す.
ここで計算条件は(\ref{equ:132_zrot})式, (\ref{equ:132_yrot})式から求めたものであり,表\ref{tab:keisanjouken}のとおりである.
図\ref{equ:132_zrot}を見ると,干渉縞は斜めに入っており,また明線の本数は$x=0\ \si{\milli\metre}$で数えると20本である.
図\ref{equ:132_yrot}を見ると,干渉縞はほぼ垂直に入っており,また明線の本数は$y=0\ \si{\milli\metre}$で数えると20本である.
\begin{table}[h]
\caption{計算条件}
\label{tab:keisanjouken}
\centering
\begin{tabular}{ccc}
\hline
条件&回転軸&角度$\theta$\\
\hline \hline
1&$z$軸回転&$2.98\times10^{-4}$\\
2&$y$軸回転&$9.27\times10^{-5}$\\
\hline
\end{tabular}
\end{table}
\mfig[width=8cm]{fig/zrot.png}{条件1($z$軸回転)}
\mfig[width=8cm]{fig/yrot.png}{条件2($y$軸回転)}
\subsection{自由課題4}
図\ref{fig:fig/free4_trim.png}に得られたホログラフィ像を示す.ただし図\ref{fig:fig/free4_trim.png}は消しゴムが写っている部分だけをトリミングしている.
図\ref{fig:fig/free4_trim.png}を見ると左右が干渉縞の間隔が狭く,中央ほど間隔が広いことがわかる.
また中央下部に暗い領域がある.
\mfig[width=8cm]{fig/free4_trim.png}{自由課題4のホログラフィ像}
\bibliography{ref.bib}
\end{document}