\documentclass[uplatex,a4j,11pt,dvipdfmx]{jsarticle}
\usepackage{listings,jvlisting}
\bibliographystyle{jplain}

\usepackage{standalone}
\usepackage{url}
\usepackage{chemfig}
\usepackage{graphicx}
\usepackage{gnuplot-lua-tikz}
\usepackage{pgfplots}
\usepackage{tikz}
\usepackage{amsmath,amsfonts,amssymb}
\usepackage{bm}
\usepackage{siunitx}
\usepackage{multirow}

% 使用する括弧を定義
\newcommand\setpolymerdelim[2]{\def\delimleft{#1}\def\delimright{#2}}
% 実際に括弧を描くコマンドの定義
\def\makebraces[#1,#2]#3#4#5{%
  \edef\delimhalfdim{\the\dimexpr(#1+#2)/2}%
  \edef\delimvshift{\the\dimexpr(#1-#2)/2}%
  \chemmove{%
    \node[at=(#4),yshift=(\delimvshift)]
    {$\left\delimleft\vrule height\delimhalfdim depth\delimhalfdim
        width0pt\right.$};%
    \node[at=(#5),yshift=(\delimvshift)]
    {$\left.\vrule height\delimhalfdim depth\delimhalfdim
        width0pt\right\delimright_{\rlap{$\scriptstyle#3$}}$};}}

\makeatletter
\def\fgcaption{\def\@captype{figure}\caption}
\makeatother
\newcommand{\setsections}[3]{
\setcounter{section}{#1}
\setcounter{subsection}{#2}
\setcounter{subsubsection}{#3}
}
\newcommand{\mfig}[3][width=15cm]{
\begin{center}
\includegraphics[#1]{#2}
\fgcaption{#3 \label{fig:#2}}
\end{center}
}
\newcommand{\mfis}[3][width=15cm]{
\begin{center}
\includestandalone[#1]{#2}
\fgcaption{#3 \label{fig:#2}}
\end{center}
}
\newcommand{\gnu}[2]{
\begin{figure}[hptb]
\begin{center}
\input{#2}
\caption{#1}
\label{fig:#2}
\end{center}
\end{figure}
}

\begin{document}
\section{目的}
スピン共鳴装置は医療や物質の分析などにおいて広く用いられる手法である.
当実験では電子スピン共鳴を用いて1-ジフェニル-2-ピクリルヒドラジルのg因子の測定,
共鳴曲線の決定,温度依存性の測定,地磁気の測定,またヘルムホルツコイル内の磁場分布の測定を行う.
また$^{1}{\rm H}$, $^{19}{\rm H}$の核磁気共鳴を測定し,それぞれのg因子を測定する.
\section{原理}
\subsection{磁気モーメント}
\subsubsection{定義}
磁気モーメント$\bm m$は図\ref{fig:fig/fig1.png}のように面積$\bm S$の周囲を囲む電流$I$を用いて
\begin{align}
  {\bm m}=I{\bm S}
\end{align}
で定義される.磁気双極子モーメントは図\ref{fig:fig/fig2.png}のように磁気単極子$\pm m$を用いて定義されたが,
磁気単極子は未発見であり,磁気モーメントがより現実に即した定義だと言える.
ここで電流$I$が点電荷の円運動によって生じていると解釈すると,電荷を$q$,
円運動の半径を$a$,速度を$v$とすると
\begin{align}
  I=\frac{v}{2\pi a}q
\end{align}
ここで角運動量は$L=mva$なので
\begin{align}
  {\bm m}=\frac{qv{\bm S}}{2\pi a}=\frac{qv\cdot\pi a^2}{2\pi a}=\frac{q{\bm L}}{2m}
\end{align}
となる.
\begin{figure}[htbp]
  \begin{minipage}{0.5\hsize}
    \mfig[width=4cm]{fig/fig1.png}{磁気モーメントの定義}
  \end{minipage}
  \begin{minipage}{0.5\hsize}
    \mfig[width=4cm]{fig/fig2.png}{磁気双極子モーメントの定義}
  \end{minipage}
\end{figure}
\subsubsection{電子,陽子,中性子の磁気モーメント}
電子や核子において,軌道角運動量$\bm L$は量子化され
\begin{align}
  {\bm L}=\hbar\sqrt{L(L+1)}
\end{align}
と表される.したがってこれらの磁気モーメントの基本量は$q\hbar/2m$となる.
特に$m$を電子の質量$m_e$としたとき$\mu_B=e\hbar/2m_e$はBohr磁子とよばれ,磁気モーメントの基本単位として使われる..
また電子や核子には軌道角運動量の他にスピン角運動量${\bm s}=\pm1/2$が存在し,
スピン角運動量に由来する磁気モーメントは
\begin{align}
  {\bm m}=g\mu_B|{\bm s}|
\end{align}
で与えられる.ここで$g$はg因子と呼ばれ量子電磁気学によれば電子の場合は$g_e\simeq-2.00232$,
陽子の場合は$g_p\simeq5.586$,中性子の場合は$g_n\simeq-3.826$となることが知られている.
すなわち軌道角運動量に由来する磁気モーメントの表式に比べて,
スピン角運動量に由来する磁気モーメントは定数倍されていることがわかる.

一般には原子核は複数の核子から構成されており,磁気モーメントはそれぞれの核子が持つスピン角運動量の和
${\bm J}=\hbar{\bm I}$に依存する.
${\bm I}$を用いて原子核の磁気モーメントは
\begin{align}
  {\bm m}_n=\gamma{\bm J}=\gamma\hbar{\bm I}
\end{align}
で与えられる.ここで$\gamma$は磁気角運動量比と呼ばれ,核種に依存する定数である.
\subsection{磁気共鳴の古典的説明}
\subsubsection{ラーモア歳差運動}
前節でみたように磁気モーメントは角運動量${\bm J}$に対して
\begin{align}
  {\bm m}=\gamma{\bm J}
\end{align}
という関係にあった.磁気モーメントに磁場${\bm B}$を掛けたとき,磁気モーメントが受けるトルク${\bm N}$は一般に
\begin{align}
  {\bm N}={\bm m}\times {\bm B}
\end{align}
である.したがって角運動量に対する運動方程式$\dot{{\bm L}}={\bm N}$を用いれば
\begin{align}
  \begin{split}
    \frac{d{\bm J}}{dt}&={\bm m}\times {\bm B}\\
    \frac{d{\bm m}}{dt}&=\gamma{\bm m}\times {\bm B}
  \end{split}
\end{align}
となることがわかる.ここで磁場を
\begin{align}
  {\bm B}=\left(\begin{array}{c}
    0\\0\\B_0
  \end{array}\right)
\end{align}
とすると
\begin{align}
  \frac{d}{dt}\left(\begin{array}{c}
    m_x\\m_y\\m_z
  \end{array}\right)=
  \gamma B_0\left(\begin{array}{c}
    m_y\\-m_x\\0
  \end{array}\right)
\end{align}
となるので一般解は
\begin{align}
  \begin{split}
    m_x&=A\cos(\omega_0 t+\alpha)\\
    m_y&=-A\sin(\omega_0 t+\alpha)\\
    m_z&={\rm Const}
  \end{split}
\end{align}
となる.ここで$\omega_0=B_0\gamma$である.
以上から磁気モーメントは$z$方向の磁場中では$z$軸回りで回転することがわかる.この運動がラーモア歳差運動である.
\mfig[width=4cm]{fig/fig3.png}{ラーモア歳差運動}
\subsection{磁気共鳴の量子論による解釈}
前節では古典的な磁気モーメントが定常磁場の元では歳差運動を示すことが示された.
これを量子系で考える.磁場${\bm B}$中にある磁気モーメント${\bm m}$のハミルトニアンは
\begin{align}
  H=-{\bm m}\cdot{\bm B}
\end{align}
である.ここで磁気モーメントとして(6)式を,磁場として(10)式を用いると
\begin{align}
  \begin{split}
    H&=-\gamma {\bm J}\cdot{\bm B}\\
    &=-\gamma B_0J_z\\
    &=-\omega_0 J_z
  \end{split}
\end{align}
となる.ここで角運動量としてスピン$s_{\pm}=\pm\hbar/2$を考えるとエネルギー固有値は
\begin{align}
  E_{\pm}=\pm\frac{\hbar\omega_0}{2}
\end{align}
となり縮退が解けている.また準位間のエネルギーは$\hbar\omega_0$となり,
これはラーモア歳差運動の各振動数と同じ角振動数の光のエネルギーに相当する.
このことから磁場中にあるスピン由来の磁気モーメントは角振動数$\omega_0$の光を吸収し,励起状態に遷移できることが示唆される.

\section{測定}
\subsection{$2\theta$-$\theta$スキャン}
これまで示した通り,この実験においては散乱角$\theta$はMiller指数と対応し,
これを正確に決めることは構造の同定において重要である.
図に実験の概略図を示す.
図において散乱角は
\begin{align}
  2\theta+\gamma=180\si{\degree}
\end{align}
となっている.
$2\theta$-$\theta$スキャンとは,試料とX線源,検出器が同一の円周上に並んでいると考えることで,
円周角の定理より$\alpha=\beta=\gamma$となり,
X線のビーム幅がある程度大きくても正確に散乱角を測定できるセットアップである.
これによってX線強度を落とさずに正確に散乱角を測定できる.

また$2\theta$-$\theta$スキャンで散乱角を変化させるとき,
試料を$\phi$回転させると散乱角は$2\phi$変化することになる.
$2\theta$-$\theta$スキャンという名前はこの散乱角と試料の回転角との対応に由来する名称である.
\subsection{実験装置}
図\ref{fig:fig/setup.png}に実験装置のセットアップを示す.
実験装置は全て定盤上に固定されている.
実験装置は中央のX線管があり,その左右に試料台とX線露光装置が設置されている.
X線管とX線露光装置の型番は表\ref{tab:souti}のとおりである.
X線管から発生したX線は光学系でビーム幅$0.1\ \si{\milli\metre}$程度に収束され,
コリメータを出て試料に照射される,試料を透過したX線は対する露光装置によって観測され2次元画像としてデータが出力される.
ただし直接透過したX線が露光装置に当たると故障の恐れがあるため,受光面の中央にビームストップを設置し,それを防いでいる.
試料は備え付けられたゴニオメータに取り付けられ,顕微鏡を覗きながら位置を調整できる.
また装置自体は放射線遮蔽ガラスの扉に覆われており,完全に扉が閉まっていなければ装置は作動しないようになっている.
更に動作中に扉が開いた場合直ちに装置がシャットダウンする.
\begin{table}[h]
  \caption{実験装置の型番}
  \label{tab:souti}
  \centering
  \begin{tabular}{cc}
  \hline
  &名称\\
  \hline \hline
  X線管&Rigaku Ultrax-18\\
  露光装置&Rigaku R-AXIS IV\\
  \hline
  \end{tabular}
\end{table}
\mfig[width=8cm]{fig/setup.png}{実験装置(実験テキストから引用)}
\subsection{X線回折法}
Schr\"{o}dinger方程式は
\begin{align}
  \left(-\frac{\hbar^2}{2\mu}\nabla^2+V(\vec{r})\right)\psi(\vec{r})=E\psi(\vec{r})
\end{align}
ここで両辺$-2\mu/\hbar^2$を掛けると
\begin{align}
  (\nabla^2+k^2)\psi(\vec{r})=U(\vec{r})\psi(\vec{r})
\end{align}
となる.ただし$k^2=2\mu E/\hbar^2$, $U(\vec{r})=2\mu V(\vec{r})/\hbar^2$とした.
この解は平面波解
\begin{align}
  \psi_{\rm in}(\vec{r})={\rm e}^{ikz}
\end{align}
及びPoisson方程式のGreen関数
\begin{align}
  G(\vec{r})=-\frac{1}{4\pi}\frac{{\rm e}^{\pm ikr}}{r}
\end{align}
を用いて
\begin{align}
  \psi(\vec{r})=\psi_{\rm in}(\vec{r})+\int d\vec{r'}G(\vec{r}-\vec{r'})U(\vec{r'})\psi(\vec{r'})
\end{align}
と表され,これはLippmann-Schwinger方程式と呼ばれる.
散乱問題では1項目が平面波である入射波と対応し,
2項目は散乱波と対応する.
ここで積分内の$\psi(\vec{r})$に左辺を逐次代入すると
\section{結果}
\subsection{g因子の測定}
図\ref{fig:graph/out/g_factor_17M.pdf}から図\ref{fig:graph/out/g_factor_23M.pdf}に$17\ \si{\mega\hertz}$, $20\ \si{\mega\hertz}$, $23\ \si{\mega\hertz}$の場合の吸収曲線を示す.
ただし$17\ \si{\mega\hertz}$の場合は雑音が大きいためウィンドウサイズ8での移動平均を掛けた場合を図\ref{fig:graph/out/g_factor_17M_mean.pdf}に示す.
各周波数について測定結果は表\ref{tab:ginsi}のようになる.
\begin{table}[h]
\caption{g因子の測定結果}
\label{tab:ginsi}
\centering
\begin{tabular}{ccccc}
\hline
\multirow{2}{*}{共振周波数 / MHz}&\multicolumn{2}{c}{ピーク電圧 / mV}&\multirow{2}{*}{ピーク電圧(外部磁場除去後) / mV}&\multirow{2}{*}{g因子 / $\si{\joule.\tesla^{-1}}$}\\
&ピーク1&ピーク2\\
\hline \hline
17.178&60&-58&59&-1.7\\
20.865&64&-70&67&-1.9\\
17.178&60&-58&59&-1.8\\
\hline
\end{tabular}
\end{table}
\newpage
\begin{figure}[htbp]
  \begin{minipage}{0.5\hsize}
    \mfig[width=8cm]{graph/out/g_factor_17M.pdf}{$17\ \si{\mega\hertz}$の時の吸収曲線}
  \end{minipage}
  \begin{minipage}{0.5\hsize}
    \mfig[width=8cm]{graph/out/g_factor_20M.pdf}{$20\ \si{\mega\hertz}$の時の吸収曲線}
  \end{minipage}
\end{figure}
\mfig[width=8cm]{graph/out/g_factor_23M.pdf}{$23\ \si{\mega\hertz}$の時の吸収曲線}
\newpage
\mfig[width=8cm]{graph/out/g_factor_17M_mean.pdf}{$17\ \si{\mega\hertz}$の時の吸収曲線(移動平均)}
\subsection{地球磁場の測定}
表に各$\theta$での測定結果を示す.ここでは地磁気以外の外部磁場は存在しないものとした.
また図に$\theta$と地磁気の関係,並びにcos関数でのfitting結果を示す.
fittingにより地磁気の絶対値は$(-2.9\pm0.2)\times10^4\ \si{\nano\tesla}$となった.
\begin{table}[h]
\caption{$\theta$と地磁気の関係}
\label{tab:tijiki}
\centering
\begin{tabular}{ccccc}
\hline
\multirow{2}{*}{$\theta$ / $\si{\degree}$}&\multicolumn{2}{c}{ピーク電圧 / mV}&\multirow{2}{*}{$\overline{V_{ch1}}$ / mV}&\multirow{2}{*}{地磁気 / nT}\\
&ピーク1&ピーク2\\
\hline \hline
0 & 64 & -70 & -3 & $-3.6\times10^4$ \\
30 & 66 & -70 & -2 & $-2.4\times10^4$ \\
60 & 66 & -70 & -2 & $-2.4\times10^4$ \\
90 & 68 & -68 & 0 & 0.0 \\
120 & 68 & -66 & 1 & $1.2\times10^4$ \\
150 & 70 & -66 & 2 & $2.4\times10^4$ \\
180 & 70 & -66 & 2 & $2.4\times10^4$ \\
\hline
\end{tabular}
\end{table}
\gnu{方位と磁場強度の関係}{graph/azimuth.tex}
\newpage
\subsection{KClのX線ピークの指数付け}
\subsubsection{Ewald球の作図}
Ewald球の作図(図\ref{fig:fig/KCL_hkl.jpg})によって各Miller指数に対する散乱角$2\theta$は表のようになった.
ただし作図においてはKClを単純格子と見なしたため,
Miller指数は2倍になっている.

\begin{table}[h]
  \caption{Miller指数と散乱角$2\theta$の対応}
  \label{tab:miller_kclhkl}
  \centering
  \begin{tabular}{cc}
    \hline
    Miller指数&$2\theta\ /\ \si{\degree}$\\
    \hline \hline
    200&27.8\\
    220&40.0\\
    222&50.5\\
    400&59.8\\
    240&66.0\\
    440&90.0\\
    \hline
  \end{tabular}
\end{table}
\mfig[width=10cm]{fig/KCL_hkl.jpg}{KClに対するEwald球の作図}
\subsubsection{ピーク強度の理論値算出}
図\ref{fig:graph/out/KCl_raw.pdf}に示したKClのピークの強度と,理論的に予測される強度を比較することでMiller指数との対応を調べる.
PDXL2による解析では(400)以降に対応するピークが検出されなかったため,
(200), (220), (222)に対応するピークを同定する.
前節で得られた散乱角に近いピークをPDXL2の出力から抽出した.
その結果は表\ref{tab:extract_peak_i}のようになった.
ここでピーク1, 2は(200)に近く,
ピーク3, 4は(220)に近く,
ピーク5は(222)に近い.

次にEwald球の作図で得た$2\theta$に対して(21)式から散乱強度の理論値を計算すると表のようになった.
ただし温度因子の計算には
\begin{align}
  M=\frac{1.15\times10^4 T}{A\Theta^2}\left(\frac{1}{x}\int_0^x\frac{\xi}{{\rm e}^\xi-1}d\xi+\frac{x}{4}\right)\left(\frac{\sin\theta}{\lambda}\right)^2
\end{align}
を用いた.\cite{alma990007897430204034}
ただし$A$は原子量, $\Theta$はDebyeの特性温度, $x=\Theta/T$である.
ここでKClのは$\Theta=230\ \si{\kelvin}$を用い, $T$は常温として$300\ \si{\kelvin}$とした.\cite{2001167}
原子量$A$にはKの原子量とClの原子量の平均値を用いた.
また散乱因子$f$はマニュアルの付録表3, 4の値を線形補間したものを用いた.

この結果から散乱光の強度は(200)の強度が最も大きく, (220)の強度はその半分程度, (222)では$0.15$倍程度と予想される.
したがって,測定データの内尤もらしいピークは(200)に対応するのがピーク1,
(220)に対応するのがピーク3,
(222)に対応するのがピーク5であると考えられる.
図\ref{fig:graph/out/KCl_peaks.pdf}に以上の結果を示す.
\begin{table}[h]
\caption{抽出したピークの強度}
\label{tab:extract_peak_i}
\centering
\begin{tabular}{ccccc}
\hline
No.&$2\theta\ /\ \si{\degree}$&積分強度&吸収補正後強度&相対強度\\
\hline \hline
1&26.476& 13446 & 67824 & 100   \\
2&27.694& 599   & 3091  & 4.58  \\
3&40.58 & 3263  & 23381 & 34.4  \\
4&41.080& 741   & 5396  & 7.95  \\
5&50.334& 1223  & 12656 & 18.6  \\
\hline
\end{tabular}
\end{table}
\begin{table}[h]
\caption{KClの散乱強度の理論値}
\label{tab:KCl_i_riron}
\centering
\begin{tabular}{ccc}
\hline
Miller指数&$2\theta\ /\ \si{\degree}$&相対強度\\
\hline \hline
200&27.8&100\\
220&40.0&52.7\\
222&50.5&15.2\\
\hline
\end{tabular}
\end{table}
\mfig[width=10 cm]{graph/out/KCl_peaks.pdf}{Miller指数とピークの対応}

\section{考察}
\subsection{格子定数の精度について}
NaClの格子定数は$20\ \si{\degreeCelsius}$において
\begin{align}
  a=5.640\ \si{\angstrom}
\end{align}
である.\cite{rikanennpyo}
したがって測定値の相対誤差は$0.19\%$となり,よく一致していることがわかる.

各ピークから計算された格子定数は表\ref{tab:kakupi-ku}のようになっている.
NaClの線膨張率は$40\times10^{-6}\ \si{\kelvin^{-1}}$である.\cite{rikanennpyo}
したがって文献地が測定された$20\ \si{\degreeCelsius}$から仮に実験時の温度が
$10\ \si{\degreeCelsius}$度高く$30\ \si{\degreeCelsius}$だったとすると,膨張後の格子定数は
\begin{align}
  a=5.640(1+40\times10^{-6}\times 10)=5.642
\end{align}
であり,温度の影響は十分小さいと考えられる.

ここで散乱角と格子定数の関係は図\ref{fig:graph/out/NaCl_correl.pdf}のようになっており,
相関係数は$R=0.882$と強い正の相関が見える.
これは(24)式を考えると,散乱角が大きいほど散乱角が小さくなるようにシフトしていると考えられる.
このように高角度側で散乱角がシフトする原因としては固溶体の影響が考えられる.\cite{nakayama}
固溶体は結晶構造の一部原子が別の原子に置換されることや,格子内部に別の原子侵入する現象であり,
試料の不純物や経年劣化により格子定数が変化した可能性がある.
\begin{table}[h]
\caption{各ピークから計算される格子定数}
\label{tab:kakupi-ku}
\centering
  \begin{tabular}{ccc}
  \hline
  Miller指数&$2\theta\ /\ \si{\degree}$&格子定数 / $\si{\angstrom}$\\
  \hline \hline
  111&27.33&5.646\\
  200&31.69&5.641\\
  220&45.33&5.653\\
  311&53.73&5.653\\
  222&56.23&5.661\\
  \hline
  \end{tabular}
\end{table}
\mfig[width=10cm]{graph/out/NaCl_correl.pdf}{散乱角と格子定数の相関}
\subsection{NaClの散乱強度}
3.3.2節で行ったのと同様に各ピークに対して散乱光の相対強度と理論値を計算すると表\ref{tab:nacl_kyoudo}のようになる.
原子散乱因子はマニュアルの付録表3, 4\cite{bussitu}を元に線形近似で求めた.
ここで横軸に測定値,縦軸に理論値の相対強度をプロットすると図\ref{fig:graph/out/NaCl_inten.pdf}のようになる.
図\ref{fig:graph/out/NaCl_inten.pdf}を見ると測定値と理論値の相対強度はある程度一致しているが,
測定値が理論値よりも小さい傾向があることがわかる.
理論値が正確であるとするならば,誤差の要因は測定値に求めるべきである.
また全ての測定値に対して同様に小さくなっている傾向が見られることから,誤差の要因は吸収因子の計算にあると考える.
吸収因子の計算にあたっては,完全な円盤試料に対してX線が垂直に入射するとしたが,実際には試料は錠剤の形をしており,
また入射角度も完全には調整していない.
試料にX線が斜めに入射した場合,X線の通る経路が長くなるので吸収の影響はより大きくなると考えられる.
これによって全ての散乱角が同様に減衰したために,図\ref{fig:graph/out/NaCl_inten.pdf}のような結果になったと考えられる.
\begin{table}[h]
\caption{NaClの散乱強度}
\label{tab:nacl_kyoudo}
\centering
\begin{tabular}{ccccc}
\hline
\multirow{2}{*}{No.}&\multirow{2}{*}{$2\theta\ /\ \si{\degree}$}&\multirow{2}{*}{Miller指数}&\multicolumn{2}{c}{相対強度}\\
&&&測定値&理論値\\
\hline \hline
1&27.3&111&6.59&8.76\\
2&31.7&200&100&100\\
3&45.3&220&57.4&59.7\\
4&53.7&311&1.41&1.61\\
5&56.2&222&14.5&17.6\\
\hline
\end{tabular}
\end{table}
\mfig[width=10cm]{graph/out/NaCl_inten.pdf}{測定値と理論値の比較}

\bibliography{ref.bib}
\end{document}