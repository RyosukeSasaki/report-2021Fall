\section{実験方法}
\subsection{ESR測定実験}
\subsubsection{ESR測定装置概略}
図\ref{fig:fig/fig4.png}にESR測定装置の概略を示す.
ヘルムホルツコイルは水平方向に自由に回転できるようになっている.
また試料台はレールになっておりヘルムホルツコイル内を軸方向に平行移動できる.
更にレールを動径方向に付け直せば動径方向にも平行移動できる.
ここでは試料としてラジカルを持ちかつ化学的に安定である
2,2-ジフェニル-1-ピクリルヒドラジル(DPPH)を用いた.
試料はヘルムホルツコイル(b)の内部に置かれ,その磁場を受ける.
正弦波発振器は低周波($数十\ \si{\hertz}$)の信号を出力し,これによってヘルムホルツコイルは磁場の大きさを掃引する.
試料に巻きつけられたコイル(a)はマージナル発振器からの信号によって一定の電磁波(角振動数$\omega$)を発生させる.
ヘルムホルツコイルの発生させる磁場が$\omega/\gamma$のとき,共鳴条件が満たされ共鳴が起きる.
またマージナル発振器の発振周波数は周波数カウンタで監視される.
またオシロスコープのCh1にはシャント抵抗の両端電圧,
Ch2にはマージナル発振器のSignal信号が入力され,それぞれヘルムホルツコイルに流れる電流とマージナル発振器の出力信号を監視することになる.
共鳴が起きる時コイル(a)の発生させる電磁波は吸収されるため,インダクタンスが低下する.
インダクタンスの低下により(a)に流れる電流が増えるため,そのピークを測定することでESRを検出できる.
\mfig[width=10cm]{fig/fig4.png}{ESR測定装置}
\subsubsection{配線,準備}
図のように各部品同士をケーブルで接続する.ケーブルはBNCコネクタの同軸ケーブルである.
次にコイル(a)をポリウレタン線を巻くことで作成した.
コイルは直径$7.95\ \si{\milli\metre}$のガラスチューブに巻きつけることで作成した.
このときマージナルコイルや同軸ケーブルの寄生容量を$100\ \si{\pico\farad}$程度とすると,
$20\ \si{\mega\hertz}$で発振させるためにはコイルの単位長さあたりの巻数$n$を
\begin{align}
  n=\frac{1}{2\pi f}\sqrt{\frac{1}{\mu_0 ClS}}\simeq 920
\end{align}
程度にすれば良いことになる.試料の長さは$12\ \si{\milli\metre}$程度なので,巻数は
\begin{align}
  920\times\frac{12}{1000}\simeq 11.0
\end{align}
程度になる.作成したコイルをコネクタに半田付けした.
次に全機器の電源を投入した.
最後にコイルと試料を装置にセットした後,
周波数カウンタの値を見て発振周波数が$20\pm1\ \si{\mega\hertz}$
程度になるようにコイルの巻き数や同軸ケーブルを調整した.
\subsubsection{g因子の測定}
ヘルムホルツコイルがその中心に作る磁場は
\begin{align}
  B=\left(\frac{4}{5}\right)^{3/2}\frac{\mu_0I}{R}
\end{align}
である.また電流測定用のシャント抵抗は$0.1\ \si{\ohm}$であり,コイルの巻数は200なのでオシロスコープのCh1の電圧を$V_{ch1}$とすれば
\begin{align}
  B=\left(\frac{4}{5}\right)^{3/2}\frac{\mu_0\cdot 200\cdot10V_{ch1}}{R}
\end{align}
となる.
また磁場の共鳴条件は(5)と$\omega_0=B_0\gamma$の関係から
\begin{align}
  |B_0|&=\frac{\omega_0\hbar}{g\mu_B}=\frac{\nu_0 h}{g\mu_B}
\end{align}
となる.
したがってピーク頂点での$V_{ch1}$とそのときの周波数カウンタの値を記録すれば(19)と(20)よりg因子を算出できる.
ここで次節の手順を用いて外部磁場の影響を取り除いた値を計算に用いた.
ここではマージナル発振器の発振周波数を$23\ \si{\mega\hertz}$, $17\ \si{\mega\hertz}$, $20\ \si{\mega\hertz}$程度に変更し,
それぞれの周波数でg因子を測定した.
\subsubsection{外部磁場の影響の除去}
実際の実験系ではヘルムホルツコイルによる磁場$B_0$の他に地磁気などの外部磁場$B_{\rm ext}$が存在する.
外部磁場は本来$V_{ch1}=0$を基準に左右対称に現れるはずだったピークを左右どちらかに平行移動させる.
したがってピークが$V_{ch1}=0$を基準に対称になるように全体を平行移動させることで外部磁場の影響を除くことができる.
これにはピークにおける$V_{ch1}$の値を$V_{ch1}'$, $V_{ch1}''$ (ただし$V_{ch1}'>V_{ch1}''$)とすれば
\begin{align}
  \pm\frac{V_{ch1}'-V_{ch1}''}{2}
\end{align}
を用いれば良い.また$V_{ch1}=0$からのピークの平行移動量
\begin{align}
  \overline{V_{ch1}}=\frac{V_{ch1}'+V_{ch1}''}{2}
\end{align}
は外部磁場の寄与に依るものだったので(19)式から
\begin{align}
  B_{\rm ext}=\left(\frac{4}{5}\right)^{3/2}\frac{\mu_0\cdot 2000\overline{V_{ch1}}}{R}
\end{align}
とすることで外部磁場を測定できる.
\subsubsection{地球磁場の測定}
マージナル発振器の発振周波数は$20\ \si{\mega\hertz}$程度とした.
磁北の方向を$0\si{\degree}$としてヘルムホルツコイルの方向$\theta$を$0\si{\degree}$から$180\si{\degree}$まで$30\si{\degree}$毎に共鳴吸収曲線を測定した.
各$\theta$について(23)式から外部磁場を計算した.
\subsubsection{ヘルムホルツコイル内の磁場分布の測定}
マージナル発振器の発振周波数は$20\ \si{\mega\hertz}$程度とした.
試料を軸方向に$-15\ \si{\centi\metre}$から$15\ \si{\centi\metre}$まで$1\ \si{\centi\metre}$刻みで移動させ,
各点での共鳴吸収曲線を測定した.次に試料台のレールを動径方向に付け直し,動径方向についても同様の測定を行った.
\subsubsection{共鳴吸収曲線の測定}
マージナル発振器の発振周波数は$20\ \si{\mega\hertz}$程度とした.
これまでの実験では共鳴時の磁場の大きさが主な測定対象であったが,ここでは共鳴吸収曲線そのものを測定した.
吸収曲線のピークの1つを拡大して記録し,ピークの高さと半値全幅を記録した.
この吸収曲線に同様のピーク高さと半値全幅のガウス曲線とローレンツ曲線を重ね,実際の吸収曲線がどちらに近いかを検討する.
同様の実験を試料の温度を$0\ \si{\degreeCelsius}$と$-196\ \si{\degreeCelsius}$に変えて行った.
このとき試料を氷水と液体窒素に浸すことで冷却した.