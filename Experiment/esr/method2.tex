\subsection{NMR測定実験}
\subsubsection{NMR測定装置概略}
図\ref{fig:fig/fig5.png}にNMR測定装置の概略を示す.基本的な構成はESR測定装置と同じだが,
ヘルムホルツコイル(b)と同軸に
直流電源と接続されたヘルムホルツコイル(c)が配置されている.
(20)式より核子の共鳴条件は
\begin{align}
  |B_0|=\frac{\omega_0\hbar}{g\mu_p}=\frac{2m_n\omega_0}{ge}
\end{align}
ここで$\mu_p$は核磁子と呼ばれ,ボーア磁子の$m_e$を核子の質量$m_p$で置き換えたものである.
以上からわかるとおり質量の大きい核子の共鳴吸収を測定するにはより強い磁場が必要になる.
これを達成するためヘルムホルツコイル(c)を用いて変動磁場をオフセットし,磁場を強くしている.
またヘルムホルツコイル(c)は大電流が流れるため冷却水を流す必要がある.
\mfig[width=10cm]{fig/fig5.png}{NMR測定装置}
\subsubsection{試料}
ここでは水素($^1{\rm H}$)とフッ素同位体($^{19}{\rm F}$)の核磁気共鳴を測定した.
それぞれ水($H_2O$)及びフッ素樹脂であるTeflon
($
\setpolymerdelim()
\chemfig{\vphantom{CF_2}-[@{op,.75}]CF_2-CF_2-[@{cl,.25}]}
\makebraces[3pt,3pt]{\!\!n}{op}{cl}
$)を試料として用いた.また純粋な水の水素の共鳴線幅は非常に狭く,検出が困難なので常磁性イオン(硫酸銅:$\rm CuSO_4\cdot5H_2O$及び鉄アルミニウムミョウバン:$\rm FeNH_4(SO_4)_2\cdot12H_2O$)
を用いて共鳴線幅を広げた.
\subsubsection{準備}
NMR測定装置の配線は事前に行われていた.
電源投入前に冷却水の循環を始める必要がある.
冷却水の循環開始後,全機器の電源を投入した.
試料は装置内に固定されている.以上で測定準備は完了した.
\subsubsection{測定手順}
直流電源の電流を$1.2\ \si{\ampere}$から徐々に減らしていき,ピークが現れる上限と下限の電流を記録した.
この中央の電流値が共鳴吸収が最大になっている電流とする.
同様の測定を$^1{\rm H}$及び$^{19}{\rm F}$それぞれのピークについて行った.
それぞれの共鳴周波数と磁場の強さからg因子を測定した.
ただし電流$I$とヘルムホルツコイル(c)の磁場$B$は以下の式で対応付けられる.
\begin{align}
  B=0.4667I+0.029I^2-0.0157I^3
\end{align}