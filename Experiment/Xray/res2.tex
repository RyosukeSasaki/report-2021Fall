\subsection{KClのX線ピークの指数付け}
\subsubsection{Ewald球の作図}
Ewald球の作図(図\ref{fig:fig/KCL_hkl.jpg})によって各Miller指数に対する散乱角$2\theta$は表のようになった.
ただし作図においてはKClを単純格子と見なしたため,
Miller指数は2倍になっている.

\begin{table}[h]
  \caption{Miller指数と散乱角$2\theta$の対応}
  \label{tab:miller_kclhkl}
  \centering
  \begin{tabular}{cc}
    \hline
    hkl&$2\theta\ /\ \si{\degree}$\\
    \hline \hline
    200&27.8\\
    220&40.0\\
    222&50.5\\
    400&59.8\\
    240&66.0\\
    440&90.0\\
    \hline
  \end{tabular}
\end{table}
\mfig[width=10cm]{fig/KCL_hkl.jpg}{KClに対するEwald球の作図}
\subsubsection{ピーク強度の理論値算出}
図\ref{fig:graph/out/KCl_raw.pdf}に示したKClのピークの強度と,理論的に予測される強度を比較することでMiller指数との対応を調べる.
PDXL2による解析では(400)以降に対応するピークが検出されなかったため,
(200), (220), (222)に対応するピークを同定する.
前節で得られた散乱角に近いピークをPDXL2の出力から抽出した.
その結果は表\ref{tab:extract_peak_i}のようになった.
ここでピーク1, 2は(200)に近く,
ピーク3, 4は(220)に近く,
ピーク5は(222)に近い.

次にEwald球の作図で得た$2\theta$に対して(21)式から散乱強度の理論値を計算すると表のようになった.
ただし温度因子の計算には
\begin{align}
  M=\frac{1.15\times10^4 T}{A\Theta^2}\left(\frac{1}{x}\int_0^x\frac{\xi}{{\rm e}^\xi-1}d\xi+\frac{x}{4}\right)\left(\frac{\sin\theta}{\lambda}\right)^2
\end{align}
を用いた.\cite{alma990007897430204034}
ただし$A$は原子量, $\Theta$はDebyeの特性温度, $x=\Theta/T$である.
ここでKClのは$\Theta=230\ \si{\kelvin}$を用い, $T$は常温として$300\ \si{\kelvin}$とした.\cite{2001167}
また散乱因子$f$はマニュアルの付録表3, 4の値を線形補間したものを用いた.

この結果から散乱光の強度は(200)の強度が最も大きく, (220)の強度はその半分程度, (222)では$0.15$倍程度と予想される.
したがって,測定データの内尤もらしいピークは(200)に対応するのがピーク1,
(220)に対応するのがピーク3,
(222)に対応するのがピーク5であると考えられる.
図\ref{fig:graph/out/KCl_peaks.pdf}に以上の結果を示す.
\begin{table}[h]
\caption{抽出したピークの強度}
\label{tab:extract_peak_i}
\centering
\begin{tabular}{ccccc}
\hline
No.&$2\theta\ /\ \si{\degree}$&積分強度&吸収補正後強度&相対強度\\
\hline \hline
1&26.476& 13446 & 67824 & 100   \\
2&27.694& 599   & 3091  & 4.58  \\
3&40.58 & 3263  & 23381 & 34.4  \\
4&41.080& 741   & 5396  & 7.95  \\
5&50.334& 1223  & 12656 & 18.6  \\
\hline
\end{tabular}
\end{table}
\begin{table}[h]
\caption{KClの散乱強度の理論値}
\label{tab:KCl_i_riron}
\centering
\begin{tabular}{ccc}
\hline
hkl&$2\theta\ /\ \si{\degree}$&相対強度\\
\hline \hline
200&27.8&100\\
220&40.0&52.7\\
222&50.5&15.2\\
\hline
\end{tabular}
\end{table}
\mfig[width=10 cm]{graph/out/KCl_peaks.pdf}{Miller指数とピークの対応}
