\section{結果}
\subsection{生データ}
図\ref{fig:graph/out/NaCl_raw.pdf}から図にNaCl粉末,KCl粉末,未知試料1,未知試料2の散乱光強度-$2\theta$グラフを示す.
また図\ref{fig:fig/polymer2.png},図\ref{fig:fig/polymer1.png}に高分子フィルムのX線画像を示す.
\mfig[width=8 cm]{graph/out/NaCl_raw.pdf}{NaCl粉末の測定データ}
\mfig[width=8 cm]{graph/out/KCl_raw.pdf}{KCl粉末の測定データ}
\mfig[width=8 cm]{graph/out/unknown1_raw.pdf}{未知試料1の測定データ}
\mfig[width=8 cm]{graph/out/unknown2_raw.pdf}{未知試料2の測定データ}
\mfig[width=8 cm]{fig/polymer2.png}{高分子フィルムのX線画像}
\mfig[width=8 cm]{fig/polymer1.png}{高分子フィルム(延伸)のX線画像}
\subsection{NaClの格子定数見積もり}
図\ref{fig:graph/out/NaCl_all.pdf}に各Miller指数に対応するピークとそれに対するFitting曲線を示す.
ただしFittingは
\begin{align}
  \begin{split}
    f(x)&=\frac{A}{\sqrt{2\pi}\sigma}\exp\left(-\frac{(x-x_0)^2}{2\sigma^2}\right)\\
    &+\frac{A}{2\sqrt{2\pi}\sigma}\exp\left(-\frac{\left(x-\frac{360}{\pi}\arcsin\left(\frac{1.5433}{1.54051}\sin\left(\frac{x_0}{2}\frac{\pi}{180}\right)\right)\right)^2}{2\sigma^2}\right)\\
    &+B+Cx
  \end{split}
\end{align}
という関数でgnuplotの非線形最小二乗Fitting機能を用いて行った.
Fittingにより各Miller指数に対応するピークの位置は表\ref{tab:miller_2theta}のようになった.
したがって
\begin{align}
  a=d_{hkl}\sqrt{h^2+k^2+l^2}\qquad(d=\lambda_{{\rm K}_{\alpha 1}}/2\sin\theta)
\end{align}
より格子定数は
\begin{align}
  a=5.651\pm 0.006\ \si{\angstrom}
\end{align}
と求まった.
\begin{table}[h]
  \caption{各Miller指数に対応するピーク位置}
  \label{tab:miller_2theta}
  \centering
  \begin{tabular}{cc}
  \hline
  Miller指数&$2\theta\ /\ \si{\degree}$\\
  \hline \hline
  111&27.33\\
  200&31.69\\
  220&45.33\\
  311&53.73\\
  222&56.23\\
  \hline
  \end{tabular}
\end{table}
\mfig[width=11 cm]{graph/out/NaCl_all.pdf}{各Miller指数に対応するピーク(上段左:(111),上段右(200),中段左:(220),中段右(311),下段左(222))}