\section{測定}
\subsection{$2\theta$-$\theta$スキャン}
これまで示した通り,この実験においては散乱角$\theta$はMiller指数と対応し,
これを正確に決めることは構造の同定において重要である.
図に実験の概略図を示す.
図において散乱角は
\begin{align}
  2\theta+\gamma=180\si{\degree}
\end{align}
となっている.
$2\theta$-$\theta$スキャンとは,試料とX線源,検出器が同一の円周上に並んでいると考えることで,
円周角の定理より$\alpha=\beta=\gamma$となり,
X線のビーム幅がある程度大きくても正確に散乱角を測定できるセットアップである.
これによってX線強度を落とさずに正確に散乱角を測定できる.

また$2\theta$-$\theta$スキャンで散乱角を変化させるとき,
試料を$\phi$回転させると散乱角は$2\phi$変化することになる.
$2\theta$-$\theta$スキャンという名前はこの散乱角と試料の回転角との対応に由来する名称である.
\subsection{実験装置}
図\ref{fig:fig/setup.png}に実験装置のセットアップを示す.
実験装置は全て定盤上に固定されている.
実験装置は中央のX線管があり,その左右に試料台とX線露光装置が設置されている.
X線管とX線露光装置の型番は表\ref{tab:souti}のとおりである.
X線管から発生したX線は光学系でビーム幅$0.1\ \si{\milli\metre}$程度に収束され,
コリメータを出て試料に照射される,試料を透過したX線は対する露光装置によって観測され2次元画像としてデータが出力される.
ただし直接透過したX線が露光装置に当たると故障の恐れがあるため,受光面の中央にビームストップを設置し,それを防いでいる.
試料は備え付けられたゴニオメータに取り付けられ,顕微鏡を覗きながら位置を調整できる.
また装置自体は放射線遮蔽ガラスの扉に覆われており,完全に扉が閉まっていなければ装置は作動しないようになっている.
更に動作中に扉が開いた場合直ちに装置がシャットダウンする.
\begin{table}[h]
  \caption{実験装置の型番}
  \label{tab:souti}
  \centering
  \begin{tabular}{cc}
  \hline
  &名称\\
  \hline \hline
  X線管&Rigaku Ultrax-18\\
  露光装置&Rigaku R-AXIS IV\\
  \hline
  \end{tabular}
\end{table}
\mfig[width=8cm]{fig/setup.png}{実験装置(実験テキストから引用)}