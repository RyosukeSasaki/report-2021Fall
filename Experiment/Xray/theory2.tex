\subsection{X線回折法}
Schr\"{o}dinger方程式は
\begin{align}
  \left(-\frac{\hbar^2}{2\mu}\nabla^2+V(\vec{r})\right)\psi(\vec{r})=E\psi(\vec{r})
\end{align}
ここで両辺$-2\mu/\hbar^2$を掛けると
\begin{align}
  (\nabla^2+k^2)\psi(\vec{r})=U(\vec{r})\psi(\vec{r})
\end{align}
となる.ただし$k^2=2\mu E/\hbar^2$, $U(\vec{r})=2\mu V(\vec{r})/\hbar^2$とした.
この解は平面波解
\begin{align}
  \psi_{\rm in}(\vec{r})={\rm e}^{i\vec{k_0}\cdot\vec{r}}
\end{align}
及びPoisson方程式のGreen関数
\begin{align}
  G(\vec{r})=-\frac{1}{4\pi}\frac{{\rm e}^{\pm i\vec{k}\cdot\vec{r}}}{r}
\end{align}
を用いて
\begin{align}
  \psi(\vec{r})=\psi_{\rm in}(\vec{r})+\int d\vec{r'}G(\vec{r}-\vec{r'})U(\vec{r'})\psi(\vec{r'})
\end{align}
と表され,これはLippmann-Schwinger方程式と呼ばれる.
散乱問題では1項目が平面波である入射波と対応し,
2項目は散乱波と対応する.
ここで積分内の$\psi(\vec{r})$に左辺を逐次代入すると
\begin{align}
  \begin{split}
    \psi(\vec{r})=\psi_{\rm in}(\vec{r})&+\int d\vec{r'}G(\vec{r}-\vec{r'})U(\vec{r'})\psi_{\rm in}(\vec{r'})\\
    &+\int\int d\vec{r'}G(\vec{r}-\vec{r'})U(\vec{r'})\int d\vec{r''}G(\vec{r'}-\vec{r''})U(\vec{r''})\psi_{\rm in}(\vec{r''})\\
  \end{split}
\end{align}
ここで$U(\vec{r})$の2次以降を無視すると
\begin{align}
  \begin{split}
    \psi(\vec{r})&=\psi_{\rm in}(\vec{r})+\int d\vec{r'}G(\vec{r}-\vec{r'})U(\vec{r'})\psi_{\rm in}(\vec{r'})\\
    &=\psi_{\rm in}(\vec{r})-\int d\vec{r'}\frac{{\rm e}^{ik|\vec{r}-\vec{r'}|}}{4\pi|\vec{r}-\vec{r'}|}U(\vec{r'})\psi_{\rm in}(\vec{r'})
  \end{split}
\end{align}
ここで$r\gg r'$とすると
\begin{align}
  \begin{split}
    |\vec{r}-\vec{r'}|&=\sqrt{r^2-2\vec{r}\cdot\vec{r'}+r'^2}\\
    &\simeq r\left(1-\frac{\vec{r}\cdot\vec{r'}}{r^2}+O\left(r^{-2}\right)\right)\\
    &=r-\bm{e}_r\cdot\vec{r'}+O\left(r^{-1}\right)
  \end{split}
\end{align}
となるので
\begin{align}
  \begin{split}
    \frac{{\rm e}^{ik(r-\bm{e}_r\cdot\vec{r'})}}{4\pi r\left(1-\frac{\vec{r}\cdot\vec{r'}}{r^2}\right)}
    &=\frac{{\rm e}^{ikr-i\vec{k}\cdot\vec{r'}}}{4\pi r}
  \end{split}
\end{align}
したがって
\begin{align}
  \begin{split}
    \psi(\vec{r})&=\psi_{\rm in}(\vec{r'})-\frac{{\rm e}^{ikr}}{4\pi r}\int d\vec{r'}{\rm e}^{-i(\vec{k}-\vec{k_0})\cdot\vec{r'}}U(\vec{r'})\\
    &=\psi_{\rm in}(\vec{r'})-\frac{{\rm e}^{ikr}}{4\pi r}\int d\vec{r'}{\rm e}^{-i\vec{K}\cdot\vec{r'}}U(\vec{r'})\\
  \end{split}
\end{align}
ここで$\vec{k}-\vec{k_0}=:\vec{K}$を散乱ベクトルとした.
(14)式はポテンシャル$U(\vec{r'})$のFourier変換となっている.
入射波が光であるとき主な散乱体は電子であると考えられるので$U(\vec{r})\propto \rho(\vec{r})$となるので,
散乱波の強度は
\begin{align}
  I\propto|\psi(\vec{r})|^2\propto \left|\int d\vec{r'}\rho(\vec{r'}){\rm e}^{-i\vec{K}\cdot\vec{r'}}\right|
\end{align}
となることがわかる.