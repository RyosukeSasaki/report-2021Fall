\subsection{試料の準備}
試料として表\ref{tab:siryou}の物を用いる.
粉末試料は乳鉢で細かくすり潰し,
錠剤成形機を用いて円盤状に固めたものをゴニオメータのヘッドに取り付けた.
未知試料1は錠剤が脆いため,カプトンテープを用いて挟んで固定した.
その他の金属箔やフィルムなどの試料は粘土を用いてゴニオメータのヘッドに取り付けた.
また高分子フィルムは延伸方向が上下方向になるように取り付けられた.
\begin{table}[h]
  \caption{試料の一覧}
  \label{tab:siryou}
  \centering
  \begin{tabular}{ccc}
  \hline
  名称&形状\\
  \hline \hline
  NaCl&粉末\\
  KCl&粉末\\
  未知試料1&粉末\\
  未知試料2&金属箔\\
  高分子(PLA)フィルム&フィルム\\
  高分子(PLA)フィルム(延伸)&フィルム\\
  \hline
  \end{tabular}
\end{table}
\subsection{測定手順}
装置稼働中,実験従事者はは個人線量計を装着し,またその旨を記録した.
装置の立ち上げは以下の手順で行った.
\begin{enumerate}
  \item 冷却水循環ポンプを稼働した
  \item 真空ポンプを稼働し,適当な真空度に至っていることを確認した
  \item X線発生装置を起動し,加速電圧を$45\ \si{\keV}$,電流を$50\ \si{\milli\ampere}$に設定した.
  \item DOORボタンを押し,キースイッチを回して放射線遮蔽扉を開いた
  \item 露光装置を操作しやすい位置まで後退させた
  \item ゴニオメータに試料を取り付け,試料の位置を粗調整した
  \item 顕微鏡のレティクルの中央に試料が来るように微調整した
  \item 顕微鏡を収納し放射線遮蔽扉を閉じた
\end{enumerate}
以上で装置の立ち上げは完了した.次に測定を以下の手順で行った.
\begin{enumerate}
  \item 制御用PCで制御ソフト(CrystalClear)を立ち上げた
  \item プロジェクトとサンプルを適当に命名し,イメージングプレートの初期化を行った
  \item 初期化完了後, Collect Imageを行い表のように測定条件を設定した
  \item Runを押し測定を開始した
\end{enumerate}
設定した照射時間が経過したら測定は完了する.
\begin{table}[h]
  \caption{測定条件}
  \label{tab:sokuteicond}
  \centering
  \begin{tabular}{cc}
  \hline
  パラメータ&値\\
  \hline \hline
  Width&$0.1\si{\degree}$\\
  End Width&$0.1\si{\degree}$\\
  Exp Time&$900\ \si{\second}$または$1800\ \si{\second}$\\
  \hline
  \end{tabular}
\end{table}