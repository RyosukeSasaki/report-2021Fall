\subsection{X線回折法}
Schr\"{o}dinger方程式は
\begin{align}
  \left(-\frac{\hbar^2}{2\mu}\nabla^2+V(\vec{r})\right)\psi(\vec{r})=E\psi(\vec{r})
\end{align}
ここで両辺$-2\mu/\hbar^2$を掛けると
\begin{align}
  (\nabla^2+k^2)\psi(\vec{r})=U(\vec{r})\psi(\vec{r})
\end{align}
となる.ただし$k^2=2\mu E/\hbar^2$, $U(\vec{r})=2\mu V(\vec{r})/\hbar^2$とした.
この解は平面波解
\begin{align}
  \psi_{\rm in}(\vec{r})={\rm e}^{ikz}
\end{align}
及びPoisson方程式のGreen関数
\begin{align}
  G(\vec{r})=-\frac{1}{4\pi}\frac{{\rm e}^{\pm ikr}}{r}
\end{align}
を用いて
\begin{align}
  \psi(\vec{r})=\psi_{\rm in}(\vec{r})+\int d\vec{r'}G(\vec{r}-\vec{r'})U(\vec{r'})\psi(\vec{r'})
\end{align}
と表され,これはLippmann-Schwinger方程式と呼ばれる.
散乱問題では1項目が平面波である入射波と対応し,
2項目は散乱波と対応する.
ここで積分内の$\psi(\vec{r})$に左辺を逐次代入すると