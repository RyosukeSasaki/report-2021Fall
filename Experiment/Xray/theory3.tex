\subsection{Miller指数}
結晶格子の面間隔$d$が指定されたとき干渉条件はBraggの公式
\begin{align}
  2d\sin\theta=n\lambda
\end{align}
で与えられる.ここで面間隔は図のように格子の中で様々なとり方ができる.
まず簡単のために2次元で考える.
ここで$a$, $b$軸は格子定数で正規化している.
このときMiller指数は$a$, $b$軸の切片の比の逆数で与えられる.
たとえば図の赤線の場合は切片の比は$1:0.5$なのでその逆数の比は$1:2=h:k$となる.
また青線の場合は切片の比は$1:3$なので$1:1/3=3:1=h:k$となる.
緑線の場合は$1:\infty$なので$1:1/\infty=1:0=h:k$となる.
このようにMiller指数が指定されたとき,格子において注目する平面が決まり,
これによって面間隔$d$が指定できることがわかる.

以下では$b=a$として考える.あるMiller指数$h,k$に対して,原点に最も近い直線は
\begin{align}
  hx+ky=a
\end{align}
であり原点との距離(すなわち面間隔$d_{hk}$は)
\begin{align}
  d_{hk}=\frac{a}{\sqrt{h^2+k^2}}
\end{align}
となることがわかる.
三次元の場合は$c$軸についての指数$l$を追加することで
\begin{align}
  h_{hkl}=\frac{a}{\sqrt{h^2+k^2+l^2}}
\end{align}
と表される.
したがって(16)式は
\begin{align}
  2d_{hkl}\sin\theta=n\lambda
\end{align}
となり,あるMiller指数に対して干渉が起きる散乱角$\theta$が決まることがわかる.
サンプルが単結晶のように異方性がある場合は$hkl$に対応する光線が点として決まるのに対して,
粉末のような等方的なサンプルに対しては図\ref{fig:fig/isotropic.png}にように光線に対して軸対称な像が得られることになる.

また,あるMiller指数に対応する散乱光の強度は
\begin{align}
  I\left(\theta_{hkl}\right)\propto |F\left(\theta_{hkl}\right)|^2\times p\times \frac{1+\cos^22\theta}{\sin^2\theta\cos\theta}\times\exp(-2M)
\end{align}
で与えられる.
1項目の$F$は結晶構造因子という結晶の構造に依る因子である.
2項目の$p$は多重度という格子面の対称性に依存する因子であり,
例えば立方晶の(100)面は6方向に等価な面があるので$p=6$となる.
3項目はLorentz因子という因子である.
図\ref{fig:fig/lorentz.jpg}のようにBraggの公式を満たす散乱角で散乱光強度は最大となるが,少しずれた角度でも散乱光が現れる.
したがって観測される散乱光強度は図\ref{fig:fig/lorentz.jpg}(b)の面積に比例する量となることが考えられ,
これがLorentz因子として現れている.
4項目はDebye-Waller因子と呼ばれ,温度に依存する因子である.
\mfig[width=6cm]{fig/isotropic.png}{等方的なサンプルに対する散乱(実験テキストから引用)}
\mfig[width=10cm]{fig/lorentz.jpg}{Bragg角周りの散乱光強度}
\subsection{単結晶}
単結晶は図のように結晶格子が周期的に並んだ構造である.
一方で単結晶が多数集合して構成している物質は多結晶と呼ばれる.
単結晶は強い力が加わったとき,ある格子面に沿って割れる.この現象は劈開と呼ばれる.
劈開する格子面は物質によって決まっており,例えば岩塩の場合は(100)面である.\cite{hekikai}
単結晶は溶媒を溶かした過飽和溶液を徐々に冷却し,水溶液中に設置した種結晶表面に析出させることで生成される.\cite{sakusei}
またタンパク質の結晶などは重力下では重力の影響で高品質な単結晶が成長しにくいことから,
国際宇宙ステーションなどの無重力環境で単結晶を作成する試みもある.\cite{jaxa:online}
一方で単結晶が複数集合してできた物質は多結晶と呼ばれる.

また1.3節で述べた通り,単結晶はその異方性からMiller指数に対して干渉が起きる散乱角が定まり,干渉による像は図\ref{fig:fig/spring8.png}のような点の集まりとなる.
\mfig[width=10cm]{fig/spring8.png}{単結晶のX線回折像\cite{spring8:online}}