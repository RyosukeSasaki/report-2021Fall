\section{入門編}
\subsection{2進表示回路}
ここではスイッチ$n$ ($n=1,2,3$)からの入力を受け$n$を2進数表示する回路を作成した.
以下ではスイッチ$n$の状態を${\rm X}_n$,
1桁目の出力の状態をA,2桁目の出力の状態をBと表記している
\subsubsection{設計}
表\ref{tab:binary}に真理値表を示す.
したがって論理式は以下で与えられる.
\begin{align}
  \begin{split}
    {\rm A} &= {\rm X}_1{\rm\overline{X}}_2{\rm\overline{X}}_3+{\rm\overline{X}}_1{\rm\overline{X}}_2{\rm X}_3\\
    {\rm B} &= {\rm\overline{X}}_1{\rm X}_2{\rm\overline{X}}_3+{\rm\overline{X}}_1{\rm\overline{X}}_2{\rm X}_3
  \end{split}
\end{align}
\begin{table}[h]
  \label{tab:binary}
  \centering
  \begin{tabular}{ccc|cc}
    \hline
    ${\rm X}_3$ & ${\rm X}_2$ & ${\rm X}_1$ & B & A\\
    \hline
    0 & 0 & 0 & 0 & 0 \\
    0 & 0 & 1 & 0 & 1 \\
    0 & 1 & 0 & 1 & 0 \\
    0 & 1 & 1 & 0 & 0 \\
    1 & 0 & 0 & 1 & 1 \\
    1 & 0 & 1 & 0 & 0 \\
    1 & 1 & 0 & 0 & 0 \\
    1 & 1 & 1 & 0 & 0 \\
    \hline
  \end{tabular}
  \caption{2進表示回路}
\end{table}
\clearpage
\subsubsection{実装及び動作確認}
図\ref{fig:fig/1-b.png}に実装した回路を示す.
トグルスイッチ$n$が${\rm X}_n$の状態を入力し,
LED AとLED Bがそれぞれ出力A, Bを表す.
全8通りの入力を試行し,期待される通りの動作をしていることを確認できた.
\mfig[width=10cm]{fig/1-b.png}{2進表示回路の実装}
\clearpage
\subsection{比較回路}
ここでは2桁の2進数${\rm X}_1{\rm X}_2$および${\rm Y}_1{\rm Y}_2$を受け,
${\rm X}_1{\rm X}_2>{\rm Y}_1{\rm Y}_2$を判定する回路を作成した.
以下では出力の状態をSとする.
\subsubsection{設計}
表\ref{tab:comparator}に真理値表を示す.これからカルノー図は図\ref{fig:karnaugh1c}のようになる.したがって論理式は以下のようになる.
\begin{align}
  S={\rm X}_1{\rm\overline{Y}}_1+{\rm X}_2{\rm\overline{Y}}_1{\rm\overline{Y}}_2+{\rm X}_1{\rm X}_2{\rm\overline{Y}}_2
\end{align}
\begin{figure}[h]
  \begin{tabular}{c}
    \begin{minipage}[t]{.48\textwidth}
      \centering
      \begin{tabular}{cccc|c}
        \hline
        ${\rm X}_1$ & ${\rm X}_2$ & ${\rm Y}_1$ & ${\rm Y}_2$ & S\\
        \hline
        0 & 0 & 0 & 0 & 0 \\
        0 & 0 & 0 & 1 & 0 \\
        0 & 0 & 1 & 0 & 0 \\
        0 & 0 & 1 & 1 & 0 \\
        0 & 1 & 0 & 0 & 1 \\
        0 & 1 & 0 & 1 & 0 \\
        0 & 1 & 1 & 0 & 0 \\
        0 & 1 & 1 & 1 & 0 \\
        1 & 0 & 0 & 0 & 1 \\
        1 & 0 & 0 & 1 & 1 \\
        1 & 0 & 1 & 0 & 0 \\
        1 & 0 & 1 & 1 & 0 \\
        1 & 1 & 0 & 0 & 1 \\
        1 & 1 & 0 & 1 & 1 \\
        1 & 1 & 1 & 0 & 1 \\
        1 & 1 & 1 & 1 & 0 \\
        \hline
      \end{tabular}
      \tblcaption{比較回路}
      \label{tab:comparator}
    \end{minipage}
    %
    \hfill
    %
    \begin{minipage}[c]{.48\textwidth}
      \centering
      \askmapiv{S}{{${\rm Y}_1$}{${\rm Y}_2$}{${\rm X}_1$}{${\rm X}_2$}}{}{0111001100010000}%
      {%
      \color{red}\put(1,1){\oval(1.8,1.8)}
      \color{green}\put(0.5,2){\oval(0.8,1.8)}
      \color{blue}\put(0,1.5){\oval(1.8,0.8)[r]}
      \color{blue}\put(4,1.5){\oval(1.8,0.8)[l]}
      }
      \caption{Sのカルノー図}
      \label{fig:karnaugh1c}
    \end{minipage}
  \end{tabular}
\end{figure}
\clearpage
\subsubsection{実装及び動作確認}
図\ref{fig:fig/1-c.png}に実装した回路を示す.
トグルスイッチ${\rm X}n$, ${\rm Y}n$が入力値,
LED Aが出力Sを表す.
全16通りの入力を試行し,期待される通りの動作をしていることを確認できた.
\mfig[width=10cm]{fig/1-c.png}{比較回路の実装}