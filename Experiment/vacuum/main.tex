\documentclass[uplatex,a4j,11pt,dvipdfmx,draft]{jsarticle}
\usepackage{listings,jvlisting}
\bibliographystyle{jplain}

\usepackage{url}

\usepackage{graphicx}
\usepackage{gnuplot-lua-tikz}
\usepackage{pgfplots}
\usepackage{tikz}
\usepackage{amsmath,amsfonts,amssymb}
\usepackage{bm}
\usepackage{siunitx}

\makeatletter
\def\fgcaption{\def\@captype{figure}\caption}
\makeatother
\newcommand{\setsections}[3]{
\setcounter{section}{#1}
\setcounter{subsection}{#2}
\setcounter{subsubsection}{#3}
}
\newcommand{\mfig}[3][width=15cm]{
\begin{center}
\includegraphics[#1]{#2}
\fgcaption{#3 \label{fig:#2}}
\end{center}
}
\newcommand{\gnu}[2]{
\begin{figure}[hptb]
\begin{center}
\input{#2}
\caption{#1}
\label{fig:#2}
\end{center}
\end{figure}
}
\makeatletter
  \renewcommand{\theequation}{%
    \arabic{equation}}
  \@addtoreset{equation}{section}
\makeatother
\makeatletter
\@addtoreset{equation}{section}
\def\theequation{\thesection.\arabic{equation}}% renewcommand でもOK
\makeatother

\begin{document}
\section{目的}
今日の研究開発において真空技術はもはや必要不可欠である.
当実験では真空に関する物理学,並びに真空を実現するための技術,装置の取り扱いについて理解する.
また真空技術の基礎概念並びに応用としてコンダクタンスの測定,吸着の測定,仕事関数の測定を行う.
\section{原理}
\subsection{X線の発生}
電磁波の発生過程には黒体放射,制動放射,遷移放射などがある.
今回の実験で用いたX線発生装置では制動放射と遷移輻射が支配的になっているのでこれらについて述べる.
\subsubsection{連続X線}
連続X線とは制動放射により生じるX線である.
連続X線は図\ref{fig:fig/XraySpectrum.jpg}のスペクトルの連続的な部分である.
図\ref{fig:fig/Xray_gen.png}にX線発生装置の概略を示す.
電子銃から放射された電子は銅のターゲットに衝突する.
ターゲットに衝突した電子は様々な方向に散乱されるが,
その程度によって制動の具合が異なってくる.これによって様々なエネルギーのX線が放射されることになる.
ここで電子銃の加速電圧を$V$としよう.
最も制動が大きいのは1回の衝突で全エネルギーを失うことに相当するので$hc/\lambda=eV$から
\begin{align}
  \lambda_{m}=\frac{hc}{eV}
\end{align}
となる.この波長が加速電圧$V$の連続X線で得られる最も高エネルギーな(すなわち波長が短い)光になる.
連続X線の全強度は管電流を$i$, ターゲット原子の原子番号を$Z$とすれば
\begin{align}
  I\propto iV^mZ
\end{align}
となり,ターゲットに重元素を用いると効率よくX線を得ることができる.\cite{XrayS}
連続X線スペクトルを求める方法としてはBirch-Marshallのモデルが用いられる.\cite{Xraymodel}
\subsubsection{特性X線}
特性X線(または固有X線)とは遷移放射により生じるX線である.
特性X線は図\ref{fig:fig/XraySpectrum.jpg}の鋭いピークの部分である.
特性X線は主量子数$n$の高い準位にいる電子が低い準位に遷移する際に放射されるX線である.
X線の波長は準位間のエネルギーの差を$\Delta E$とすれば
\begin{align}
  \lambda=\frac{hc}{\Delta E}
\end{align}
となり,同じ遷移による放射は常に同じ波長となることから鋭いピークが現れる.
原子では主量子数が0の軌道をK殻, 1の軌道をL殻, 2の軌道をM殻と呼ぶが,
それぞれの遷移について名前がついている.
図\ref{fig:fig/KLM.jpg}のようにL殻からK殻への遷移によるX線を$\rm K_{\alpha}$線,
M殻からK殻への遷移によるX線を$\rm K_{\beta}$線と呼ぶ.また角運動量量子数$j$の違いなどによってL, M殻は微細構造を持つので波長が僅かに異なるX線が発生する.
特に$\rm L_{III}$殻からKへの遷移によるX線は$\rm K_{\alpha 1}$線, 特に$\rm L_{II}$殻からKへの遷移によるX線は$\rm K_{\alpha 2}$線などと呼ばれる.
これらの遷移確率は約$2:1$であり,すなわち強度比も$2:1$となる.
これらのX線の波長は非常に近接しており,実験的には分離が困難なため,その波長を加重平均した
\begin{align}
  \lambda_{\rm K_{\alpha}}=\frac{2\lambda_{{\rm K}_{\alpha 1}}+\lambda_{{\rm K}_{\alpha 2}}}{3}
\end{align}
を用いる場合が多い.\cite{XrayS}
ターゲットとして銅を用いた場合,これらのX線の波長は表\ref{tab:lambda_xray}のようになる.
\begin{table}[h]
\caption{X線の波長}
\label{tab:lambda_xray}
\centering
\begin{tabular}{cc}
\hline
&$波長\ /\ \si{\angstrom}$\\
\hline \hline
${\rm K}_{\alpha 1}$&1.5405\\
${\rm K}_{\alpha 2}$&1.5443\\
${\rm K}_{\alpha}$&1.5418\\
${\rm K}_{\beta}$&1.3922\\
\hline
\end{tabular}
\end{table}
\mfig[width=8cm]{fig/XraySpectrum.jpg}{モリブデンターゲットによるX線のスペクトル\cite{alma990007897430204034}}
\mfig[width=6cm]{fig/Xray_gen.png}{X線発生装置の概略(実験テキストから引用)}
\mfig[width=10cm]{fig/KLM.jpg}{準位間の遷移\cite{XrayS}}
\newpage
\subsection{X線回折法}
Schr\"{o}dinger方程式は
\begin{align}
  \left(-\frac{\hbar^2}{2\mu}\nabla^2+V(\vec{r})\right)\psi(\vec{r})=E\psi(\vec{r})
\end{align}
ここで両辺$-2\mu/\hbar^2$を掛けると
\begin{align}
  (\nabla^2+k^2)\psi(\vec{r})=U(\vec{r})\psi(\vec{r})
\end{align}
となる.ただし$k^2=2\mu E/\hbar^2$, $U(\vec{r})=2\mu V(\vec{r})/\hbar^2$とした.
この解は平面波解
\begin{align}
  \psi_{\rm in}(\vec{r})={\rm e}^{i\vec{k_0}\cdot\vec{r}}
\end{align}
及びPoisson方程式のGreen関数
\begin{align}
  G(\vec{r})=-\frac{1}{4\pi}\frac{{\rm e}^{\pm i\vec{k}\cdot\vec{r}}}{r}
\end{align}
を用いて
\begin{align}
  \psi(\vec{r})=\psi_{\rm in}(\vec{r})+\int d\vec{r'}G(\vec{r}-\vec{r'})U(\vec{r'})\psi(\vec{r'})
\end{align}
と表され,これはLippmann-Schwinger方程式と呼ばれる.
散乱問題では1項目が平面波である入射波と対応し,
2項目は散乱波と対応する.
ここで積分内の$\psi(\vec{r})$に左辺を逐次代入すると
\begin{align}
  \begin{split}
    \psi(\vec{r})=\psi_{\rm in}(\vec{r})&+\int d\vec{r'}G(\vec{r}-\vec{r'})U(\vec{r'})\psi_{\rm in}(\vec{r'})\\
    &+\int\int d\vec{r'}G(\vec{r}-\vec{r'})U(\vec{r'})\int d\vec{r''}G(\vec{r'}-\vec{r''})U(\vec{r''})\psi_{\rm in}(\vec{r''})\\
  \end{split}
\end{align}
ここで$U(\vec{r})$の2次以降を無視すると
\begin{align}
  \begin{split}
    \psi(\vec{r})&=\psi_{\rm in}(\vec{r})+\int d\vec{r'}G(\vec{r}-\vec{r'})U(\vec{r'})\psi_{\rm in}(\vec{r'})\\
    &=\psi_{\rm in}(\vec{r})-\int d\vec{r'}\frac{{\rm e}^{ik|\vec{r}-\vec{r'}|}}{4\pi|\vec{r}-\vec{r'}|}U(\vec{r'})\psi_{\rm in}(\vec{r'})
  \end{split}
\end{align}
ここで$r\gg r'$とすると
\begin{align}
  \begin{split}
    |\vec{r}-\vec{r'}|&=\sqrt{r^2-2\vec{r}\cdot\vec{r'}+r'^2}\\
    &\simeq r\left(1-\frac{\vec{r}\cdot\vec{r'}}{r^2}+O\left(r^{-2}\right)\right)\\
    &=r-\bm{e}_r\cdot\vec{r'}+O\left(r^{-1}\right)
  \end{split}
\end{align}
となるので
\begin{align}
  \begin{split}
    \frac{{\rm e}^{ik(r-\bm{e}_r\cdot\vec{r'})}}{4\pi r\left(1-\frac{\vec{r}\cdot\vec{r'}}{r^2}\right)}
    &=\frac{{\rm e}^{ikr-i\vec{k}\cdot\vec{r'}}}{4\pi r}
  \end{split}
\end{align}
したがって
\begin{align}
  \begin{split}
    \psi(\vec{r})&=\psi_{\rm in}(\vec{r'})-\frac{{\rm e}^{ikr}}{4\pi r}\int d\vec{r'}{\rm e}^{-i(\vec{k}-\vec{k_0})\cdot\vec{r'}}U(\vec{r'})\\
    &=\psi_{\rm in}(\vec{r'})-\frac{{\rm e}^{ikr}}{4\pi r}\int d\vec{r'}{\rm e}^{-i\vec{K}\cdot\vec{r'}}U(\vec{r'})\\
  \end{split}
\end{align}
ここで$\vec{k}-\vec{k_0}=:\vec{K}$を散乱ベクトルとした.
(14)式はポテンシャル$U(\vec{r'})$のFourier変換となっている.
入射波が光であるとき主な散乱体は電子であると考えられるので$U(\vec{r})\propto \rho(\vec{r})$となるので,
散乱波の強度は
\begin{align}
  I\propto|\psi(\vec{r})|^2\propto \left|\int d\vec{r'}\rho(\vec{r'}){\rm e}^{-i\vec{K}\cdot\vec{r'}}\right|
\end{align}
となることがわかる.
\section{実験方法}
\subsection{実験装置}
\subsubsection{真空装置の構成}
図\ref{fig:fig/fig7.png}に真空装置の構成図を示す.
RPは油回転ポンプDPは油拡散ポンプである.また${\rm V}_n$はバルブである.
コンダクタンス管はフレキシブルパイプで接続され,交換が可能である.
真空チャンバーは内径$216.3\ \si{\milli\metre}$,高さ$250\ \si{\milli\metre}$の円筒容器である.
またコンダクタンス管は内径$10\ \si{\milli\metre}$のものと$15\ \si{\milli\metre}$のものがあり,
共に長さは$200\ \si{\milli\metre}$である.
フランジは旧JIS規格のVF/VGフランジでありOリングと共に用いることで$10\ \si{\micro\pascal}$程度の高真空を実現できる.\cite{VacuumTe71:online}
油回転ポンプは大気圧から$1\ \si{\pascal}$程度まで減圧することができる.
一方で油拡散ポンプは低圧($6\ \si{\pascal}$程度)から高真空まで減圧することができる.
ただし停止中にチャンバー側が真空になっていると油が逆流するため,吸気側をリークする必要がある.
油拡散ポンプを使用する際は${\rm V}_2$を開き,排気された気体を油回転ポンプで更に排気する必要がある.
また電離真空計は$0.6\ \si{\pascal}$以上で使用することはできない.
\mfig[width=12cm]{fig/fig7.png}{真空装置の構成}
\subsubsection{熱電子電流測定装置の構成}
図\ref{fig:fig/fig12.png}に熱電子放出の実験系の構成を示す.
フィラメントはTaであり,電流$I_f$を流すとジュール熱により発熱する.
アノードには引き込み電圧$V_A$が掛けられている.
熱電子がアノードに到達すると電流計が熱電子電流を検出する.
\mfig[width=6cm]{fig/fig12.png}{熱電子電流測定装置の構成}
\subsection{実験手順}
\subsubsection{排気手順}
図\ref{fig:fig/fig8.png}に粗引き手順,図\ref{fig:fig/fig9.png}に油回転ポンプの起動手順
図\ref{fig:fig/fig10.png}に真空引きの手順のフローチャートを示す.
油回転ポンプによる粗引きの後油拡散ポンプを起動,高真空までの真空引きを行っている.
これにより油拡散ポンプへ大気圧を流入させることなく高真空を実現できる.
\begin{figure}[htbp]
  \begin{minipage}{0.5\hsize}
    \centering
    \mfig[width=3cm]{fig/fig8.png}{粗引き手順}
  \end{minipage}
  \begin{minipage}{0.5\hsize}
    \centering
    \mfig[width=3cm]{fig/fig9.png}{油回転ポンプの起動手順}
  \end{minipage}
\end{figure}
\mfig[width=6cm]{fig/fig10.png}{真空引き手順}
\newpage
\subsubsection{真空装置のシャットダウン手順}
図\ref{fig:fig/fig11.png}に真空装置のシャットダウン手順を示す.
シャットダウンの順番は真空引きの際の逆順である.
\mfig[width=6cm]{fig/fig11.png}{シャットダウン手順}
\subsubsection{コンダクタンスの測定}
コンダクタンス管がない場合と2種のコンダクタンス管それぞれについて図\ref{fig:fig/fig13.png}の手順で圧力の時間変化を各5回ずつ測定した.
(\ref{equ:216-pt})によれば圧力の時間変化は指数関数的であり,その時定数は排気速度に依存した.
したがって得られた圧力の時間変化を指数関数でfittingし,その時定数を測定することで排気速度を得ることができる.
コンダクタンス管を取り付けた場合,合成コンダクタンスは(\ref{equ:216-conduct})で表されるので,
コンダクタンス管がない場合の排気速度からコンダクタンス管のコンダクタンスを求める.
以上からコンダクタンスの半径依存性を求め,チャンバーの気体が粘性流か分子流かを判定する.
\mfig[width=6cm]{fig/fig13.png}{コンダクタンスの測定}
\subsubsection{熱電子電流の測定}
図\ref{fig:fig/fig14.png}の手順で熱電子電流$I_P$を測定した.
また有限要素法解析によりフィラメント電流$I_f$とフィラメント温度の関係を計算した.
これによりSchottkyプロットを作成し,外挿により$I_S$を求めた.
(\ref{equ:221-RDmann})から$\log(I_S/T^2)=a-\phi_0/k_BT$なので縦軸を$\log(I_ST^2)$,
横軸を$1/T$とすればその傾きが$-\phi_0/k_B$となり,仕事関数を求められる.
\mfig[width=3cm]{fig/fig14.png}{熱電子電流の測定}
\newpage
\subsubsection{質量分析}
図\ref{fig:fig/fig15.png}の手順で質量スペクトル及び吸着,脱離時の圧力変化を測定した.
質量スペクトルから残留気体の組成を同定した.
\mfig[width=7cm]{fig/fig15.png}{熱電子電流の測定}
\newpage
\section{結果}
\subsection{ホログラフィー干渉}
pythonスクリプトを用いて生成した平板回転による干渉のシミュレーション結果を図\ref{fig:fig/zrot.png},図\ref{fig:fig/yrot.png}に示す.
ここで計算条件は(\ref{equ:132_zrot})式, (\ref{equ:132_yrot})式から求めたものであり,表\ref{tab:keisanjouken}のとおりである.
図\ref{equ:132_zrot}を見ると,干渉縞は斜めに入っており,また明線の本数は$x=0\ \si{\milli\metre}$で数えると20本である.
図\ref{equ:132_yrot}を見ると,干渉縞はほぼ垂直に入っており,また明線の本数は$y=0\ \si{\milli\metre}$で数えると20本である.
\begin{table}[h]
\caption{計算条件}
\label{tab:keisanjouken}
\centering
\begin{tabular}{ccc}
\hline
条件&回転軸&角度$\theta$\\
\hline \hline
1&$z$軸回転&$2.98\times10^{-4}$\\
2&$y$軸回転&$9.27\times10^{-5}$\\
\hline
\end{tabular}
\end{table}
\mfig[width=8cm]{fig/zrot.png}{条件1($z$軸回転)}
\mfig[width=8cm]{fig/yrot.png}{条件2($y$軸回転)}
\subsection{自由課題4}
図\ref{fig:fig/free4_trim.png}に得られたホログラフィ像を示す.ただし図\ref{fig:fig/free4_trim.png}は消しゴムが写っている部分だけをトリミングしている.
図\ref{fig:fig/free4_trim.png}を見ると左右が干渉縞の間隔が狭く,中央ほど間隔が広いことがわかる.
また中央下部に暗い領域がある.
\mfig[width=8cm]{fig/free4_trim.png}{自由課題4のホログラフィ像}\newpage
\section{考察}
\subsection{コンダクタンスの測定}
\subsubsection{コンダクタンスの管径依存性}
図\ref{fig:graph/r_depend.tex}にコンダクタンスの管径依存を示す.
(\ref{equ:215-conduct})からコンダクタンスの管径依存は$y=ax^n$という形を取ると予想されるので
$y=ax^n$でFittingを行った.
Fitting曲線は以下のようになった.
\begin{align}
  y=(2.6\times10^{-4})\times x^{3.2}
\end{align}
したがってコンダクタンスの管径依存は$x^3$に近く,チャンバー内は分子流的であると考えられる.

また(2.23)と(2.27)から温度$T$における平均自由行程は
\begin{align}
  \lambda=\frac{k_BT}{\sqrt{2}\pi d^2 p}
\end{align}
となる.特に室温($25\ \si{\degreeCelsius}$)で圧力が$10^{-3}$程度,気体が窒素分子で$d=3.75\ \si{\AA}$とすると
\begin{align}
  \lambda\sim 6.6\ \si{\metre}
\end{align}
程度になりチャンバーの典型的な大きさ($200\ \si{\milli\metre}$)程度に比べて十分大きい.
したがって実測された流体の挙動と理論からの予測は一致する.

また気体が完全に分子流であるとすると(2.36)と気体分子の平均速度が$\sqrt{8k_BT/\pi m}$であることから
\begin{align}
  C=\frac{2\pi a^3}{3L}\sqrt{\frac{8k_BT}{\pi m}}
\end{align}
となり,各コンダクタンス管のコンダクタンスは表\ref{tab:con_theo}のようになる.
ただし気体の温度は室温($25\ \si{\degreeCelsius}$)とし,
空気の平均分子量は窒素が$80\%$, 酸素が$20\%$と仮定し$28\times 0.8+32\times0.2=28.8$と置いた.
表\ref{tab:con_theo}から理論値と実測値は桁で一致しているが20\%から30\%ほど実測値の方が小さいとわかる.
このためには(5.4)式から
\begin{itemize}
  \item $a$, $L$, $T$, $m$などの量が誤差を含んでいる
  \item 粘性流の寄与が入っている
\end{itemize}
などが考えられる.ここで$a$, $L$, $T$は容易に測定可能であり,大きな誤差は無いと考える.
ここで質量測定の結果から$m$を改めて考える.
図\ref{fig:graph/mass/mass.tex}からチャンバー内の気体の組成が大まかに表\ref{tab:heikin}のようになっていると考える.
この時平均分子量は$19.4$となり空気に比べて軽くなり,コンダクタンスの理論値は更に上昇させるよう働くことがわかる.
以上からコンダクタンスの誤差は$a$, $L$, $T$, $m$の誤差に起因するものではなく,粘性流の寄与だと考えられる.
大気が窒素80\%と酸素20\%の混合気体であるならばその粘性率$\eta$は$18.2\ \si{\micro\pascal.\second}$である.\cite{kagakuNensei:online}
また気体の平均圧力を$10^{-3}\ \si{\pascal}$程度とするなら,
(2.36)から粘性流の場合のコンダクタンスは$10^{-9}$から$10^{-10}$程度のオーダーになる.
したがって粘性流の場合のコンダクタンスは分子流に比べて非常に小さく,この寄与によってコンダクタンスの実測値は小さくなったと考えられる.
\begin{table}[h]
\caption{コンダクタンスの理論値と実測値}
\label{tab:con_theo}
\centering
\begin{tabular}{c|ccc}
\hline
&\multicolumn{2}{c}{コンダクタンス / $\si{\liter.\second^{-1}}$}\\
管径 / $\si{\milli\metre}$&実測値&理論値&相対誤差 / \%\\
\hline \hline
10&0.4437 & 0.6126 & 27\\
15&1.643 & 2.067 & 20\\
\hline
\end{tabular}
\end{table}
\begin{table}[h]
  \caption{平均分子量の推定}
  \label{tab:heikin}
  \centering
  \begin{tabular}{cc}
  \hline
  分子量&割合 / $vol\%$\\
  \hline \hline
  2&20\\
  18&50\\
  28&20\\
  44&10\\
  \hline
  \end{tabular}
\end{table}
\gnu{コンダクタンスの管径依存}{graph/r_depend.tex}
\subsubsection{排気速度}
またコンダクタンス管のない場合の排気速度$2.31\ \si{\liter.\second^{-1}}$は
真空ポンプの排気速度$250\ \si{\liter.\min^{-1}}=4.17\ \si{\liter.\second^{-1}}$
に比べて小さい.これはバルブや配管などの装置全体のコンダクタンスによるものと考えられる.
ここで装置全体のコンダクタンスを$C$,真空ポンプの排気速度を$S_0$,排気速度の実測値を$S$として,
$S_0$と$C$が直列に接続されていると考えると(2.34)式から装置全体のコンダクタンスは
\begin{align}
  C=\frac{S_0S}{S_0-S}=5.17\ \si{\liter.\second^{-1}}
\end{align}
程度と見積もられる.また(2.34)式からコンダクタンス$S_1$と$S_2$を直列に接続した場合は
\begin{align}
  S=\frac{S_1S_2}{S_1+S_2}
\end{align}
となり$S_1\rightarrow\infty$とすると$S\rightarrow S_2$となる.
すなわち直列のコンダクタンスは最もコンダクタンスが小さい部分に律速されるとわかる.
\subsection{熱電子電流の測定}
\subsubsection{仕事関数について}
Taの仕事関数の文献値は$4.25\ \si{\electronvolt}$であるのに対して実測値は$4.39\ \si{\electronvolt}$であり相対誤差は$3\%$程度とよく一致している.
誤差の原因としては
\begin{itemize}
  \item 有限要素法のモデルと実物との温度差
  \item 電流測定値の誤差
  \item 気体分子による熱電子の阻害
\end{itemize}
などが考えられる.
電流測定値の誤差について,電流値は有効数字5桁から6桁なので,電流計が適切に校正されておりドリフトなどがなければ$3\%$もの誤差が乗るとは考えにくい.
また気体分子による熱電子の阻害は$I_S$を下げる方向に寄与するので,仕事関数の実測値に対しては減少させるように働くはずである.
したがってこの実測値に対しては気体分子による熱電子の阻害は主要な誤差原因ではないと考えられる.
以上から電流値に大きな誤差が乗ることは考えにくく,温度に誤差要因を求めるのが妥当と考えられる.
\subsection{質量分析}
\subsubsection{残留気体の組成}
残留気体には水素,水,一酸化炭素,二酸化炭素などが含まれていた.
真空ポンプがすべての気体に対して同様に排気を行うならば,これは吸着やアウトガス,漏れに由来するものと考えるべきである.
特に油回転ポンプと油拡散ポンプはどちらも油を用いることから,油の一部が加熱により分解し,炭化水素が容器内に拡散する.\cite{pump18:online}
残留気体はすべて有機化合物の分解により発生しうるものであり,ポンプに由来すると考えられる.
\bibliography{ref.bib}
\end{document}