\section{考察}
\subsection{コンダクタンスの測定}
\subsubsection{コンダクタンスの管径依存性}
図\ref{fig:graph/r_depend.tex}にコンダクタンスの管径依存を示す.
(\ref{equ:215-conduct})からコンダクタンスの管径依存は$y=ax^n$という形を取ると予想されるので
$y=ax^n$でFittingを行った.
Fitting曲線は以下のようになった.
\begin{align}
  y=(2.6\times10^{-4})\times x^{3.2}
\end{align}
したがってコンダクタンスの管径依存は$x^3$に近く,チャンバー内は分子流的であると考えられる.

また(2.23)と(2.27)から温度$T$における平均自由行程は
\begin{align}
  \lambda=\frac{k_BT}{\sqrt{2}\pi d^2 p}
\end{align}
となる.特に室温($25\ \si{\degreeCelsius}$)で圧力が$10^{-3}$程度,気体が窒素分子で$d=3.75\ \si{\AA}$とすると
\begin{align}
  \lambda\sim 6.6\ \si{\metre}
\end{align}
程度になりチャンバーの典型的な大きさ($200\ \si{\milli\metre}$)程度に比べて十分大きい.
したがって実測された流体の挙動と理論からの予測は一致する.

また気体が完全に分子流であるとすると(2.36)と気体分子の平均速度が$\sqrt{8k_BT/\pi m}$であることから
\begin{align}
  C=\frac{2\pi a^3}{3L}\sqrt{\frac{8k_BT}{\pi m}}
\end{align}
となり,各コンダクタンス管のコンダクタンスは表\ref{tab:con_theo}のようになる.
ただし気体の温度は室温($25\ \si{\degreeCelsius}$)とし,
空気の平均分子量は窒素が$80\%$, 酸素が$20\%$と仮定し$28\times 0.8+32\times0.2=28.8$と置いた.
表\ref{tab:con_theo}から理論値と実測値は桁で一致しているが20\%から30\%ほど実測値の方が小さいとわかる.
このためには(5.4)式から
\begin{itemize}
  \item $a$, $L$, $T$, $m$などの量が誤差を含んでいる
  \item 粘性流の寄与が入っている
\end{itemize}
などが考えられる.ここで$a$, $L$, $T$は容易に測定可能であり,大きな誤差は無いと考える.
ここで質量測定の結果から$m$を改めて考える.
図\ref{fig:graph/mass/mass.tex}からチャンバー内の気体の組成が大まかに表\ref{tab:heikin}のようになっていると考える.
この時平均分子量は$19.4$となり空気に比べて軽くなり,コンダクタンスの理論値は更に上昇させるよう働くことがわかる.
以上からコンダクタンスの誤差は$a$, $L$, $T$, $m$の誤差に起因するものではなく,粘性流の寄与だと考えられる.
大気が窒素80\%と酸素20\%の混合気体であるならばその粘性率$\eta$は$18.2\ \si{\micro\pascal.\second}$である.\cite{kagakuNensei:online}
また気体の平均圧力を$10^{-3}\ \si{\pascal}$程度とするなら,
(2.36)から粘性流の場合のコンダクタンスは$10^{-9}$から$10^{-10}$程度のオーダーになる.
したがって粘性流の場合のコンダクタンスは分子流に比べて非常に小さく,この寄与によってコンダクタンスの実測値は小さくなったと考えられる.
\begin{table}[h]
\caption{コンダクタンスの理論値と実測値}
\label{tab:con_theo}
\centering
\begin{tabular}{c|ccc}
\hline
&\multicolumn{2}{c}{コンダクタンス / $\si{\liter.\second^{-1}}$}\\
管径 / $\si{\milli\metre}$&実測値&理論値&相対誤差 / \%\\
\hline \hline
10&0.4437 & 0.6126 & 27\\
15&1.643 & 2.067 & 20\\
\hline
\end{tabular}
\end{table}
\begin{table}[h]
  \caption{平均分子量の推定}
  \label{tab:heikin}
  \centering
  \begin{tabular}{cc}
  \hline
  分子量&割合 / $vol\%$\\
  \hline \hline
  2&20\\
  18&50\\
  28&20\\
  44&10\\
  \hline
  \end{tabular}
\end{table}
\gnu{コンダクタンスの管径依存}{graph/r_depend.tex}
\subsubsection{排気速度}
またコンダクタンス管のない場合の排気速度$2.31\ \si{\liter.\second^{-1}}$は
真空ポンプの排気速度$250\ \si{\liter.\min^{-1}}=4.17\ \si{\liter.\second^{-1}}$
に比べて小さい.これはバルブや配管などの装置全体のコンダクタンスによるものと考えられる.
ここで装置全体のコンダクタンスを$C$,真空ポンプの排気速度を$S_0$,排気速度の実測値を$S$として,
$S_0$と$C$が直列に接続されていると考えると(2.34)式から装置全体のコンダクタンスは
\begin{align}
  C=\frac{S_0S}{S_0-S}=5.17\ \si{\liter.\second^{-1}}
\end{align}
程度と見積もられる.また(2.34)式からコンダクタンス$S_1$と$S_2$を直列に接続した場合は
\begin{align}
  S=\frac{S_1S_2}{S_1+S_2}
\end{align}
となり$S_1\rightarrow\infty$とすると$S\rightarrow S_2$となる.
すなわち直列のコンダクタンスは最もコンダクタンスが小さい部分に律速されるとわかる.
\subsection{熱電子電流の測定}
\subsubsection{仕事関数について}
Taの仕事関数の文献値は$4.25\ \si{\electronvolt}$であるのに対して実測値は$4.39\ \si{\electronvolt}$であり相対誤差は$3\%$程度とよく一致している.
誤差の原因としては
\begin{itemize}
  \item 有限要素法のモデルと実物との温度差
  \item 電流測定値の誤差
  \item 気体分子による熱電子の阻害
\end{itemize}
などが考えられる.
電流測定値の誤差について,電流値は有効数字5桁から6桁なので,電流計が適切に校正されておりドリフトなどがなければ$3\%$もの誤差が乗るとは考えにくい.
また気体分子による熱電子の阻害は$I_S$を下げる方向に寄与するので,仕事関数の実測値に対しては減少させるように働くはずである.
したがってこの実測値に対しては気体分子による熱電子の阻害は主要な誤差原因ではないと考えられる.
以上から電流値に大きな誤差が乗ることは考えにくく,温度に誤差要因を求めるのが妥当と考えられる.
\subsection{質量分析}
\subsubsection{残留気体の組成}
残留気体には水素,水,一酸化炭素,二酸化炭素などが含まれていた.
真空ポンプがすべての気体に対して同様に排気を行うならば,これは吸着やアウトガス,漏れに由来するものと考えるべきである.
特に油回転ポンプと油拡散ポンプはどちらも油を用いることから,油の一部が加熱により分解し,炭化水素が容器内に拡散する.\cite{pump18:online}
残留気体はすべて有機化合物の分解により発生しうるものであり,真空オイルに由来すると考えられる.
\subsubsection{吸着,脱離について}
吸着反応が自発的に起きる時,ギブスの自由エネルギーの変化量は負でなければならないので
\begin{align}
  \Delta G=\Delta H-T\Delta S<0
\end{align}
ここで分子が物質表面に吸着したならば,分子は固体表面に束縛されるためエントロピーの変化量$\Delta S<0$となる.
以上から$-T\Delta S>0$なのでエンタルピーの変化量は
\begin{align}
  \Delta H<0
\end{align}
となる.これにより吸着反応は発熱反応であることがわかる.
ここで分子1つが吸着した際に放出するエネルギーを$\varepsilon$とする.
また固体表面に$M$個の吸着可能なサイトがあるとすると温度$T$において被覆率$\sigma$は大正準分布を用いて
\begin{align}
  \sigma=\frac{1}{A{\rm e}^{-\varepsilon/k_BT}+1}
\end{align}
で与えられる.\cite{StatPhys36:online}これはLangmuirの吸着等温式と呼ばれる.
(5.9)からある温度$T$において被覆率は一定値であり吸着と脱離が平衡状態にある,
$T$が小さい時$\sigma$は1に近づき多くの分子が吸着するのに対して,
$T$が大きくなると$\sigma$は減少する.
以上から温度変化に伴い吸着,脱離反応が起きることがわかる.