\section{原理}
\subsection{ホログラフィの特徴}
ホログラフィとは写真技術の一種である.
写真技術にはホログラフィの他にネガフィルムやCMOSなど撮像方法の異なるものが存在するが,これらは基本的に光の強度を記録する撮像方法である.
一方でホログラフィでは光の振幅情報と同時に位相情報を記録できる.
ホログラフィでは被写体表面からの散乱光と,位相が既知で揃った参照光を干渉させることで位相情報を記録する.
特にコヒーレントな光源としてレーザーが用いられる.
これによりホログラフィには以下のような特徴がある.
\begin{description}
  \item [3次元画像の記録,再生] \text{ }\\
  ホログラフィでは再生時に位相を含めた撮影時の電場を再現することができる.
  これを受けた人間の脳はホログラフィの像が3次元であるかのように認識する.
  \item [分散記録性] \text{ }\\
  従来のネガフィルムによる撮像ではフィルム上の1点が1つの画素に相当したため,
  フィルムが欠落すると画像のその部分は喪失した.
  対してホログラフィではフィルム上の各点が散乱光の情報を持っており,フィルムが一部欠落しても
  同じ像を再生できる.
  \item [多重記録] \text{ }\\
  波数ベクトルの異なる参照光を用いることで同一フィルム状に複数の情報を記録できる.
\end{description}
これらの特徴を用いてホログラフィは貨幣の偽造防止や物体の測定などに応用されている.
\subsection{フレネルホログラムの原理}