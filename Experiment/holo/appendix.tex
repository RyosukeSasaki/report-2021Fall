\appendix
\def\thesection{補遺\Alph{section}}
\renewcommand*{\thelstlisting}{\Alph{section}.\arabic{lstlisting}}
\section{ソースコード}
\begin{lstlisting}[caption=$z$軸回転関連のソースコード,label=src_zrot]
from math import pi
import skimage
from PIL import Image
from PIL import ImageDraw
import numpy as np
from matplotlib import pyplot as plt


def main():
  # 画像を読み込む
  img = Image.open('yrot_trim.png')
  # グレイスケールに変換する
  gray_img = img.convert('L')
  # NumPy 配列にする
  f_xy = np.asarray(gray_img)

  # 2 次元高速フーリエ変換で周波数領域の情報を取り出す
  f_uv = np.fft.fft2(f_xy)
  # 画像の中心に低周波数の成分がくるように並べかえる
  shifted_f_uv = np.fft.fftshift(f_uv)

  # フィルタ (ローパス) を用意する
  x_pass_filter = Image.new(mode='L',  # 8-bit pixels, black and white
                            size=(shifted_f_uv.shape[1],
                                  shifted_f_uv.shape[0]),
                            color=0,  # default black
                            )
  # 中心に円を描く
  draw = ImageDraw.Draw(x_pass_filter)
  # 円の半径
  ellipse_r = 30
  # 画像の中心
  center = (shifted_f_uv.shape[1] // 2,
            shifted_f_uv.shape[0] // 2)
  # 円の座標
  ellipse_pos = (center[0] - ellipse_r,
                  center[1] - ellipse_r,
                  center[0] + ellipse_r,
                  center[1] + ellipse_r)
  draw.ellipse(ellipse_pos, fill=255)
  # フィルタ
  filter_array = np.asarray(x_pass_filter)

  # フィルタを適用する
  filtered_f_uv = np.multiply(shifted_f_uv, filter_array)

  # パワースペクトルに変換する
  magnitude_spectrum2d = 20 * np.log(np.absolute(filtered_f_uv))

  # 元の並びに直す
  unshifted_f_uv = np.fft.fftshift(filtered_f_uv)
  # 2 次元逆高速フーリエ変換で空間領域の情報に戻す
  i_f_xy_complex = np.fft.ifft2(unshifted_f_uv)
  i_f_xy = abs(i_f_xy_complex)
  f_xy_complex = np.fft.ifft2(i_f_xy)
  i_f_phase = np.angle(f_xy_complex)
  i_f_phase = skimage.restoration.unwrap_phase(i_f_phase)
  #i_f_phase = abs(i_f_phase)

  x = np.array([x for x in range(shifted_f_uv.shape[1])])
  y = np.array([x for x in range(shifted_f_uv.shape[0])])
  X, Y = np.meshgrid(x,y)

  fig = plt.figure(figsize = (8, 8))
  ax = fig.add_subplot(111, projection="3d")
  ax.set_xlabel("x")
  ax.set_ylabel("y")
  ax.set_zlabel("Phase Shift")
  ax.plot_wireframe(X,Y,i_f_phase,rstride=10,cstride=50)
  plt.show()

  fig, axes = plt.subplots(2, 2, figsize=(12, 12))
  for axe in axes:
    for ax in axe:
      for spine in ax.spines.values():
        spine.set_visible(False)
      ax.set_xticks([])
      ax.set_yticks([])
  # 元画像
  axes[0,0].imshow(f_xy, cmap='gray')
  axes[0,0].set_title('Input Image')
  ## フィルタ画像
  #axes[0,1].imshow(filter_array, cmap='gray')
  #axes[0,1].set_title('Filter Image')
  # フィルタされた周波数領域のパワースペクトル
  axes[0,1].imshow(magnitude_spectrum2d, cmap='gray')
  axes[0,1].set_title('Filtered Magnitude Spectrum')
  # FFT -> Band-pass Filter -> IFFT した画像
  axes[1,0].imshow(i_f_xy, cmap='gray')
  axes[1,0].set_title('Inverted FFT Image')
  # FFT -> Band-pass Filter -> IFFT した画像
  axes[1,1].imshow(i_f_phase, cmap='gray')
  axes[1,1].set_title('Phase Image')
  # グラフを表示する
  plt.show()

if __name__ == '__main__':
  main()
\end{lstlisting}