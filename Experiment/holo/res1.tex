\section{結果}
\subsection{ホログラフィー干渉}
pythonスクリプトを用いて生成した平板回転による干渉のシミュレーション結果を図\ref{fig:fig/zrot.png},図\ref{fig:fig/yrot.png}に示す.
ここで計算条件は(\ref{equ:132_zrot})式, (\ref{equ:132_yrot})式から求めたものであり,表\ref{tab:keisanjouken}のとおりである.
図\ref{equ:132_zrot}を見ると,干渉縞は斜めに入っており,また明線の本数は$x=0\ \si{\milli\metre}$で数えると20本である.
図\ref{equ:132_yrot}を見ると,干渉縞はほぼ垂直に入っており,また明線の本数は$y=0\ \si{\milli\metre}$で数えると20本である.
\begin{table}[h]
\caption{計算条件}
\label{tab:keisanjouken}
\centering
\begin{tabular}{ccc}
\hline
条件&回転軸&角度$\theta$\\
\hline \hline
1&$z$軸回転&$2.98\times10^{-4}$\\
2&$y$軸回転&$9.27\times10^{-5}$\\
\hline
\end{tabular}
\end{table}
\mfig[width=8cm]{fig/zrot.png}{条件1($z$軸回転)}
\mfig[width=8cm]{fig/yrot.png}{条件2($y$軸回転)}
\subsection{自由課題4}
図\ref{fig:fig/free4_trim.png}に得られたホログラフィ像を示す.ただし図\ref{fig:fig/free4_trim.png}は消しゴムが写っている部分だけをトリミングしている.
図\ref{fig:fig/free4_trim.png}を見ると左右が干渉縞の間隔が狭く,中央ほど間隔が広いことがわかる.
また中央下部に暗い領域がある.
\mfig[width=8cm]{fig/free4_trim.png}{自由課題4のホログラフィ像}