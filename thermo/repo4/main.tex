\documentclass[uplatex,a4j,11pt,dvipdfmx]{jsarticle}
\usepackage{listings,jvlisting}
\bibliographystyle{jplain}

\usepackage{url}

\usepackage{graphicx}
\usepackage{gnuplot-lua-tikz}
\usepackage{pgfplots}
\usepackage{tikz}
\usepackage{amsmath,amsfonts,amssymb}
\usepackage{bm}
\usepackage{siunitx}

\makeatletter
\def\fgcaption{\def\@captype{figure}\caption}
\makeatother
\newcommand{\setsections}[3]{
\setcounter{section}{#1}
\setcounter{subsection}{#2}
\setcounter{subsubsection}{#3}
}
\newcommand{\mfig}[3][width=15cm]{
\begin{center}
\includegraphics[#1]{#2}
\fgcaption{#3 \label{fig:#2}}
\end{center}
}
\newcommand{\gnu}[2]{
\begin{figure}[hptb]
\begin{center}
\input{#2}
\caption{#1}
\label{fig:#2}
\end{center}
\end{figure}
}

\begin{document}
\title{熱統計力学2 レポートNo.4}
\author{佐々木良輔}
\date{}
\maketitle
\subsubsection*{問1}
$\log(2\cosh f(x))$を$x$で微分すると
\begin{align}
  \frac{d}{dx}\log(2\cosh f(x))=\frac{df(x)}{dx}\frac{1}{2\cosh f(x)}\frac{d(2\cosh f(x))}{dx}=\frac{df(x)}{dx}\tanh f(x)
\end{align}
となるので自由エネルギーの極値条件は
\begin{align}
  \begin{split}
    0&=JNd\langle S_z\rangle-Nk_BT\times\frac{\beta g\mu_B}{2}\frac{Jd}{g\mu_B}\tanh\left(\frac{1}{2}\beta g\mu_B\left(H+\frac{Jd}{g\mu_B}\langle S_z\rangle\right)\right)\\
    &=\langle S_z\rangle-\frac{1}{2}\tanh\left(\frac{1}{2}\beta g\mu_B\left(H+\frac{Jd}{g\mu_B}\langle S_z\rangle\right)\right)\\
    \therefore\ \langle S_z\rangle&=\frac{1}{2}\tanh\left(\frac{1}{2}\beta g\mu_B\left(H+\frac{Jd}{g\mu_B}\langle S_z\rangle\right)\right)
  \end{split}
\end{align}
となり自己無撞着方程式が得られた.
\subsubsection*{問2}
低席比熱は
\begin{align}
  \begin{split}
    C_V=\frac{\alpha^2}{2b}T
  \end{split}
\end{align}
であった.ここで
\begin{align}
  \alpha&=\frac{JNd}{T_C}\\
  b&=\frac{1}{48}\frac{NJ^4d^4}{k_B^3T^3}
\end{align}
なので,代入すると
\begin{align}
  C_V=24\frac{Nk_B^3}{J^2d^2}\frac{T^4}{T_C^2}
\end{align}
ここで$T\rightarrow T_C-0$とすると$T_C=Jd/4k_B$より
\begin{align}
  \begin{split}
    C_V&\rightarrow24\frac{Nk_B^3}{J^2d^2}T_c^2\\
    &=24\frac{Nk_B^3}{J^2d^2}\frac{J^2d^2}{16k_B^2}\\
    &=\frac{3}{2}Nk_B
  \end{split}
\end{align}
また$T\rightarrow T_C+0$では$C_V=0$なので
\begin{align}
  \Delta C_V=\frac{3}{2}Nk_B
\end{align}
\subsubsection*{問3}
Ising模型でのHamiltonianは
\begin{align}
  H=-\frac{J}{2}\sum_{(i,j)}\sigma_i\sigma_j-h\sum_{i=1}^N\sigma_i
\end{align}
である.
各サイトでの$\sigma$を$\sigma_i=\langle\sigma\rangle+\delta\sigma_i$と表す.
このとき相互作用項は
\begin{align}
  \begin{split}
    \sum_{(i,j)}\sigma_i\sigma_j&=\sum_{(i,j)}(\langle\sigma_i\rangle+\delta\sigma_i)(\langle\sigma_i\rangle+\delta\sigma_j)\\
    &=\sum_{(i,j)}\left(\langle\sigma\rangle^2+\langle\sigma\rangle(\delta\sigma_i+\delta\sigma_j)+\delta\sigma_i\delta\sigma_j\right)\\
    &\simeq\sum_{(i,j)}\left(\langle\sigma\rangle^2+\langle\sigma\rangle(\sigma_i-\langle\sigma\rangle+\sigma_j-\langle\sigma_j\rangle)\right)\\
    &=\sum_{(i,j)}\left(\langle\sigma\rangle(\sigma_i+\sigma_j)-\langle\sigma\rangle^2\right)
  \end{split}
\end{align}
ここで$\delta\sigma_i$の2乗項は無視した.また和は近接する対を重複を許して足し合わせているので
\begin{align}
  \sum_{(i,j)}=Nd,\qquad \sum_{(i,j)}=d\sum_{i=1}^N
\end{align}
とできるので
\begin{align}
  \sum_{(i,j)}\sigma_i\sigma_j&=2d\langle\sigma\rangle\sum_{i=1}^N\sigma_i-Nd\langle\sigma\rangle^2
\end{align}
したがって平均場近似を用いたHamiltonianは
\begin{align}
  \begin{split}
    H_{平}&=-Jd\langle\sigma\rangle\sum_{i=1}^N\sigma_i+\frac{1}{2}JNd\langle\sigma\rangle^2-h\sum_{i=1}^N\sigma_i\\
    &=:-h_{\rm eff}\sum_{i=1}^N\sigma_i+\frac{1}{2}JdN\langle\sigma\rangle^2
  \end{split}
\end{align}
したがって分配関数は
\begin{align}
  \begin{split}
    Z_{平}&=\sum_{\{\sigma_i\}=\pm1}{\rm e}^{-\beta H_{平}}\\
    &={\rm e}^{-\beta JdN\langle\sigma\rangle^2/2} \sum_{\{\sigma_i\}=\pm1} {\rm e}^{\beta h_{\rm eff}(\sigma_1+\sigma_2+\cdots+\sigma_N)}\\
    &={\rm e}^{-\beta JdN\langle\sigma\rangle^2/2}\left({\rm e}^{\beta h_{\rm eff}}+{\rm e}^{-\beta h_{\rm eff}}\right)^N\\
    &={\rm e}^{-\beta JdN\langle\sigma\rangle^2/2}\left(2\cosh (\beta h_{\rm eff})\right)^N
  \end{split}
\end{align}
これを用いて$\sigma_l$の期待値を計算する
\begin{align}
  \begin{split}
    \langle\sigma_l\rangle&=\frac{1}{Z_{平}}\sum_{\{\sigma_i\}=\pm1}\sigma_l{\rm e}^{-\beta H_{平}}\\
    &=\frac{{\rm e}^{-\beta JdN\langle\sigma\rangle^2/2}}{{\rm e}^{-\beta JdN\langle\sigma\rangle^2/2}\left(2\cosh (\beta h_{\rm eff})\right)^N}\sum_{\{\sigma_i\}=\pm1}\sigma_l{\rm e}^{\beta h_{\rm eff}(\sigma_1+\sigma_2+\cdots+\sigma_N)}\\
    &=\frac{1}{(2\cosh(\beta h_{\rm eff}))^N}\left({\rm e}^{\beta h_{\rm eff}}+{\rm e}^{-\beta h_{\rm eff}}\right)^{N-1}\left({\rm e}^{\beta h_{\rm eff}}-{\rm e}^{-\beta h_{\rm eff}}\right)\\
    &=\frac{(2\cosh(\beta h_{\rm eff}))^{N-1}(2\sinh(\beta h_{\rm eff}))}{(2\cosh(\beta h_{\rm eff}))^N}\\
    &=\tanh(\beta h_{\rm eff})=\tanh(\beta(h+Jd\langle\sigma\rangle))
  \end{split}
\end{align}
ここでサイト$l$における期待値が平均場に等しいとすると
\begin{align}
  \langle\sigma\rangle=\tanh(\beta(h+Jd\langle\sigma\rangle))
\end{align}
となり,自己無撞着方程式を得る.
ここで$x\ll1$で$\tanh x\simeq x$なので, $h=0$かつ$\langle\sigma\rangle\rightarrow0$で(16)は
\begin{align}
  \begin{split}
    \langle\sigma\rangle&=\beta Jd\langle\sigma\rangle\\
    \therefore\ T&=\frac{Jd}{k_B}:=T_C
  \end{split}
\end{align}
\end{document}
