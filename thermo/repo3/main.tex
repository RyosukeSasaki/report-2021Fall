\documentclass[uplatex,a4j,11pt,dvipdfmx]{jsarticle}
\usepackage{listings,jvlisting}
\bibliographystyle{jplain}

\usepackage{url}

\usepackage{graphicx}
\usepackage{gnuplot-lua-tikz}
\usepackage{pgfplots}
\usepackage{tikz}
\usepackage{amsmath,amsfonts,amssymb}
\usepackage{bm}
\usepackage{siunitx}

\makeatletter
\def\fgcaption{\def\@captype{figure}\caption}
\makeatother
\newcommand{\setsections}[3]{
\setcounter{section}{#1}
\setcounter{subsection}{#2}
\setcounter{subsubsection}{#3}
}
\newcommand{\mfig}[3][width=15cm]{
\begin{center}
\includegraphics[#1]{#2}
\fgcaption{#3 \label{fig:#2}}
\end{center}
}
\newcommand{\gnu}[2]{
\begin{figure}[hptb]
\begin{center}
\input{#2}
\caption{#1}
\label{fig:#2}
\end{center}
\end{figure}
}

\begin{document}
\title{熱統計力学2 レポートNo.3}
\author{佐々木良輔}
\date{}
\maketitle
\subsubsection*{問1}
$\Gamma_\triangle$は
\begin{align}
  \Gamma_\triangle=-\frac{\beta^3\rho^2}{6}\int d{\bm r}_2d{\bm r}_3\ v({\bm r}_1-{\bm r}_2)v({\bm r}_2-{\bm r}_3)v({\bm r}_3-{\bm r}_1)
\end{align}
である.ここで${\bm R}_{12}={\bm r}_1-{\bm r}_2$, ${\bm R}_{23}={\bm r}_2-{\bm r}_3$として,また$v$をFourier変換すると積分項は
\begin{align}
  \begin{split}
    &\int d{\bm R}_{12}d{\bm R}_{23}\ \sum_{\bm p}\frac{v_{\bm p}}{V}{\rm e}^{i{\bm p}\cdot{\bm R}_{12}}
    \sum_{\bm p'}\frac{v_{\bm p}'}{V}{\rm e}^{i{\bm p''}\cdot{\bm R}_{23}}
    \sum_{\bm p''}\frac{v_{\bm p}''}{V}{\rm e}^{i{\bm p''}\cdot({\bm R}_{23}-{\bm R}_{12})}\\
    =&\frac{1}{V^3}\sum_{{\bm p},{\bm p}',{\bm p}''}v_{\bm p}v_{\bm p'}v_{\bm p''}
    \int d{\bm R}_{12}d{\bm R}_{23}\ {\rm e}^{i({\bm p}-{\bm p''}){\bm R}_{12}}{\rm e}^{i({\bm p'}+{\bm p''}){\bm R}_{23}}\\
    =&\frac{1}{V}\sum_{{\bm p},{\bm p}',{\bm p}''}v_{\bm p}v_{\bm p'}v_{\bm p''}\delta_{{\bm p},{\bm p}''}\delta_{{\bm p'},-{\bm p}''}\\
    =&\frac{1}{V}\sum_{\bm p}v_{\bm p}^3
  \end{split}
\end{align}
以上から
\begin{align}
  \Gamma_\triangle=-\frac{\beta^3\rho^2}{6}\frac{1}{V}\sum_{\bm p}v_{\bm p}^3
\end{align}
となる.
\subsubsection*{問2}
\begin{align}
  \Gamma=\frac{1}{4\rho\pi^2}\int_0^\infty dp\ p^2\left(\frac{\kappa^2}{p^2}-\log\left(1+\frac{\kappa^2}{p^2}\right)\right)
\end{align}
であった.ここで積分範囲を$[0,D]$までとし,計算後に$D\rightarrow\infty$とすることで計算を行う.すると積分項は
\begin{align}
  \begin{split}
    &\lim_{D\rightarrow\infty}\left(\int_0^D dp\ \kappa^2-\int_0^D dp\ p^2\log\left(1+\frac{\kappa^2}{p^2}\right)\right)\\
    =&\lim_{D\rightarrow\infty}\left(\kappa^2D-\int_0^D dp\ p^2\log\left(1+\frac{\kappa^2}{p^2}\right)\right)
  \end{split}
\end{align}
ここで第2項について部分積分を行う.
\begin{align}
  \begin{split}
    \int_0^D dp\ p^2\log\left(1+\frac{\kappa^2}{p^2}\right)=&\left[\frac{p^3}{3}\log\left(1+\frac{\kappa^2}{p^2}\right)\right]_0^D+\int_0^Ddp\ \frac{2}{3}\frac{\kappa^2p^2}{\kappa^2+p^2}\\
    =&\frac{D^3}{3}\log\left(1+\frac{\kappa^2}{D^2}\right)+\frac{2\kappa^2}{3}\left[p-\kappa\arctan\frac{p}{\kappa}\right]_0^D\\
    =&\frac{D^3}{3}\log\left(1+\frac{\kappa^2}{D^2}\right)+\frac{2\kappa^2D}{3}-\frac{2\kappa^3}{3}\arctan\frac{D}{\kappa}
  \end{split}
\end{align}
第1項を展開すると
\begin{align}
  \begin{split}
    \frac{D^3}{3}\log\left(1+\frac{\kappa^2}{D^2}\right)=&\frac{D^3}{3}\left(\frac{\kappa^2}{D^2}+O(D^{-4})\right)\\
    =&\frac{\kappa^2D}{3}+O(D^{-1})
  \end{split}
\end{align}
したがって(5)式は
\begin{align}
  \begin{split}
    &\lim_{D\rightarrow\infty}\left(k^2D-\frac{\kappa^2D}{3}-\frac{2\kappa^2D}{3}+\frac{2\kappa^3}{3}\arctan\frac{D}{\kappa}+O(D^{-1})\right)\\
    =&\frac{\kappa^3\pi}{3}
  \end{split}
\end{align}
以上より
\begin{align}
  \Gamma=\frac{1}{4\rho\pi^2}\frac{\kappa^3\pi}{3}=\frac{\kappa^3}{12\rho\pi}
\end{align}
となる.
\end{document}
