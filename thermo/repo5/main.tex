\documentclass[uplatex,a4j,11pt,dvipdfmx]{jsarticle}
\usepackage{listings,jvlisting}
\bibliographystyle{jplain}

\usepackage{url}

\usepackage{graphicx}
\usepackage{gnuplot-lua-tikz}
\usepackage{pgfplots}
\usepackage{tikz}
\usepackage{amsmath,amsfonts,amssymb}
\usepackage{bm}
\usepackage{siunitx}
\usepackage{empheq}
\usepackage{cases}

\makeatletter
\def\fgcaption{\def\@captype{figure}\caption}
\makeatother
\newcommand{\setsections}[3]{
\setcounter{section}{#1}
\setcounter{subsection}{#2}
\setcounter{subsubsection}{#3}
}
\newcommand{\mfig}[3][width=15cm]{
\begin{center}
\includegraphics[#1]{#2}
\fgcaption{#3 \label{fig:#2}}
\end{center}
}
\newcommand{\gnu}[2]{
\begin{figure}[hptb]
\begin{center}
\input{#2}
\caption{#1}
\label{fig:#2}
\end{center}
\end{figure}
}

\begin{document}
\title{熱統計力学2 レポートNo.5}
\author{佐々木良輔}
\date{}
\maketitle
\subsubsection*{(1)}
van der waals状態方程式から圧力$p$は
\begin{align}
  p=\frac{k_BT}{v-b}-\frac{a}{v^2}
\end{align}
であったので$A$点の条件は
\begin{align}
  \begin{split}
    0&=\cfrac{\partial p}{\partial v}=-\cfrac{k_BT_A}{(v_A-b)^2}+2\cfrac{a}{v_A^3}\\
    \iff
    0&=k_BT_Av_A^3-2a(v_A-b)^2
  \end{split}
\end{align}
\begin{align}
  \begin{split}
    0&=\cfrac{\partial^2 p}{\partial v^2}=\cfrac{2k_BT_A}{(v_A-b)^3}-\cfrac{6a}{v_A^4}\\
    \iff
    0&=k_BT_Av_A^4-3a(v_A-b)^3
  \end{split}
\end{align}
まず$(2)\times-3(v-b)/2+(3)$から
\begin{align}
  \begin{split}
    0&=k_BT_A(v^4-\frac{3}{2}v_A^3(v_A-b))\\
    \iff
    0&=v_A-\frac{3}{2}(v_A-b)\\
    \therefore\ 
    v_A&=3b
  \end{split}
\end{align}
また(2)に(4)を代入すれば
\begin{align}
  \begin{split}
    0&=k_BT_A(3b)^3-2a(2b^2)\\
    \therefore\ 
    T_A&=\frac{8}{27}\frac{a}{k_Bb}    
  \end{split}
\end{align}
以上の結果を(1)に代入すれば
\begin{align}
  \begin{split}
    p_A&=k_B\frac{8a}{27k_Bb}\frac{1}{3b-b}-\frac{a}{(3b)^2}\\
    &=\frac{1}{27}\frac{a}{b^2}
  \end{split}
\end{align}
となり$v_A$, $T_A$, $p_A$が求まった.
また
\begin{align}
  T_A=v_A\times p_A\times\frac{8}{3k_B}
\end{align}
となることがわかる.
これを用いて状態方程式の両辺を$v_A\times p_A$で割ると
\begin{align}
  \begin{split}
    \left(\frac{p}{p_A}+\frac{a}{v^2}\times\frac{v_A^2}{p_A}\times v_A^2\right)\left(\frac{v}{v_A}-\frac{b}{v_A}\right)&=k_BT\times\frac{8}{3k_BT_A}\\
    \left(\overline{p}+\frac{a}{\overline{v}^2}\times\frac{27b^2}{a}\times 9b^2\right)\left(\overline{v}-\frac{b}{3b}\right)&=\frac{8}{3}\overline{T}\\
    \left(\overline{p}+\frac{3}{\overline{v}^2}\right)\left(\overline{v}-\frac{1}{3}\right)&=\frac{8}{3}\overline{T}
  \end{split}
\end{align}
を得る.
\subsubsection*{(2)}
Clapeyron-Clausiusの式は
\begin{align}
  \left.\frac{dP}{dT}\right|_{相境界}=\frac{S_{液}-S_{固}}{V_{液}-V_{固}}
\end{align}
であった.
$T$-$P$相図において傾きが負であることは(9)の右辺が負であることに相当するが,
$S_{液}-S_{固}>0$より
\begin{align}
  \begin{split}
    V_{液}&-V_{固}<0\\
    \therefore\ V_{液}&<V_{固}
  \end{split}
\end{align}
となる.したがって水では固体の体積が液体よりも大きく,質量が保存することから
固体の密度が液体よりも低いことになる.
これによって氷はより密度の大きい水に浮かぶことになる.
\subsubsection*{(3)}
実現する磁化は$F$が極小となる場合なので,
Landauの理論から
\begin{align}
  \begin{split}    
    0&=\frac{\partial F}{\partial M}=M(a+bM^2)\\
    \therefore\ 
    M&=\left\{
      \begin{array}{l}
        0\\
        \pm\sqrt{\cfrac{-a}{b}}=\pm\sqrt{\cfrac{\alpha}{b}}\sqrt{T_C-T}
    \end{array}
    \right.
  \end{split}
\end{align}
$T>T_C$のとき下の解は実数でなくなるので$M=0$となることがわかる.
一方で$T\leq T_C$のとき
\begin{align}
  F(\pm\sqrt{-a/b})=F_0-\frac{1}{2}\frac{a^2}{b}+\frac{1}{4}\frac{a^2}{b}=F_0-\frac{1}{4}\frac{a^2}{b}<F_0=F(0)
\end{align}
となり下の解が実現する.以上から
\begin{align}
  M=\left\{
    \begin{array}{ll}
      0&(T>T_C)\\
      \pm\sqrt{\cfrac{\alpha}{b}}\sqrt{T_C-T}&(T\leq T_C)
  \end{array}\right.
\end{align}
また
\begin{align}
  F(M)=\left\{
    \begin{array}{ll}
      F(0)=F_0&(T>T_C)\\
      F(\pm\sqrt{-a/b})=F_0-\cfrac{\alpha^2}{4b}(T-T_C)^2&(T\leq T_C)
    \end{array}\right.
\end{align}
である.
これを用いるとエントロピー$S$は$S_0=-\partial F_0/\partial T$を用いて
\begin{align}
  S(M)=-\frac{\partial F}{\partial T}=\left\{
    \begin{array}{ll}
      S_0&(T>T_C)\\
      S_0+\cfrac{\alpha^2}{2b}(T-T_C)&(T\leq T_C)
    \end{array}\right.
\end{align}
また定積比熱$C_V$は$C_{V0}=T(\partial S/\partial T)$を用いて
\begin{align}
  C_V(M)=T\frac{\partial S}{\partial T}=\left\{
    \begin{array}{ll}
      C_{V0}&(T>T_C)\\
      C_{V0}+\cfrac{\alpha^2}{2b}T&(T\leq T_C)
    \end{array}\right.
\end{align}
となる.
\end{document}
  