\documentclass[uplatex,a4j,11pt,dvipdfmx]{jsarticle}
\usepackage{listings,jvlisting}
\bibliographystyle{jplain}

\usepackage{url}

\usepackage{graphicx}
\usepackage{gnuplot-lua-tikz}
\usepackage{pgfplots}
\usepackage{tikz}
\usepackage{amsmath,amsfonts,amssymb}
\usepackage{bm}
\usepackage{siunitx}

\makeatletter
\def\fgcaption{\def\@captype{figure}\caption}
\makeatother
\newcommand{\setsections}[3]{
\setcounter{section}{#1}
\setcounter{subsection}{#2}
\setcounter{subsubsection}{#3}
}
\newcommand{\mfig}[3][width=15cm]{
\begin{center}
\includegraphics[#1]{#2}
\fgcaption{#3 \label{fig:#2}}
\end{center}
}
\newcommand{\gnu}[2]{
\begin{figure}[hptb]
\begin{center}
\input{#2}
\caption{#1}
\label{fig:#2}
\end{center}
\end{figure}
}

\begin{document}
\title{熱統計力学2 レポートNo.2}
\author{佐々木良輔}
\date{}
\maketitle
\subsubsection*{問1}
$\langle x_1x_2\rangle_C$は$n_1=n_2=1$なので
\begin{align}
  \begin{split}
    1&=\sum_jk_jm_{j1}=k_1m_{11}+k_2m_{21}\\
    1&=\sum_jk_jm_{j2}=k_1m_{12}+k_2m_{22}
  \end{split}
\end{align}
$\sum_jk_j=k_1+k_2=1$に対しては
\begin{align}
  \begin{split}
    (k_1,k_2)&=(1,0)\\
    {\bm m}_1=(m_{11},m_{12})&=(1,1)
  \end{split}
\end{align}
$\sum_jk_j=k_1+k_2=2$に対しては
\begin{align}
  \begin{split}
    (k_1,k_2)&=(1,1)\\
    {\bm m}_1=(m_{11},m_{12})&=(1,0)\\
    {\bm m}_2=(m_{21},m_{22})&=(0,1)
  \end{split}
\end{align}
したがって
\begin{align}
  \begin{split}
    \langle x_1x_2\rangle_C&=(-1)^0(0)!\frac{1}{1!}\frac{\langle x_1^1x_2^1\rangle}{1!\cdot 1!}+(-1)^1(1)!\frac{1}{1!}\frac{\langle x_1^1x_2^0\rangle}{1!\cdot 0!}\frac{1}{1!}\frac{\langle x_1^0x_2^1\rangle}{0!\cdot 1!}\\
    &=\langle x_1x_2\rangle-\langle x_1\rangle\langle x_2\rangle
  \end{split}
\end{align}
\subsubsection*{問2}
$\langle x_1x_2x_3\rangle_C$は$n_1=n_2=n_3=1$なので
\begin{align}
  \begin{split}
    1&=\sum_jk_jm_{j1}=k_1m_{11}+k_2m_{21}+k_2m_{31}\\
    1&=\sum_jk_jm_{j2}=k_1m_{12}+k_2m_{22}+k_2m_{32}\\
    1&=\sum_jk_jm_{j3}=k_1m_{13}+k_2m_{23}+k_2m_{33}
  \end{split}
\end{align}
$\sum_jk_j=k_1+k_2+k_3=1$に対しては
\begin{align}
  \begin{split}
    (k_1,k_2,k_3)&=(1,0,0)\\
    {\bm m}_1=(m_{11},m_{12},m_{13})&=(1,1,1)
  \end{split}
\end{align}
$\sum_jk_j=k_1+k_2+k_3=2$に対しては
\begin{align}
  \begin{split}
    \left\{
      \begin{array}{c}
      (k_1,k_2,k_3)=(1,1,0)\\
      {\bm m}_1=(1,1,0)\\
      {\bm m}_2=(0,0,1)
    \end{array}\right.
  \end{split}\\
  \begin{split}
    \left\{
      \begin{array}{c}
        (k_1,k_2,k_3)=(1,0,1)\\
        {\bm m}_3=(1,0,1)\\
        {\bm m}_4=(0,1,0)
      \end{array}\right.
  \end{split}\\
  \begin{split}
    \left\{
      \begin{array}{c}
        (k_1,k_2,k_3)=(0,1,1)\\
        {\bm m}_5=(0,1,1)\\
        {\bm m}_6=(1,0,0)
      \end{array}\right.
  \end{split}
\end{align}
また$\sum_jk_j=k_1+k_2+k_3=3$に対しては
\begin{align}
  \begin{split}
    (k_1,k_2,k_3)&=(1,1,1)\\
    {\bm m}_1&=(1,0,0)\\
    {\bm m}_2&=(0,1,0)\\
    {\bm m}_3&=(0,0,1)
  \end{split}
\end{align}
以上から
\begin{align}
  \begin{split}
    \langle x_1x_2x_3\rangle_C&=(-1)^0(0)!\frac{1}{1!}\frac{\langle x_1x_2x_3\rangle}{1!\cdot1!\cdot1!}\\
    &+(-1)^1(1)!\frac{1}{1!}\frac{\langle x_1^1x_2^1x_3^0\rangle}{1!\cdot1!\cdot0!}\frac{1}{1!}\frac{\langle x_1^0x_2^0x_3^1\rangle}{0!\cdot0!\cdot1!}\\
    &+(-1)^1(1)!\frac{1}{1!}\frac{\langle x_1^1x_2^0x_3^1\rangle}{1!\cdot0!\cdot1!}\frac{1}{1!}\frac{\langle x_1^0x_2^1x_3^0\rangle}{0!\cdot1!\cdot0!}\\
    &+(-1)^1(1)!\frac{1}{1!}\frac{\langle x_1^0x_2^1x_3^1\rangle}{0!\cdot1!\cdot1!}\frac{1}{1!}\frac{\langle x_1^1x_2^0x_3^0\rangle}{1!\cdot0!\cdot0!}\\
    &+(-1)^2(2)!\frac{1}{1!}\frac{\langle x_1^1x_2^0x_3^0\rangle}{1!\cdot0!\cdot0!}
    \frac{1}{1!}\frac{\langle x_1^0x_2^1x_3^0\rangle}{0!\cdot1!\cdot0!}
    \frac{1}{1!}\frac{\langle x_1^0x_2^0x_3^1\rangle}{0!\cdot0!\cdot1!}
  \end{split}
\end{align}
\begin{align}
  \begin{split}
    \therefore\ \langle x_1x_2x_3\rangle_C&=\langle x_1x_2x_3\rangle-\langle x_1x_2\rangle\langle x_3\rangle-\langle x_1x_3\rangle\langle x_2\rangle\\
    &-\langle x_2x_3\rangle\langle x_1\rangle+2\langle x_1\rangle\langle x_2\rangle\langle x_3\rangle
  \end{split}
\end{align}
\subsubsection*{問3}
$U(\beta)$を展開した際の3次項は
\begin{align}
  \begin{split}
    T_\tau\left(-\frac{1}{3!}\int_0^\beta{\rm d}\tau\int_0^\beta{\rm d}\tau'\int_0^\beta{\rm d}\tau''V(\tau)V(\tau')V(\tau'')\right)
    =&-\frac{1}{6}\int_0^\beta{\rm d}\tau\int_0^\tau{\rm d}\tau'\int_0^{\tau'}{\rm d}\tau''V(\tau)V(\tau')V(\tau'')\\
    &-\frac{1}{6}\int_0^\beta{\rm d}\tau\int_0^\tau{\rm d}\tau''\int_0^{\tau''}{\rm d}\tau'V(\tau)V(\tau'')V(\tau')\\
    &-\frac{1}{6}\int_0^\beta{\rm d}\tau'\int_0^{\tau'}{\rm d}\tau\int_0^{\tau}{\rm d}\tau''V(\tau')V(\tau)V(\tau'')\\
    &-\frac{1}{6}\int_0^\beta{\rm d}\tau'\int_0^{\tau'}{\rm d}\tau''\int_0^{\tau''}{\rm d}\tau V(\tau')V(\tau'')V(\tau)\\
    &-\frac{1}{6}\int_0^\beta{\rm d}\tau''\int_0^{\tau''}{\rm d}\tau\int_0^{\tau}{\rm d}\tau' V(\tau'')V(\tau)V(\tau')\\
    &-\frac{1}{6}\int_0^\beta{\rm d}\tau''\int_0^{\tau''}{\rm d}\tau'\int_0^{\tau'}{\rm d}\tau V(\tau'')V(\tau')V(\tau)
  \end{split}
\end{align}
積分の仮変数は任意なので
\begin{align}
  T_\tau\left(-\frac{1}{3!}\int_0^\beta{\rm d}\tau\int_0^\beta{\rm d}\tau'\int_0^\beta{\rm d}\tau''V(\tau)V(\tau')V(\tau'')\right)
  =-\int_0^\beta{\rm d}\tau\int_0^\tau{\rm d}\tau'\int_0^{\tau'}{\rm d}\tau''V(\tau)V(\tau')V(\tau'')
\end{align}
\end{document}