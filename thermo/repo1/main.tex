\documentclass[uplatex,a4j,11pt,dvipdfmx]{jsarticle}
\usepackage{listings,jvlisting}
\bibliographystyle{jplain}

\usepackage{url}

\usepackage{graphicx}
\usepackage{gnuplot-lua-tikz}
\usepackage{pgfplots}
\usepackage{tikz}
\usepackage{amsmath,amsfonts,amssymb}
\usepackage{bm}
\usepackage{siunitx}

\makeatletter
\def\fgcaption{\def\@captype{figure}\caption}
\makeatother
\newcommand{\setsections}[3]{
\setcounter{section}{#1}
\setcounter{subsection}{#2}
\setcounter{subsubsection}{#3}
}
\newcommand{\mfig}[3][width=15cm]{
\begin{center}
\includegraphics[#1]{#2}
\fgcaption{#3 \label{fig:#2}}
\end{center}
}
\newcommand{\gnu}[2]{
\begin{figure}[hptb]
\begin{center}
\input{#2}
\caption{#1}
\label{fig:#2}
\end{center}
\end{figure}
}

\begin{document}
\title{熱統計力学2 レポートNo.1}
\author{佐々木良輔}
\date{}
\maketitle
図のように熱浴と接している系を考える.
系と熱浴を合わせたエネルギーは保存し,これらは熱力学的に平衡であるとする.
このとき系のエネルギーが$E$から$E+\Delta E$の間にある確率は状態密度$N$を用いて
\begin{align*}
  P(E,N)\Delta E=\frac{N(E)\Delta E\cdot N_r(E_r)\Delta E}{N_{全}(E_{全})\Delta E}
\end{align*}
である.ただし$E_r=E_{全}-E$である.ここで系と熱浴を切り離し,それぞれを孤立系にする.
ただし系と熱浴のエネルギーのゆらぎは十分小さいとする.
ここで熱浴側のエントロピーは
\begin{align*}
  S_r(E_r)=S_r(E_{全}-E)&=k_B\log(N_r(E_r)\Delta E)\\
  &=S_r(E_{全})-\frac{\partial S_r}{\partial E_{全}}E
\end{align*}
ここで$\partial S/\partial E=1/T$なので
\begin{align*}
  k_B\log(N_r(E_r)\Delta E)&=S_r(E_{全})-\frac{1}{T}E\\
  &=k_B\log\left(N_r(E_{全}){\rm e}^{-\beta E}\Delta E\right)
\end{align*}
したがって
\begin{align*}
  N_r(E_r)=N_r(E_{全}){\rm e}^{-\beta E}
\end{align*}
よって確率密度は
\begin{align*}
  P(E,V,N)=\frac{N_r(E_{全})}{N_{全}(E_{全})}N(E){\rm e}^{-\beta E}
\end{align*}
ここで$N_r(E_{全})/N_{全}(E_{全})$は定数なので$1/Z$とすると
\begin{align*}
  P(E,V,N)=\frac{1}{Z}N(E){\rm e}^{-\beta E}
\end{align*}
ここで規格化条件から
\begin{align*}
  1&=\frac{1}{Z}\int{\rm d}E\ N(E){\rm e}^{-\beta E}\\
  Z&=\int{\rm d}E\ N(E){\rm e}^{-\beta E}b
\end{align*}
エネルギーが離散的ならば
\begin{align*}
  Z&=\sum_\nu{\rm e}^{-\beta E_\nu}
\end{align*}
となる.ここでヘルムホルムの自由エネルギーは$F=E-TS$なのでボルツマンの関係式を用いれば
\begin{align*}
  F&=\sum_\nu E_\nu P_\nu+Tk_B\sum_\nu P_\nu\log P_\nu\\
  &=\sum_\nu E_\nu P_\nu+Tk_B\sum_\nu P_\nu\left(\frac{{\rm e}^{-\beta E_\nu}}{Z}\right)\\
  &=\sum_\nu E_\nu P_\nu-Tk_B\beta\sum_\nu P_\nu E_\nu-Tk_B\sum_\nu P_\nu\log Z\\
  &=-k_BT\langle\log Z\rangle=-k_BT\log Z
\end{align*}
を得る.
\bibliography{ref.bib}
\end{document}