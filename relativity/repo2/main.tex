\documentclass[uplatex,a4j,11pt,dvipdfmx]{jsarticle}
\usepackage{listings,jvlisting}
\bibliographystyle{jplain}

\usepackage{url}

\usepackage{graphicx}
\usepackage{gnuplot-lua-tikz}
\usepackage{pgfplots}
\usepackage{tikz}
\usepackage{amsmath,amsfonts,amssymb}
\usepackage{bm}
\usepackage{siunitx}

\makeatletter
\def\fgcaption{\def\@captype{figure}\caption}
\makeatother
\newcommand{\setsections}[3]{
\setcounter{section}{#1}
\setcounter{subsection}{#2}
\setcounter{subsubsection}{#3}
}
\newcommand{\mfig}[3][width=15cm]{
\begin{center}
\includegraphics[#1]{#2}
\fgcaption{#3 \label{fig:#2}}
\end{center}
}
\newcommand{\gnu}[2]{
\begin{figure}[hptb]
\begin{center}
\input{#2}
\caption{#1}
\label{fig:#2}
\end{center}
\end{figure}
}

\begin{document}
\title{相対性理論 レポートNo.2}
\author{佐々木良輔}
\date{}
\maketitle
\subsection*{Q11.}
どの慣性系からみても波の位相は一致すべきなので
\begin{align}
  \begin{split}
    -\omega t+k_x\cdot x=-\omega' t'+k'_x\cdot x'\\
    -\omega t+k_y\cdot y=-\omega' t'+k'_y\cdot y'\\
    -\omega t+k_z\cdot z=-\omega' t'+k'_z\cdot z'
  \end{split}
\end{align}
ここで$S'$から$S$へのLorentz変換は
\begin{align*}
  \left(\begin{array}{c}
    ct\\x
  \end{array}\right)
  =\gamma\left(\begin{array}{cc}
    1 & \beta\\
    \beta & 1\\
  \end{array}\right)\left(
    \begin{array}{c}
      ct'\\x'
    \end{array}
  \right)
\end{align*}
なので
\begin{align}
  \begin{split}
    t&=\gamma\left(t'+\frac{\beta}{c}x'\right)\\
    x&=\gamma\left(x'+\beta ct'\right)\\
    y&=y'\\
    z&=z'
  \end{split}
\end{align}
である.これを(1)に代入すると$y,z$成分については直ちに
\begin{align*}
  k_y'=k_y,\ k_z'=k_z
\end{align*}
とわかる.
また$t,z$成分については
\begin{align}
  \begin{split}
    -\omega' t'+k'_x\cdot x'&=-\omega\gamma\left(t'+\frac{\beta}{c}x'\right)+k_x\gamma\left(x'+\beta ct'\right)\\
    &=t'\gamma\left(k_x\beta c-\omega\right)+x'\gamma\left(k_x-\frac{\beta}{c}\omega\right)
  \end{split}
\end{align}
$t',x'$について係数を比較すれば
\begin{align}
  \begin{split}
    \frac{\omega'}{c}&=\gamma\left(\frac{\omega}{c}-k_x\beta\right)\\
    k_x'&=\gamma\left(k_x-\frac{\beta}{c}\omega\right)
  \end{split}
\end{align}
を得る.
\subsection*{Q16.}
\subsubsection*{(1)}
相対論的な速度の合成則において$v/c\ll1$, $u_x/c\ll1$, $u_x'/c\ll1$としてこれらの2次以上の微小項を無視すると
\begin{align}
  u_x=\frac{u_x'+v}{1+\frac{u_x'}{c}\frac{u_x}{c}}\simeq\frac{u_x'+v}{1+O(\frac{v^2}{c^2})}=u_x'+v
\end{align}
また
\begin{align}
  \gamma=\left(1+\left(\frac{v}{c}\right)^2\right)^{-\frac{1}{2}}\simeq1-\frac{1}{2}\left(\frac{v}{c}\right)^2
\end{align}
なので
\begin{align}
  \begin{split}
    u_i&=\frac{u_i'}{\gamma\left(1+\frac{u_x'}{c}\frac{v}{c}\right)}\\
    &\simeq\frac{u_i'}{\left(1-\frac{1}{2}\left(\frac{v}{c}\right)^2\right)\left(1+\frac{u_x'}{c}\frac{v}{c}\right)}\\
    &\simeq\frac{u_i'}{1+O(\frac{v^2}{c^2})}=u_i'\qquad (i=y,z)    
  \end{split}
\end{align}
となる.
\subsubsection*{(2)}
$S'$で見たときの$y,z$成分が0なので$u_y'=u_z'=0$
また$x$成分に関して$u_x'/c=:\beta'$とすると相対論的な速度の合成則から
\begin{align}
  u_x=c\frac{\beta'+\beta}{1+\beta'\beta}
\end{align}
ここで$\beta$を定数と見て$\beta'$について微分すると
\begin{align}
  \begin{split}
    \frac{\partial u_x}{\partial \beta'}=\frac{1-\beta^2}{(1+\beta'\beta)^2}
  \end{split}
\end{align}
$0\leq\beta<1$なので(9)は常に正であり$u_x$は$\beta'$について単調増加である.
ここで$0\leq\beta'<1$なので$u_x<u_x|_{\beta'=1}$であるので
\begin{align}
  u_x<c\frac{1+\beta}{1+\beta}=c
\end{align}
であり$u_x<c$が示された.
\subsubsection*{(3)}
$(u_x,u_y)=c(\cos\theta,\sin\theta)$とする.また$(u_x',u_y')=c(\cos\theta',\sin\theta')$とする.
したがって$\tan\theta=u_y/u_x$, $\tan\theta'=u_y'/u_x'$である.
速度の合成則を逆に解くには$S$と$S'$を入れ替えればよいので
\begin{align}
  \begin{split}
    u_x'&=\frac{u_x-v}{1-\frac{u_x}{c}\frac{v}{c}}\\
    u_i'&=\frac{u_i}{\gamma\left(1-\frac{u_x}{c}\frac{v}{c}\right)}\qquad(i=y,z)
  \end{split}
\end{align}
したがって
\begin{align}
  \begin{split}
    \tan\theta'&=\frac{u_y'}{u_x'}
    =\frac{\frac{u_y}{\gamma\left(1-\frac{u_x}{c}\frac{v}{c}\right)}}{\frac{u_x-v}{1-\frac{u_x}{c}\frac{v}{c}}}\\
    &=\frac{u_y}{\gamma(u_x-v)}\\
    &=\frac{1}{\gamma}\frac{u_y}{u_x}\frac{1}{1-\frac{v}{u_x}}
  \end{split}
\end{align}
ここで$c=\sqrt{u_x^2+u_y^2}$より
\begin{align}
  \begin{split}
    \frac{v}{u_x}&=\frac{v}{u_x}\sqrt{\frac{u_x^2+u_y^2}{c^2}}\\
    &=\frac{v}{c}\sqrt{1+\left(\frac{u_y}{u_x}\right)^2}\\
    &=\beta\sqrt{1+\tan^2\theta}=\frac{\beta}{\cos\theta}
  \end{split}
\end{align}
なので
\begin{align}
  \begin{split}
    \tan\theta'&=\frac{1}{\gamma}\tan\theta\frac{1}{1-\beta\cos\theta}\\
    &=\frac{1}{\gamma}\frac{\sin\theta}{\cos\theta-\beta}
  \end{split}
\end{align}
を得る.
\end{document}