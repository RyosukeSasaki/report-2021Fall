\documentclass[uplatex,a4j,11pt,dvipdfmx]{jsarticle}
\usepackage{listings,jvlisting}
\bibliographystyle{jplain}

\usepackage{url}

\usepackage{graphicx}
\usepackage{gnuplot-lua-tikz}
\usepackage{pgfplots}
\usepackage{tikz}
\usepackage{amsmath,amsfonts,amssymb}
\usepackage{bm}
\usepackage{siunitx}

\makeatletter
\def\fgcaption{\def\@captype{figure}\caption}
\makeatother
\newcommand{\setsections}[3]{
\setcounter{section}{#1}
\setcounter{subsection}{#2}
\setcounter{subsubsection}{#3}
}
\newcommand{\mfig}[3][width=15cm]{
\begin{center}
\includegraphics[#1]{#2}
\fgcaption{#3 \label{fig:#2}}
\end{center}
}
\newcommand{\gnu}[2]{
\begin{figure}[hptb]
\begin{center}
\input{#2}
\caption{#1}
\label{fig:#2}
\end{center}
\end{figure}
}

\begin{document}
\title{相対性理論 レポートNo.1}
\author{佐々木良輔}
\date{}
\maketitle
\subsection*{Q3.}
\subsubsection*{(1)}
Lorentz変換は1次変換なので$x',y',z',t'$は以下のように表されるべきである
\begin{align*}
  x'&=a_{11}x+a_{12}y+a_{13}z+a_{14}t\\
  y'&=a_{21}x+a_{22}y+a_{23}z+a_{24}t\\
  z'&=a_{31}x+a_{32}y+a_{33}z+a_{34}t\\
  t'&=a_{41}x+a_{42}y+a_{43}z+a_{44}t
\end{align*}
これらの係数は$x,y,z,t$には依存すべきではないが$v$には依存しても良い.
$y',z'$が陽に$t$に依存するとき,これらは時間経過とともに移動することになる.
これは系S'が$x$軸方向にのみ移動しているという仮定に反するので$a_{24}=a_{34}=0$となるべきである.
また$y'$が$x,z$, $z'$が$x,y$に依存するとき$y',z'$は$y,z$に対して傾いた軸になるため,
$a_{21}=a_{31}=a_{23}=a_{32}=0$となるべきである.ここまでで$y',z'$は
\begin{align*}
  y'=a_{22}(v)y\\
  z'=a_{33}(v)z
\end{align*}
ここで空間の等方性から$y',z'$は対等であるべきなので$a_{22}=a_{33}$である.
また空間の反転に対して対称であるべきなのでこれらの係数は$|v|$にのみ依るべきである.
また$S'$から$S$をみたとき同様に
\begin{align*}
  y=a_{22}(|-v|)y'\\
  z=a_{22}(|-v|)z'
\end{align*}
であり,以上から
\begin{align*}
  &a_{22}(|v|)^2=1\\
  \therefore\ &a_{22}(|v|)=\pm1
\end{align*}
さらに$v\rightarrow 0$で明らかに$y'=y$, $z'=z$であるべきなので
\begin{align*}
  a_{22}(|v|)=1
\end{align*}
である.以上から
\begin{align}
  y'=y\\
  z'=z
\end{align}
となる.
\subsubsection*{(2)}
まず$x'$について$x'$軸と$x$軸は一致すべきなので$a_{12}=a_{13}=0$である.
また$x'=0$は$x=vt$と一致すべきなので定数$A$を用いて
\begin{align}
  x'=A(v)(x-vt)
\end{align}
という形であるべきである.また$s^2=0\Leftrightarrow s'^2=0$より比例関係
\begin{align*}
  s'^2=\alpha(v)s^2
\end{align*}
が成り立つべきである.また空間の反転に対して対称であるべきなので$\alpha$は$|v|$にのみ依るべきである.
また前問と同様に
\begin{align*}
  s^2=\alpha(|v|)s'^2
\end{align*}
であるべきなので
\begin{align*}
  \alpha(|v|)=\pm1
\end{align*}
また$v\rightarrow 0$で$s'^2=s^2$であることから
\begin{align*}
  \alpha(|v|)=1
\end{align*}
となる.以上から
\begin{align*}
  s'^2&=s^2\\
  x'^2+y'^2+z'^2-(ct')^2&=x^2+y^2+z^2-(ct)^2
\end{align*}
(1), (2), (3)から
\begin{align*}
  A(v)^2(x-vt)^2-c^2(a_{41}x+a_{42}y+a_{43}z+a_{44}t)^2=x^2-(ct)^2
\end{align*}
右辺に$y,z$が現れないことから$a_{42}=a_{43}=0$である.したがって新たな定数$B, D$を用いて
\begin{align*}
  x^2-(ct)^2&=A(v)^2(x-vt)^2-c^2(B(v)x+D(v)t)^2\\
  &=(A^2-c^2B^2)x^2-(c^2D^2-v^2A^2)t^2+(-2vA^2-2c^2BD)xt
\end{align*}
よって係数を比較すれば
\begin{align}
  A^2-c^2B^2&=1\\
  c^2D^2-v^2A^2&=c^2\\
  vA^2+c^2BD&=0
\end{align}
を得る.まず(6)から
\begin{align*}
  B=-\frac{vA^2}{c^2D}
\end{align*}
これを(4)に代入すると
\begin{align*}
  A^2-\frac{v^2}{c^2D^2}A^4=1
\end{align*}
また(5)から
\begin{align}
  A^2=\frac{c^2}{v^2}(D^2-1)
\end{align}
これらを連立すると
\begin{align*}
  \frac{c^2}{v^2}(D^2-1)-\frac{v^2}{c^2D^2}\left(\frac{c^2}{v^2}(D^2-1)\right)^2&=1\\
  (D^2-1)\left(1-\frac{D^2-1}{D^2}\right)&=\frac{v^2}{c^2}
\end{align*}
\begin{align*}
  D=\pm\frac{1}{\sqrt{1-\frac{v^2}{c^2}}}
\end{align*}
これと(7)を連立すると
\begin{align*}
  A=D=\pm\frac{1}{\sqrt{1-\frac{v^2}{c^2}}}
\end{align*}
ここで(3)において$v\rightarrow0$のとき$x'=x$となるべきなので符号が確定し
\begin{align}
  A=D=\frac{1}{\sqrt{1-\frac{v^2}{c^2}}}
\end{align}
となる.また(4)から
\begin{align*}
  B=\pm\frac{v/c^2}{\sqrt{1-\frac{v^2}{c^2}}}
\end{align*}
であり(6)が成り立つことから符号が確定し
\begin{align*}
  B=-\frac{v/c^2}{\sqrt{1-\frac{v^2}{c^2}}}
\end{align*}
となる.以上から$x',t'$と$x,t$の間の関係は
\begin{align*}
  x'&=\gamma(x-vt)\\
  t'&=\gamma(-vx/c^2+t)
\end{align*}
すなわち
\begin{align*}
  \left(\begin{array}{c}
    ct'\\x'
  \end{array}\right)
  =\gamma\left(\begin{array}{cc}
    1&-\beta\\
    -\beta&1
  \end{array}\right)
  \left(\begin{array}{c}
    ct\\x
  \end{array}\right)
\end{align*}
を得る.
\subsection*{Q4.}
明らかに
\begin{align*}
  y=y',z=z'
\end{align*}
である.また$x$については
\begin{align}
  x'&=\gamma(x-vt)\nonumber\\
  x&=x'\frac{1}{\gamma}+vt
\end{align}
同様に$t$については
\begin{align}
  t=t'\frac{1}{\gamma}+\frac{v}{c^2}x
\end{align}
(9)に(10)を代入すると
\begin{align*}
  x&=x'\frac{1}{\gamma}+t'\frac{v}{\gamma}+\beta^2x\\
  x(1-\beta^2)&=x'\frac{1}{\gamma}+t'\frac{v}{\gamma}\\
  x\frac{1}{\gamma^2}&=x'\frac{1}{\gamma}+t'\frac{v}{\gamma}\\
  x&=\gamma(x'+vt')
\end{align*}
同様に(10)に(9)を代入すると
\begin{align*}
  t&=t'\frac{1}{\gamma}+\frac{v}{\gamma c^2}x'+\frac{v^2}{c^2}t\\
  t(1-\beta^2)&=t'\frac{1}{\gamma}+\frac{v}{\gamma c^2}x'\\
  t&=\gamma(vx'/c^2+t')
\end{align*}
以上からLorentz変換を逆に解くと$S$と$S'$を入れ替え,
$\beta$と$-\beta$に置き換えたものになっている.
\bibliography{ref.bib}
\end{document}