\documentclass[uplatex,a4j,11pt,dvipdfmx]{jsarticle}
\usepackage{listings,jvlisting}
\bibliographystyle{jplain}

\usepackage{url}

\usepackage{graphicx}
\usepackage{gnuplot-lua-tikz}
\usepackage{pgfplots}
\usepackage{tikz}
\usepackage{amsmath,amsfonts,amssymb}
\usepackage{bm}
\usepackage{siunitx}
\usepackage[thicklines]{cancel}

\renewcommand{\CancelColor}{\color{red}}
\makeatletter
\def\fgcaption{\def\@captype{figure}\caption}
\makeatother
\newcommand{\setsections}[3]{
\setcounter{section}{#1}
\setcounter{subsection}{#2}
\setcounter{subsubsection}{#3}
}
\newcommand{\mfig}[3][width=15cm]{
\begin{center}
\includegraphics[#1]{#2}
\fgcaption{#3 \label{fig:#2}}
\end{center}
}
\newcommand{\gnu}[2]{
\begin{figure}[hptb]
\begin{center}
\input{#2}
\caption{#1}
\label{fig:#2}
\end{center}
\end{figure}
}

\begin{document}
\title{相対性理論 レポートNo.12}
\author{佐々木良輔}
\date{}
\maketitle
3次元球座標における線素ベクトルは
\begin{align}
  d{\bm r}=dr{\bm e}_r+rd\theta{\bm e}_\theta+r\sin\theta d\phi{\bm e}_\phi
\end{align}
であり基底ベクトルが直交することから線素は
\begin{align}
  ds^2=(d{\bm r})^2=dr^2+r^2d\theta^2+r^2\sin^2\theta d\phi^2
\end{align}
である.ここで半径$a$の2次元球面座標上での微小変位を考えると,
球面上での移動においては動径方向の変位が無いので(2)において
\begin{align}
  dr=0
\end{align}
とすればよい.したがって2次元球面上での線素$ds^2$は
\begin{align}
  ds^2=a^2(d\theta^2+\sin^2\theta d\phi^2)
\end{align}
となる.ここで
\begin{align}
  ds^2=g_{\mu\nu}dx^\mu dx^\nu
\end{align}
なので
\begin{align}
  \begin{split}
    g_{\theta\theta}&=a^2\\
    g_{\phi\phi}&=a^2\sin^2\theta\\
    g_{\theta\phi}=g_{\phi\theta}&=0
  \end{split}
\end{align}
であり
\begin{align}
  \partial_\lambda g_{\mu\nu}=\left\{
    \begin{array}{ll}
      2a^2\sin\theta\cos\theta&(\lambda=\theta,\mu=\nu=\phi)\\
      0&({\rm otherwise})
    \end{array}
    \right.
\end{align}
したがって
\begin{align}
  \begin{split}
    \Gamma_{\theta\phi\phi}&=\frac{1}{2}(-\partial_\theta g_{\phi\phi}+\partial_\phi g_{\theta\phi}+\partial_\phi g_{\phi\theta})\\
    &=-a^2\sin\theta\cos\theta\\
    &=-\Gamma_{\phi\theta\phi}=-\Gamma_{\phi\phi\theta}
  \end{split}
\end{align}
それ以外の$\Gamma_{\lambda\mu\nu}$は$0$である.また計量を行列表示すると
\begin{align}
  g_{\mu\nu}=\left(
    \begin{array}{cc}
      a^2&0\\
      0&a^2\sin^2\theta\\
    \end{array}
  \right)
\end{align}
したがって
\begin{align}
  \begin{split}
    g^{\mu\nu}&=\left(
      \begin{array}{cc}
        a^2&0\\
        0&a^2\sin^2\theta\\
      \end{array}
      \right)^{-1}\\
      &=\left(
        \begin{array}{cc}
          1/a^2&0\\
          0&1/a^2\sin^2\theta\\
        \end{array}
        \right)
  \end{split}
\end{align}
以上からChristoffel記号は
\begin{align}
  \begin{split}
    \Gamma^\theta_{\ \phi\phi}=g^{\theta\rho}\Gamma_{\rho\phi\phi}&=g^{\theta\theta}\Gamma_{\theta\phi\phi}+g^{\theta\phi}\Gamma_{\phi\phi\phi}\\
    &=\frac{1}{a^2}(-a^2\sin\theta\cos\theta)\\
    &=-\sin\theta\cos\theta
  \end{split}
\end{align}
\begin{align}
  \begin{split}
    \Gamma^\phi_{\ \theta\phi}=g^{\phi\rho}\Gamma_{\rho\theta\phi}&=g^{\phi\theta}\Gamma_{\theta\theta\phi}+g^{\phi\phi}\Gamma_{\phi\theta\phi}\\
    &=\frac{1}{a^2\sin^2\theta}(a^2\sin\theta\cos\theta)\\
    &=\cot\theta=\Gamma^\phi_{\ \phi\theta}
  \end{split}
\end{align}
それ以外の$\Gamma^{\lambda}_{\ \mu\nu}$は$0$である.

次に曲率テンソルを考える.対称性から
\begin{align}
  R_{\alpha\alpha\mu\nu}=R{\alpha\beta\mu\mu}=0
\end{align}
となるので,残るのは$\alpha\neq\beta$かつ$\mu\neq\nu$である
\begin{align}
  R_{\theta\phi\theta\phi},R_{\theta\phi\phi\theta},R_{\phi\theta\theta\phi},R_{\phi\theta\phi\theta}
\end{align}
のみである,更に対称性から
\begin{align}
  \begin{split}
    R_{\theta\phi\theta\phi}&=-R_{\theta\phi\phi\theta}\\
    &=-R_{\phi\theta\theta\phi}\\
    &=-(-R_{\phi\theta\phi\theta})
  \end{split}
\end{align}
となるので1つの成分だけ考えれば十分である.したがって
\begin{align}
  \begin{split}
    R^\theta_{\ \phi\theta\phi}&=\partial_\theta\Gamma^{\theta}_{\ \phi\phi}
    -\partial_\phi\Gamma^\theta_{\ \theta\phi}
    +\Gamma^\theta_{\ \theta\tau}\Gamma^\tau_{\ \phi\phi}
    -\Gamma^\theta_{\ \phi\tau}\Gamma^\tau_{\ \theta\phi}\\
    &=\partial_\theta(-\sin\theta\cos\theta)-0+(0+0)-(0+(-\sin\theta\cos\theta)\cot\theta)\\
    &=-\cos^2\theta+\sin^2\theta+\cos^2\theta\\
    &=\sin^2\theta
  \end{split}
\end{align}
$R^\alpha_{\ \beta\mu\nu}$は$\mu\nu$について反対称なので
\begin{align}
  R^\theta_{\ \phi\phi\theta}=-\sin^2\theta
\end{align}
である.また
\begin{align}
  \begin{split}
    R_{\theta\phi\theta\phi}&=g_{\theta\rho}R^\rho_{\ \phi\theta\phi}\\
    &=g_{\theta\theta}R^\theta_{\ \phi\theta\phi}+0\\
    &=a^2\sin^2\theta=-R_{\phi\theta\theta\phi}
  \end{split}
\end{align}
よって
\begin{align}
  \begin{split}
    R^\phi_{\ \theta\theta\phi}&=g^{\phi\rho}R_{\rho\theta\theta\phi}\\
    &=\frac{1}{a^2\sin^2\theta}(-a^2\sin^2\theta)+0\\
    &=-1
  \end{split}
\end{align}
であり,また
\begin{align}
  R^\phi_{\ \theta\phi\theta}=1
\end{align}
となる.次にRicciテンソルは
\begin{align}
  R_{\theta\theta}=R^\alpha_{\ \theta\theta\alpha}=-1+0=-1
\end{align}
\begin{align}
  R_{\theta\phi}=R^\theta_{\ \theta\phi\theta}+R^\phi_{\ \theta\phi\phi}=0=R_{\phi\theta}
\end{align}
\begin{align}
  R_{\phi\phi}=-\sin^2\theta
\end{align}
となる.次にスカラー曲率は$g^{\theta\phi}=g^{\phi\theta}=0$から
\begin{align}
  \begin{split}
    R&=g^{\mu\nu}R_{\mu\nu}\\
    &=g^{\theta\theta}R_{\theta\theta}+g^{\phi\phi}R_{\phi\phi}\\
    &=\frac{1}{a^2}(-1)+\frac{1}{a^2\sin^2\theta}(-\sin^2\theta)\\
    &=-\frac{2}{a^2}
  \end{split}
\end{align}
となる.
\end{document}