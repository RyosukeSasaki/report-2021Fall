\documentclass[uplatex,a4j,11pt,dvipdfmx]{jsarticle}
\usepackage{listings,jvlisting}
\bibliographystyle{jplain}

\usepackage{url}

\usepackage{graphicx}
\usepackage{gnuplot-lua-tikz}
\usepackage{pgfplots}
\usepackage{tikz}
\usepackage{amsmath,amsfonts,amssymb}
\usepackage{bm}
\usepackage{siunitx}

\makeatletter
\def\fgcaption{\def\@captype{figure}\caption}
\makeatother
\newcommand{\setsections}[3]{
\setcounter{section}{#1}
\setcounter{subsection}{#2}
\setcounter{subsubsection}{#3}
}
\newcommand{\mfig}[3][width=15cm]{
\begin{center}
\includegraphics[#1]{#2}
\fgcaption{#3 \label{fig:#2}}
\end{center}
}
\newcommand{\gnu}[2]{
\begin{figure}[hptb]
\begin{center}
\input{#2}
\caption{#1}
\label{fig:#2}
\end{center}
\end{figure}
}

\begin{document}
\title{相対性理論 レポートNo.6}
\author{佐々木良輔}
\date{}
\maketitle
\subsubsection*{(1)}
荷電粒子の受ける力は
\begin{align}
  \vec{F}=q\vec{E}+q\vec{v}\times\vec{B}
\end{align}
であった.慣性系$S'$において速度の空間成分は$0$であった.したがって4元力の空間成分は
\begin{align}
  F'^i=eE'^i
\end{align}
となる.また時間成分は定義から
\begin{align}
  F'^0=0
\end{align}
である.
\subsubsection*{(2)}
$f'_{\mu\nu}$は
\begin{align}
  f'_{\mu\nu}=\left(\begin{array}{cccc}
    0     & -E_x'/c  & -E_y'/c  & -E_z'/c\\
    E_x'/c & 0       & B_z'     & -B_y'\\
    E_y'/c & -B_z'    & 0       & B_x'\\
    E_z'/c & B_y'     & -B_x'    & 0\\
  \end{array}\right)
\end{align}
なので$f'^{\mu\nu}=f'_{\rho\sigma}\eta'^{\rho\mu}\eta'^{\sigma\nu}$は
\begin{align}
  f'^{\mu\nu}=\left(\begin{array}{cccc}
    0     & E_x'/c   & E_y'/c   & E_z'/c\\
    -E_x'/c& 0       & B_z'     & -B_y'\\
    -E_y'/c& -B_z'    & 0       & B_x'\\
    -E_z'/c& B_y'     & -B_x'    & 0\\
  \end{array}\right)
\end{align}
したがって(2)式, (3)式は
\begin{align}
  F'^\mu=-ecf'^{\mu0}=ecf'^{0\mu}
\end{align}
である.
\subsubsection*{(3)}
$S\rightarrow S'$のLorentz変換係数を$a^\mu_{\ \nu}$,逆変換を$b^\mu_{\ \nu}$とする.
このとき$\delta^\mu_{\ \nu}=a^\mu_{\ \rho}b^{\rho}_{\ \nu}$である.以上より(6)式は
\begin{align}
  \begin{split}
    F^\mu&=b^\mu_{\ \nu}F'^{\nu}\\
    &=ecb^\mu_{\ \nu}a^0_{\ \rho}a^\nu_{\ \sigma}f^{\rho\sigma}\\
    &=ec\delta^\mu_{\ \sigma}a^0_{\ \rho}f^{\rho\sigma}\\
    &=eca^0_{\ \rho}f^{\rho\mu}
  \end{split}
\end{align}
ここで$u^\mu=ca_0^{\ \mu}$だった.
$a_0^{\ \mu}=-a^0_{\ \mu}$より
\begin{align}
  u_\mu=-ca^0_{\ \mu}
\end{align}
なので
\begin{align}
  \begin{split}
    F^\mu&=-ef^{\rho\mu}u_\rho\\
    &=ef^{\mu\rho}u_\rho
  \end{split}
\end{align}
となる.
\subsubsection*{(4)}
(9)式の空間成分は
\begin{align}
  F^i=ef^{i0}u_0+ef^{i1}u_1+ef^{i2}u_2+ef^{i3}u_3
\end{align}
ここで$u_\mu=\eta_{\mu\nu}u^\nu$より
\begin{align}
  \begin{split}
    u_0&=-u^0=-\gamma c\\
    u_i&=u^i=\gamma v_i\ (i=1,2,3)
  \end{split}
\end{align}
であったので
\begin{align}
  F^i=e\gamma(E_i+\epsilon_{ijk}v_jB_k)
\end{align}
となる.
\subsubsection*{(5)}
古典的な陽子の受ける力は(1)式から
\begin{align}
  \vec{K}=e(\vec{E}+\vec{v}\times\vec{B})
\end{align}
であった.
\subsubsection*{(6)}
(12)式と(13)式から
\begin{align}
  F^i=\gamma K^i
\end{align}
となる.
\subsubsection*{(7)}
Lorentz力による仕事率$P$は
\begin{align}
  \begin{split}
    P=\vec{K}\cdot\vec{v}
  \end{split}
\end{align}
\subsubsection*{(8)}
$u_\mu F^\mu=0$及び(11)式から
\begin{align}
  -cF^0+\sum_{i=1}^3v_iF^i=0
\end{align}
ここで(14)式より
\begin{align}
  \vec{K}\cdot\vec{v}=\sum_{i=1}^3v_iK^i=\frac{1}{\gamma}\sum_{i=1}^3v_iF^i
\end{align}
以上から
\begin{align}
  \frac{1}{\gamma}\sum_{i=1}^3v_iF^i=\frac{c}{\gamma}F^0
\end{align}
したがって
\begin{align}
  P=\frac{c}{\gamma}F^0=cF^0\sqrt{1-\left(\frac{v}{c}\right)^2}
\end{align}
\end{document}