\documentclass[uplatex,a4j,11pt,dvipdfmx]{jsarticle}
\usepackage{listings,jvlisting}
\bibliographystyle{jplain}

\usepackage{url}

\usepackage{graphicx}
\usepackage{gnuplot-lua-tikz}
\usepackage{pgfplots}
\usepackage{tikz}
\usepackage{amsmath,amsfonts,amssymb}
\usepackage{bm}
\usepackage{siunitx}

\makeatletter
\def\fgcaption{\def\@captype{figure}\caption}
\makeatother
\newcommand{\setsections}[3]{
\setcounter{section}{#1}
\setcounter{subsection}{#2}
\setcounter{subsubsection}{#3}
}
\newcommand{\mfig}[3][width=15cm]{
\begin{center}
\includegraphics[#1]{#2}
\fgcaption{#3 \label{fig:#2}}
\end{center}
}
\newcommand{\gnu}[2]{
\begin{figure}[hptb]
\begin{center}
\input{#2}
\caption{#1}
\label{fig:#2}
\end{center}
\end{figure}
}

\begin{document}
\title{相対性理論 レポートNo.3}
\author{佐々木良輔}
\date{}
\maketitle
\subsection*{Q20.}
(5.3)式から
\begin{align}
  \begin{split}
    \eta_{00}=-1&=\sum_{\rho,\sigma}a^\rho_{\ 0}\eta_{\rho\sigma}a^\sigma_{\ 0}\\
    &=\sum_\sigma\left(a^0_{\ 0}\eta_{0\sigma}a^\sigma_{\ 0}+
    a^1_{\ 0}\eta_{1\sigma}a^\sigma_{\ 0}+
    a^2_{\ 0}\eta_{2\sigma}a^\sigma_{\ 0}+
    a^3_{\ 0}\eta_{3\sigma}a^\sigma_{\ 0}
    \right)\\
    &=-(a^0_{\ 0})^2+(a^1_{\ 0})^2+(a^2_{\ 0})^2+(a^3_{\ 0})^2
  \end{split} 
\end{align}
したがって
\begin{align}
  (a^0_{\ 0})^2=1+(a^1_{\ 0})^2+(a^2_{\ 0})^2+(a^3_{\ 0})^2\geq 1
\end{align}
以上から
\begin{align}
  |a^0_{\ 0}|\geq 1
\end{align}
\subsection*{Q23.}
Lorentz変換の条件は(5.3)から
\begin{align}
  {\bm Y}=\bm{A}^T\bm{YA}
\end{align}
であった.ここで
\begin{align}
  \bm{Y}^T\bm{Y}=\left(\begin{array}{cccc}
    -1 & 0 & 0 & 0\\
    0 & 1 & 0 & 0\\
    0 & 0 & 1 & 0\\
    0 & 0 & 0 & 1\\
  \end{array}\right)
  \left(\begin{array}{cccc}
    -1 & 0 & 0 & 0\\
    0 & 1 & 0 & 0\\
    0 & 0 & 1 & 0\\
    0 & 0 & 0 & 1\\
  \end{array}\right)
  ={\bm I}
\end{align}
から$Y$は直交行列である.したがって(4)を両辺転置すると
\begin{align}
  \begin{split}
    \bm{Y}^T&=({\bm A}^T)^T{\bm Y}^T{\bm A}^T\\
    \bm{Y}^{-1}&=\bm{A}\bm{Y}^{-1}\bm{A}^T
  \end{split}
\end{align}
これをテンソルで表記すると
\begin{align}
  \eta^{\mu\nu}=a^\mu_{\ \rho}\eta^{\rho\sigma}a^\nu_{\ \sigma}
\end{align}
\end{document}