\documentclass[uplatex,a4j,11pt,dvipdfmx]{jsarticle}
\usepackage{listings,jvlisting}
\bibliographystyle{jplain}

\usepackage{url}

\usepackage{graphicx}
\usepackage{gnuplot-lua-tikz}
\usepackage{pgfplots}
\usepackage{tikz}
\usepackage{amsmath,amsfonts,amssymb}
\usepackage{bm}
\usepackage{siunitx}
\usepackage[thicklines]{cancel}

\renewcommand{\CancelColor}{\color{red}}
\makeatletter
\def\fgcaption{\def\@captype{figure}\caption}
\makeatother
\newcommand{\setsections}[3]{
\setcounter{section}{#1}
\setcounter{subsection}{#2}
\setcounter{subsubsection}{#3}
}
\newcommand{\mfig}[3][width=15cm]{
\begin{center}
\includegraphics[#1]{#2}
\fgcaption{#3 \label{fig:#2}}
\end{center}
}
\newcommand{\gnu}[2]{
\begin{figure}[hptb]
\begin{center}
\input{#2}
\caption{#1}
\label{fig:#2}
\end{center}
\end{figure}
}

\begin{document}
\title{相対性理論 レポートNo.10}
\author{佐々木良輔}
\date{}
\maketitle
\subsection*{Q74.}
$A^\mu B_\mu$は縮約されてスカラーになっている.
したがってスカラーに対する共変微分の要請から
\begin{align}
  \begin{split}
    \nabla_\nu(A^\mu B_\mu)&=\partial_\nu(A^\mu B_\mu)
  \end{split}
\end{align}
また左辺に対して縮約と微分が可換であることとLeibniz則から
\begin{align}
  \begin{split}
    \nabla_\nu(A^\mu B_\mu)&=\nabla_\nu A^\mu B_\mu\\
    &=(\nabla_\nu A^\mu)B_\mu+A^\mu(\nabla_\nu B_\mu)
  \end{split}
\end{align}
(1), (2)式から
\begin{align}
  \partial_\nu(A^\mu B_\mu)=(\nabla_\nu A^\mu)B_\mu+A^\mu(\nabla_\nu B_\mu)
\end{align}
ここで反変ベクトルに関する共変微分
\begin{align}
  \nabla_\nu A^\mu=\partial_\nu A^\mu-X^\mu_{\ \nu\lambda}A^\lambda
\end{align}
を用いると
\begin{align}
  \begin{split}
    \partial_\nu(A^\mu B_\mu)&=(\partial_\nu A^\mu-X^\mu_{\ \nu\lambda}A^\lambda)B_\mu+A^\mu(\nabla_\nu B_\mu)\\
    A^\mu(\nabla_\nu B_\mu)&=\partial_\nu(A^\mu B_\mu)-(\partial_\nu A^\mu)B_\mu+X^\mu_{\ \nu\lambda}A^\lambda B_\mu\\
    A^\mu(\nabla_\nu B_\mu)&=A^\mu\partial_\nu B_\mu+X^\mu_{\ \nu\lambda}A^\lambda B_\mu
  \end{split}
\end{align}
この両辺が任意の$A^\mu$について等しくあるべきなので
\begin{align}
  \nabla_\nu B_\mu=\partial_\nu B_\mu+X^\lambda_{\ \nu\mu}B_\lambda
\end{align}
となる.
\end{document}