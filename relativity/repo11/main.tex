\documentclass[uplatex,a4j,11pt,dvipdfmx]{jsarticle}
\usepackage{listings,jvlisting}
\bibliographystyle{jplain}

\usepackage{url}

\usepackage{graphicx}
\usepackage{gnuplot-lua-tikz}
\usepackage{pgfplots}
\usepackage{tikz}
\usepackage{amsmath,amsfonts,amssymb}
\usepackage{bm}
\usepackage{siunitx}
\usepackage[thicklines]{cancel}

\renewcommand{\CancelColor}{\color{red}}
\makeatletter
\def\fgcaption{\def\@captype{figure}\caption}
\makeatother
\newcommand{\setsections}[3]{
\setcounter{section}{#1}
\setcounter{subsection}{#2}
\setcounter{subsubsection}{#3}
}
\newcommand{\mfig}[3][width=15cm]{
\begin{center}
\includegraphics[#1]{#2}
\fgcaption{#3 \label{fig:#2}}
\end{center}
}
\newcommand{\gnu}[2]{
\begin{figure}[hptb]
\begin{center}
\input{#2}
\caption{#1}
\label{fig:#2}
\end{center}
\end{figure}
}

\begin{document}
\title{相対性理論 レポートNo.11}
\author{佐々木良輔}
\date{}
\maketitle
\subsection*{Q80.}
共変微分は計量性から
\begin{align}
  \begin{split}
    \nabla_\lambda g_{\mu\nu}&=\partial_\lambda g_{\mu\nu}+X^\alpha_{\ \lambda\mu}g_{\alpha\nu}+X^\beta_{\ \lambda\nu}g_{\mu\beta}\\
    &=\partial_\lambda g_{\mu\nu}+X_{\nu\lambda\mu}+X_{\mu\lambda\nu}=0\\
    \therefore\ \partial_\lambda g_{\mu\nu}&=-X_{\nu\lambda\mu}-X_{\mu\lambda\nu}
  \end{split}
\end{align}
同様に
\begin{align}
  \begin{split}
    \nabla_\mu g_{\nu\lambda}&=\partial_\mu g_{\nu\lambda}+X^\alpha_{\ \mu\nu}g_{\alpha\lambda}+X^\beta_{\ \mu\lambda}g_{\nu\beta}\\
    &=\partial_\mu g_{\nu\lambda}+X_{\lambda\mu\nu}+X_{\nu\mu\lambda}=0\\
    \therefore\ \partial_\mu g_{\nu\lambda}&=-X_{\lambda\mu\nu}-X_{\nu\mu\lambda}
  \end{split}
\end{align}
\begin{align}
  \begin{split}
    \nabla_\nu g_{\lambda\mu}&=\partial_\nu g_{\lambda\mu}+X^\alpha_{\ \nu\lambda}g_{\alpha\mu}+X^\beta_{\ \nu\mu}g_{\lambda\beta}\\
    &=\partial_\nu g_{\lambda\mu}+X_{\mu\nu\lambda}+X_{\lambda\nu\mu}=0\\
    \therefore\ \partial_\nu g_{\lambda\mu}&=-X_{\mu\nu\lambda}-X_{\lambda\nu\mu}=0
  \end{split}
\end{align}
ここで$X_{\alpha\beta\gamma}=X_{\alpha\gamma\beta}$を用いて$(2)+(3)-(1)$を計算すると
\begin{align}
  \begin{split}
    &\partial_\mu g_{\nu\lambda}+\partial_\nu g_{\lambda\mu}-\partial_\lambda g_{\mu\nu}\\
    =&X_{\nu\lambda\mu}+X_{\mu\lambda\nu}-X_{\lambda\mu\nu}-X_{\nu\mu\lambda}-X_{\mu\nu\lambda}-X_{\lambda\nu\mu}\\
    =&-2X_{\lambda\mu\nu}\\
    \therefore\ X_{\lambda\mu\nu}=&X^{\alpha}_{\ \mu\nu}g_{\alpha\lambda}=-\frac{1}{2}(\partial_\mu g_{\nu\lambda}+\partial_\nu g_{\lambda\mu}-\partial_\lambda g_{\mu\nu})
  \end{split}
\end{align}
(4)の両辺に$g^{\lambda\rho}$を掛けると$g_{\alpha\lambda}g^{\lambda\rho}=\delta^\rho_{\ \alpha}$より
\begin{align}
  \begin{split}
    X^{\rho}_{\ \mu\nu}&=-\frac{1}{2}g^{\lambda\rho}(\partial_\mu g_{\nu\lambda}+\partial_\nu g_{\lambda\mu}-\partial_\lambda g_{\mu\nu})
  \end{split}
\end{align}
添字を$\rho\rightarrow\mu$, $\mu\rightarrow\rho$, $\nu\rightarrow\sigma$, $\lambda\rightarrow\nu$と置き直せば
\begin{align}
  \begin{split}
    X^\mu_{\ \rho\sigma}&=-\frac{1}{2}g^{\nu\mu}(\partial_\sigma g_{\nu\rho}+\partial_\rho g_{\sigma\nu}-\partial_\nu g_{\rho\sigma})\\
    &=-\Gamma^\mu_{\ \rho\sigma}
  \end{split}
\end{align}
を得る.
\end{document}