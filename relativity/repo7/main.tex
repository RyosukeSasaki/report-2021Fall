\documentclass[uplatex,a4j,11pt,dvipdfmx]{jsarticle}
\usepackage{listings,jvlisting}
\bibliographystyle{jplain}

\usepackage{url}

\usepackage{graphicx}
\usepackage{gnuplot-lua-tikz}
\usepackage{pgfplots}
\usepackage{tikz}
\usepackage{amsmath,amsfonts,amssymb}
\usepackage{bm}
\usepackage{siunitx}

\makeatletter
\def\fgcaption{\def\@captype{figure}\caption}
\makeatother
\newcommand{\setsections}[3]{
\setcounter{section}{#1}
\setcounter{subsection}{#2}
\setcounter{subsubsection}{#3}
}
\newcommand{\mfig}[3][width=15cm]{
\begin{center}
\includegraphics[#1]{#2}
\fgcaption{#3 \label{fig:#2}}
\end{center}
}
\newcommand{\gnu}[2]{
\begin{figure}[hptb]
\begin{center}
\input{#2}
\caption{#1}
\label{fig:#2}
\end{center}
\end{figure}
}

\begin{document}
\title{相対性理論 レポートNo.7}
\author{佐々木良輔}
\date{}
\maketitle
\subsubsection*{Q59.}
\begin{align}
  W=c\sqrt{m^2c^2+|p|^2}=mc^2\sqrt{1+\frac{|p|^2}{m^2c^2}}
\end{align}
ここで$|p|/mc\ll 1$であるならば2次まで展開すると
\begin{align}
  W\simeq mc^2(1+\frac{|p|^2}{2m^2c^2})=mc^2+\frac{|p|^2}{2m}
\end{align}
となる.
\subsubsection*{Q62.}
Lagrangian $L_0$は
\begin{align}
  L_0=-\sum_i m_ic^2\sqrt{1-\left|\frac{v(i)}{c}\right|^2}
\end{align}
であったのでHamiltonian $H_0$は
\begin{align}
  \begin{split}
    H_0&=\sum_ip_k(i)v^k(i)+\sum_im_ic^2\sqrt{1-\left|\frac{v(i)}{c}\right|^2}\\
    &=\sum_ip_k(i)v^k(i)+m_ic^2\sqrt{1-\left|\frac{v(i)}{c}\right|^2}
  \end{split}
\end{align}
となり,各$i$について計算すれば十分である.
$i$に関するHamiltonianを$H_0(i)$とすると
\begin{align}
  \begin{split}
    H_0(i)&=p_kv^k+mc^2\sqrt{1-\left|\frac{v}{c}\right|^2}\\
    &=c\left(\frac{p^2}{mc}+mc\right)\sqrt{1-\left|\frac{v}{c}\right|^2}
  \end{split}
\end{align}
ここでQ61.の結果を用いると
\begin{align}
  \begin{split}
    H_0(i)&=c\left(\frac{p^2}{mc}+mc\right)\frac{mc}{\sqrt{p^2+m^2c^2}}\\
    &=c\sqrt{p^2+m^2c^2}
  \end{split}
\end{align}
以上から
\begin{align}
  H_0=\sum_iH_0(i)=\sum_ic\sqrt{p^2+m^2c^2}
\end{align}
\end{document}