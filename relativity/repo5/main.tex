\documentclass[uplatex,a4j,11pt,dvipdfmx]{jsarticle}
\usepackage{listings,jvlisting}
\bibliographystyle{jplain}

\usepackage{url}

\usepackage{graphicx}
\usepackage{gnuplot-lua-tikz}
\usepackage{pgfplots}
\usepackage{tikz}
\usepackage{amsmath,amsfonts,amssymb}
\usepackage{bm}
\usepackage{siunitx}

\makeatletter
\def\fgcaption{\def\@captype{figure}\caption}
\makeatother
\newcommand{\setsections}[3]{
\setcounter{section}{#1}
\setcounter{subsection}{#2}
\setcounter{subsubsection}{#3}
}
\newcommand{\mfig}[3][width=15cm]{
\begin{center}
\includegraphics[#1]{#2}
\fgcaption{#3 \label{fig:#2}}
\end{center}
}
\newcommand{\gnu}[2]{
\begin{figure}[hptb]
\begin{center}
\input{#2}
\caption{#1}
\label{fig:#2}
\end{center}
\end{figure}
}

\begin{document}
\title{相対性理論 レポートNo.5}
\author{佐々木良輔}
\date{}
\maketitle
\subsubsection*{Q44.}
$f_{\mu\nu}$は反対称テンソルなので$f_{\mu\nu}=-f_{\nu\mu}$, $f_{\mu\mu}=0$である.
したがって$\mu<\nu$だけ考えれば良い.
まず$\mu=0$かつ$\nu\neq 0$のとき
\begin{align}
  \begin{split}
    f_{0\nu}&=\partial_0A_\nu-\partial_\nu A_0\\
    &=\frac{1}{c}\frac{\partial A_\nu}{\partial t}-\frac{\partial }{\partial x_\nu}\eta_{0\rho}A^\rho\\
    &=\frac{1}{c}\left(\eta_{\nu\rho}\frac{\partial A^\rho}{\partial t}+\frac{\partial \phi}{\partial x_\nu}\right)\\
    &=\frac{1}{c}\left(\frac{\partial A^\nu}{\partial t}+\frac{\partial \phi}{\partial x_\nu}\right)
  \end{split}
\end{align}
ここで$E=-\frac{\partial {\bm A}}{\partial t}-{\rm grad}\phi$だったので
\begin{align}
  \begin{split}
    f_{0\nu}&=\frac{1}{c}\left(\frac{\partial {\bm A}}{\partial t}+{\rm grad}\phi\right)_\nu\\
    &=-\frac{E_\nu}{c}
  \end{split}
\end{align}
$\mu=1$, $\nu=2$のとき
\begin{align}
  \begin{split}
    f_{12}&=\partial_1A_2-\partial_2A_1\\
    &=\frac{\partial}{\partial x}\eta_{2\rho}A^\rho-\frac{\partial}{\partial y}\eta_{1\sigma}A^\sigma\\
    &=\frac{\partial}{\partial x}({\bm A})_y-\frac{\partial}{\partial y}({\bm A})_x\\
    &=({\rm rot}{\bm A})_z=B_z
  \end{split}
\end{align}
同様に$\mu=1$, $\nu=3$のとき
\begin{align}
  \begin{split}
    f_{13}&=\partial_1A_3-\partial_3A_1\\
    &=-\left(\frac{\partial}{\partial z}({\bm A})_x-\frac{\partial}{\partial x}({\bm A})_z\right)\\
    &=-({\rm rot}{\bm A})_y=-B_y
  \end{split}
\end{align}
$\mu=2$, $\nu=3$のとき
\begin{align}
  \begin{split}
    f_{23}&=\partial_2A_3-\partial_3A_2\\
    &=\frac{\partial}{\partial y}({\bm A})_z-\frac{\partial}{\partial z}({\bm A})_y\\
    &=B_x
  \end{split}
\end{align}
となる.
\subsubsection*{Q52.}
%\begin{align}
%  \begin{split}
%    &E_x=E_x',\ E_y=\gamma(E_y'+\beta cB_z'),\ E_z=\gamma(E_z'-\beta cB_y')\\
%    &B_x=B_x',\ B_y=\gamma(B_y'-\frac{\beta}{c}E_z'),\ B_z=\gamma(B_z'+\frac{\beta}{c}E_y')
%  \end{split}
%\end{align}
$S'$系の原点に固定された電荷$e$の作る電場$\bm E'$は
\begin{align}
  {\bm E}'=\frac{e}{4\pi\varepsilon_0r'^2}({\bm e}_x',{\bm e}_y',{\bm e}_z')
\end{align}
である.また$f'^{\mu\nu}$は$\alpha=E'/c={e}/{4\pi\varepsilon_0cr'^2}$を用いて
\begin{align}
  f'^{\mu\nu}\overset{\text{r}}{=}\left(\begin{array}{cccc}
    0&\alpha&\alpha&\alpha\\
    -\alpha&0&0&0\\
    -\alpha&0&0&0\\
    -\alpha&0&0&0
  \end{array}\right)={\bm F}'
\end{align}
である.
ここで$S$系が$S'$系に対して$-x$方向に速度$v$で移動する場合の特殊Lorentz変換において変換は
\begin{align}
  a^\mu_{\ \nu}\overset{\text{r}}{=}\left(\begin{array}{cc}
    \gamma\left(\begin{array}{cc}
      1&\beta\\
      \beta&1
    \end{array}\right)&{\bm 0}\\
    {\bm 0}&{\bm 1}
  \end{array}\right)={\bm A}
\end{align}
で表される.このとき$S$系では$f^{\mu\nu}=a^\mu_{\ \rho}a^\nu_{\ \sigma}f^{\rho\sigma}\overset{\text{r}}{=}{\bm F}={\bm A}^T{\bm F}'{\bm A}$なので
\begin{align}
  \begin{split}
    &f^{\mu\nu}\overset{\text{r}}{=}{\bm A}^T{\bm F}'{\bm A}\\
    =&\left(\begin{array}{cc}
      \gamma\left(\begin{array}{cc}
        1&\beta\\
        \beta&1
      \end{array}\right)&{\bm 0}\\
      {\bm 0}&{\bm 1}
    \end{array}\right)
    \left(\begin{array}{cccc}
      0&\alpha&\alpha&\alpha\\
      -\alpha&0&0&0\\
      -\alpha&0&0&0\\
      -\alpha&0&0&0
    \end{array}\right)
    \left(\begin{array}{cc}
      \gamma\left(\begin{array}{cc}
        1&\beta\\
        \beta&1
      \end{array}\right)&{\bm 0}\\
      {\bm 0}&{\bm 1}
    \end{array}\right)\\
    =&
    \left(\begin{array}{cccc}
      -\alpha\beta\gamma&\alpha\gamma&\alpha\gamma&\alpha\gamma\\
      -\alpha\gamma&\alpha\beta\gamma&\alpha\beta\gamma&\alpha\beta\gamma\\
      -a&0&0&0\\
      -a&0&0&0
    \end{array}\right)
    \left(\begin{array}{cc}
      \gamma\left(\begin{array}{cc}
        1&\beta\\
        \beta&1
      \end{array}\right)&{\bm 0}\\
      {\bm 0}&{\bm 1}
    \end{array}\right)\\
    &=\left(\begin{array}{cccc}
      0&\alpha\gamma^2(1-\beta^2)&\alpha\gamma&\alpha\gamma\\
      -\alpha\gamma^2(1-\beta^2)&0&\alpha\gamma\beta&\alpha\gamma\beta\\
      -\alpha\gamma&-\alpha\beta\gamma&0&0\\
      -\alpha\gamma&-\alpha\beta\gamma&0&0\\
    \end{array}\right)\\
    &=\left(\begin{array}{cccc}
      0&\alpha&\alpha\gamma&\alpha\gamma\\
      -\alpha&0&\alpha\gamma\beta&\alpha\gamma\beta\\
      -\alpha\gamma&-\alpha\beta\gamma&0&0\\
      -\alpha\gamma&-\alpha\beta\gamma&0&0\\
    \end{array}\right)
  \end{split}
\end{align}
よって
\begin{align}
  {\bm E}=\left(\begin{array}{c}
    E'\\\gamma E'\\\gamma E'
  \end{array}\right),\ 
  {\bm B}=\left(\begin{array}{c}
    0\\-\frac{E'}{c}\gamma\beta\\\frac{E'}{c}\gamma\beta    
  \end{array}\right)
\end{align}
なので
\begin{align}
  {\bm B}=\frac{1}{c^2}{\bm E}\times{\bm v}
\end{align}
\end{document}
