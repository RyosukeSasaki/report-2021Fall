\documentclass[uplatex,a4j,11pt,dvipdfmx]{jsarticle}
\usepackage{listings,jvlisting}
\bibliographystyle{jplain}

\usepackage{url}

\usepackage{graphicx}
\usepackage{gnuplot-lua-tikz}
\usepackage{pgfplots}
\usepackage{tikz}
\usepackage{amsmath,amsfonts,amssymb}
\usepackage{bm}
\usepackage{siunitx}

\makeatletter
\def\fgcaption{\def\@captype{figure}\caption}
\makeatother
\newcommand{\setsections}[3]{
\setcounter{section}{#1}
\setcounter{subsection}{#2}
\setcounter{subsubsection}{#3}
}
\newcommand{\mfig}[3][width=15cm]{
\begin{center}
\includegraphics[#1]{#2}
\fgcaption{#3 \label{fig:#2}}
\end{center}
}
\newcommand{\gnu}[2]{
\begin{figure}[hptb]
\begin{center}
\input{#2}
\caption{#1}
\label{fig:#2}
\end{center}
\end{figure}
}

\begin{document}
\title{相対性理論 レポートNo.4}
\author{佐々木良輔}
\date{}
\maketitle
\subsubsection*{Q27.(1)}
示すべき式は$\eta_{\mu\nu}$が座標系に依らないことから
\begin{align}
  \eta_{\mu\nu}'=\eta_{\mu\nu}=\eta_{\rho\sigma}\frac{\partial x^{\rho}}{\partial x'^{\mu}}\frac{\partial x^{\sigma}}{\partial x'^{\nu}}
\end{align}
である.
また$(5.1)'''$から
\begin{align}
  \frac{\partial x^{\rho}}{\partial x'^{\mu}}=a_\mu^{\ \rho}
\end{align}
なので(1)は
\begin{align}
  \eta_{\mu\nu}=\eta_{\rho\sigma}a_\mu^{\ \rho}a_\nu^{\ \sigma}
\end{align}
である.また
\begin{align}
  a^{\gamma}_{\ \rho}a_\mu^{\ \rho}=\delta^{\gamma}_{\ \mu}
\end{align}
であったことから(3)式の両辺に$a^{\gamma}_{\ \rho}a^{\kappa}_{\ \sigma}$を掛けると
\begin{align}
  \eta_{\mu\nu}a^{\gamma}_{\ \rho}a^{\kappa}_{\ \sigma}=\eta_{\rho\sigma}\delta^{\gamma}_{\ \mu}\delta^{\kappa}_{\ \nu}
\end{align}
となる.したがって
\begin{align}
  \eta_{\mu\nu}a^{\mu}_{\ \rho}a^{\nu}_{\ \sigma}=\eta_{\rho\sigma}
\end{align}
(5.3)式からLorentz変換に対して(6)は成り立つ.
したがって(1)は成り立ち,
$\eta_{\mu\nu}$は共変ベクトルであることが示された.
\subsubsection*{Q33.}
$A^\mu$をLorentz変換すると
\begin{align}
  A'^\mu=a^\mu_{\ \rho}A^\rho
\end{align}
であった.
したがってLorentz変換した先でのノルムは
\begin{align}
  |A'^2|=\eta_{\mu\nu}'a^\mu_{\ \rho}a^\nu_{\ \sigma}A^\rho A^\sigma
\end{align}
である.ここで$\eta_{\mu\nu}'a^\mu_{\ \rho}a^\nu_{\ \sigma}=\eta_{\rho\sigma}$なので
\begin{align}
  |A'^2|=\eta_{\rho\sigma}A^\rho A^\sigma=|A^2|
\end{align}
となり,ノルムは保存した.
\subsubsection*{Q41.}
定義から
\begin{align}
  *f^{\mu\nu}=\frac{1}{2!}E^{\mu\nu\rho\sigma}f_{\rho\sigma}
\end{align}
したがって完全反対称テンソルの性質から$\mu=\nu$のとき
\begin{align*}
  *f^{\mu\mu}=\frac{1}{2}E^{\mu\mu\rho\sigma}f_{\rho\sigma}=0
\end{align*}
同様に
\begin{align*}
  *f^{01}&=\frac{1}{2}E^{01\rho\sigma}f_{\rho\sigma}\\
  &=\frac{1}{2}\left(E^{0123}f_{23}+E^{0132}f_{32}\right)\\
  &=f_{23}
\end{align*}
\begin{align*}
  *f^{02}&=\frac{1}{2}E^{02\rho\sigma}f_{\rho\sigma}\\
  &=\frac{1}{2}\left(E^{0213}f_{13}+E^{0231}f_{31}\right)\\
  &=f_{31}
\end{align*}
\begin{align*}
  *f^{03}&=\frac{1}{2}E^{03\rho\sigma}f_{\rho\sigma}\\
  &=\frac{1}{2}\left(E^{0312}f_{12}+E^{0321}f_{21}\right)\\
  &=f_{12}
\end{align*}
\begin{align*}
  *f^{10}=f_{32}
\end{align*}
\begin{align*}
  *f^{12}&=\frac{1}{2}E^{12\rho\sigma}f_{\rho\sigma}\\
  &=\frac{1}{2}\left(E^{1203}f_{03}+E^{1230}f_{30}\right)\\
  &=f_{03}
\end{align*}
\begin{align*}
  *f^{13}&=\frac{1}{2}E^{13\rho\sigma}f_{\rho\sigma}\\
  &=\frac{1}{2}\left(E^{1302}f_{02}+E^{1320}f_{20}\right)\\
  &=f_{20}
\end{align*}
\begin{align*}
  *f^{20}=f_{13}
\end{align*}
\begin{align*}
  *f^{21}=f_{30}
\end{align*}
\begin{align*}
  *f^{23}&=\frac{1}{2}E^{23\rho\sigma}f_{\rho\sigma}\\
  &=\frac{1}{2}\left(E^{2301}f_{01}+E^{2310}f_{10}\right)\\
  &=f_{01}
\end{align*}
\begin{align*}
  *f^{30}=f_{21}
\end{align*}
\begin{align*}
  *f^{31}=f_{02}
\end{align*}
\begin{align*}
  *f^{32}=f_{10}
\end{align*}
\end{document}