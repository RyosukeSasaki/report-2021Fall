\documentclass[uplatex,a4j,11pt,dvipdfmx]{jsarticle}
\usepackage{listings,jvlisting}
\bibliographystyle{jplain}

\usepackage{url}

\usepackage{graphicx}
\usepackage{gnuplot-lua-tikz}
\usepackage{pgfplots}
\usepackage{tikz}
\usepackage{amsmath,amsfonts,amssymb}
\usepackage{bm}
\usepackage{siunitx}
\usepackage[thicklines]{cancel}

\renewcommand{\CancelColor}{\color{red}}
\makeatletter
\def\fgcaption{\def\@captype{figure}\caption}
\makeatother
\newcommand{\setsections}[3]{
\setcounter{section}{#1}
\setcounter{subsection}{#2}
\setcounter{subsubsection}{#3}
}
\newcommand{\mfig}[3][width=15cm]{
\begin{center}
\includegraphics[#1]{#2}
\fgcaption{#3 \label{fig:#2}}
\end{center}
}
\newcommand{\gnu}[2]{
\begin{figure}[hptb]
\begin{center}
\input{#2}
\caption{#1}
\label{fig:#2}
\end{center}
\end{figure}
}

\begin{document}
\title{相対性理論 レポートNo.9}
\author{佐々木良輔}
\date{}
\maketitle
\subsection*{Q70.(3)}
等価原理から局所Lorentz系での質点の運動方程式は
\begin{align}
  \frac{d^2X^\mu}{d\tau^2}=0
\end{align}
であった.連鎖律から
\begin{align}
  \frac{d}{d\tau}X^\mu=\frac{dx^\rho}{d\tau}\frac{\partial X^\mu}{\partial x^\rho}
\end{align}
同様に
\begin{align}
  \begin{split}
    0=\frac{d^2X^\mu}{d\tau^2}&=\frac{d}{d\tau}\left(\frac{dx^\rho}{d\tau}\frac{\partial X^\mu}{\partial x^\rho}\right)\\
    &=\frac{d^2 x^\rho}{d\tau^2}\frac{\partial X^\mu}{\partial x^\rho}
    +\frac{dx^\nu}{d\tau}\left(\frac{d}{d\tau}\frac{\partial X^\mu}{\partial x^\nu}\right)
  \end{split}
\end{align}
ここで連鎖律から
\begin{align}
  \frac{d}{d\tau}=\frac{dx^\sigma}{d\tau}\frac{\partial}{\partial x^\sigma}
\end{align}
より
\begin{align}
  \begin{split}
    0=\frac{d^2X^\mu}{d\tau^2}&=\frac{d^2 x^\rho}{d\tau^2}\frac{\partial X^\mu}{\partial x^\rho}
    +\frac{dx^\nu}{d\tau}\frac{dx^\sigma}{d\tau}\frac{\partial X^\mu}{\partial x^\nu\partial x^\sigma}\\
    &=\ddot{x}^\rho\frac{\partial X^\mu}{\partial x^\rho}+\dot{x}^\nu\dot{x}^\sigma\frac{\partial X^\mu}{\partial x^\nu\partial x^\sigma}
  \end{split}
\end{align}
ここで両辺に$\partial x^\rho/\partial X^\mu$を掛けると
\begin{align}
  0=\ddot{x^\rho}+\frac{\partial x^\rho}{\partial X^\mu}\dot{x}^\nu\dot{x}^\sigma\frac{\partial X^\mu}{\partial x^\nu\partial x^\sigma}
\end{align}
ここで$\rho\rightarrow\mu$, $\mu\rightarrow\nu$, $\nu\rightarrow\rho$と置き換えれば
\begin{align}
  0=\ddot{x}^\mu+\frac{\partial x^\mu}{\partial X^\nu}\frac{\partial X^\nu}{\partial x^\rho\partial x^\sigma}\dot{x}^\rho\dot{x}^\sigma
\end{align}
を得る.
\subsection*{Q71.(2)}
$\partial_\sigma g_{\nu\rho}$は
\begin{align}
  \begin{split}
    \frac{\partial}{\partial x_\sigma}g_{\nu\rho}&=\frac{\partial}{\partial x_\sigma}\eta_{\alpha\beta}\frac{\partial X^{\alpha}}{\partial x^\nu}\frac{\partial X^{\beta}}{\partial x^\rho}\\
    &=\eta_{\alpha\beta}\left(\frac{\partial X^\alpha}{\partial x^\sigma\partial x^\nu}\frac{\partial X^\beta}{\partial x^\rho}+\frac{\partial X^\alpha}{\partial x^\nu}\frac{\partial X^\beta}{\partial x^\sigma\partial x^\rho}\right)
  \end{split}
\end{align}
より
\begin{align}
  \begin{split}
    \partial_\sigma g_{\nu\rho}+\partial_\rho g_{\sigma\nu}+\partial_\nu g_{\rho\sigma}
    &=\eta_{\alpha\beta}\left(\cancel{\frac{\partial X^\alpha}{\partial x^\sigma\partial x^\nu}\frac{\partial X^\beta}{\partial x^\rho}} + \frac{\partial X^\alpha}{\partial x^\nu}\frac{\partial X^\beta}{\partial x^\sigma\partial x^\rho}\right)\\
    &\qquad+\eta_{\alpha\beta}\left(\frac{\partial X^\alpha}{\partial x^\rho\partial x^\sigma}\frac{\partial X^\beta}{\partial x^\nu} + \cancel{\frac{\partial X^\alpha}{\partial x^\sigma}\frac{\partial X^\beta}{\partial x^\rho\partial x^\nu}}\right)\\
    &\qquad+\eta_{\alpha\beta}\left(\cancel{\frac{\partial X^\alpha}{\partial x^\nu\partial x^\rho}\frac{\partial X^\beta}{\partial x^\sigma}} + \cancel{\frac{\partial X^\alpha}{\partial x^\rho}\frac{\partial X^\beta}{\partial x^\nu\partial x^\sigma}}\right)\\
    &=\eta_{\alpha\beta}\left(\frac{\partial X^\alpha}{\partial x^\nu}\frac{\partial X^\beta}{\partial x^\sigma\partial x^\rho} + \frac{\partial X^\alpha}{\partial x^\rho\partial x^\sigma}\frac{\partial X^\beta}{\partial x^\nu}\right)\\
    &=2\eta_{\alpha\beta}\frac{\partial X^\alpha}{\partial x^\nu}\frac{\partial X^\beta}{\partial x^\sigma\partial x^\rho}\\
    &=2\eta_{\alpha\beta}\frac{\partial X^\alpha}{\partial x^\nu}\frac{\partial X^\beta}{\partial x^\gamma}\frac{\partial x^\gamma}{\partial X^\beta}\frac{\partial X^\beta}{\partial x^\sigma\partial x^\rho}\\
    &=2g_{\nu\gamma}\frac{\partial x^\gamma}{\partial X^\beta}\frac{\partial X^\beta}{\partial x^\sigma\partial x^\rho}
  \end{split}
\end{align}
したがって
\begin{align}
  \begin{split}  
    \Gamma^{\mu}_{\ \rho\sigma}&=\frac{1}{2}g^{\mu\nu}\left(\partial_\sigma g_{\nu\rho}+\partial_\rho g_{\sigma\nu}+\partial_\nu g_{\rho\sigma}\right)\\
    &=g^{\mu\nu}g_{\nu\gamma}\frac{\partial x^\gamma}{\partial X^\beta}\frac{\partial X^\beta}{\partial x^\sigma\partial x^\rho}\\
    &=\delta^{\mu}_{\ \gamma}\frac{\partial x^\gamma}{\partial X^\beta}\frac{\partial X^\beta}{\partial x^\sigma\partial x^\rho}\\
    &=\frac{\partial x^\mu}{\partial X^\beta}\frac{\partial X^\beta}{\partial x^\sigma\partial x^\rho}\\
  \end{split}
\end{align}
となる.
\end{document}