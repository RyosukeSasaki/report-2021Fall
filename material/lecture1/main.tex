\documentclass[uplatex,a4j,11pt,dvipdfmx]{jsarticle}
\usepackage{listings,jvlisting}
\bibliographystyle{jplain}

\usepackage{url}

\usepackage{graphicx}
\usepackage{gnuplot-lua-tikz}
\usepackage{pgfplots}
\usepackage{tikz}
\usepackage{amsmath,amsfonts,amssymb}
\usepackage{bm}
\usepackage{siunitx}

\makeatletter
\def\fgcaption{\def\@captype{figure}\caption}
\makeatother
\newcommand{\setsections}[3]{
\setcounter{section}{#1}
\setcounter{subsection}{#2}
\setcounter{subsubsection}{#3}
}
\newcommand{\mfig}[3][width=15cm]{
\begin{center}
\includegraphics[#1]{#2}
\fgcaption{#3 \label{fig:#2}}
\end{center}
}
\newcommand{\gnu}[2]{
\begin{figure}[hptb]
\begin{center}
\input{#2}
\caption{#1}
\label{fig:#2}
\end{center}
\end{figure}
}
\renewcommand{\thesubsection}{問\arabic{subsection}}

\begin{document}
\title{物性物理学1 レポートNo.1}
\author{61908697 佐々木良輔}
\date{}
\maketitle
\setcounter{section}{1}
\subsection{}
\gnu{$V(x)$のグラフ}{q1.tex}
\subsection{}
$V_0=0$のときSchr\"{o}dinger方程式に$W_k(x)$を代入すると
\begin{align*}
  -\frac{\hbar^2}{2m}\frac{{\rm d}^2}{{\rm d}x^2}W_k(x)&=EW_k(x)\\
  \frac{\hbar^2k^2}{2m}W_k(x)&=EW_k(x)\\
  \therefore\ E&=\frac{\hbar^2k^2}{2m}
\end{align*}
\subsection{}
$V_0>0$のとき$HW_k(x)$は
\begin{align*}
  &HW_k(x)\\
  =&-\frac{\hbar^2}{2m}\frac{{\rm d}^2}{{\rm d}x^2}W_k(x)+V_0({\rm e}^{i{2\pi x}/{a}}+{\rm e}^{-i{2\pi x}/{a}})W_k(x)\\
\end{align*}
このとき第1項は
\begin{align*}
  (第1項)=\frac{\hbar^2k^2}{2m}W_k(x)
\end{align*}
また第2項は
\begin{align*}
  (第2項)&=V_0({\rm e}^{i{2\pi x}/{a}}+{\rm e}^{-i{2\pi x}/{a}})\frac{1}{\sqrt{L}}{\rm e}^{ikx}\\
  &=V_0\frac{1}{\sqrt{L}}({\rm e}^{i(k+2\pi/a)x}+{\rm e}^{i(k-2\pi/a)x})\\
  &=V_0\left(W_{k+h_1}(x)+W_{k+h_{-1}}(x)\right)
\end{align*}
より$HW_k(x)$は
\begin{align*}
  HW_k(x)=\frac{\hbar^2k^2}{2m}W_k(x)+V_0\left(W_{k+h_1}(x)+W_{k+h_{-1}}(x)\right)
\end{align*}
となる.したがってSchr\"{o}dinger方程式は
\begin{align*}
  \frac{\hbar^2k^2}{2m}W_k(x)+V_0\left(W_{k+h_1}(x)+W_{k+h_{-1}}(x)\right)=EW_k(x)
\end{align*}
となる.
$W_k(x)$と$W_{k+h_1}(x)$, $W_{k+h_{-1}}(x)$は直行するため,
$W_k(x)$は上のSchr\"{o}dinger方程式の固有関数ではない.
\subsection{}
前問と同様に
\begin{align*}
  &HW_{k+h_n}(x)\\
  =&-\frac{\hbar^2}{2m}\frac{{\rm d}^2}{{\rm d}x^2}W_{k+h_n}(x)+V_0({\rm e}^{i{2\pi x}/{a}}+{\rm e}^{-i{2\pi x}/{a}})W_{k+h_n}(x)\\
  =&\frac{\hbar^2(k+h_n)^2}{2m}W_{k+h_n}(x)+V_0\left(W_{k+h_{n+1}}(x)+W_{k+h_{n-1}}(x)\right)
\end{align*}
\subsection{}
$\psi_k(x)$をSchr\"{o}dinger方程式に代入すると,その左辺は
\begin{align*}
  H\psi_k(x)&=-\frac{\hbar^2}{2m}\frac{{\rm d}^2}{{\rm d}x^2}\psi_k(x)+V_0({\rm e}^{i{2\pi x}/{a}}+{\rm e}^{-i{2\pi x}/{a}})\psi_k(x)\\
\end{align*}
ここで右辺第1項は
\begin{align*}
  (第1項)&=-\frac{\hbar^2}{2m}\frac{{\rm d}^2}{{\rm d}x^2}\sum_nc_n(k)W_{k+h_n}(x)\\
  &=\frac{\hbar^2}{2m}\sum_nc_n(k)(k+h_n)^2W_{k+h_n}(x)
\end{align*}
また第2項は
\begin{align*}
  (第2項)&=V_0({\rm e}^{i{2\pi x}/{a}}+{\rm e}^{-i{2\pi x}/{a}})\sum_nc_n(k)W_{k+h_n}(x)\\
  &=V_0\sum_nc_n(k)\left(W_{k+h_{n+1}}(x)+W_{k+h_{n-1}}(x)\right)
\end{align*}
したがってSchr\"{o}dinger方程式は
\begin{align*}
  \frac{\hbar^2}{2m}\sum_nc_n(k)(k+h_n)^2W_{k+h_n}(x)+V_0\sum_nc_n(k)\left(W_{k+h_{n+1}}(x)+W_{k+h_{n-1}}(x)\right)=E\sum_nc_n(k)W_{k+h_n}(x)
\end{align*}
であり$W_{k+h_n}(x)$に関する項をまとめると
\begin{align*}
  \frac{\hbar^2}{2m}c_n(k)(k+h_n)^2+V_0\left(c_{n-1}(k)+c_{n+1}(k)\right)&=Ec_n(k)
\end{align*}
\subsection{}
前問の結果を用いると
\begin{align*}&\left(
  \begin{array}{ccccccc}
    \ddots\\
    &\ddots&&&&\text{\huge{0}}\\
    &V_0&{\hbar^2(k+h_{n-1})^2}/{2m}&V_0\\
    &&V_0&{\hbar^2(k+h_{n})^2}/{2m}&V_0\\
    &&&V_0&{\hbar^2(k+h_{n+1})^2}/{2m}&V_0\\
    &\text{\huge{0}}&&&&\ddots\\
    &&&&&&\ddots
  \end{array}\right)
  \left(
  \begin{array}{c}
    \vdots\\
    c_{n-2}(k)\\
    c_{n-1}(k)\\
    c_{n}(k)\\
    c_{n+1}(k)\\
    c_{n+2}(k)\\
    \vdots
  \end{array}
  \right)\\
  =&E
  \left(
    \begin{array}{c}
      \vdots\\
      c_{n-2}(k)\\
      c_{n-1}(k)\\
      c_{n}(k)\\
      c_{n+1}(k)\\
      c_{n+2}(k)\\
      \vdots
    \end{array}
  \right)
\end{align*}
\subsection{}
$k=-\pi/a$なので
\begin{align*}
  \psi_k(x)=W_{-\pi/a}(x)+W_{\pi/a}(x)
\end{align*}
したがって$H\psi_k(x)$は
\begin{align*}
  H\psi_k(x)&=-\frac{\hbar^2}{2m}\frac{{\rm d}^2}{{\rm d}x^2}\psi_k(x)+V(x)\psi_k(x)\\
  &=-\frac{\hbar^2}{2m}\left(\frac{\pi}{a}\right)^2\psi_k(x)+\frac{V_0}{\sqrt{L}}\left({\rm e}^{i\pi x/a}+{\rm e}^{-i\pi x/a}+{\rm e}^{i3\pi x/a}+{\rm e}^{-i3\pi x/a}\right)\\
  &=-\frac{\hbar^2}{2m}\left(\frac{\pi}{a}\right)^2\psi_k(x)+\frac{2V_0}{\sqrt{L}}\left(\cos\left(\frac{\pi x}{a}\right)+\cos\left(\frac{3\pi x}{a}\right)\right)
\end{align*}
\subsection{}
\begin{align*}
  \psi_k(x)&=W_{-\pi/a}(x)+W_{\pi/a}(x)\\
  &=\frac{2}{\sqrt{L}}\cos\left(\frac{\pi x}{a}\right)
\end{align*}
\gnu{$|\psi_k(x)|^2$のグラフ}{q2.tex}
\subsection{}
前問で得た$H\psi_k(x)$より$\cos(3\pi x/a)$の成分が十分小さければ$\psi_k(x)$は固有関数となりうる.
そのためには$V_0$が小さいか$L$が十分大きければ良いと考えられる.
\bibliography{ref.bib}
\end{document}