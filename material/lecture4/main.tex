\documentclass[uplatex,a4j,11pt,dvipdfmx]{jsarticle}
\usepackage{listings,jvlisting}
\bibliographystyle{jplain}

\usepackage{url}
\usepackage{braket}
\usepackage{graphicx}
\usepackage{gnuplot-lua-tikz}
\usepackage{pgfplots}
\usepackage{tikz}
\usepackage{amsmath,amsfonts,amssymb}
\usepackage{bm}
\usepackage{siunitx}

\makeatletter
\def\fgcaption{\def\@captype{figure}\caption}
\makeatother
\newcommand{\setsections}[3]{
\setcounter{section}{#1}
\setcounter{subsection}{#2}
\setcounter{subsubsection}{#3}
}
\newcommand{\mfig}[3][width=15cm]{
\begin{center}
\includegraphics[#1]{#2}
\fgcaption{#3 \label{fig:#2}}
\end{center}
}
\newcommand{\gnu}[2]{
\begin{figure}[hptb]
\begin{center}
\input{#2}
\caption{#1}
\label{fig:#2}
\end{center}
\end{figure}
}

\begin{document}
\title{物性物理学 No.4}
\author{61908697 佐々木良輔}
\date{}
\maketitle
\subsubsection*{問1}
\mfig[width=6cm]{fig/fig1.png}{$H_1$のグラフ}
\subsubsection*{問2}
$V=0$のとき$H_1=0$なのでSchr\"{o}dinger方程式は
\begin{align}
  \begin{split}
    -\frac{\hbar^2}{2m}\frac{{\rm d}^2}{{\rm d}x^2}\psi_{\pm}(x)&=-\frac{\hbar^2}{2m}\frac{{\rm d}^2}{{\rm d}x^2}\frac{1}{\sqrt{L}}{\rm e}^{\pm i(\pi/a)x}\\
    &=\frac{\hbar^2}{2m}\left(\frac{\pi}{a}\right)^2\frac{1}{\sqrt{L}}{\rm e}^{\pm i(\pi/a)x}\\
    &=E\psi_{\pm}(x)
  \end{split}
\end{align}
よって
\begin{align}
  E=\frac{\hbar^2}{2m}\left(\frac{\pi}{a}\right)^2
\end{align}
\newpage
\subsubsection*{問3}
\begin{align}
  \begin{split}
    0&=(H_0+H_1-E)(A\psi_-(x)+B\psi_+(x))\\
    &=\left(E_0-E+2V\left(1+\sin\frac{2\pi}{a}x\right)\right)(A\psi_-(x)+B\psi_+(x))\\
    &=\left(E_0-E+2V+\frac{V}{i}({\rm e}^{i(2\pi/a)x}-{\rm e}^{-i(2\pi/a)x})\right)(A\psi_-(x)+B\psi_+(x))\\
    &=(E_0-E+2V)(A\psi_-(x)+B\psi_+(x))\\
    &\qquad+\frac{V}{i\sqrt{L}}\left(A{\rm e}^{i(\pi/a)x}+B{\rm e}^{i(3\pi/a)x}-A{\rm e}^{-i(3\pi/a)x}-B{\rm e}^{-i(\pi/a)x}\right)
  \end{split}
\end{align}
ここで${\rm e}^{ik(\pi/a)x}/\sqrt{L}=\ket{k}$と表記する.このとき$\braket{k'|k}=\delta_{kk'}$である.
\begin{align}
  0&=(E_0-E+2V)(A\ket{-1}+B\ket{1})+\frac{V}{i}(A\ket{1}+B\ket{3}-A\ket{-3}-B\ket{-1})
\end{align}
両辺に$\bra{-1}$を掛けると
\begin{align}
  0=(E_0-E+2V)A-\frac{V}{i}B
\end{align}
両辺に$\bra{1}$を掛けると
\begin{align}
  0=(E_0-E+2V)B+\frac{V}{i}A
\end{align}
すなわち
\begin{align}
  \left(
    \begin{array}{cc}
      E_0-E+2V&-V/i\\
      V/i&E_0-E+2V
  \end{array}\right)\left(
    \begin{array}{c}
      A\\B
    \end{array}
  \right)={\bm 0}
\end{align}
これが非自明な解を持つ時
\begin{align}
  \left|
  \begin{array}{cc}
    E_0-E+2V&-V/i\\
    V/i&E_0-E+2V
  \end{array}
  \right|=0
\end{align}
\begin{align}
  \begin{split}
    E_1&=E_0+V\\
    E_2&=E_0+3V
  \end{split}
\end{align}
\subsubsection*{問4}
$E_1$に対応する固有ベクトルは
\begin{align}
  \left(\begin{array}{c}
    A\\B
  \end{array}\right)=
  \left(\begin{array}{c}
    1\\i
  \end{array}\right)
\end{align}
このとき固有関数は
\begin{align}
  \psi_1(x)=\frac{1}{\sqrt{L}}({\rm e}^{-i(\pi/a)x}+i{\rm e}^{i(\pi/a)x})
\end{align}
したがって
\begin{align}
  |\psi_1(x)|^2&=\frac{1}{L}({\rm e}^{-i(\pi/a)x}+i{\rm e}^{i(\pi/a)x})({\rm e}^{i(\pi/a)x}-i{\rm e}^{-i(\pi/a)x})\\
  &=2\left(1-\sin\frac{2\pi}{a}x\right)
\end{align}
同様に$E_2$に対応する固有ベクトルは
\begin{align}
  \left(\begin{array}{c}
    A\\B
  \end{array}\right)=
  \left(\begin{array}{c}
    1\\-i
  \end{array}\right)
\end{align}
このとき固有関数は
\begin{align}
  \psi_2(x)=\frac{1}{\sqrt{L}}({\rm e}^{-i(\pi/a)x}-i{\rm e}^{i(\pi/a)x})
\end{align}
したがって
\begin{align}
  |\psi_2(x)|^2&=2\left(1+\sin\frac{2\pi}{a}x\right)
\end{align}
である.
$|\psi(x)|^2$のグラフは下図のようになる.
\mfig[width=6cm]{fig/fig2.png}{$|\psi(x)|^2$のグラフ(青:$|\psi_1|^2$,赤:$|\psi_2|^2$)}
\subsubsection*{問5}
図1と図2から$|\psi_2|^2$と$H_1$は同位相,
$|\psi_1|^2$と$H_1$は逆位相とわかる.
すなわち$|\psi_1|^2$はエネルギーが高い場所に局在し,
$|\psi_2|^2$はエネルギーが低い場所に局在している.
これによって2つの準位にエネルギー差が生じていると言える.
\end{document}