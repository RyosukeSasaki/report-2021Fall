\documentclass[uplatex,a4j,11pt,dvipdfmx]{jsarticle}
\usepackage{listings,jvlisting}
\bibliographystyle{jplain}

\usepackage{url}

\usepackage{graphicx}
\usepackage{gnuplot-lua-tikz}
\usepackage{pgfplots}
\usepackage{tikz}
\usepackage{amsmath,amsfonts,amssymb}
\usepackage{bm}
\usepackage{siunitx}

\makeatletter
\def\fgcaption{\def\@captype{figure}\caption}
\makeatother
\newcommand{\setsections}[3]{
\setcounter{section}{#1}
\setcounter{subsection}{#2}
\setcounter{subsubsection}{#3}
}
\newcommand{\mfig}[3][width=15cm]{
\begin{center}
\includegraphics[#1]{#2}
\fgcaption{#3 \label{fig:#2}}
\end{center}
}
\newcommand{\gnu}[2]{
\begin{figure}[hptb]
\begin{center}
\input{#2}
\caption{#1}
\label{fig:#2}
\end{center}
\end{figure}
}

\begin{document}
\title{物性物理学 No.11}
\author{61908697 佐々木良輔}
\date{}
\maketitle
\subsubsection*{問1}
図\ref{fig:fig/fig1.png}のように並進ベクトルを定義する.
すなわち格子定数$a$を用いて
\begin{align}
  \vec{a}_1&=\frac{a}{2}(\hat{x}+\hat{y})\\
  \vec{a}_2&=\frac{a}{2}(\hat{y}+\hat{z})\\
  \vec{a}_3&=\frac{a}{2}(\hat{z}+\hat{x})
\end{align}
となる.このとき逆格子ベクトルは
\begin{align}
  \vec{b}_1&=2\pi\frac{\vec{a}_2\times\vec{a}_3}{\vec{a}_1\cdot(\vec{a}_2\times\vec{a}_3)}
\end{align}
ここで
\begin{align}
  \begin{split}
    \vec{a}_2\times\vec{a}_3&=\frac{a^2}{4}(\hat{y}+\hat{z})\times(\hat{z}+\hat{x})\\
    &=\frac{a^2}{4}(\hat{x}+\hat{y}-\hat{z})
  \end{split}
\end{align}
また
\begin{align}
  \vec{a}_1\cdot(\vec{a}_2\times\hat{a}_3)=\frac{a}{2}(\hat{x}+\hat{y})\cdot\frac{a^2}{4}(\hat{x}+\hat{y}-\hat{z})=\frac{a^3}{4}
\end{align}
したがって
\begin{align}
  \vec{b}_1=\frac{2\pi}{a}(\hat{x}+\hat{y}-\hat{z})
\end{align}
同様に
\begin{align}
  \vec{a}_3\times\vec{a}_1=-\vec{x}+\vec{y}+\vec{z}
\end{align}
より
\begin{align}
  \vec{b}_2=\frac{2\pi}{a}(-\hat{x}+\hat{y}+\hat{z})
\end{align}
同様に
\begin{align}
  \vec{a}_1\times\vec{a}_2=\vec{x}-\vec{y}+\vec{z}
\end{align}
より
\begin{align}
  \vec{b}_3=\frac{2\pi}{a}(\hat{x}-\hat{y}+\hat{z})
\end{align}
を得る.
\mfig[width=8cm]{fig/fig1.png}{面心立方格子での並進ベクトルの定義}
\subsubsection*{問2}
図\ref{fig:fig/fig2.png}のように並進ベクトルを定義する.
このとき並進ベクトルは
\begin{align}
  \vec{a}_1&=\frac{a}{2}(\hat{x}+\hat{y}-\hat{z})\\
  \vec{a}_2&=\frac{a}{2}(-\hat{x}+\hat{y}+\hat{z})\\
  \vec{a}_3&=\frac{a}{2}(\hat{x}-\hat{y}+\hat{z})
\end{align}
これは前問で得られた逆格子ベクトルと定数を除いて一致している.
ここで逆格子ベクトルが
\begin{align}
  \vec{a}_i\cdot\vec{b_j}=2\pi\delta_{ij}
\end{align}
で定義されたことを考えれば体心立方格子の逆格子ベクトルは
面心立方格子の並進ベクトルと平行になるべきである.
また$\vec{a_i}$と$\vec{b_i}$の内積が$2\pi$となるように係数を定めれば逆格子ベクトルは
\begin{align}
  \vec{b}_1&=\frac{2\pi}{a}(\hat{x}+\hat{y})\\
  \vec{b}_2&=\frac{2\pi}{a}(\hat{y}+\hat{z})\\
  \vec{b}_3&=\frac{2\pi}{a}(\hat{z}+\hat{x})
\end{align}
となる.
\mfig[width=8cm]{fig/fig2.png}{体心立方格子での並進ベクトルの定義}
\end{document}