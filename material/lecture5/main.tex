\documentclass[uplatex,a4j,11pt,dvipdfmx]{jsarticle}
\usepackage{listings,jvlisting}
\bibliographystyle{jplain}

\usepackage{url}

\usepackage{graphicx}
\usepackage{gnuplot-lua-tikz}
\usepackage{pgfplots}
\usepackage{tikz}
\usepackage{amsmath,amsfonts,amssymb}
\usepackage{bm}
\usepackage{siunitx}

\makeatletter
\def\fgcaption{\def\@captype{figure}\caption}
\makeatother
\newcommand{\setsections}[3]{
\setcounter{section}{#1}
\setcounter{subsection}{#2}
\setcounter{subsubsection}{#3}
}
\newcommand{\mfig}[3][width=15cm]{
\begin{center}
\includegraphics[#1]{#2}
\fgcaption{#3 \label{fig:#2}}
\end{center}
}
\newcommand{\gnu}[2]{
\begin{figure}[hptb]
\begin{center}
\input{#2}
\caption{#1}
\label{fig:#2}
\end{center}
\end{figure}
}

\begin{document}
\title{物性物理学 No.5}
\author{61908697 佐々木良輔}
\date{}
\maketitle
\subsubsection*{(1)}
$E(k)=E_0-\alpha-2\gamma\cos kb$より
\begin{align}
  \begin{split}
    \langle v(k)\rangle&=\frac{1}{\hbar}\frac{d E(k)}{d k}\\
    &=\frac{2\gamma b}{\hbar}\sin kb
  \end{split}
\end{align}
\mfig[width=6cm]{fig/fig1.jpg}{$k$-$\langle v(k)\rangle$グラフ}
\subsubsection*{(2)}
\begin{align}
  \begin{split}
    \langle a(k)\rangle&=\frac{d\langle v(k)\rangle}{dt}\\
    &=\frac{1}{\hbar}\frac{dk}{dt}\frac{d}{dk}\left(\frac{dE(k)}{dk}\right)\\
    &=\frac{1}{\hbar}\left(-\frac{eF}{\hbar}\right)\frac{d^2E(k)}{dk^2}\\
    &=-eF\cdot\frac{2\gamma b^2}{\hbar^2}\cos kb
  \end{split}
\end{align}
\mfig[width=6cm]{fig/fig2.jpg}{$k$-$\langle a(k)\rangle$グラフ}
\subsubsection*{(3)}
(2)式から
\begin{align}
  \begin{split}
    \langle a(k)\rangle\frac{\hbar^2}{2\gamma b^2\cos kb}=-eF
  \end{split}
\end{align}
なので
\begin{align}
  m^*(k)=\frac{\hbar^2}{2\gamma b^2\cos kb}
\end{align}
\mfig[width=6cm]{fig/fig3.jpg}{$k$-$m^*$グラフ}
\subsubsection*{(4)}
(1)式を$t$で積分すると
\begin{align}
  \begin{split}
    \langle x(k)\rangle&=\int\langle v(k)\rangle\ dk\\
    &=-\frac{2\gamma}{eF}\cos kb+x_0
  \end{split}
\end{align}
ここで$k(t)=k_0-eFt/\hbar$なので
\begin{align}
  \begin{split}
    \langle x(t)\rangle=-\frac{2\gamma}{eF}\cos \left(k_0-\frac{eF}{\hbar}t\right)b+x_0
  \end{split}
\end{align}
となり電子が振動することがわかる.
$k(t)\propto t$であることから図\ref{fig:fig/fig3.jpg}において$m^*$は時間経過と共に左へ単調に進行することがわかる.
すなわち有効質量は$k_0$で最小をとり,その後$t$の経過と共に増加,
$k(t)=\pi/b$で正から負に転じる,すなわち運動の方向が逆転する.
$m^*$は周期関数なので以上を繰り返すことにより,電子が振動することがわかる.
\subsubsection*{(5)}
(6)式から振動の角振動数$\omega$は
\begin{align}
  \omega=\frac{eF}{\hbar}b
\end{align}
したがって周期$T$は
\begin{align}
  T=\frac{2\pi\hbar}{eFb}
\end{align}
\subsubsection*{(6)}
(8)式からBloch振動が安定して起きるためには
\begin{align}
  b\geq\frac{2\pi\hbar}{eFT_{lim}}
\end{align}
ここで$T_{lim}\simeq 10\ \si{\femto\second}$は散乱が起きるまでの平均時間である.
これを計算すると
\begin{align}
  b\geq2.6\times10^{-6}\ \si{\metre}
\end{align}
となる.
\subsubsection*{(7)}
電子は原子核の熱振動によって散乱される.
\bibliography{ref.bib}
\end{document}