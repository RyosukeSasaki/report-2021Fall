\documentclass[uplatex,a4j,11pt,dvipdfmx]{jsarticle}
\usepackage{listings,jvlisting}
\bibliographystyle{jplain}

\usepackage{url}

\usepackage{graphicx}
\usepackage{gnuplot-lua-tikz}
\usepackage{pgfplots}
\usepackage{tikz}
\usepackage{amsmath,amsfonts,amssymb}
\usepackage{bm}
\usepackage{siunitx}

\makeatletter
\def\fgcaption{\def\@captype{figure}\caption}
\makeatother
\newcommand{\setsections}[3]{
\setcounter{section}{#1}
\setcounter{subsection}{#2}
\setcounter{subsubsection}{#3}
}
\newcommand{\mfig}[3][width=15cm]{
\begin{center}
\includegraphics[#1]{#2}
\fgcaption{#3 \label{fig:#2}}
\end{center}
}
\newcommand{\gnu}[2]{
\begin{figure}[hptb]
\begin{center}
\input{#2}
\caption{#1}
\label{fig:#2}
\end{center}
\end{figure}
}

\begin{document}
\title{物性物理学 No.7}
\author{61908697 佐々木良輔}
\date{}
\maketitle
\subsubsection*{(1-1)}
系の面積を$S$とすると波数空間において1状態が占める面積は
\begin{align}
  \frac{(2\pi)^2}{S}
\end{align}
となるので,微小面積$2\pi kdk$中に存在する状態数$dN$は
\begin{align}
  dN=2\times\frac{S}{(2\pi)^2}2\pi kdk=\frac{S}{\pi}kdk
\end{align}
となる.
\subsubsection*{(1-2)}
\begin{align}
  \frac{dE}{dk}=\frac{\hbar^2 k}{m}
\end{align}
より
\begin{align}
  dN=\frac{S}{\pi}\frac{m}{\hbar^2}dE
\end{align}
\subsubsection*{(1-3)}
$D(E)=dN/dE$と書けるので(4)より
\begin{align}
  D(E)=\frac{dN}{dE}=\frac{S}{\pi}\frac{m}{\hbar^2}
\end{align}
\subsubsection*{(2)}
波数空間で1状態が占める長さは
\begin{align}
  \frac{2\pi}{L}
\end{align}
である.
したがって波数空間での微小長さ$k$から$k+dk$中に存在する状態数$dN$は
\begin{align}
  dN=2\times\frac{L}{2\pi}dk=\frac{L}{\pi}dk
\end{align}
また$E=\hbar^2k^2/2m$より
\begin{align}
  \sqrt{E}=\frac{\hbar k}{\sqrt{2m}}
\end{align}
両辺$k$で微分すると
\begin{align}
  \begin{split}
    \frac{d\sqrt{E}}{dE}\frac{dE}{dk}&=\frac{\hbar}{\sqrt{2m}}\\
    \frac{1}{2\sqrt{E}}\frac{dE}{dk}&=\frac{\hbar}{\sqrt{2m}}\\
    \therefore\ dk&=dE\frac{1}{\hbar}\sqrt{\frac{m}{2E}}
  \end{split}
\end{align}
よって
\begin{align}
  dN=\frac{L}{\pi}\frac{1}{\hbar}\sqrt{\frac{m}{2E}}dE
\end{align}
したがって状態密度は
\begin{align}
  D(E)=\frac{dN}{dE}=\frac{L}{\hbar\pi}\sqrt{\frac{m}{2E}}
\end{align}
\end{document}