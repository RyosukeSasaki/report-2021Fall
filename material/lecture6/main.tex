\documentclass[uplatex,a4j,11pt,dvipdfmx]{jsarticle}
\usepackage{listings,jvlisting}
\bibliographystyle{jplain}

\usepackage{url}

\usepackage{graphicx}
\usepackage{gnuplot-lua-tikz}
\usepackage{pgfplots}
\usepackage{tikz}
\usepackage{amsmath,amsfonts,amssymb}
\usepackage{bm}
\usepackage{siunitx}

\makeatletter
\def\fgcaption{\def\@captype{figure}\caption}
\makeatother
\newcommand{\setsections}[3]{
\setcounter{section}{#1}
\setcounter{subsection}{#2}
\setcounter{subsubsection}{#3}
}
\newcommand{\mfig}[3][width=15cm]{
\begin{center}
\includegraphics[#1]{#2}
\fgcaption{#3 \label{fig:#2}}
\end{center}
}
\newcommand{\gnu}[2]{
\begin{figure}[hptb]
\begin{center}
\input{#2}
\caption{#1}
\label{fig:#2}
\end{center}
\end{figure}
}

\begin{document}
\title{物性物理学 No.6}
\author{61908697 佐々木良輔}
\date{}
\maketitle
\subsubsection*{(1)}
電子数密度を$n$とすると
\begin{align}
  n=\frac{2}{a^3}=4.7\times10^{28}\ \si{\metre^{-3}}
\end{align}
\subsubsection*{(2)}
$1/n=4\pi(a_Br_s)^3/3$であるので
\begin{align}
  \begin{split}
    r_s&=\frac{1}{a_B}\left(\frac{3}{4\pi n}\right)^{1/3}\\
    &\simeq 3.3
  \end{split}
\end{align}
\subsubsection*{(3)}
$\alpha=e^2/4\pi\varepsilon_0\hbar c$とすると
\begin{align}
  a_B=\frac{4\pi\varepsilon_0\hbar^2}{me^2}=\frac{\hbar}{\alpha cm}
\end{align}
\subsubsection*{(4)}
\begin{align}
  \begin{split}
    v_F&=\frac{\hbar}{m}(3\pi^2n)^{1/3}\\
    &=\frac{\hbar}{\alpha cm}\cdot\alpha c\cdot a_B\left(\frac{4\pi n}{3}\right)^{1/3}\cdot\frac{1}{a_B}\cdot\left(\frac{9}{4}\pi\right)^{1/3}\\
    &=\frac{\alpha c}{r_s}\left(\frac{9}{4}\pi\right)^{1/3}
  \end{split}
\end{align}
\subsubsection*{(5)}
(4)より光速に対するFermi速度の比は
\begin{align}
  \begin{split}
    \frac{\alpha}{r_s}\left(\frac{9}{4}\pi\right)^{1/3}&\simeq\frac{1}{137\cdot3.252}\left(\frac{9}{4}\pi\right)^{1/3}\\
    &=4.3\times10^{-3}
  \end{split}
\end{align}
\subsubsection*{(6)}
\begin{align}
  \begin{split}
    E_F&=\frac{\hbar^2}{2m}(3\pi^2n)^{2/3}\\
    &\simeq7.6\times10^{-19}\ \si{\joule}\\
    &=4.7\ \si{\electronvolt}
  \end{split}
\end{align}
\subsubsection*{(7)}
\begin{align}
  T_F=5.5\times10^4\ \si{\kelvin}
\end{align}
\end{document}