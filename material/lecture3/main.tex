\documentclass[uplatex,a4j,11pt,dvipdfmx]{jsarticle}
\usepackage{listings,jvlisting}
\bibliographystyle{jplain}

\usepackage{url}

\usepackage{graphicx}
\usepackage{gnuplot-lua-tikz}
\usepackage{pgfplots}
\usepackage{tikz}
\usepackage{amsmath,amsfonts,amssymb}
\usepackage{bm}
\usepackage{siunitx}

\makeatletter
\def\fgcaption{\def\@captype{figure}\caption}
\makeatother
\newcommand{\setsections}[3]{
\setcounter{section}{#1}
\setcounter{subsection}{#2}
\setcounter{subsubsection}{#3}
}
\newcommand{\mfig}[3][width=15cm]{
\begin{center}
\includegraphics[#1]{#2}
\fgcaption{#3 \label{fig:#2}}
\end{center}
}
\newcommand{\gnu}[2]{
\begin{figure}[hptb]
\begin{center}
\input{#2}
\caption{#1}
\label{fig:#2}
\end{center}
\end{figure}
}

\begin{document}
\title{物性物理学 No.3}
\author{61908697 佐々木良輔}
\date{}
\maketitle
\subsection*{問1}
$E_k^0$は波数$k$の平面波のエネルギーに等しいので
\begin{align}
  \begin{split}
    E_k^0-E_{k-h_1}^0&=\frac{\hbar^2k^2}{2m}-\frac{\hbar^2(k-h_1)^2}{2m}\\
    &=\frac{\hbar^2}{2m}\left(\left(\frac{\pi}{a}-\Delta k\right)^2-\left(-\frac{\pi}{a}-\Delta k\right)^2\right)\\
    &=-\frac{\hbar^2}{2m}\frac{4\pi}{a}\Delta k=\frac{2\hbar^2\pi}{ma}\Delta k=:\Delta E
  \end{split}
\end{align}
\subsection*{問2}
\begin{align}
  \begin{split}
    E_{\pm}&=\frac{1}{2}(E_k^0+E_{k-h_1}^0)\pm |V_1|\sqrt{1+\left(\frac{\Delta E}{2|V_1|}\right)^2}\\
    &=\frac{\hbar^2}{2m}\left(\left(\frac{\pi}{a}\right)^2+\left(\Delta k\right)^2\right)\pm |V_1|\sqrt{1+\left(\frac{\Delta E}{2|V_1|}\right)^2}
  \end{split}
\end{align}
ここで$\Delta E\propto\Delta k$であることから$(\Delta E/2|V_1|)^2\ll 1$とすると
\begin{align}
  \begin{split}
    E_{\pm}&=\frac{\hbar^2}{2m}\left(\left(\frac{\pi}{a}\right)^2+\left(\Delta k\right)^2\right)\pm |V_1|\left(1+\frac{1}{2}\left(\frac{\Delta E}{2|V_1|}\right)^2\right)\\
    &\simeq \frac{\hbar^2}{2m}\left(\frac{\pi}{a}\right)^2\pm |V_1|+\left(\frac{\hbar^2}{2m}\pm\frac{\hbar^4\pi^2}{2m^2a^2|V_1|}\right)(\Delta k)^2\\
    &=E_{\frac{\pi}{a}}^0\pm|V_1|+\frac{\hbar^2}{2m}\left(1\pm\frac{\hbar^2\pi^2}{ma^2|V_1|}\right)(\Delta k)^2\\
    &=E_{\frac{\pi}{a}}^0\pm|V_1|+\frac{\hbar^2}{2m}\left(1\pm 2\frac{E_{\frac{\pi}{a}}^0}{|V_1|}\right)(\Delta k)^2
  \end{split}
\end{align}
\subsection*{問3}
(3)式から分散関係は$k=\pi/a$近傍において図\ref{fig:1.jpg}のように2次曲線的な振る舞いをすると考えられる.
また$2E_{\frac{\pi}{a}}^0/|V_1|>0$なので$E_{-}$の曲率は$E_{+}$よりも小さいことがわかる.
また基底状態の分散関係は図\ref{fig:2.jpg}のようになめらかに繋がっているべきなので$E_{-}$の$(\Delta k)^2$の係数は負となるべきである.
すなわち$|V_1|$の大きさは$E_{\frac{\pi}{a}}^0$に比べて十分小さいか,せいぜい同程度であるべきであり,
そうでなければ摂動近似が成り立たないと考えられる.
\mfig[width=6cm]{1.jpg}{$k=\pi/a$近傍での振る舞い}
\mfig[width=6cm]{2.jpg}{基底状態の分散関係}
\end{document}