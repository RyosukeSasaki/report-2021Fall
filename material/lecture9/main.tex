\documentclass[uplatex,a4j,11pt,dvipdfmx]{jsarticle}
\usepackage{listings,jvlisting}
\bibliographystyle{jplain}

\usepackage{url}

\usepackage{graphicx}
\usepackage{gnuplot-lua-tikz}
\usepackage{pgfplots}
\usepackage{tikz}
\usepackage{amsmath,amsfonts,amssymb}
\usepackage{bm}
\usepackage{siunitx}

\makeatletter
\def\fgcaption{\def\@captype{figure}\caption}
\makeatother
\newcommand{\setsections}[3]{
\setcounter{section}{#1}
\setcounter{subsection}{#2}
\setcounter{subsubsection}{#3}
}
\newcommand{\mfig}[3][width=15cm]{
\begin{center}
\includegraphics[#1]{#2}
\fgcaption{#3 \label{fig:#2}}
\end{center}
}
\newcommand{\gnu}[2]{
\begin{figure}[hptb]
\begin{center}
\input{#2}
\caption{#1}
\label{fig:#2}
\end{center}
\end{figure}
}

\begin{document}
\title{物性物理学 No.9}
\author{61908697 佐々木良輔}
\date{}
\maketitle
\subsubsection*{問1}
$\langle v(\vec{k})\rangle$は
\begin{align}
  \langle v(\vec{k})\rangle=\frac{\hbar}{m}\vec{k}
\end{align}
より$\vec{k}$に対して奇関数である.
一方で$f_0(\vec{k})$は
\begin{align}
  f_0(\vec{k})=\frac{1}{\exp((E-E_F)\beta+1)}
\end{align}
であるが
\begin{align}
  E=\frac{\hbar^2}{2m}|\vec{k}|^2
\end{align}
から$E$は偶関数であるので$f_0(\vec{k})$も偶関数である.したがって問題の(6)式の被積分関数は奇関数となり,
全空間で積分すると0になる.
\subsubsection*{問2}
ドリフト速度$\vec{v}$について
\begin{align}
  \begin{split}
    -en\vec{v}&=\frac{ne^2\tau}{m}\vec{F}\\
    \therefore\ \vec{v}&=-\frac{e\tau}{m}\vec{F}
  \end{split}
\end{align}
となる.
\subsubsection*{問3}
自由電子の質量は$m_0=9.109\times10^{-31}\ \si{\kilo\gram}$,電気素量は$e=1.602\times10^{-19}$なので
\begin{align}
  |\vec{v}|=\frac{(1.602\times10^{-19})\times(1\times10^{-14})}{9.109\times10^{-31}}1\times10^3=1.759\ \si{\metre.\second^{-1}}
\end{align}
となる.
\end{document}