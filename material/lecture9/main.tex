\documentclass[uplatex,a4j,11pt,dvipdfmx]{jsarticle}
\usepackage{listings,jvlisting}
\bibliographystyle{jplain}

\usepackage{url}

\usepackage{graphicx}
\usepackage{gnuplot-lua-tikz}
\usepackage{pgfplots}
\usepackage{tikz}
\usepackage{amsmath,amsfonts,amssymb}
\usepackage{bm}
\usepackage{siunitx}

\makeatletter
\def\fgcaption{\def\@captype{figure}\caption}
\makeatother
\newcommand{\setsections}[3]{
\setcounter{section}{#1}
\setcounter{subsection}{#2}
\setcounter{subsubsection}{#3}
}
\newcommand{\mfig}[3][width=15cm]{
\begin{center}
\includegraphics[#1]{#2}
\fgcaption{#3 \label{fig:#2}}
\end{center}
}
\newcommand{\gnu}[2]{
\begin{figure}[hptb]
\begin{center}
\input{#2}
\caption{#1}
\label{fig:#2}
\end{center}
\end{figure}
}

\begin{document}
\title{物性物理学 No.8}
\author{61908697 佐々木良輔}
\date{}
\maketitle
\subsubsection*{問1}
$U(R)$の極値は
\begin{align}
  \begin{split}
    \frac{dU(R)}{dR}=-12\varepsilon\sigma^6\left(\frac{\sigma^6}{R^{13}}-\frac{1}{R^7}\right)&=0\\
    \therefore\ R&=\pm\sigma
  \end{split}
\end{align}
ただし
\begin{align}
  U(\sigma)=-\varepsilon
\end{align}
また$R\rightarrow\infty$では$U(R)\rightarrow 0$, $R\rightarrow 0$では$U(R)\rightarrow\infty$なので
ポテンシャルは図\ref{fig:fig/fig1.png}のような形になる.
\mfig[width=8cm]{fig/fig1.png}{ポテンシャルの概形}
\subsubsection*{問2}
図\ref{fig:fig/fig2.png}のように面心立方格子の単位格子を考えると,
青い原子の周りの赤い原子が最近接である.
これは$(1/2,1/2,0)$にある原子であるので
\begin{align}
  N_1=_3C_1\times 2^2=12
\end{align}
となる.
\mfig[width=8cm]{fig/fig2.png}{面心立方格子の配位数}
\subsubsection*{問3}
第2近接原子は$(1,0,0)$にある原子であるので,その数は
\begin{align}
  N_2=_3C_1\times2=6
\end{align}
また$\sqrt{(a/2)^2+(a/2)^2+0^2}=a/\sqrt{2}=R$としたので,
\begin{align}
  R_2=\sqrt{a^2+0^2+0^2}=a=\sqrt{2}R
\end{align}
となる.
\subsubsection*{問4}
第3近接原子は$(1,1/2,1/2)$にある原子であるので,その数は
\begin{align}
  N_3=_3C_1\times 2^3=24
\end{align}
またその距離は
\begin{align}
  R_3=\sqrt{1^2+\left(\frac{1}{2}\right)^2+\left(\frac{1}{2}\right)^2}a=\sqrt{\frac{3}{2}}a=\sqrt{3}R
\end{align}
となる.
\subsubsection*{問5}
(2)式の第1項について
\begin{align}
  \begin{split}
    \sum_{j\neq i}\frac{\sigma^{12}}{R_{ij}^{12}}&=\sigma^{12}\left(12\frac{1}{R^{12}}+6\frac{1}{(\sqrt{2}R)^{12}}+24\frac{1}{(\sqrt{3}R)^{12}}\right)\\
    &\simeq12.13\frac{\sigma^{12}}{R^{12}}
  \end{split}
\end{align}
また第2項について
\begin{align}
  \begin{split}
    \sum_{j\neq i}\frac{\sigma^{6}}{R_{ij}^{6}}&=\sigma^{6}\left(12\frac{1}{R^{6}}+6\frac{1}{(\sqrt{2}R)^{6}}+24\frac{1}{(\sqrt{3}R)^{6}}\right)\\
    &\simeq13.64\frac{\sigma^{6}}{R^{6}}
  \end{split}
\end{align}
となる.
\subsubsection*{問6}
問1と同様に極値を求めると
\begin{align}
  \begin{split}
    \frac{dU_S(R)}{dR}=-12\frac{N\varepsilon\sigma^6}{2}\left(A_{12}\frac{\sigma^6}{R^{13}}-A_6\frac{1}{R^7}\right)=0\\
    \therefore\ R=\pm\left(\frac{A_{12}}{A_6}\right)^{1/6}\sigma\simeq\pm0.9806\sigma
  \end{split}
\end{align}
また$U_S(R)$の極値は
\begin{align}
  \begin{split}
    U_S(R)&=\frac{N\varepsilon}{2}\left(A_{12}\left(\frac{A_6}{A_{12}}\right)^2-2A_6\left(\frac{A_6}{A_{12}}\right)^1\right)\\
    &=-\frac{N\varepsilon}{2}\left(\frac{A_6^2}{A_{12}}\right)\simeq-7.669N\varepsilon
  \end{split}
\end{align}
\subsubsection*{問7}
もっともポテンシャルが低い状態が実現するならば(10)式から相互作用下での$R$は希ガス原子対に比べて小さいことがわかる.
\end{document}