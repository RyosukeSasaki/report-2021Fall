\documentclass[uplatex,a4j,11pt,dvipdfmx]{jsarticle}
\usepackage{listings,jvlisting}
\bibliographystyle{jplain}

\usepackage{url}

\usepackage{graphicx}
\usepackage{gnuplot-lua-tikz}
\usepackage{pgfplots}
\usepackage{tikz}
\usepackage{amsmath,amsfonts,amssymb}
\usepackage{bm}
\usepackage{siunitx}

\makeatletter
\def\fgcaption{\def\@captype{figure}\caption}
\makeatother
\newcommand{\setsections}[3]{
\setcounter{section}{#1}
\setcounter{subsection}{#2}
\setcounter{subsubsection}{#3}
}
\newcommand{\mfig}[3][width=15cm]{
\begin{center}
\includegraphics[#1]{#2}
\fgcaption{#3 \label{fig:#2}}
\end{center}
}
\newcommand{\gnu}[2]{
\begin{figure}[hptb]
\begin{center}
\input{#2}
\caption{#1}
\label{fig:#2}
\end{center}
\end{figure}
}

\begin{document}
\title{物性物理学1 レポートNo.2}
\author{61908697 佐々木良輔}
\date{}
\maketitle
\subsection*{問1}
$V(x)=\infty\ (x\leq 0,a\leq x)$より境界では波動関数は0になっているべきなので,以下の境界条件を課す.
\begin{align}
  \psi(x)|_{x=0}=\psi(x)|_{x=a}=0
\end{align}
また$V(x)=0\ (0<x<a)$なのでSchr\"{o}dinger方程式は
\begin{align}
  -\frac{\hbar^2}{2m}\psi(x)=E\psi(x)
\end{align}
ここで$\psi(x)=C{\rm e}^{\lambda x}$と置くと
\begin{align}
  -\frac{\hbar^2}{2m}\lambda\psi(x)&=E\psi(x)\nonumber\\
  \therefore\ \lambda&=\pm i\frac{\sqrt{2mE}}{\hbar}=:\pm ik
\end{align}
となる.したがって$\psi(x)$は定数$A,B$を用いて
\begin{align}
  \psi(x)=A{\rm e}^{ikx}+B{\rm e}^{-ikx}
\end{align}
となる.ここで境界条件(1)を用いると
\begin{align}
  \psi(0)&=A+B=0\nonumber\\
  \therefore\ B&=-A
\end{align}
\begin{align}
  \psi(a)&=A({\rm e}^{ika}-{\rm e}^{-ika})\nonumber\\
  &=2iA\sin ka=0\nonumber\\
  \therefore\ k&=\frac{n\pi}{a}
\end{align}
(3)と(5)より
\begin{align}
  \frac{n\pi}{a}&=\frac{\sqrt{2mE}}{\hbar}\nonumber\\
  \therefore\ E_n&=\frac{\pi^2\hbar^2}{2ma^2}n^2
\end{align}
以上から$E_1$, $E_2$はそれぞれ以下のようになる.
\begin{align}
  E_1=\frac{\pi^2\hbar^2}{2ma^2}\\
  E_2=\frac{4\pi^2\hbar^2}{2ma^2}
\end{align}
\subsection*{問2}
$\alpha=\sqrt{2mE}/\hbar,\beta=\sqrt{2m(V_0-E)}/\hbar$である.
ここで$Vb$を一定にしたまま$V\rightarrow\infty,b\rightarrow 0$とすると
\begin{align*}
  \alpha(a-b)&\rightarrow\alpha a\\
  \beta b&\rightarrow 0\\
\end{align*}
また$\beta b\rightarrow 0$より
\begin{align*}
  \sinh\beta b&\simeq\beta b\\
  \cosh\beta b&\simeq 1
\end{align*}
である.したがって
\begin{align*}
  \frac{\beta^2-\alpha^2}{\beta}\sinh\beta b&\simeq b(\beta^2-\alpha^2)\\
  &=b\frac{2mV_0-4mE}{\hbar^2}\\
  &\rightarrow \frac{2mV_0b}{\hbar^2}
\end{align*}
ただし最後の極限では$V_0\gg E$とした.
したがって\textcircled{\scriptsize 7}式は
\begin{align*}
  {\rm e}^{2ika}+1&={\rm e}^{ika}\left(\frac{2mV_0b}{\alpha\hbar^2}\sin\alpha a+2\cos\alpha a\right)
\end{align*}
両辺を${\rm e}^{ika}$で割ると
\begin{align}
  {\rm e}^{ika}+{\rm e}^{-ika}&=\frac{2mV_0b}{\alpha\hbar^2}\sin\alpha a+2\cos\alpha a\nonumber\\
  \cos ka&=\frac{mV_0b}{\alpha\hbar^2}\sin\alpha a+\cos\alpha a=:f(\alpha a)
\end{align}
\newpage
\subsubsection*{(1)}
$V_0b\rightarrow 0$とすると(10)は
\begin{align}
  \cos ka&=\cos\alpha a\nonumber\\
  k&=\alpha=\frac{\sqrt{2mE}}{\hbar}\nonumber\\
  \therefore\ E&=\frac{\hbar^2k^2}{2m}
\end{align}
となり自由電子のエネルギー分散関係が得られた.
\subsubsection*{(2)}
$f(x)$をプロットすると図1のようになる.
(10)の左辺は$\cos$関数であり$-1$から$1$までの値しか取れないため,
図1の赤ハッチの領域のように$|f(\alpha a)|>1$となるエネルギー領域では解がないことがわかる.
ここで${mV_0b}/{\alpha\hbar^2}$を徐々に大きくしたときのグラフは図2のようになる.
図2から$V_0b$が大きくなるとグラフが縦に鋭くなり,解を持つエネルギー領域が狭くなっていくことがわかる.
ここで$V_0b\rightarrow\infty$とすると解は$\alpha a=n\pi$のみになる.このときのエネルギー準位は
\begin{align}
  \alpha a&=n\pi\nonumber\\
  \frac{\sqrt{2mE}}{\hbar}&=\frac{n\pi}{a}\nonumber\\
  \therefore\ E_n&=\frac{\pi^2\hbar^2}{2ma^2}n^2
\end{align}
となり,井戸型ポテンシャルと同様のエネルギー準位を得る.
\mfig[width=10cm]{fig/materialNo2-1.png}{$f(x)$のグラフ(${mV_0b}/{\alpha\hbar^2}=1$のとき)}
\mfig[width=10cm]{fig/materialNo2-2.png}{$f(x)$のグラフ(青→緑→赤の順で${mV_0b}/{\alpha\hbar^2}$が大きい)}
\end{document}