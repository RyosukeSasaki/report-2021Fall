\documentclass[uplatex,a4j,11pt,dvipdfmx]{jsarticle}
\usepackage{listings,jvlisting}
\bibliographystyle{junsrt}

\usepackage{url}

\usepackage{graphicx}
\usepackage{gnuplot-lua-tikz}
\usepackage{pgfplots}
\usepackage{tikz}
\usepackage{amsmath,amsfonts,amssymb}
\usepackage{bm}
\usepackage{siunitx}

\makeatletter
\def\fgcaption{\def\@captype{figure}\caption}
\makeatother
\newcommand{\setsections}[3]{
\setcounter{section}{#1}
\setcounter{subsection}{#2}
\setcounter{subsubsection}{#3}
}
\newcommand{\mfig}[3][width=15cm]{
\begin{center}
\includegraphics[#1]{#2}
\fgcaption{#3 \label{fig:#2}}
\end{center}
}
\newcommand{\gnu}[2]{
\begin{figure}[hptb]
\begin{center}
\input{#2}
\caption{#1}
\label{fig:#2}
\end{center}
\end{figure}
}

\begin{document}
\title{量子力学3 期末レポート}
\author{61908697 佐々木良輔}
\date{}
\maketitle
\section*{1.}
\subsection*{(a)}
動径方向の波動関数$R_l(r)$は
\begin{align}
  \left(\frac{d^2}{dr^2}+\frac{2}{r}\frac{d}{dr}-\frac{l(l+1)}{r^2}-U(r)+k^2\right)R_l(r)=0
\end{align}
\begin{align}
  U(r)=\frac{2\mu}{\hbar^2}V(r),\qquad \mu=\frac{MM}{M+M}=\frac{M}{2}
\end{align}
を満たす.まず$b<r$のとき$U(r)=0$より
\begin{align}
  \left(\frac{d^2}{dr^2}+\frac{2}{r}\frac{d}{dr}-\frac{l(l+1)}{r^2}+k^2\right)R_l(r)=0
\end{align}
ここでs波($l=0$)では
\begin{align}
  \left(\frac{1}{r}\left(\frac{d}{dr}\right)^2r+k^2\right)R_0(r)=0
\end{align}
$rR_0(r)=u_0(r)$とおくと
\begin{align}
  \left(\left(\frac{d^2}{dr^2}\right)+k^2\right)u_0(r)=0
\end{align}
したがって一般解は
\begin{align}
  u_0(r)=B\sin k(r+\alpha_1)
\end{align}
となる.
次に$c<r<b$においては$U(r)=-2\mu V_0/\hbar^2=:-U_0$より$k'^2=2\mu(E+V_0)/\hbar^2$とすると
\begin{align}
  \left(\frac{1}{r}\left(\frac{d}{dr}\right)^2r+k'^2\right)R_0(r)=0
\end{align}
で同型の方程式になので一般解は
\begin{align}
  u_0(r)=A\sin k'(r+\alpha_2)
\end{align}
である.ここで$r=c$において$u_0(r)=0$となるべきなので
\begin{align}
  u_0(r)=A\sin k'(r-c)
\end{align}
となる.
以上から一般解は
\begin{align}
  R_0(r)=\left\{
  \begin{array}{ll}
    A\sin k'(r-c)/r&(c<r<b)\\
    B\sin (kr+\delta_0)/r&(b<r)
  \end{array}
  \right.
\end{align}
\subsection*{(b)}
$r=b$において
\begin{align}
  \frac{1}{R_0}\frac{d R_0}{dr}
\end{align}
が一致すべきである.
$c<r<b$では
\begin{align*}
  \left.\frac{1}{R_0}\frac{d R_0}{dr}\right|_{r=b}&=\frac{k'b\cos k'(b-c)-\sin k'(b-c)}{b\sin k'(b-c)}\\
  &=k'\cot k'(b-c)-\frac{1}{b}=:\gamma_0
\end{align*}
となり$\gamma_0$は$k$に依らない.
一方で$b<r$では
\begin{align*}
  \left.\frac{1}{R_0}\frac{d R_0}{dr}\right|_{r=b}&=\frac{\frac{d}{dr}\left(\frac{\sin kr\cos\delta_0+\cos kr\sin\delta_0}{r}\right)}{\frac{\sin kr\cos\delta_0+\cos kr\sin\delta_0}{r}}
\end{align*}
これが$\gamma_0$に一致するので
\begin{align*}
  \gamma_0\frac{\sin kb\cos\delta_0+\cos kb\sin\delta_0}{b}&=\frac{d}{dr}\left(\frac{\sin kb\cos\delta_0+\cos kb\sin\delta_0}{b}\right)\\
  \gamma_0\left(j_0(kb)+n_0(kb)\tan\delta_0\right)&=j'_0(kb)+n'_0(kb)\tan\delta_0\\
\end{align*}
したがって
\begin{align*}
  \tan\delta_0&=-\frac{j'_0(kb)-\gamma_0j_0(kb)}{n'_0(kb)-\gamma_0n_0(kb)}\\
  &=-\frac{kb\cos kb-\sin kb-\gamma_0b\sin kb}{kb\sin kb+\cos kb+\gamma_0b\cos kb}
\end{align*}
以上から
\begin{align}
  \cot\delta_0&=-\frac{kb\sin kb+\cos kb+\gamma_0b\cos kb}{kb\cos kb-\sin kb-\gamma_0b\sin kb}
\end{align}
%$r=b$において$R_0(r)$及びその一階微分が連続であるべきなので
%\begin{align}
%  B\sin (kb+\delta)=A\sin k'(b-c)
%\end{align}
%\begin{align}
%  B(kb\cos (kb+\delta)-\sin (kb+\delta))=A(k'b\cos k'(b-c)-\sin k'(b-c))
%\end{align}
%連立して$A$, $B$を削除すると
%\begin{align*}
%  \frac{k'b\cos k'(b-c)-\sin k'(b-c)}{kb\cos (kb+\delta)-\sin (kb+\delta)}=\frac{\sin k'(b-c)}{\sin (kb+\delta)}\\[8pt]
%  \frac{k'b\cos k'(b-c)-\sin k'(b-c)}{\sin k'(b-c)}=\frac{kb\cos (kb+\delta)-\sin (kb+\delta)}{\sin (kb+\delta)}
%\end{align*}
%ここで左辺を$\gamma_0$とすると
%\begin{align*}
%  \gamma_0=\frac{kb\cos (kb+\delta)-\sin (kb+\delta)}{\sin (kb+\delta)}
%\end{align*}
\subsection*{(c)}
(12)を$k\simeq 0$で近似すると
\begin{align*}
  \cot\delta_0\simeq-\frac{k^2b^2+1+\gamma_0b}{kb-kb-\gamma_0kb^2}\simeq\frac{1+\gamma_0b}{\gamma_0kb^2}
\end{align*}
両辺に$k$をかければ
\begin{align*}
  k\cot\delta_0\simeq\frac{1+\gamma_0b}{\gamma_0b^2}=-\frac{1}{a}
\end{align*}
したがって
\begin{align*}
  a=-\frac{\gamma_0b^2}{1+\gamma_0b}=-\frac{k'b\cot k'(b-c)-1}{k'\cot k'(b-c)}
\end{align*}
ここで
\begin{align*}
  k'=\sqrt{\frac{M(E_0+V_0)}{\hbar^2}}
\end{align*}
だが$E_0\ll V_0$なので
\begin{align*}
  k'=K_0
\end{align*}
とできる.したがって
\begin{align}
  a=\frac{1-K_0b\cot K_0(b-c)}{K_0\cot K_0(b-c)}
\end{align}
また(12)を$kb=x$としてテイラー展開すると
\begin{align}
  \cot\delta_0=-\frac{x(x-x^3/6)+1-x^2/2+\gamma_0b(1-x^2/2)}{x(1-x^2/2)-x+x^3/6-\gamma_0b(x-x^3/6)}
\end{align}
したがって
\begin{align*}
  k\cot\delta_0=-\frac{1}{b}\left(\frac{1+\gamma_0b}{-\gamma_0b+x^2(\gamma_0b/6-1/3)}
  +\frac{(1/2-\gamma_0b/2)x^2}{-\gamma_0b+x^2(\gamma_0b/6-1/3)}+O(x^3)
  \right)
\end{align*}
各項をテイラー展開して整理すると
\begin{align*}
  k\cot\delta_0=\frac{1+\gamma_0b}{\gamma_0b^2}-\frac{1-(\gamma_0b)^2}{2b(\gamma_0b)^2}x^2+O(x^3)
\end{align*}
したがって
\begin{align}
  \begin{split}
    r_{\rm eff}&=\frac{1-(\gamma_0b)^2}{b(\gamma_0b)^2}\\
    &=\frac{1-(K_0\cot K_0(b-c)-1)^2}{b(K_0\cot K_0(b-c)-1)^2}
  \end{split}
\end{align}
となる.
\subsection*{(d)}
散乱振幅は
\begin{align*}
  f(\theta)=\sum_l(2l+1)f_lP_l(\cos\theta),\qquad f_l=\frac{{\rm e}^{i\delta_l}\sin\delta_l}{k}
\end{align*}
ここで
\begin{align*}
  f_l=\frac{1}{k}\frac{\sin\delta_l}{\cos\delta_l-i\sin\delta_l}=\frac{1}{k\cot\delta_l-ik}
\end{align*}
であり,今はs波のみを考えているので
\begin{align*}
  f_0=\frac{1}{-\frac{1}{a}+\frac{1}{2}r_{\rm eff}k^2-ik}=\frac{-\frac{1}{a}+\frac{1}{2}r_{\rm eff}k^2+ik}{\left(-\frac{1}{a}+\frac{1}{2}r_{\rm eff}k^2\right)^2+k^2}
\end{align*}
\begin{align*}
  P_0(\cos\theta)=1
\end{align*}
より
\begin{align}
  f(\theta)=\frac{1}{-\frac{1}{a}+\frac{1}{2}r_{\rm eff}k^2-ik}
\end{align}
したがって微分断面積は
\begin{align}
  \begin{split}
  \frac{d\sigma}{d\Omega}&=|f(\theta)|^2\\
  &=\frac{1}{\left(-\frac{1}{a}+\frac{1}{2}r_{\rm eff}k^2\right)^2+k^2}
  \end{split}
\end{align}
であり,また全断面積は
\begin{align}
  \sigma=\int\frac{d\sigma}{d\Omega}d\Omega=\frac{4\pi}{\left(-\frac{1}{a}+\frac{1}{2}r_{\rm eff}k^2\right)^2+k^2}
\end{align}
となる.
\section*{2.}
\subsection*{(a)}
ベクトルポテンシャルを用いると
\begin{align*}
  \vec{B}&=\nabla\times\vec{A}\\
  \vec{E}&=-\nabla\phi-\frac{\partial\vec{A}}{\partial t}
\end{align*}
と表される.
ここでMaxwell方程式
\begin{align*}
  \nabla\times\vec{B}=\frac{\partial\vec{E}}{\partial t}
\end{align*}
においてCoulombゲージを用いれば
\begin{align*}
  \nabla\times(\nabla\times\vec{A})=-\nabla^2\vec{A}=\partial_t\left(-\nabla\phi-\partial_t\vec{A}\right)
\end{align*}
特に真空中では$\phi=0$なので
\begin{align*}
  (\partial_t^2-\nabla^2)\vec{A}=0
\end{align*}
となり$\vec{A}$が波動方程式に従うことがわかる.
したがって$\vec{A}$を波数$k$の平面波で展開すると
\begin{align}
  \vec{A}=\sum_{\vec{k},\lambda}\hat{e}_{\vec{k},\lambda}\left(A_{\vec{k},\lambda}{\rm e}^{i(\vec{k}\cdot\vec{r}-\omega_kt)}+A_{\vec{k},\lambda}^\dagger{\rm e}^{-i(\vec{k}\cdot\vec{r}-\omega_kt)}\right)
\end{align}
ここで$\lambda=1,2$は$\vec{k}$に直行する2方向を示す添字である.
このとき
\begin{align}
  \vec{E}=\sum_{\vec{k},\lambda}\hat{e}_{\vec{k},\lambda}\left(i\omega_k A_{\vec{k},\lambda}{\rm e}^{i(\vec{k}\cdot\vec{r}-\omega_kt)}-i\omega_k A_{\vec{k},\lambda}^\dagger{\rm e}^{-i(\vec{k}\cdot\vec{r}-\omega_kt)}\right)
\end{align}
\begin{align}
  \vec{B}=\nabla\times\vec{A}=\sum_{\vec{k},\lambda}\left(A_{\vec{k},\lambda}(i\vec{k}\times\hat{e}_{\vec{k},\lambda}){\rm e}^{i(\vec{k}\cdot\vec{r}-\omega_kt)}
  A_{\vec{k},\lambda}^\dagger(-i\vec{k}\times\hat{e}_{\vec{k},\lambda}){\rm e}^{-i(\vec{k}\cdot\vec{r}-\omega_kt)}\right)
\end{align}
となる.
ここで調和振動子の生成,消滅演算子を用いると
\begin{align*}
  \hat{x}=\sqrt{\frac{\hbar}{2m\omega}}(\hat{a}+\hat{a}^\dagger),\qquad\hat{p}=\frac{1}{\sqrt{2m\hbar\omega}}\frac{\hat{a}-\hat{a}^\dagger}{i}
\end{align*}
であったが,これらと(19), (20)は同様の形をしている.
このことから調和振動子と電磁場には類似性があり,電磁場もまた生成,消滅演算子を用いて表すことができることを示唆している.
ここで電磁場のエネルギーは
\begin{align*}
  U=\int\varepsilon\vec{E}^2dV
\end{align*}
ここで$\exp(2i(\vec{k}\cdot\vec{r}-\omega_kt))$の項は全空間で積分した時に0になり,クロスタームのみが残るので
\begin{align*}
  \begin{split}
    U&=\varepsilon\int dV\sum_{\vec{k},\lambda}\omega_k^2(A_{\vec{k},\lambda}A_{\vec{k},\lambda}^\dagger+A_{\vec{k},\lambda}^\dagger A_{\vec{k},\lambda})\\
    &=\sum_{\vec{k},\lambda}V\varepsilon\omega_k^2(A_{\vec{k},\lambda}A_{\vec{k},\lambda}^\dagger+A_{\vec{k},\lambda}^\dagger A_{\vec{k},\lambda})
  \end{split}
\end{align*}
ここで
\begin{align*}
  \sqrt{V\varepsilon}\omega_kA_{\vec{k},\lambda}=\sqrt{\frac{\hbar\omega_k}{2}}\hat{a}_{\vec{k},\lambda}
\end{align*}
と置き直せば
\begin{align*}
  \sum_{\vec{k},\lambda}\frac{\hbar\omega_k}{2}(\hat{a}_{\vec{k},\lambda}\hat{a}_{\vec{k},\lambda}^\dagger+\hat{a}_{\vec{k},\lambda}^\dagger\hat{a}_{\vec{k},\lambda})
\end{align*}
となり,これは調和振動子のハミルトニアンそのものである.ここで生成消滅演算子の交換関係
\begin{align*}
  [\hat{a}_{\vec{k},\lambda},\hat{a}_{\vec{k},\lambda}^\dagger]=1
\end{align*}
を用いれば
\begin{align*}
  \sum_{\vec{k},\lambda}\hbar\omega\left(\hat{a}_{\vec{k},\lambda}\hat{a}_{\vec{k},\lambda}^\dagger+\frac{1}{2}\right)
\end{align*}
とできる.
ここで$\hat{a}_{\vec{k},\lambda}\hat{a}_{\vec{k},\lambda}^\dagger=\hat{N}$の個数演算子であるのでハミルトニアンを固有状態に作用すると
\begin{align}
  \sum_{\vec{k},\lambda}\hbar\omega\left(\hat{a}_{\vec{k},\lambda}\hat{a}_{\vec{k},\lambda}^\dagger+\frac{1}{2}\right)|n\rangle=\sum_{\vec{k},\lambda}\hbar\omega\left(n_{\vec{k},\lambda}+\frac{1}{2}\right)
\end{align}
となり$n$が光子数に対応する.このことから$n=0$の真空状態でもエネルギーは0にならず,零点振動のようにエネルギーを持っていることがわかる.
このことを真空場ゆらぎと呼ぶ.
古典電磁気学では光は波として扱われたが,以上によって光が量子化され光子という粒子性が現れた.
\subsection*{(b)}
\begin{align*}
  \hat{A}=\sum_{\vec{k},\lambda}\sqrt{\frac{\hbar}{2\omega_kL^3}}{\bm \epsilon}_{\vec{k},\lambda}\left(\hat{a}_{\vec{k},\lambda}{\rm e}^{i(kx-\omega_xt)}+\hat{a}_{\vec{k},\lambda}^\dagger{\rm e}^{-i(kx-\omega_xt)}\right)
\end{align*}
とすると
\begin{align*}
  \hat{E}=-\partial_t\hat{A}=\sum_{\vec{k},\lambda}\sqrt{\frac{\hbar}{2\omega_kL^3}}{\bm \epsilon}_{\vec{k},\lambda}\left(i\omega_k\hat{a}_{\vec{k},\lambda}{\rm e}^{i(kx-\omega_xt)}-i\omega_k\hat{a}_{\vec{k},\lambda}^\dagger{\rm e}^{-i(kx-\omega_xt)}\right)
\end{align*}
なので
\begin{align*}
  \hat{E}(t=0)\times\hat{A}(t=0)=&\sum_{\vec{k}}\frac{\hbar}{2\omega_kL^3}\sum_\lambda{\bm \epsilon}_{\vec{k},\lambda}\left(i\omega_k\hat{a}_{\vec{k},\lambda}{\rm e}^{ikx}-i\omega_k\hat{a}_{\vec{k},\lambda}^\dagger{\rm e}^{-ikx}\right)\\
  &\times\sum_{\lambda'}{\bm \epsilon}_{\vec{k},\lambda'}\left(\hat{a}_{\vec{k},\lambda'}{\rm e}^{ikx}+\hat{a}_{\vec{k},\lambda'}^\dagger{\rm e}^{-ikx}\right)\\
  =&\sum_{\vec{k}}\frac{i\hbar}{2L^3}\biggl(0\\
  &+{\bm \epsilon}_{\vec{k},1}\times{\bm \epsilon}_{\vec{k},2}\left(\hat{a}_{\vec{k},1}{\rm e}^{ikx}-\hat{a}_{\vec{k},1}^\dagger{\rm e}^{-ikx}\right)\left(\hat{a}_{\vec{k},2}{\rm e}^{ikx}+\hat{a}_{\vec{k},2}^\dagger{\rm e}^{-ikx}\right)\\
  &+{\bm \epsilon}_{\vec{k},2}\times{\bm \epsilon}_{\vec{k},1}\left(\hat{a}_{\vec{k},2}{\rm e}^{ikx}-\hat{a}_{\vec{k},2}^\dagger{\rm e}^{-ikx}\right)\left(\hat{a}_{\vec{k},1}{\rm e}^{ikx}+\hat{a}_{\vec{k},1}^\dagger{\rm e}^{-ikx}\right)\\
  &+0\biggr)
\end{align*}
ここで前問と同様に$\exp(2i(\vec{k}\cdot\vec{r}-\omega_kt))$の残った項は全空間で積分した時に0になるので
\begin{align}
  \begin{split}
    \hat{S}=&\sum_{\vec{k}}\frac{i\hbar}{2L^3}\int dV\hat{E}(t=0)\times\hat{A}(t=0)\\
    =&\sum_{\vec{k}}\frac{i\hbar}{2L^3}\int dV\hat{k}\biggl(\left(\hat{a}_{\vec{k},1}\hat{a}_{\vec{k},2}^\dagger-\hat{a}_{\vec{k},1}^\dagger\hat{a}_{\vec{k},2}\right)\\
    &-\left(\hat{a}_{\vec{k},2}\hat{a}_{\vec{k},1}^\dagger-\hat{a}_{\vec{k},2}^\dagger\hat{a}_{\vec{k},1}\right)\biggr)\\
    =&\sum_{\vec{k}}\frac{i\hbar\hat{k}}{L^3}\int dV\left(\hat{a}_{\vec{k},1}\hat{a}_{\vec{k},2}^\dagger-\hat{a}_{\vec{k},1}^\dagger\hat{a}_{\vec{k},2}\right)\\
    =&i\sum_{\vec{k}}\hbar\hat{k}\left(\hat{a}_{\vec{k},1}\hat{a}_{\vec{k},2}^\dagger-\hat{a}_{\vec{k},1}^\dagger\hat{a}_{\vec{k},2}\right)
  \end{split}
\end{align}
となる.
\subsection*{(c)}
交換関係を計算する.
\begin{align}
  \begin{split}
    [\hat{a}_{\vec{k},+},\hat{a}_{\vec{k}',+}^\dagger]&=\frac{1}{2}\left([\hat{a}_{\vec{k},1},\hat{a}_{\vec{k}',1}^\dagger]-i[\hat{a}_{\vec{k},2},\hat{a}_{\vec{k}',1}^\dagger]+i[\hat{a}_{\vec{k},1},\hat{a}_{\vec{k}',2}^\dagger]+[\hat{a}_{\vec{k},2},\hat{a}_{\vec{k}',2}^\dagger]\right)\\
    &=\delta_{\vec{k},\vec{k}'}
  \end{split}
\end{align}
\begin{align}
  \begin{split}
    [\hat{a}_{\vec{k},+},\hat{a}_{\vec{k}',-}]&=\frac{1}{2}\left([\hat{a}_{\vec{k},1},\hat{a}_{\vec{k}',1}]-i[\hat{a}_{\vec{k},2},\hat{a}_{\vec{k}',1}]+i[\hat{a}_{\vec{k},1},\hat{a}_{\vec{k}',2}]+[\hat{a}_{\vec{k},2},\hat{a}_{\vec{k}',2}]\right)\\
    &=0
  \end{split}
\end{align}
ここで
\begin{align*}
  \hat{a}_{\vec{k},+}\hat{a}_{\vec{k},+}^\dagger=\hat{a}_{\vec{k},-}\hat{a}_{\vec{k},-}^\dagger=\frac{1}{2}\left(\hat{a}_{\vec{k},1}\hat{a}_{\vec{k},1}^\dagger+i\hat{a}_{\vec{k},1}\hat{a}_{\vec{k},2}^\dagger-i\hat{a}_{\vec{k},1}^\dagger\hat{a}_{\vec{k},2}+\hat{a}_{\vec{k},2}\hat{a}_{\vec{k},2}^\dagger\right)
\end{align*}
より
\begin{align*}
  i\left(\hat{a}_{\vec{k},1}\hat{a}_{\vec{k},2}^\dagger-\hat{a}_{\vec{k},1}^\dagger\hat{a}_{\vec{k},2}\right)=2\left(\hat{a}_{\vec{k},+}\hat{a}_{\vec{k},+}^\dagger-\frac{1}{2}\right)
\end{align*}
\begin{align}
  \hat{S}=2\sum_{\vec{k}}\hbar\vec{k}\left(\hat{a}_{\vec{k},+}\hat{a}_{\vec{k},+}^\dagger-\frac{1}{2}\right)
\end{align}
となる.これは調和振動子のエネルギーであるため$\hat{S}$は電磁場のエネルギーであると考えられる.
\section*{(3)}
\subsection*{(a)}
\subsubsection*{(1)}
ディラック方程式から
\begin{align*}
  i\gamma^0\partial_0&=-i\gamma^1\partial_1+-i\gamma^2\partial_2+-i\gamma^3\partial_3+m
\end{align*}
ここで
\begin{align*}
  {\bm \alpha}=\gamma_0\left(
  \begin{array}{ccc}
    \gamma_1&\gamma_2&\gamma_3
  \end{array}
\right)
\end{align*}
とすれば
\begin{align*}
  i\partial_0={\bm \alpha}\cdot{\bm P}+\gamma_0 m
\end{align*}
であり$-\partial_t=H$なので
\begin{align}
  H_{\rm Dirac}={\bm \alpha}\cdot{\bm P}+\gamma_0 m
\end{align}
を得る.
\subsubsection*{(2)}
交換関係を計算する.
$L^i=\varepsilon_{ijk}x_jP_k$から
\begin{align*}
  [H_{\rm Dirac},L^i]&=[\alpha_lP_l,\varepsilon_{ijk}x_jP_k]\\
  &=\varepsilon_{ijk}\alpha_l\left(x_j[P_l,P_k]+[P_l,x_j]P_k\right)\\
  &=-i\varepsilon_{ijk}\alpha_jP_k
\end{align*}
である.また
\begin{align*}
  \gamma_0\gamma^i=\left(
    \begin{array}{cc}
      {\bm 1}&{\bm 0}\\
      {\bm 0}&-{\bm1}
    \end{array}
  \right)\left(
    \begin{array}{cc}
      {\bm 0}&\sigma^i\\
      -\sigma^i&{\bm 0}\\
    \end{array}
  \right)=\left(
    \begin{array}{cc}
      {\bm 0}&\sigma^i\\
      \sigma^i&{\bm 0}\\
    \end{array}
  \right)
\end{align*}
より
\begin{align*}
  \alpha^i=\left(
    \begin{array}{cc}
      {\bm 0}&\sigma^i\\
      \sigma^i&{\bm 0}\\
    \end{array}
  \right)
\end{align*}
である.ここで
\begin{align*}
  [\sigma_i,\sigma_j]=2i\varepsilon_{ijk}\sigma_k
\end{align*}
また
\begin{align*}
  \alpha^iS^j=S^j\alpha^i=\frac{1}{2}\left(
    \begin{array}{cc}
      {\bm 0}&\sigma_i\sigma_j\\
      \sigma_i\sigma_j&{\bm 0}\\
    \end{array}
  \right)
\end{align*}
から
\begin{align}
  \begin{split}
    [H_{\rm Dirac},S^j]&=[\alpha_iP_i,S^j]\\
    &=[\alpha_i,S^j]P_i\\
    &=\frac{1}{2}\left(
      \begin{array}{cc}
        {\bm 0}&\sigma_i\sigma_j-\sigma_j\sigma_i\\
        \sigma_i\sigma_j-\sigma_j\sigma_i&{\bm 0}\\
      \end{array}
    \right)P_i\\
    &=i\varepsilon_{ijk}\alpha_kP_i\\
    &=i\varepsilon_{jki}\alpha_kP_i
  \end{split}
\end{align}
となる.以上から
\begin{align}
  [H_{\rm Dirac},J_i+S_i]i\varepsilon_{ijk}(\alpha_jP_k-\alpha_jP_k)=0
\end{align}
となり,全角運動量が保存していることがわかる.
\subsection*{(b)}
\begin{align*}
  \vec{p}=p\left(
    \begin{array}{c}
      \sin\theta\cos\phi\\
      \sin\theta\sin\phi\\
      \cos\theta
    \end{array}
  \right)
\end{align*}
とおくと
\begin{align*}
  {\bm \sigma}\cdot\vec{p}=p\left(
    \begin{array}{cc}
      \cos\theta&\sin\theta{\rm e}^{-i\phi}\\
      \sin\theta{\rm e}^{i\phi}&-\cos\theta\\
    \end{array}
  \right)
\end{align*}
このとき固有値方程式
\begin{align*}
  0=&|{\bm \sigma}\cdot\vec{p}-{\bm I}\lambda|\\
  =&\lambda^2-p^2\cos^2\theta-p^2\sin^2\theta
\end{align*}
より
\begin{align}
  \lambda=\pm p
\end{align}
$\lambda=1$に対応する固有ベクトルは
\begin{align*}
  \begin{split}
    \vec{x}_1&=(p\cos\theta-p,p\sin\theta{\rm e}^{i\phi})\\
    &=(p\sin\theta{\rm e}^{-i\phi},p\cos\theta+p)
  \end{split}
\end{align*}
$\lambda=-1$に対応する固有ベクトルは
\begin{align*}
  \vec{x}_2&=(p\cos\theta+p,p\sin\theta{\rm e}^{i\phi})
\end{align*}
である.それぞれ規格化すれば
\begin{align}
  \psi_1=\frac{1}{\sqrt{2(1+\cos\theta)}}(\sin\theta{\rm e}^{-i\phi},\cos\theta+1)
\end{align}
\begin{align}
  \psi_2=\frac{1}{\sqrt{2(1+\cos\theta)}}(\cos\theta+1,\sin\theta{\rm e}^{i\phi})
\end{align}
したがってベリー接続は
\begin{align*}
  a_{1,p}=i\langle\psi_1|\frac{d}{dr}|\psi_1\rangle=0
\end{align*}
\begin{align*}
  a_{1,\theta}&=i\langle\psi_1|\frac{1}{p}\frac{d}{d\theta}|\psi_1\rangle\\
  &=\frac{i\sin\theta}{4r(1+\cos\theta)^2}\left(\begin{array}{c}
    \sin^2\theta\\
    (\cos\theta+1)^2
  \end{array}\right)+
  \frac{i}{2r(1+\cos\theta)}\left(\begin{array}{c}
    \sin\theta\cos\theta\\
    (1-\sin\theta)(1+\cos\theta)
  \end{array}\right)
\end{align*}
\begin{align*}
  a_{1,\phi}&=i\langle\psi_1|\frac{1}{r\sin\theta}\frac{d}{d\phi}|\psi_1\rangle\\
  &=\frac{-1}{2r(1+\cos\theta)}\left(\begin{array}{c}
    \sin\theta\\0
  \end{array}\right)
\end{align*}
したがってベリー曲率は
\begin{align*}
  {\bm b}_p&=\nabla_p\times{\bm a}_p\\
  &=
\end{align*}
\bibliography{ref.bib}
\end{document}