\documentclass[uplatex,a4j,11pt,dvipdfmx]{jsarticle}
\usepackage{listings,jvlisting}
\bibliographystyle{jplain}

\usepackage{url}

\usepackage{graphicx}
\usepackage{gnuplot-lua-tikz}
\usepackage{pgfplots}
\usepackage{tikz}
\usepackage{amsmath,amsfonts,amssymb}
\usepackage{bm}
\usepackage{siunitx}

\makeatletter
\def\fgcaption{\def\@captype{figure}\caption}
\makeatother
\newcommand{\setsections}[3]{
\setcounter{section}{#1}
\setcounter{subsection}{#2}
\setcounter{subsubsection}{#3}
}
\newcommand{\mfig}[3][width=15cm]{
\begin{center}
\includegraphics[#1]{#2}
\fgcaption{#3 \label{fig:#2}}
\end{center}
}
\newcommand{\gnu}[2]{
\begin{figure}[hptb]
\begin{center}
\input{#2}
\caption{#1}
\label{fig:#2}
\end{center}
\end{figure}
}

\begin{document}
\subsection*{3.2.5 補足}
元の問題
\begin{align}
  {\bm S}\cdot\hat{\bm n}|\alpha\rangle=\frac{\hbar}{2}|\alpha\rangle
\end{align}
はスピン1/2が$\hat{\bm n}$の方向を向いている固有状態$|\alpha\rangle$を求めることにほかならない.
したがってこの問題を行列表現した
\begin{align}
  {\bm \sigma}\cdot\hat{\bm n}\chi=\chi
\end{align}
においても$\chi$は$\hat{\bm n}$を向いているべきである.したがって$\chi$は教科書(2.52)の方法で求まる.
\bibliography{ref.bib}
\end{document}